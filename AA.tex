 
We will need few integral formulas based on the Bochner formula.
We made these calculations based on \cite[Chapter II]{lawson-michelsohn}.
First let us state the Bochner formula for function with Dirichlet boundary condition.

\begin{thm}{Proposition}\label{prop:bochner-dirichle-old}
Assume $\Omega$ is a compact domain with smooth boundary $\partial \Omega$ in a Riemannian manifold
and $f$ is a smooth function that vanish on $\partial \Omega$.
Then
\[\int\limits_\Omega |\Delta f|^2
-|\mathrm{Hess}f|^2
-\langle\mathrm{Ric}(\nabla f),\nabla f\rangle
=\int\limits_{\partial\Omega}
H\cdot|\nabla f|^2,\]
where $H$ denotes mean curvature of $\partial \Omega$.
\end{thm}
 
\begin{thm}{Corollary}
Assume $\Omega$ is a compact domain with smooth boundary $\partial \Omega$ in a 2-dimensional Riemannian manifold with nonnegaive curvature
and $f$ is a smooth concave function that vanish on $\partial \Omega$.
Then
\[\int\limits_\Omega 
\det(\mathrm{Hess}f)
\le\pi\cdot\sup_{x\in\partial\Omega}|\nabla_x f|^2.\]

\end{thm}






Let $M$ be a Riemannian manifold.
Denote by $\nabla$ the Levi-Cevitta connection on $M$.
The bundle $\LT M$ of multivectors over $M$ is equipped with Clifford product, denoted by $\,\bullet \,$.
We will denote by $e_i$ is an orthonormal frame at a point; the following definitions will not depend on its choice.

\parbf{Laplasians.}
The Dirac operator on differential forms forms will be denoted by $D$;
it is defined as
\[D=\sum_i e_i\bullet \nabla_{e_i}.\]
Its square 
\[D^2=\sum_i e_i\bullet e_j\bullet \nabla^2_{e_i,e_j}\]
is called Hodge laplacian.

The Dirac operator if \emph{formally self-adjoint}, in particular,
\[\int_M \langle D^2\phi,\psi\rangle=\int_M \langle D\phi,D\psi\rangle\]
for any two vector fields $\phi$ and $\psi$ with compact support.

Further, define the connection laplacian
\[\nabla^*\nabla\phi =-\sum_i\nabla^2_{e_i,e_i}\phi\]
and the gradient
\[\nabla \phi=\sum e_i\otimes \nabla_{e_i}\phi.\]

For the connection laplacian we also have the identity
\[\int_M \langle \nabla^*\nabla\phi,\psi\rangle
=
\int_M \langle \nabla\phi,\nabla\psi\rangle.\]
If $\phi$ and $\psi$ have support in the domain where the frame is defined, then the right handside can be written as 
\[\int_M \langle \nabla\phi,\nabla\psi\rangle=\sum_i\int_M\langle \nabla_{e_i}\phi,\nabla_{e_i}\psi\rangle;\]
using the a partition of unity, one can use the latter expression to redefine the left hand side. 

\parbf{Bochner formula.}
The difference $D^2-\nabla^*\nabla$ between two laplasians described above is a 0-order differential operator which can be written in terms of curvature.
For a vector field $v$, the formula is 
\[D^2v-\nabla^*\nabla v=\Ric(v).\]

Using the identities above, the formula can be written in an integral form
\[\int_M \langle Dv,Dw\rangle -\langle \nabla v,\nabla w\rangle=\langle \Ric(v),w\rangle\]
In particular, if $w=\phi \cdot v$ for a smooth function $\phi$;
we get
\[\int_M \langle Dv,D(\phi\cdot v)\rangle -\langle \nabla v,\nabla(\phi\cdot v) \rangle=\phi\cdot\langle \Ric(v), v\rangle,\]
or, equivalently
\[\int_M \phi\cdot(\langle Dv,D v\rangle -\langle \nabla v,\nabla v \rangle)
+
\int_M (\langle Dv,\nabla \phi \bullet v\rangle -\langle \nabla v,\nabla v \rangle)
=
\phi\cdot\langle \Ric(v), v\rangle,\]
\parbf{Relative formulas.}
For a domain $\Omega$ with boundary $\partial \Omega$, the formula above takes form
\[\int_\Omega (\langle D^2\phi,\psi\rangle- \langle D\phi,D\psi\rangle)
=
\int_{\partial \Omega}\langle \nu\bullet D\phi,\psi\rangle,\]
where $\nu$ is the outer normal field on $\partial \Omega$.

The square $D^2$ of the Dirac operator is called Hodge laplacian.


For a domain $\Omega$ with boundary $\partial \Omega$, the formula above takes form
\[\int_\Omega (\langle \nabla^*\nabla\phi,\psi\rangle-\langle \nabla\phi,\nabla\psi\rangle)
=
\int_{\partial \Omega}\langle \nabla_\nu \phi,\psi\rangle,\]
where $\nu$ is the outer normal field on $\partial \Omega$.

The Bochner's identity for a vector field $u$ can be written as
\[D^2 u-\nabla^*\nabla u=\Ric(u).\]
relates the Dirac's laplasian $D^2$ and the connection laplasian $\nabla^*\nabla$

In this section we will write an integral version of the Bochner formula \cite[8.3]{lawson-michelsohn}
\[D^2-\nabla^*\nabla=\Ric\]
in such a way that each term has geometric meaning.


Further $\nabla^*\nabla=-\sum_i\nabla^2_{e_i,e_i}$ is the connection Laplacian.
