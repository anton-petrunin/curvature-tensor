\section{Gauss curvature}
\label{sec:gauss}

\begin{thm}{Theorem}\label{thm:lens}
Let $\eps$ be a sufficiently small positive number and 
$M$ be a 3-dimensional manifold with nonnegative sectional curvature.
Assume two smooth discs $\Delta_1$ and $\Delta_2$ in $M$ bound a convex set and meet the common boundary at angle at most $\varepsilon$.
Then 
\[\int_{\Delta_1}k_1\cdot k_2\le 10\cdot \eps^2,\]
where $k_1,k_2$ denote the principle curvatures of $\Delta_1$.
\end{thm}

\begin{thm}{Lemma}\label{lem:bochner}
Assume $\Omega$ is a compact domain with smooth boundary $\partial \Omega$ in a 2-dimensional Riemannian manifold with nonnegaive curvature
and $f$ is a smooth concave function that vanish on $\partial \Omega$.
Then
\[\int\limits_\Omega 
\det(\mathrm{Hess}f)
\le\pi\cdot\sup_{x\in\partial\Omega}|\nabla_x f|^2.\]

\end{thm}

The proof of the lemma is based on the 2-dimesional case  following statement,
which is the integral Bochner formula written for function with Dirichlet boundary condition.

\begin{thm}{Proposition}\label{prop:bochner-dirichle}
Assume $\Omega$ is a compact domain with smooth boundary $\partial \Omega$ in a Riemannian manifold
and $f$ is a smooth function that vanish on $\partial \Omega$.
Then
\[\int\limits_\Omega |\Delta f|^2
-|\mathrm{Hess}f|^2
-\langle\mathrm{Ric}(\nabla f),\nabla f\rangle
=\int\limits_{\partial\Omega}
H\cdot|\nabla f|^2,\]
where $H$ denotes mean curvature of $\partial \Omega$.
\end{thm}

\parit{Proof of the lemma.}
Denote by  $\lambda_1,\lambda_2$ the eigenvalues of  $\Hess f$.
Then
\begin{align*}
\trace(\Hess f)&=\Delta f=\lambda_1+\lambda_2,
\\
|\mathrm{Hess}f|^2&=\lambda_1^2+\lambda_2^2,
\\
\det(\Hess f)&=\lambda_1\cdot\lambda_2.
\end{align*}
Therefore 
\[2\cdot\det(\Hess f)
=|\Delta f|^2
-|\mathrm{Hess}f|^2.\] 

Since $\Omega$ is two-dimensional, the calculations above and Proposition~\ref{prop:bochner-dirichle} imply that
\[\int\limits_\Omega 
2\cdot\det(\mathrm{Hess}f)
=\int\limits_\Omega 
K\cdot|\nabla f|^2
+
\int\limits_{\partial\Omega}
\kappa\cdot|\nabla f|^2,\]
where $\kappa$ denotes the geodesic curvature of $\partial \Omega$ and $K$ the curvature of the manifold; the sign of $\kappa$ is chosen so that $\kappa\ge 0$ if $\Omega$ is convex.

Since $f$ is concave and vanish on the boundary of $\Omega$,
it has unique critical point, which is its maximum.
By Morse lemma, the closure $\bar\Omega$ is a disc.
Therefore, by Gauss--Bonnet formula, we get that
\[\int_\Omega K+\int_{\partial\Omega}\kappa=2\cdot\pi.\]
Therefore 
\[\int\limits_\Omega 
\det(\mathrm{Hess}f)
\le\pi\cdot\sup_{x\in\Omega}|\nabla_x f|^2.\]

Given $x\in\Omega$ consider the geodesic $\gamma$ starting in $x$ in the direction of $-\nabla_xf$.
Since $f$ is concave, $|\nabla_yf|>|\nabla_xf|$ for any $y\in \gamma\backslash\{x\}$.
Whence the maximum of $|\nabla_xf|$ is admitted only on the boundary.
That is,
\[\sup_{x\in\partial\Omega}|\nabla_x f|^2=\sup_{x\in\Omega}|\nabla_x f|^2,\]
hence the result.
\qeds


\parit{Proof.}
Denote by $L$ the set (lens) bounded by $\Delta_1$ and $\Delta_2$.
Let us show that the function $f=\dist_{\Delta_2}$ is concave in $L$.

Given $p\in L$ denote by $\bar p$ a closest point in $\Delta_2$;
since $L$ is concave, geodesic $[\bar pp]$ lies in $L$.
The angle between the geodesic $[\bar pp]$ and any direction in $\Delta_1$ has to be at least $\tfrac\pi2$; otherwise one could decrease the distance $|p-\bar p|$ by moving $\bar p$ in such a direction.

If $\bar p\in\partial \Delta_2$, then by assumption the angle to one of such directions has to be smaller than $\eps$, which contradicts the assumption if $\eps<\tfrac\pi2$.

In the remaining case $\bar p\not\in\partial \Delta_2$, by second variation formula, $f$ admits an upper barrier at $p$;
that is, given $\eps>0$, there is a smooth function $h$ defined in a neighborhood of $p$ such that $h''\le \eps$ and
$h(x)\ge f(x)$ if $h(x)$ is defined and the equality holds for $x=p$.
The latter implies that $f$ is concave.

Next note that the restriction $\bar f=f|_{\Delta_1}$ is concave.
Denote by $\nu$ the orthonormal vector field to $\Delta_1$ pointing outside $L$.
Since $L$ is convex any geodesic $[p\bar p]\subset L$ for any $p\in \Delta_1$.
It follows that $f$ increase in the direction of $\nu$; that is 
\[df(\nu)\ge0.
\eqlbl{nuf>=0}\] 
Any geodesic in $\Delta_1$ curves in the direction of $-\nu$;
by \ref{nuf>=0}, 
since $f$ is concave, so is $\bar f$. %???

Next let us show that
\[df(\nu)\ge \cos\eps.\eqlbl{nuf}\]
Indeed, if at some point $p\in \Delta_1$ the inequality \ref{nuf} does not hold,
then for some unit vector $u$ tangent to $\Delta_1$ at $p$ we have 
\[df(u)<-\sin\eps.\]
Consider the unit speed geodesic $\gamma$ in $\Delta_1$ starting from $p$ in the direction $u$.
Since $f$ is concave, the inequality \ref{nuf>=0} implies that $t\mapsto f\circ\gamma(t)$ is concave and therefore function $t\mapsto (f\circ\gamma)'(t)$ is nonincreasing.
Whence at the moment $t_{{\max}}$, 
when $\gamma$ hits the boundary $\partial\Delta_1$,
we have that 
\begin{align*}
(f\circ\gamma)'(t_{{\max}})&\le(f\circ\gamma)'(0)=
\\&=df(u)<
\\&<-\sin\eps.
\end{align*}
It follows that, $\gamma$ hits the common boundary $\partial\Delta_2=\partial\Delta_1$ at the angle larger than $\eps$ to $\Delta_2$.
The latter contradicts the assumption.

If $\bar f$ is smooth, then \ref{nuf} implies that 
\[\Hess f\ge \cos\eps\cdot \mathrm{I\!I},\]
where $\mathrm{I\!I}$ denotes the second fundamental form of $\Delta_1$.
In particular,
\[\det[\Hess f]\ge (\cos\eps)^2\cdot k_1\cdot k_2.\]
In this case, applying Lemma~\ref{lem:bochner} we get the needed estimate.

In case if $\bar f$ is not smooth, one can apply smoothing by convolution and pass to the limit to conclude the same.
\qeds
