\section{Gauss and mean curvature estimate}

\begin{thm}{Theorem}\label{thm:extimage-of-G-and-H}
Let $f$, $h$ be a pair of strongly convex smooth 1-Lipschitz functions defined on an open set of a 3-dimensional Riemannian manifold.
Suppose that (1)
\[|\nabla (f+h)|<\eps\cdot|\nabla f|\quad\text{and}\quad |\nabla f|\ge 1\] 
for some fixed positive $\eps<\tfrac12$
and (2) for some $a,b\in \RR$, the set
\[W_{a,b}=\set{p\in M}{f(p)=a, h(p)\le b}\]
is compact.
Denote by $k_1(p)\le k_2(p)$ the principle curvatures of $W_{a,b}$ at a point $p\in W_{a,b}$.
Let $H(p)=k_1(p)+ k_2(p)$ and
$G(p)=k_1(p)\cdot k_2(p)$ its mean and Gauss curvatures.
Then
\[\int_{W_{a,b}}G\le 100\cdot\eps
\eqlbl{intG<100eps}\]
and 
\[\int_{W_{a,b}}H\le 10\cdot \sqrt{\eps}\cdot \length(\partial{W_{a,b}}).
\eqlbl{intH<eps}\]
\end{thm}

The proof is based on the 2-dimesional case  following statement,
which is the integral Bochner formula written for function with Dirichlet boundary condition.

\begin{thm}{Proposition}\label{prop:bochner-dirichle}
Assume $\Omega$ is a compact domain with smooth boundary $\partial \Omega$ in a Riemannian manifold
and $f$ is a smooth function that vanish on $\partial \Omega$.
Then
\[\int\limits_\Omega |\Delta f|^2
-|\Hess f|^2
-\langle\mathrm{Ric}(\nabla f),\nabla f\rangle
=\int\limits_{\partial\Omega}
H\cdot|\nabla f|^2,\]
where $H$ denotes mean curvature of $\partial \Omega$.
\end{thm}

\parit{Proof.}
Equip $W_{a,b}$ with unit normal vector field $n=\tfrac{\nabla f}{|\nabla f|}$.
Let $S_p\:\T_pW_{a,b}\zz\to \T_pW_{a,b}$ be the corresponding shape operator, so $S_p\:v\mapsto\nabla_vn$.
Since $f$ is strongly convex, we have that 
\[\langle S_p(v),v\rangle\ge \delta\cdot|v|^2\]
for a fixed value $\delta>0$ and any tangent vector $v\in \T_pW_{a,b}$. 

Note that the restriction of $u=h|_{W_{a,b}}$ is strongly convex.
Moreover 
\[\Hess_pu(v,v)\ge (1-\eps)\cdot \langle S_p(v),v\rangle\eqlbl{Hess=<shape}\]
for any $p\in W_{a,b}$ and $v\in\T_pW_{a,b}$.

Indeed, consider the geodesic $\gamma$ in $W_{a,b}$ such $\gamma(0)=p$ and $\gamma'(0)=v$.
Set $a=\gamma''(t)$.
Note that 
\begin{align*}
a&=-\langle S_p(v),v\rangle \cdot n,
\intertext{Since $h$ is is strongly convex, $\Hess_p h\ge 0$; therefore}
(\Hess_pu)(v,v)&=(\Hess_p h)(v,v)+\langle \nabla_p h,a\rangle\ge
\\
&\ge-\tfrac{\langle\nabla_p h,\nabla_p f\rangle}{|\nabla_p f|}\cdot\langle S_p(v),v\rangle\ge
\\
&\ge (1-\eps)\cdot|\nabla_p f|\cdot\langle S_p(v),v\rangle.
\end{align*}
Since $\nabla f\ge 1$, \ref{Hess=<shape} follows.

Since $\langle S_p(v),v\rangle\ge 0$ and $\eps<\tfrac12$, the inequality \ref{Hess=<shape} implies that 
\[4\cdot \det(\Hess_pu)\ge G(p)
\eqlbl{eq:det>=G}\]
and
\[-2\cdot \Delta u\ge H(p)
\eqlbl{eq:trace>=H}\]
for any $p\in W_{a,b}$.

Denote by  $\lambda_1(p),\lambda_2(p)$ the eigenvalues of  $\Hess_p u$, so
\begin{align*}
\trace(\Hess u)&=\Delta u=\lambda_1+\lambda_2,
\\
|\Hess u|^2&=\lambda_1^2+\lambda_2^2,
\\
\det(\Hess u)&=\lambda_1\cdot\lambda_2.
\end{align*}
Therefore 
\[2\cdot\det(\Hess u)
=|\Delta u|^2
-|\Hess u|^2.\] 

Since $W_{a,b}$ is two-dimensional, by Proposition~\ref{prop:bochner-dirichle} we get that
\[\int\limits_{W_{a,b}} 
2\cdot\det(\Hess u)
=\int\limits_{W_{a,b}} 
K\cdot|\nabla u|^2
+
\int\limits_{\partial W_{a,b}}
\kappa\cdot|\nabla u|^2,\]
where $\kappa\ge 0$ is the geodesic curvature of $\partial W_{a,b}$
and $K$ is the curvature of $W_{a,b}$.

Since $u$ is convex and vanish on the boundary of $W_{a,b}$,
it has unique critical point, which is its minimum.
By Morse lemma,  $W_{a,b}$ is a disc.
Therefore, by Gauss--Bonnet formula, we get that
\[\int_{W_{a,b}} K+\int_{\partial{W_{a,b}}}\kappa=2\cdot\pi.\]
Whence 
\[\int\limits_{W_{a,b}} 
\det(\Hess u)
\le\pi\cdot\sup_{p\in{W_{a,b}}}|\nabla_p u|^2.\]

Note that $\nabla_p u$ is projection of $\nabla_ph$ to $\T_pW_{a,b}$.
Therefore
\begin{align*}
|\nabla_p u|^2&=|\nabla_p h|^2-\langle\nabla_p h,n\rangle^2\le
\\
&\le1-(1-\eps)^2<
\\
&<2\cdot\eps.
\end{align*}
It follows that 
\[\int\limits_{W_{a,b}} 
\det(\Hess u)
\le2\cdot \pi\cdot\eps.\]
Applying \ref{eq:det>=G}, we obtain \ref{intG<100eps}.

Similarly,  by the divergence theorem, we get that
\[-\int\limits_{W_{a,b}} \Delta u=\int\limits_{\partial{W_{a,b}}} |\nabla u|.\]
Whence \ref{eq:trace>=H} implies 
\[\int\limits_{W_{a,b}} H\le 10\cdot \sqrt{\eps}\cdot \length(\partial{W_{a,b}}).\]
\qeds

