\section{Gauss curvature}
\label{sec:gauss}

\begin{thm}{Theorem}\label{thm:lens}
Let $\eps$ be a sufficiently small positive number and 
$M$ be a 3-dimensional manifold with nonnegative sectional curvature.
Assume two smooth discs $\Delta_1$ and $\Delta_2$ in $M$ bound a convex set and meet the the common boundary at angle at most $\varepsilon$.
Then 
\[\int_{\Delta_1}k_1\cdot k_2\le 10\cdot \eps^2,\]
where $k_1,k_2$ denote the principle curvatures of $\Delta_1$.
\end{thm}

\begin{thm}{Lemma}\label{lem:bochner}
Assume $\Omega$ is a compact domain with smooth boundary $\partial \Omega$ in a 2-dimensional Riemannian manifold with nonnegaive curvature
and $f$ is a smooth concave function that vanish on $\partial \Omega$.
Then
\[\int\limits_\Omega 
\det(\mathrm{Hess}f)
\le\pi\cdot\sup_{x\in\partial\Omega}|\nabla_x f|^2.\]

\end{thm}

The proof of the lemma is based on the 2-dimesional case  following statement,
which is the integral Bochner formula written for function with Dirichlet boundary condition.

\begin{thm}{Proposition}\label{prop:bochner-dirichle}
Assume $\Omega$ is a compact domain with smooth boundary $\partial \Omega$ in a Riemannian manifold
and $f$ is a smooth function that vanish on $\partial \Omega$.
Then
\[\int\limits_\Omega |\Delta f|^2
-|\mathrm{Hess}f|^2
-\langle\mathrm{Ric}(\nabla f),\nabla f\rangle
=\int\limits_{\partial\Omega}
H\cdot|\nabla f|^2,\]
where $H$ denotes mean curvature of $\partial \Omega$.
\end{thm}

\parit{Proof of the lemma.}
Denote by  $\lambda_1,\lambda_2$ the eigenvalues of  $\Hess f$.
Then
\begin{align*}
\trace(\Hess f)&=\Delta f=\lambda_1+\lambda_2,
\\
|\mathrm{Hess}f|^2&=\lambda_1^2+\lambda_2^2,
\\
\det(\Hess f)&=\lambda_1\cdot\lambda_2.
\end{align*}
Therefore 
\[2\cdot\det(\Hess f)
=|\Delta f|^2
-|\mathrm{Hess}f|^2.\] 
It follows that the formula in Proposition~\ref{prop:bochner-dirichle} can be simplified the following way:
\[\int\limits_\Omega 
2\cdot\det(\mathrm{Hess}f)
=\int\limits_\Omega 
K\cdot|\nabla f|^2
+
\int\limits_{\partial\Omega}
\kappa\cdot|\nabla f|^2,\]
where $\kappa$ denotes the geodesic curvature of $\partial \Omega$ and $K$ the curvature of the manifold; the sign of $\kappa$ is chosen so that $\kappa\ge 0$ if $\Omega$ is convex.

Since $f$ is concave, smooth and vanish on the boundary of $\Omega$,
by Morse lemma, the closure $\bar\Omega$ is a disc.
Therefore, by Gauss--Bonnet formula, we get that
\[\int_\Omega K+\int_{\partial\Omega}\kappa=2\cdot\pi.\]
Therefore 
\[\int\limits_\Omega 
\det(\mathrm{Hess}f)
\le\pi\cdot\sup_{x\in\Omega}|\nabla_x f|^2.\]

Given $x\in\Omega$ consider the geodesic $\gamma$ starting in $x$ in the direction of $-\nabla_xf$.
Since $f$ is concave, $|\nabla_yf|>|\nabla_xf|$ for any $y\in \gamma\backslash\{x\}$.
Whence the maximum of $|\nabla_xf|$ is admitted only on the boundary.
That is,
\[\sup_{x\in\partial\Omega}|\nabla_x f|^2=\sup_{x\in\Omega}|\nabla_x f|^2,\]
hence the result.
\qeds


\parit{Proof.}
We will construct a convex function $f\:\Delta_1\to \RR$ with the following properties:
$2\cdot\det(\mathrm{Hess}f)\ge k_1\cdot k_2$,
$f(x)=0$ and
$\nabla_x f<\eps$ for any $x\in \partial\Delta_1$.

Note that once the function $f$ is constructed, the theorem~\ref{thm:lens} will follow from Lemma~\ref{lem:bochner}.


Let $f$ be the restriction of $\dist_{\Delta_2}$ on $\Delta_1$.
By assumption $f\:\Delta_1\to \RR$ vanish on the boundary $\partial\Delta_1$
and
$|\nabla_x f|\le \eps$ for any $x\in \partial \Delta_1$.
Let us show that
\begin{enumerate}[(i)]
\item $\langle\nu,\nabla_x f\rangle\ge \cos\eps$,
for any $x\in\Delta_1$, where $\nu$ denotes the outer unit normal vector to $\Delta_1$. 
\item $f\:\Delta_1\to \RR$ is concave function,
\end{enumerate}






\qeds



\begin{thm}{Theorem}
Let $S$ be a convex surface in a three-dimensional Riemannian manifold $M$.
Denote by $\nu$ the normal vector to $S$ in the outward direction.
Assume $f$ is a function defined in a neigborhood of $S$ such that 
$\langle \nabla f,\nu\rangle$
\end{thm}



\begin{thm}{Theorem}
Let $M$ be a 3-dimensional manifold with sectional curvature at least $\kappa$ and $L$ be a smooth convex surface in $M$ with boundary, denote by $n$ the field of unit normal vectors on $L$ that points outside of the convex region that $L$ cuts from $M$.
Assume there is a smooth concave function $f$ such that $f$ vanish at the boundary $\partial L$,
the gradient $\<\nabla f,n\>\ge 1$ at any point of $L$.
Then the integral of Gauss curvature on $L$ does not exceed???
\end{thm}

We will need an other integral reformation of Bochner's identity.

\begin{thm}{Bound on determinant of Hessian}
Let $g$ be a Riemannian metric with nonnegative curvature on the 2-dimensional disc $\DD$.
Assume $f\:\DD\zz\to \RR$ is a smooth concave function that vanish on the boundary $\partial \DD$.
Then 
\[\int_\DD \det(\Hess f)\le \pi\cdot \sup\{|\nabla f|\}. \]
\end{thm}

\parit{Proof.}
By relative Bochner's formula, we have
\[\int_\DD(\Delta f)^2-|\Hess f|^2-K\cdot |\nabla f|^2=\int_{\partial\DD} \kappa\cdot |\nabla f|^2,\]
where $K$ denotes the curvature of $g$ and $\kappa$ --- geodesic curvature of $\partial\DD$.

If $\lambda_1\le \lambda_2$ denote the eigenvalues of $\Hess f$,
then 
\[\Delta f=\lambda_1+\lambda_2\quad\text{and}\quad |\Hess f|^2=\lambda_1^2+\lambda_2^2;\]
Therefore 
\begin{align*}
(\Delta f)^2-|\Hess f|^2&=2\cdot\lambda_1\cdot \lambda_2=
\\
&=2\cdot\det(\Hess f).
\end{align*}

Since $f$ is concave and vanish on $\partial\DD$, the disc $\DD$ has convex boundary, that is, $\kappa\ge 0$.
Further, from by Gauss--Bonnet formula, we get that
\[\int_\DD K+\int_{\partial\DD} \kappa =2\cdot \pi.\]
Whence, 
\begin{align*}
\int_\DD 2\cdot(\det\Hess f)
&=
\int_\DD K\cdot |\nabla f|^2
+
\int_{\partial\DD} \kappa\cdot |\nabla f|^2\le 
\\
&\le 2\cdot \pi \cdot \sup\{\,|\nabla_x f|^2\,\},
\end{align*}
hence the result. 
\qeds
