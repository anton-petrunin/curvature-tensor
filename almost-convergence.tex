\section{Rough convergence}
\label{sec:rough-convergence}

Assume $f$ is a semiconcave function defined on an open subset $\Dom f$ of Alexandrov space $A$.
For a point $p\in \Dom f$ set 
\[\delta_f(p)=\limsup_{x\to p}\{|\nabla_x f|-|\nabla_p f|\}.\]
The value $\delta_f(p)$ should be considered as a measure of nonsmoothness of the function $\tfrac{f}{|\nabla_p f|}$ at $p$.
Namely we say that a function $f$ is \emph{differentiabale} at $p\in\Dom f$
if $d_pf$ is a linear function of $\T_p$, that is, $d_pf$ is concave and convex at the same time.
In this case %???ref
\[|\nabla_p f|=\limsup_{x\to p}\{|\nabla_x f|\]
or equvaently $\delta_f(p)=0$.
In particular, $\delta_f(p)=0$ almost everywhere. %???ref


In this section we will use the Bochner formula (\ref{thm:bochner-formula}) to prove the following weaker version of the main theorem.
It tells that the two partial limits of strange curvature are close to each other. 

\begin{thm}{Proposition}
Let $(M_n)$ be a sequence of $m$-dimensional Riemannian manifolds with curvature $\ge\kappa$
that converges to an Alexandrov space $A$ without collapse.
Assume a sequence of points $p_n\in M_n$ converges to a point $p\in A$ 
and $f_n$ is a sequence of concave functions without critical points defined in $B(p_n,r)_{M_n}$ that converge to a function $f$ defined in  $B(p,r)_{A}$.
Set $u_n=\nabla f_n/|\nabla f_n|$.  

Consier the sequence of measure $\mu_n=\Str(u_n)\cdot \vol$ in $B(p_n,r)_{M_n}$.
Assume $\mu_n$ has two partial limits $\mu$ and $\nu$ on $B(p,r)_{A}$.
Then 
\[|\mu(S)-\nu(S)|
<
(\tfrac{\delta_f}{|\nabla f|}\cdot\mu)(S),\]
where $S\subset B(p,r)_{A}$ is any Borel set; we assume that $\frac{\delta_f}{|\nabla f|}=\infty$ at the points where $|\nabla f|=0$.
\end{thm}


