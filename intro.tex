\section{Introduction}

The meaning of weak convergence and measure valued tenosor used in the following theorem are introduced in the next section.

\begin{thm}{Main theorem}
Let $\kappa$ be a real number and 
$M_n$ be a sequence of $m$-dimensional Riemannian manifolds with sectional curvature bounded below by~$\kappa$.
Assume $M_n$ converges to an Alexandrov space $A$ of the same dimension.
Then the curvature tenors of $M_n$ converge weakly to a measured-valued tensor on $A$.
\end{thm}


Unlike the main theorem, the following corollary requires no new definitions; we denote by $\Sc$ the scalar curvature of Riemannian manifold.

\begin{thm}{Corollary}
In the assumption of the main theorem,
the measures $\Sc\cdot\vol_m$ on $M_n$ weakly converges to a locally finite measure on $A$.

In particular, if $A$ is compact, then the sequence
\[s_n=\int_{M_n}\Sc\]
converges.
\end{thm}

Note that the limits above, it particular $s$, depend only on $A$ and do not depend on the choice of the sequence $M_n$.
Indeed, if for an other sequence $M_n'$ satisfying the assumptions of the corollary and converging to the same Alexandrov space $A$ we would get a different limit, 
then a contradiction would occur for the alternated sequence $M_1,M_1',M_2,M_2',\dots$

Also let us mention that the main theorem in \cite{petrunin-SC} states in particular that if a sequence of complete $m$-dimensional Riemannian  manifolds $M_n$ has uniformly bounded diameter and uniform lower curvature bound, then 
the corresponding sequence $s_n$ is bounded; in particular, it has a converging subsequence.
However if $M_n$ is collapsing this sequence may not converge; 
for example an alternating sequence of flat 2-toruses and round 2-spheres might collapse to the one-point space, in this case the sequence $s_n$ is $0,4\cdot\pi,0,4\cdot\pi,\dots$.




\section{Weak convergence}

Let $\bm{f}=f_1,\dots,f_{m-2}$, 
$\bm{h}=h_1,\dots,h_{m-2}$ be two $(m-2)$-arrays of smooth functions on a Riemannian manifold $M$.

Given a point $p\in M$ consider a pair of tangent vectors $X,Y\in \T_p$ such that 
$d_pf_i(X)=d_pf_i(Y)=0$ for any $i$ and $|X\wedge Y|=|df_1\wedge \dots\wedge f_{m-2}|$.
Note that these two conditions completely determine the bivector $X\wedge Y$.

Let $M_n$ be a sequence of $m$-dimensional Riemannian manifolds with sectional curvature bounded below by $\kappa$.
Assume the sequence $M_n$ converges to an Alexandrov space $A$ of the spame dimension.
We will show that the curvature tensors of $M_n$ weakly converge to a measured valued tensor on $A$.

A formal definition of weak convergence will be given in the preliminaries,
but informally it could be described the following way.
We introduce a sufficiently large class of test sequences of functions $f_n\:M_n\to\RR$ and $f\:A\to \RR$ such that $f_n$ converges to $f$ in a strong sense.

Given a collection of $(m-2)$-array of smooth functions $\bm{f}=(f_1,\dots f_{m-2})$ of $M_n$,
there is unique bivector field $\bm{f}^*=X\wedge Y$ such that $df_i(X)=df_i(Y)=0$ for any $i$ and 
$|X\wedge Y|=|df_1\wedge \dots\wedge f_{m-2}|$.
For two $(m-2)$-arrays $\bm{f}=(f_1,\dots f_{m-2})$ and $\bm{h}=(h_1,\dots h_{m-2})$, consider the corresponding bivector fields $\bm{f}^*=X\wedge Y$ and $\bm{h}^*=V\wedge W$ and set
\[R_{\bm{f},\bm{h}}:=\langle R(X,Y)V,W\rangle.\]
The function $R_{\bm{f},\bm{h}}\:M\to \RR$

The weak convergence has to be defined.
To define weak convergence we introduce test function
