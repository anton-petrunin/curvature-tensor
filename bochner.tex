\section{Bochner formula}
\label{sec:Bochner}

We will need few integral formulas based on the Bochner formula.
We made these calculations based on \cite[Chapter II]{lawson-michelsohn}.
First let us state the Bochner formula for function with Dirichlet boundary condition.

\begin{thm}{Proposition}\label{prop:bochner-dirichle-old}
Assume $\Omega$ is a compact domain with smooth boundary $\partial \Omega$ in a Riemannian manifold
and $f$ is a smooth function that vanish on $\partial \Omega$.
Then
\[\int\limits_\Omega |\Delta f|^2
-|\mathrm{Hess}f|^2
-\langle\mathrm{Ric}(\nabla f),\nabla f\rangle
=\int\limits_{\partial\Omega}
H\cdot|\nabla f|^2,\]
where $H$ denotes mean curvature of $\partial \Omega$.
\end{thm}
 
\begin{thm}{Corollary}
Assume $\Omega$ is a compact domain with smooth boundary $\partial \Omega$ in a 2-dimensional Riemannian manifold with nonnegaive curvature
and $f$ is a smooth concave function that vanish on $\partial \Omega$.
Then
\[\int\limits_\Omega 
\det(\mathrm{Hess}f)
\le\pi\cdot\sup_{x\in\partial\Omega}|\nabla_x f|^2.\]

\end{thm}




In this section we give necessary versions of the Bochner formula.
The calculations are based on \cite[Chapter II]{lawson-michelsohn}.
We use Riemannian metric to identify differential forms and multivector fields on $M$.
Therefore the statement about differential forms can be also formulated in terms of multivector fields and the other way around.


Let $M$ be a Riemannian manifold.
Denote by $\nabla$ the Levi-Cevitta connection on $M$.
The bundle $\LT M$ of multivectors over $M$ is equipped with Clifford product, denoted by $\,\bullet \,$.
We will denote by $e_i$ is an orthonormal frame at a point; the following definitions will not depend on its choice.

\parbf{Laplasians.}
The Dirac operator on differential forms forms will be denoted by $D$;
it is defined as
\[D=\sum_i e_i\bullet \nabla_{e_i}.\]
Its square 
\[D^2=\sum_i e_i\bullet e_j\bullet \nabla^2_{e_i,e_j}\]
is called Hodge laplacian.

The Dirac operator if \emph{formally self-adjoint}, in particular,
\[\int_M \langle D^2\phi,\psi\rangle=\int_M \langle D\phi,D\psi\rangle\]
for any two vector fields $\phi$ and $\psi$ with compact support.

Further, define the connection laplacian
\[\nabla^*\nabla\phi =-\sum_i\nabla^2_{e_i,e_i}\phi\]
and the gradient
\[\nabla \phi=\sum e_i\otimes \nabla_{e_i}\phi.\]

For the connection laplacian we also have the identity
\[\int_M \langle \nabla^*\nabla\phi,\psi\rangle
=
\int_M \langle \nabla\phi,\nabla\psi\rangle.\]
If $\phi$ anad $\psi$ have support in the domain where the frame is defined, then the right handside can be written as 
\[\int_M \langle \nabla\phi,\nabla\psi\rangle=\sum_i\int_M\langle \nabla_{e_i}\phi,\nabla_{e_i}\psi\rangle;\]
using the a partition of unity, one can use the latter expression to redefine the left hand side. 

\parbf{Bochner formula.}
The difference $D^2-\nabla^*\nabla$ between two laplasians described above is a 0-order differential operator which can be written in terms of curvature.
For a vector field $v$, the formula is 
\[D^2v-\nabla^*\nabla v=\Ric(v).\]

Using the identities above, the formula can be written in an integral form
\[\int_M \langle Dv,Dw\rangle -\langle \nabla v,\nabla w\rangle=\langle \Ric(v),w\rangle\]
In particular, if $w=\phi \cdot v$ for a smooth function $\phi$;
we get
\[\int_M \langle Dv,D(\phi\cdot v)\rangle -\langle \nabla v,\nabla(\phi\cdot v) \rangle=\phi\cdot\langle \Ric(v), v\rangle,\]
or, equivalently
\[\int_M \phi\cdot(\langle Dv,D v\rangle -\langle \nabla v,\nabla v \rangle)
+
\int_M (\langle Dv,\nabla \phi \bullet v\rangle -\langle \nabla v,\nabla v \rangle)
=
\phi\cdot\langle \Ric(v), v\rangle,\]
\parbf{Relative formulas.}
For a domain $\Omega$ with boundary $\partial \Omega$, the formula above takes form
\[\int_\Omega (\langle D^2\phi,\psi\rangle- \langle D\phi,D\psi\rangle)
=
\int_{\partial \Omega}\langle \nu\bullet D\phi,\psi\rangle,\]
where $\nu$ is the outer normal field on $\partial \Omega$.

The square $D^2$ of the Dirac operator is called Hodge laplacian.


For a domain $\Omega$ with boundary $\partial \Omega$, the formula above takes form
\[\int_\Omega (\langle \nabla^*\nabla\phi,\psi\rangle-\langle \nabla\phi,\nabla\psi\rangle)
=
\int_{\partial \Omega}\langle \nabla_\nu \phi,\psi\rangle,\]
where $\nu$ is the outer normal field on $\partial \Omega$.

The Bochner's identity for a vector field $u$ can be written as
\[D^2 u-\nabla^*\nabla u=\Ric(u).\]
relates the Dirac's laplasian $D^2$ and the connection laplasian $\nabla^*\nabla$

In this section we will write an integral version of the Bochner formula \cite[8.3]{lawson-michelsohn}
\[D^2-\nabla^*\nabla=\Ric\]
in such a way that each term has geometric meaning.


Further $\nabla^*\nabla=-\sum_i\nabla^2_{e_i,e_i}$ is the connection Laplacian.



Let $M$ be Riemannian $m$-manifold and $f\:M\to\R$ be a smooth function without critical points on an open domain $\Omega\i M$.
Assume $\phi\:\Omega\to\R$ be a smooth function with compact support.
Set $u=\nabla f/|\nabla f|$.
Let us define $\Int_f(x)$ (or just $\Int$) to be scalar curvature of the level set $L_x=f^{-1}(f(x))$ at $x\in L_x\i M$.
Set
\begin{enumerate}
 \item $\kappa_1(x)\le\kappa_2(x)\le\dots\le\kappa_{m-1}(x)$ the principle curvatures of $L_x$ at $x$;
 \item $H_f=H_f(x)=\kappa_1+\kappa_2+\dots+\kappa_{m-1}$ is mean curvature of $L_x$ at $x$
\item $G_f=G_f(x)=2\sum_{i<j}\kappa_i\cdot\kappa_j$ is the extrinsic term
 in the Gauss formula for $\Int_f(x)$. 
\end{enumerate}

Let us define the strange curvature as
\[\Str(u)=\Sc-\<\Ric(u),u\>,\]
where $\Sc$ and $\Ric$ denote scalar and Ricci curvature correspondingly.

\begin{thm}{Bochner's formula}\label{thm:bochner-formula}
Let $M$ be an $m$-dimensional Riemannian manifold,
$f\:M\to\R$ be a smooth function without critical points on an open domain $\Omega\subset M$ and $u=\nabla f/|\nabla f|$.
Assume $\phi\:\Omega\to\R$ is a smooth function with compact support.
Then 
\[\int\limits_\Omega \phi\cdot \langle \Ric u,u\rangle =
\int\limits_\Omega [\phi \cdot G_f+H_f\cdot\<u,\nabla\phi\>-\<\nabla\phi,\nabla_u u\>]
\]
and, therefore,
$$\int\limits_\Omega \phi\cdot \Str(u)
=
\int\limits_\Omega \l[H\cdot\<u,\nabla\phi\>- \<\nabla\phi,\nabla_u u\> \r]+
\int\limits_\Omega \phi\cdot \Int_f.
\eqlbl{Bochner}$$
\end{thm}

In the proof of the main result we will 


\parit{Proof.}
Assume $b_1,\dots, b_m$ is an orthonormal frame such that $b_m=u$, 
then 
\[\Sc-2\cdot \<\Ric(u),u\>=2\cdot \sum_{i<j<m} \sec(b_i\wedge b_j).\] 
Therefore the Gauss formula can be written as
\[
\begin{aligned}
\Int_f&=G_f+\Sc-2\cdot \<\Ric(u),u\>=
\\
&=G_f+\Str(u)+ \<\Ric(u),u\>.
\end{aligned}
\eqlbl{eq:gauss}
\]

We can choose the frame $b_i$ so that $b_m=u$ and such that $b_i$ points in the principle directions of the level set $L_x$ for $i<m$.
Note that $\<\nabla_u u,u\>=0$, therefore
\begin{align*}
Du&=\sum_{i} b_i\bullet  \nabla_{b_i}u=
\\
&=\sum_{i<m}\kappa_i\cdot  b_i\bullet  b_i+u\bullet  \nabla_{u}u=
\\
&=
\sum_{i=1}^{m}\kappa_i+u\wedge\nabla_{u}u=H+u\wedge\nabla_{u}u,
\end{align*}
here ``$\,\bullet \,$'' denotes the Clifford multiplication.
Applying again that $\<\nabla_u u,u\>=0$, we get that
$$ \langle Du,Du \rangle=
\l(\sum_{i<m}\kappa_i\r)^2+|\nabla_{u}u|^2=H_f^2+|\nabla_{u}u|^2.$$
On the other hand
$$\nabla u=\sum_{i<m}\kappa_i\cdot b_i\otimes b_i+\nabla_u u\otimes u,$$
hence
$$\langle\nabla u,\nabla u\rangle =
\sum_{i<m}\kappa_i^2+|\nabla_{u}u|^2.$$

Therefore
$$\langle D u,D u\rangle-\langle \nabla u,\nabla u \rangle =2\cdot\sum_{i<j}\kappa_i\cdot\kappa_j=G_f.$$

Further,
\begin{align*}
\int\limits_\Omega\phi\cdot\l[\<D u,D u\>-\<D^2 u, u\>\r]
&=
\int\limits_\Omega\<\nabla\phi\bullet u,D u\>
=
\\
&=
-\int\limits_\Omega\l[H\cdot\<\nabla\phi,u\>- \<\nabla\phi,\nabla_u u\> \r],
\end{align*}
here ``$\bullet $'' denotes Kliford's multiplication.

Since $| u|\equiv 1$, we have $\<\nabla_{\nabla\phi}  u, u\>=0$.
Therefore
$$\int\limits_\Omega\phi\cdot\l[\<\nabla u,\nabla u\>-\<\nabla^*\nabla u, u\>\r]
=
\int\limits_\Omega\<\nabla_{\nabla\phi}  u, u\>=0.$$

Let us write Bochner formula \cite[8.3]{lawson-michelsohn} for field $u$:
$$D^2u-\nabla^*\nabla u=\Ric(u);$$
in particular, 
$$\phi\cdot \<D^2u,u\>-\phi\cdot \<\nabla^*\nabla u,u\>=\phi\cdot \<\Ric(u),u\>.\eqlbl{eq:prebochner}$$
Integrating \ref{eq:prebochner}, using the given calculations we get that
\begin{align*}
\int\limits_\Omega \phi \cdot G_f
&=\int\limits_\Omega\phi\cdot(\<D u,D u\>-\<\nabla u,\nabla u\>)
=
\\
&=\int\limits_\Omega (\phi\cdot \Ric(u,u) 
-
H_f\cdot\<u,\nabla\phi\>+ \<\nabla\phi,\nabla_u u\>).
\end{align*}

It remains to apply the Gauss formula \ref{eq:gauss}.
\qeds
