\section{Bochner formula}
\label{sec:Bochner}

In this section we will rewrite the Bochner formula, in such a way that each term has geometric meaning.

Let $M$ be Riemannian $m$-manifold and $f\:M\to\R$ be a smooth function without critical points on an open domain $\Omega\i M$.
Assume $\phi\:\Omega\to\R$ be a smooth function with compact support.
Set $u=\nabla f/|\nabla f|$.
Let us define $\Int_f(x)$ (or just $\Int$) to be scalar curvature of the level set $L_x=f^{-1}(f(x))$ at $x\in L_x\i M$.
Set
\begin{enumerate}
 \item $\kappa_1(x)\le\kappa_2(x)\le\dots\le\kappa_{m-1}(x)$ the principle curvatures of $L_x$ at $x$;
 \item $H_f=H_f(x)=\kappa_1+\kappa_2+\dots+\kappa_{m-1}$ is mean curvature of $L_x$ at $x$
\item $G_f=G_f(x)=2\sum_{i<j}\kappa_i\cdot\kappa_j$ is the extrinsic term
 in the Gauss formula for $\Int_f(x)$. 
\end{enumerate}

Let us define the strange curvature as
\[\Str(u)=\Sc-\<\Ric(u),u\>,\]
where $\Sc$ and $\Ric$ denote scalar and Ricci curvature correspondingly.

\begin{thm}{Bochner's formula}\label{thm:bochner-formula}
Let $M$ be an $m$-dimensional Riemannian manifold,
$f\:M\to\R$ be a smooth function without critical points on an open domain $\Omega\subset M$ and $u=\nabla f/|\nabla f|$.
Assume $\phi\:\Omega\to\R$ is a smooth function with compact support.
Then 
$$\int\limits_\Omega \phi\cdot \Str(u)
=
\int\limits_\Omega \l[H\cdot\<u,\nabla\phi\>- \<\nabla\phi,\nabla_u u\> \r]+
\int\limits_\Omega \phi\cdot \Int_f.
\eqlbl{Bochner}$$
\end{thm}

In the proof of the main result we will 


\parit{Proof.}
Assume $b_1,\dots, b_m$ is an orthonormal frame such that $b_m=u$, 
then 
\[\Sc-2\cdot \<\Ric(u),u\>=2\cdot \sum_{i<j<m} \sec(b_i\wedge b_j).\] 
Therefore the Gauss formula can be written as
\[
\begin{aligned}
\Int_f&=G_f+\Sc-2\cdot \<\Ric(u),u\>=
\\
&=G_f+\Str(u)+ \<\Ric(u),u\>.
\end{aligned}
\eqlbl{eq:gauss}
\]

We can choose the frame $b_i$ so that $b_m=u$ and such that $b_i$ points in the principle directions of the level set $L_x$ for $i<m$.
Note that $\<\nabla_u u,u\>=0$, therefore
\begin{align*}
Du&=\sum_{i} b_i\bullet  \nabla_{b_i}u=
\\
&=\sum_{i<m}\kappa_i\cdot  b_i\bullet  b_i+u\bullet  \nabla_{u}u=
\\
&=
\sum_{i=1}^{m}\kappa_i+u\wedge\nabla_{u}u=H+u\wedge\nabla_{u}u,
\end{align*}
here ``$\,\bullet \,$'' denotes the Clifford multiplication.
Applying again that $\<\nabla_u u,u\>=0$, we get that
$$ \langle Du,Du \rangle=
\l(\sum_{i<m}\kappa_i\r)^2+|\nabla_{u}u|^2=H_f^2+|\nabla_{u}u|^2.$$
On the other hand
$$\nabla u=\sum_{i<m}\kappa_i\cdot b_i\otimes b_i+\nabla_u u\otimes u,$$
hence
$$\langle\nabla u,\nabla u\rangle =
\sum_{i<m}\kappa_i^2+|\nabla_{u}u|^2.$$

Therefore
$$\langle D u,D u\rangle-\langle \nabla u,\nabla u \rangle =2\cdot\sum_{i<j}\kappa_i\cdot\kappa_j=G_f.$$

Further,
\begin{align*}
\int\limits_\Omega\phi\cdot\l[\<D u,D u\>-\<D^2 u, u\>\r]
&=
\int\limits_\Omega\<\nabla\phi\bullet u,D u\>
=
\\
&=
-\int\limits_\Omega\l[H\cdot\<\nabla\phi,u\>- \<\nabla\phi,\nabla_u u\> \r],
\end{align*}
here ``$\bullet $'' denotes Kliford's multiplication.

Since $| u|\equiv 1$, we have $\<\nabla_{\nabla\phi}  u, u\>=0$.
Therefore
$$\int\limits_\Omega\phi\cdot\l[\<\nabla u,\nabla u\>-\<\nabla^*\nabla u, u\>\r]
=
\int\limits_\Omega\<\nabla_{\nabla\phi}  u, u\>=0.$$

Let us write Bochner formula \cite[8.3]{lawson-michelsohn} for field $u$:
$$D^2u-\nabla^*\nabla u=\Ric(u);$$
in particular, 
$$\phi\cdot \<D^2u,u\>-\phi\cdot \<\nabla^*\nabla u,u\>=\phi\cdot \<\Ric(u),u\>.\eqlbl{eq:prebochner}$$
Integrating \ref{eq:prebochner}, using the given calculations we get that
\begin{align*}
\int\limits_\Omega \phi \cdot G_f
&=\int\limits_\Omega\phi\cdot(\<D u,D u\>-\<\nabla u,\nabla u\>)
=
\\
&=\int\limits_\Omega \phi\cdot \Ric(u,u) 
-
\int\limits_\Omega H_f\cdot\l[\<u,\nabla\phi\>- \<\nabla\phi,\nabla_u u\> \r].
\end{align*}

It remains to apply the Gauss formula \ref{eq:gauss}.
\qeds