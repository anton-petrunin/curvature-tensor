\section{Cone}

Let $M$ be a 3-dimensional Riemannian manifold.
Given a smooth function $f$ denote, denote by $L_c$ the level set of $f$;
that is,
\[L_{f=c}=\set{x\in M}{f(x)=c}.\]
If $L_c$ is a smooth surfaces in a neighborhood of $x\in L_c$,
then denote $k_1(x)\le k_2(x)$ the principle curvatures of $L_c$ at $x$
and set 
\begin{align*}
G_f(x)&=k_1(x)\cdot k_2(x),
\\
H_f(x)&=k_1(x)+ k_2(x).
\end{align*}


\begin{thm}{Theorem}
Let $M_n$ be a sequence of $3$-dimensional Riemannian manifolds
and $f_n\: M_n\to \RR$ be a sequence of strongly concave functions.
Assume $M_n\to A$ and $f_n\to f\:A\to \RR$ as $n\to \infty$.
Suppose that $f$ is affine; that is, $f$ is convex and concave at the same time.
Then $G_{f_n}$ and $H_{f_n}$ weakly converge to zero.
\end{thm}

\parit{Proof.}
Fix $p\in A$; set $a=f(p)$.
By \cite{LS}, a neighborhood of $p$ can be embedded into product $\RR\times A'$ where $A'$ is another Alexandrov space and $f$ is the first coordinate function.

Let us apply Perelman's construction of strongly concave function in a neighborhod of $p$ for points lying almost above $p$ in the product $A'\times \RR$.
This can be done in such a way that the obtained strongly concave function $h\:A\to \RR$ has maximum at $p$ in the set $f(x)\ge a$ and
is close to $-f$.
The latter means that
given $\eps>0$, we can assume that 
\[\<\nabla f,\nabla h\>+1<\eps,\]
$h$ is $(-\eps)$-concave.

Choose $b<h(p)$ so that the set 
\[W=\set{x\in A}{f(x)=a, h(x)\le b}\]
is compact.
Set $\ell=\length\partial W$.

Denote by $h_n$ the liftings of $h$ to $M_n$; since $h$ is given by Perelman's construction we can assume that they are $(-\eps)$-concave for all large~$n$.
Set 
\begin{align*}
W_n&=\set{x\in A}{f_n(x)=a, h_n(x)\le b},
\\
\ell_n&=\length\partial W_n.
\end{align*}

Note that the set $W_n$ is compact for large $n$ and $\ell_n\to \ell$ as $n\to\infty$.
The latter statement can be checked directly,
or one can apply \cite[Theorem 1.2]{petrunin-GC} for the sequence 
\[A_n=\set{x\in M_n}{f_n(x)\ge a,\, h_n(x)\ge b}.\]
Applying \ref{thm:extimage-of-G-and-H}, we get that 
\begin{align*}
\int_{W_n}G_{f_n}&\le 100\cdot\eps,
\\
\int_{W_n}H_{f_n}&\le 10\cdot \sqrt{\eps}\cdot \ell_n.
\end{align*}

Set $L_{a,n}=\set{}{}$
Fix sufficiently small $r>0$.
Observe that the value $\eps$ above can be chosen arbitrary close to $0$.
Further the value $b$ can be chosen so that $W_n\supset B(p,r)\cap L_{f_n=a}$.
It follows that the integrals of $G_{f_n}$ and $H_{f_n}$ over $B(p,r)\cap L_{f_n=a}$ converge to zero as $n\to\infty$.

Since $f_n$ is convex, and close to an affine function,
its gradient is bounded away from zero for all large $n$.
By the coarea formula, the statement follows.
\qeds
