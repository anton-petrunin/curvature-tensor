\section{Cone}

Let $M$ be a 3-dimensional Riemannian manifold.
Given a smooth function $f$ denote, denote by $L_{(f=c)}$ the level set of $f$;
that is,
\[L_{(f=c)}=\set{x\in M}{f(x)=c}.\]
If $L_{(f=c)}$ is a smooth surfaces in a neighborhood of $x\in L_{(f=c)}$,
then denote $k_1(x)\le k_2(x)$ the principle curvatures of $L_{(f=c)}$ at $x$
and set 
\begin{align*}
G_f(x)&=k_1(x)\cdot k_2(x),
\\
H_f(x)&=k_1(x)+ k_2(x);
\end{align*}
that is, $G_f(x)$ and $H_f(x)$ are Gauss and mean curvature of $L_{(f=c)}$ at $x$.

\begin{thm}{Theorem}
Let $M_n$ be a sequence of $3$-dimensional Riemannian manifolds
and $f_n\: M_n\to \RR$ be a sequence of strongly concave functions.
Suppose that $\sec M_n\ge -\tfrac1n$ for each $n$, $M_n\zz\to \RR\times A$ without collapsing and $f_n$ converges to the $\RR$-coordinate in $\RR\times A$ as $n\to \infty$.
Then $G_{f_n}$ and $H_{f_n}$ weakly converge to zero.
\end{thm}

\parit{Proof.}
Choose $p\in \RR\times A$; set $a=f(p)$.

By theorem of Artem Nepechiy \cite{Nepechiy},
there is a $(-2)$-concave function $\rho$ defined in an $r$-neighborhood of $p$ such that $\rho(x)=-|p-x|^2+o(|p-x|^2)$.
Moreover, the function $\rho$ is \emph{liftable};
that is, there is a sequence of $(-2)$-concave $\rho_n\:M_n\to\RR$ that converges to $\rho$.

Consider a point $q\in \RR\times A$ above $p$; that is, its $\RR$-coordinate of is larger and its $A$-coordinate is the same.
If $\RR$-coordinate of $q$ is large then $\dist_q+f$ is $\lambda$-concave for small $\lambda>0$ and it has a nonstrict minimum at $p$.
Therefore, given $\lambda>0$, we can find $q$ so that the sum $s=f+\dist_q+\lambda\cdot \rho$ is $(-\lambda)$-concave and has strict maximum at $p$.
Moreover
\[-\lambda\cdot|p-x|_A^2\ge s(x)-s(p)\ge -\tfrac12\cdot\lambda\cdot|p-x|_A^2\]
and therefore
\[|\nabla_xs|\le 10\cdot\lambda\cdot|p-x|_A
\eqlbl{O(p-x)}\]
if $|p-x|_A$ is sufficiently small; say if $|p-x|_A\le \tfrac r{10}$.

Choose a sequence of points $q_n\in M_n$ that converges to $q$ and set $h_n=\dist_{q_n}+\lambda\cdot \rho_n$;
this is a sequence of liftings of $h=\dist_q+\lambda\cdot \rho$. 
Observe that \ref{O(p-x)} implies that the first condition in \ref{thm:extimage-of-G-and-H} is met for all large $n$ in an $\tfrac r2$-neighborhod of $p_n$ with $\eps=10\cdot\lambda\cdot r$.
Moreover one can choose $b$ so that the second condition is satisfied and $B_n=B(p_n,r/10)\cap L_{(f=a)}\subset W_{a,b}$.
Applying \ref{thm:extimage-of-G-and-H}, we get that for any $\delta>0$, we have 
\[
\int_{B_n}G_{f_n}<\delta,
\quad\text{and}\quad
\int_{B_n}H_{f_n}<\delta.
\]
for all large $n$.
It remains to integrate the obtained inequalities by $a$ and pass to a limit as $n\to\infty$.
\qeds
