\section{Approximation}


Let $A$ be a 3-dimensional Alexandrov space without boundary. 
Let us partition $A$ into three subsets $A^\circ$, $A'$ and $A''$:
\begin{itemize}
\item $A^\circ$ is the set of regular points in $A$; that is, the set of points with tangent cone isometric to the Euclidean space.
\item $A'$ --- the set of points in $A\backslash A^\circ$ with a line in their tangent space; that is, for any $p\in A'$, the tangent space $\T_p$ is isometric to the product of the real line and a two-dimensional cone with total angle $\theta(p)<2\cdot \pi$.
\item $A''$ --- the remaining set; this is the set of points with tangent space that does not contain a line.
\end{itemize}
According to \cite{li-naber}, $A'$ is countably 1-rectifiable and $A''$ is a countable set. 

Consider the measure $\omega$ on $A'$ defined by
\[\omega=(2\cdot\pi-\theta)\cdot \mathcal{H}_1,\eqlbl{eq:omega}\]
where $\mathcal{H}_\alpha$ denotes $\alpha$-dimensional Hausdorff measure.
Let us extend $\omega$ to the whole space by setting $\omega(S)=\omega(S\cap A')$ for any Borel set $S\subset A$.

The function $\theta$ can be extended to the whole $A$ by setting
\[\theta(p)=2\cdot\pi\cdot \tfrac{\area \Sigma_p}{\area \mathbb{S}^2}\] for any $p\in A'$.
According to \cite[7.14]{BGP}, $\theta\:A\to \RR$ is upper-semicontinuous.
The right-hand side of the last equality vanish in $A^\circ$.
Further since $\mathcal{H}_1(A'')=0$, the measure $\omega$ could be also defined as a measure on whole $A$ that is defined by \ref{eq:omega}.

Suppose $p\in A'$.
Choose a unit \emph{vertical} vector $u(p)\in \T_p$;
that is, $u(p)$ in the $\RR$-factor of $\T_p$.
Note that $u$ is uniquely defined up to sign.

\begin{thm}{Proposition}\label{prop:str-converge}
Let $M_n$ be a sequence of $3$-dimensional Riemannian manifolds such that $M_n\zz\to A$ without collapsing and 
$u=u(p)$ is a choice of vertical unit vectors at any point $p\in A'$.
Suppose $h_n\:M_n\to\RR$ be a sequence of $\lambda$-concave $L$-Lipscitz smooth functions that converges to a function $h\:A\to R$.
Let $\mu$ be a weak partial limit of the sequence of measures $q_{M_n}(\nabla h_n,\nabla h_n)$ on $A$.
Then 
\[\mu|_{A'}=\omega\cdot |dh(u)|^2.\]
\end{thm}


\parit{Proof.}
Let us split $\mu$ into negative and positive part $\mu=\mu^+-\mu^-$:
\[\mu^\pm(X)\df\sup\set{\pm\mu(Y)}{Y\subset X}.\]
Since the sectional curvature of $M_n$ is bounded below, we get that $\mu^-$ has bounded density; in other words $\mu^-$ is regular measure on $A$.
Since $A'$ has vanishing volume, we get $\mu^-(A')=0$.

Set $\nu=\mu|_{A'}$; from above we have $\nu\ge 0$.
By \cite{petrunin-SC}, $\nu$ is a regular with respect to $\mathcal{H}_1$ on $A'$.
Therefore it is sufficient to show that 
\[|d_ph(u)|^2\cdot (2\pi-\theta)\]
is the $\mathcal{H}_1$-density of $\nu$
at $\mathcal{H}_1$-almost all $p\in A'$.

Since $A'$ is countably 1-rectifiable \cite{li-naber}, we can cover it by a countable number of images of Lipschitz maps $s_i\:\mathbb{I}_i\to A$, where $\mathbb{I}_i$ is a real interval for each $i$.
Since $s_i$ and $h\circ s_i$ are Lipschits for any $i$, for $\mathcal{H}_1$-almost all $p\in A'$ we can choose $s_i$ such that $p\in s_i(x)$ for some $x\in \mathbb{I}_i$.
By Rademacher's theorem we can assume that $s_i$ and $h\circ s_i$ are differentiable at $x$.
Moreover we can assume that $d_xs_i=\lambda\cdot u$ for some $\lambda\ne 0$ and the $\mathcal{H}_1$-density of $\nu$ at $p$ is defined.

Shifting and scaling the interval $\mathbb{I}_i$, we may assume that $x=0$ and $\lambda=1$.
In this case $|d_ph(u)|=|d_0(h\circ s_i)|$.

Note that we can choose a sequence of points $p_n\in M_n$ and a sequence of factors $c_n\to \infty$ such that $(c_n\cdot M_n,p_n)$ converges to to the tangent space~$\T_p$.
Since $p\in A'$, the space $\T_p$ is a product space $\RR\times K_p$, where $K_p$ is a 2-dimensional cone with total angle $\theta(p)$.

Let $f_n\:c_n\cdot M_n\to \RR$ be as in \ref{thm:HG-converge} and $u_n=\nabla f_n/|\nabla f_n|$.
Consider the sequence of functions $\hat h_n\:c_n\cdot M_n\to \RR$ defined by 
\[\hat h_n(x)=c_n\cdot(h_n(x)-h_n(p_n)).\]
Note that $\langle\nabla \hat h_n,u_n\rangle$ uniformly converges to  $|d_0(h\circ s_i)|$.
By \ref{cor:Ricci}(\ref{cor:Ricci:vw}), the sequence of measures $q(\nabla\hat h_n, \nabla\hat h_n)$ on $c_n\cdot M_n\to \RR$ weakly converges to $\mathcal{H}_1\otimes\kappa $ on $\T_p\zz=\RR\times K_p$, 
where $\kappa$ is the curvature measure of $K_p$; 
the measure $\kappa$ is supported at the tip $o$ of $K_p$
and $\kappa\{o\}=2\cdot \pi-\theta(p)$.

Observe that 
\[q(\nabla\hat h_n, \nabla\hat h_n)[B(p_n,1)_{c_n\cdot M_n}]
=
c_n\cdot q(\nabla h_n, \nabla h_n)[B(p_n,\tfrac1{c_n})_{M_n}]\]
Whence $2\cdot \pi-\theta(p)$ is the $\mathcal{H}_1$-density of $\nu$ at $p$ as required.
\qeds
