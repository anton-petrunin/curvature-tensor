\section{Introduction}

The weak convergence and measure-valued tensor used in the following theorem are defined in the next section;
for a more precise formulation see \ref{main}.

\begin{thm}{Main theorem}
Let $\kappa$ be a real number and 
$M_n$ be a sequence of $m$-dimensional Riemannian manifolds with sectional curvature bounded below by~$\kappa$.
Assume that $M_n$ converges to an Alexandrov space $A$ of the same dimension.
Then the curvature tensors of $M_n$ weakly converge to a measured-valued tensor on $A$.
\end{thm}

This result is new and nontrivial even in the case if the limit space is a Riemannian manifold.

The second statement in the following corollary requires no new definitions.
It looks like a direct corollary of the main theorem, 
and indeed, it follows from its proof, but strictly speaking it can not be deduced directly from the main theorem alone.

Suppose $\Sc$ denotes the scalar curvature of Riemannian manifold and $\vol^m$ the $m$-dimensional volume; that is $m$-dimensional Hausdorff measure calibrated so that unit $m$-dimensional cube has unit measure.

\begin{thm}{Corollary}\label{cor:Sc}
In the assumption of the main theorem,
the measures $\Sc\cdot \vol^m$ on $M_n$ weakly converges to a locally finite signed measure $\mathfrak m$  on $A$.

In particular, if $A$ is compact, then the sequence
\[s_n=\int_{M_n}\Sc \cdot\vol^m\]
converges.
\end{thm}

The following corollary seems to be trivial, but it is not.

\begin{thm}{Corollary}
Let $\mathfrak{K}$ be a convex closed subset of curvature tensors on $\RR^m$ 
such that 
all sectional curvatures of tensors in $\mathfrak{K}$ are at least $-1$.
Assume that $\mathfrak{K}$ is invariant with respect to the rotations of $\RR^m$.
(For example, one can take $\mathfrak{K}$ the set of all curvature tensors with nonnegative curvature operator.)

Suppose $M_n$ is a sequence of $m$-dimensional Riemannian manifolds that converges to a Riemannian manifold $M$ of the same dimension.
Assume that all curvature tensors of $M_n$ belong to $\mathfrak{K}$ for any $n$,
then the same holds for the curvature tensors of $M$.
\end{thm}



\paragraph{Remarks.}
The limit measure $\mathfrak m$ in \ref{cor:Sc} has some specific properties;
let us describe couple of them:
\begin{itemize}
\item The measure $\mathfrak m$ vanish on any subset of $A$ with vanishing $(m-2)$-dimensional Hausdorff measure.
In particular, the $\mathfrak m$-measure of the set of singularities of codimension 3.

\item The measure can be explicitely described 
on the set $A'$ of singularities
of codimension $2$. That is, the set $A'\subset A$
contains all points $x$ with tangent space
$\T_xA=\R^{m-2}\times \Cone(\theta)$,
where $\Cone(\theta)$ denotes a $2$-dimensional cone
with the total angle $\theta=\theta(x)<2\cdot\pi$.
Then 
$$\mathfrak m|_{A'}=(2\cdot \pi-\theta)\cdot \vol^{m-2};$$
the latter follows from \ref{prop:3parts:codim2+}.
\end{itemize}


Note that the limit tensor of the sequence depends only on $A$ and does not depend on the choice of the sequence $M_n$.
Indeed, if for another sequence $M_n'$ satisfying the assumptions of the corollary and converging to the same Alexandrov space $A$ we would get a different limits, 
then a contradiction would occur for the alternated sequence $M_1,M_1',M_2,M_2',\dots$. This allows to define a curvature
tensor for every \emph{smoothable} Alexandrov space.
It is expected that the same can be done for general Alexandrov space.
The construction of harmonic coordinates in a neighborhood of regular point of $\rcd$-space (in particular, Alexandrov space) given by Elia Bruè, Aaron Naber, and Daniele Semola \cite{BNS} might help to do it.

The main theorem in \cite{petrunin-SC} implies that if a sequence of complete $m$-dimensional Riemannian  manifolds $M_n$ has uniformly bounded diameter and uniform lower curvature bound, then 
the corresponding sequence $s_n$ is bounded;
in particular, it has a converging subsequence.
However, if $M_n$ is collapsing then this sequence may not converge.
For example, an alternating sequence of flat 2-toruses and round 2-spheres might collapse to the one-point space, in this case the sequence $s_n$ is $0,4\cdot\pi,0,4\cdot\pi,\dots$.

The result of the main theorem in dimension 2 is well
 known (Theorem 8.4.2 \cite{Resh}, \cite{AZ} ).

The problem of introducing Ricci tensor
was studied in far more general settings of 
 $\rcd$-spaces \cite{G1,St,H}
 and
for some singular  spaces as well \cite{L}.
Curvature tensor for  $\rcd$-spaces was defined by Nicola Gigli \cite{G}.
It is expected that our definitions agree.

The precise geometric meaning of our curvature tensor is not quite clear. 
In particular, we do not know the answer to the following question which is closely related to imposed in \cite[Conjecture~1.1]{G}.

\begin{thm}{Question}
Suppose that Alexandrov space $A$ as in the main theorem has sectional curvature bounded below by $K>\kappa$.
Does it mean that $A$ is an Alexandrov space with curvature bounded below by $K$?
\end{thm}

\paragraph{About the proof.}
As it was stated, the 2-dimensional case is well known.
The proof of the 3-dimensional case is the main step in the proof;
the higher dimensional case is reduced to it.

We subdivide the limit space $A$ into
three subsets: $A^\circ$ --- the subset of regular 
points, $A'$ --- points with singularities of codimension 2,
$A''$ points with higher codimension singularities at least 3.
These sets are treated independently.

First we show that limit curvature vanish on $A''$; 
this part is an application of the main result in \cite{petrunin-SC}.

The $A'$-case is reduced to its partial case, when the limit is isometric to the product of two-dimensional cone and Euclidean spase.
In the proof we use a Bochner-type formula proved in Section~\ref{sec:bochner}.

The $A^\circ$-case is proved by induction.
The base is $2$-dimensional case; in the prove for dimension $3$ we apply the main theorem to level sets of special concave functions.
These level sets have the same lower curvature bound provided by Gauss theorem. 
In the proof we use the Bochner-type formula as well.
The higher-dimensional case is reduced to dimension $3$.
