\section{Introduction}

The weak convergence and measure-valued tensor used in the following theorem are defined in the next section;
for a more precise formulation see \ref{main}.

\begin{thm}{Main theorem}
Let $\kappa$ be a real number and 
$M_n$ be a sequence of $m$-dimensional Riemannian manifolds with sectional curvature bounded below by~$\kappa$.
Assume that $M_n$ converges to an Alexandrov space $A$ of the same dimension.
Then the curvature tensors of $M_n$ converge weakly to a measured-valued tensor on $A$.
\end{thm}

This result is new and nontrivial even in the case if the limit space is a Riemannian manifold.


The following statement looks like a direct corollary of the main theorem, 
and indeed, it follows from its proof, but strictly it can not deduce it directly from its statement.
Unlike the main theorem, the following corollary requires no new definitions.
 
Let us denote by $\Sc$ the scalar curvature of Riemannian manifold.


\begin{thm}{Corollary}\label{cor:Sc}
In the assumption of the main theorem,
the measures $\Sc\cdot \vol_m$ on $M_n$ weakly converges to a locally finite signed measure $\mu$  on $A$.

In particular, if $A$ is compact, then the sequence
\[s_n=\int_{M_n}\Sc d\vol_m\]
converges.
\end{thm}


\paragraph{Remarks.}
The limit measure $\mu$ in \ref{cor:Sc} has some specific properties;
let us describe some of them:
The measure $\mu$ vanish on any subset of $A$ with vanishing $(m-2)$-dimensional Hausdorff measure.
In particular, the $\mu$-measure of the set of singularities
of codimension $\ge 3$ vanishes.

he measure can be explicitely described 
on the set $A'$ of singularities
of codimension $2$. That is, the set $A'\subset A$
contains all points $x$ with tangent space
$T_xA=\R^{m-2}\times \C$
where $\C$ is a $2$-dimensional cone
with the total angle $2\pi-\omega(x)$.
Then 
$$\mu|_{A'}=\omega\cdot h_{m-2},$$
where $h_{m-2}$ denotes $(m-2)$-dimensional Hausdorff measure.


Note that the limit tensor of the sequence depends only on $A$ and does not depend on the choice of the sequence $M_n$.
Indeed, if for another sequence $M_n'$ satisfying the assumptions of the corollary and converging to the same Alexandrov space $A$ we would get a different limits, 
then a contradiction would occur for the alternated sequence $M_1,M_1',M_2,M_2',\dots$. This allows to define a curvature
tensor for every smoothable Alexandrov space. 

The main theorem in \cite{petrunin-SC} states in particular that if a sequence of complete $m$-dimensional Riemannian  manifolds $M_n$ has uniformly bounded diameter and uniform lower curvature bound, then 
the corresponding sequence $s_n$ is bounded; in particular it has a converging subsequence.
However if $M_n$ is collapsing this sequence may not converge
without the non-collapsing assumption.
For example an alternating sequence of flat 2-toruses and round 2-spheres might collapse to the one-point space, in this case the sequence $s_n$ is $0,4\cdot\pi,0,4\cdot\pi,\dots$.

The result of the Main theorem in dimension 2 is well
 known (Theorem 8.4.2 \cite{Resh}, \cite{AZ} ).

The problem of introducing Ricci tensor
was studied in far more general settings of 
 $\rcd$-spaces \cite{G1,St,H}
 and
for some singular  spaces as well \cite{L}.
Curvature tensor for  $\rcd$-spaces was defined by Nicola Gigli \cite{G}.
It is expected that our definitions agree.

The precise geometric meaning of our curvature tensor is not quite clear. 
In particular, we do not know the answer to the following question which is closely related to imposed in \cite[Conjecture~1.1]{G}.

Suppose that Alexandrov space $A$ as in the main theorem has sectional curvature bounded below by $K>\kappa$.
Does it mean that $A$ is an Alexandrov space with curvature bounded below by $K$?

\paragraph{About the proof.}
The most substantial case is for dimension $3$, higher dimensional case is reduced to it.

We subdivide the limit Alexandrov space into
three subsets: $A^\circ$ --- the subset of regular 
points, $A'$ --- points with singularities of codimension 2,
$A''$ points with higher codimension singularities at least 3.
These sets are treated independently.

First we show that limit curvature vanish on $A''$; 
this part is an application of the main result in \cite{petrunin-SC}.

The $A'$-case is reduced to its partial case, when the limit is isometric to the product of two-dimensional cone and Euclidean spase.
In the proof we use a Bochner-type formula proved in Section~\ref{sec:bochner}.

The $A^\circ$-case is proved by induction.
The base is $2$-dimensional case; in the prove for dimension $3$ we apply the main theorem to level sets of special concave functions.
These level sets have the same lower curvature bound provided by Gauss theorem. 
In the proof we use a Bochner-type formula as well.
Multidimensional case is reduced to dimension $3$, using special $3$-dimensional intersections of concave functions.
