\section{Introduction}

The weak convergence and measure-valued tensor used in the following theorem are defined in the next section;
a more precise formulation is given in \ref{main}.

\begin{thm}{Main theorem}
Let $\kappa$ be a real number and 
$M_n$ be a sequence of $m$-dimensional Riemannian manifolds with sectional curvature bounded below by~$\kappa$.
Assume that $M_n$ converges to an Alexandrov space $A$ of the same dimension.
Then the curvature tensors of $M_n$ weakly converge to a measured-valued tensor on $A$.
\end{thm}

This result is nontrivial, even if the limit space $A$ is a Riemannian manifold.
Note that the limit tensor of the sequence depends only on $A$ and does not depend on the choice of the sequence $M_n$.
Indeed, suppose another sequence $M_n'$ satisfies the assumptions of the theorem.
If the limit tensor is different, 
then a contradiction would occur for the alternated sequence $M_1,M_1',M_2,M_2',\dots$ 
In particular, if the limit space is Riemannian, then the limit curvature tensor is the curvature tensor of the limit space.

Analogous statements about metric tensor and Levi-Citita connection were essentially proved by Perelman \cite{PerDC},
we only had to tie his argument with appropriate type of convergence.
This part is one of the main ingradients in the proof;
it is discussed in Section~\ref{sec:DC}. 
It provides a technique that could be useful elsewhere as well.
For curvature tensor (which has higher order of derivative), this argument can not extended directly;
we found a way around applying a Bochner-type formulas as in \cite{petrunin-SC}.

The following statement looks like a direct corollary of the main theorem, 
and indeed, it follows from its proof, but strictly speaking it can not be deduced directly from the main theorem alone.
We will denote by $\Sc$ the scalar curvature and $\vol^m$ the $m$-dimensional volume; that is, $m$-dimensional Hausdorff measure calibrated so that the unit $m$-dimensional cube has unit measure.

\begin{thm}{Corollary}\label{cor:Sc}
In the assumption of the main theorem,
the measures $\Sc\cdot \vol^m$ on $M_n$ weakly converges to a locally finite signed measure $\mathfrak m$  on $A$.
\end{thm}

The following subcorollary requires no new definitions.

\begin{thm}{Subcorollary}\label{cor:cor:Sc}
In the assumption of the main theorem, suppose $A$ is compact.
Then the sequence
\[s_n=\int_{M_n}\Sc \cdot\vol^m\]
converges.
\end{thm}

The main theorem in \cite{petrunin-SC} implies that if a sequence of complete $m$-dimensional Riemannian  manifolds $M_n$ has uniformly bounded diameter and uniform lower curvature bound, then 
the corresponding sequence $s_n$ is bounded;
in particular, it has a converging subsequence.
However, if $M_n$ is collapsing, then this sequence may not converge.
For example, an alternating sequence of flat 2-toruses and round 2-spheres might collapse to the one-point space, in this case the sequence $s_n$ is $0,4\cdot\pi,0,4\cdot\pi,\dots$.

From the main theorem (and the definition of weak convergence) we get the following corollary.

\begin{thm}{Corollary}
Let $\mathfrak{K}$ be a convex closed subset of curvature tensors on $\RR^m$ 
such that 
all sectional curvatures of tensors in $\mathfrak{K}$ are at least $-1$.
Assume that $\mathfrak{K}$ is invariant with respect to the rotations of $\RR^m$.
(For example, one can take as $\mathfrak{K}$ the set of all curvature tensors with nonnegative curvature operator.)

Suppose $M_n$ is a sequence of $m$-dimensional Riemannian manifolds that converges to a Riemannian manifold $M$ of the same dimension.
Assume that for any $n$, all curvature tensors of $M_n$ belong to $\mathfrak{K}$, 
then the same holds for the curvature tensors of $M$.
\end{thm}



\paragraph{Remarks.}
The limit measure $\mathfrak m$ in \ref{cor:Sc} has some specific properties;
let us describe a couple of them:
\begin{itemize}
\item The measure $\mathfrak m$ vanish on any subset of $A$ with vanishing $(m-2)$-dimensional Hausdorff measure.
In particular, $\mathfrak m$ vanishes on the set of singularities of codimension 3.
This is an easy corollary of \cite{petrunin-SC}.

\item The measure can be explicitly described on the set $A'$ of singularities
of codimension $2$ as 
$$\mathfrak m|_{A'}=(2\cdot \pi-\theta)\cdot \vol^{m-2},$$
here $A'\subset A$
denotess the set of all points $x$ with tangent space
$\T_xA\zz=\R^{m-2}\zz\times \Cone(\theta)$,
where $\Cone(\theta)$ is a $2$-dimensional cone
with the total angle $\theta=\theta(x)<2\cdot\pi$.
This statement follows from \ref{prop:3parts:codim2+}.
\end{itemize}

A precise geometric meaning of our curvature tensor is not quite clear. 
In particular, we do not see a solution to the following problem;
compare to \cite[Conjecture~1.1]{G}.

\begin{thm}{Problem}
Suppose that the limit curvature tensor of Alexandrov space $A$ as in the main theorem has sectional curvature bounded below by $K>\kappa$.
Show that $A$ is an Alexandrov space with curvature bounded below by~$K$.
\end{thm}

The theorem makes possible to define a curvature tensor for every \emph{smoothable} Alexandrov space.
It is expected that the same can be done for general Alexandrov space; so the following problem has to have a solution:

\begin{thm}{Problem}\label{prob:curvature}
Extend the definition of measure-valued curvature tensor to general Alexandrov space.
\end{thm}

If this is the case, then one may expect to have a generalization of Gauss formula for curvature of convex hypersurface, which in turn might lead to a solution of the following open problems in Alexandrov geometry.
This conjecture is open even for convex sets in \emph{smoothable} Alexandrov space.

\begin{thm}{Conjecture}
Boundary of Alexandrov space equipped with its intrinsic metric is an Alexandrov space with the same lower curvature bound.
\end{thm}

More importantly, a solution of \ref{prob:curvature} might provide nontrivial ways to deform Alexandrov space; see \cite[Section 9]{petrunin-conc}. 

\parbf{Related results.}
The result of the main theorem in dimension 2 is well known \cite[VII \S13]{AZ}.

The construction of harmonic coordinates in a neighborhood of regular point of $\rcd$-space (in particular, Alexandrov space) given by Elia Bruè, Aaron Naber, and Daniele Semola \cite{BNS} might help to solve \ref{prob:curvature}.

The problem of introducing Ricci tensor
was studied in far more general settings \cite{G1,St,H,L}.
Curvature tensor for  $\rcd$-spaces was defined by Nicola Gigli \cite{G};
it works for more general class of spaces, but this approach does not see the curvature of singularities.
It is expected that our definitions agree on the regular locus.

\paragraph{About the proof.}
As it was stated, the 2-dimensional case is well known.
The proof of the 3-dimensional case is the main step in the proof;
the higher dimensional require only minor modifications.

We subdivide the limit space $A$ into
three subsets: $A^\circ$ --- the subset of regular 
points, $A'$ --- points with singularities of codimension 2,
$A''$ --- singularities of higher codimension.
These sets are treated independently.

First, we show that limit curvature vanish on $A''$; 
this part is an easy application of the main result in \cite{petrunin-SC}.

The $A'$-case is reduced to its partial case, when the limit is isometric to the product of two-dimensional cone and Euclidean space.
In the proof we use a Bochner-type formula proved in Section~\ref{sec:bochner}.

The $A^\circ$-case is proved by induction.
The base is $2$-dimensional case.
Further, we apply the main theorem to level sets of special concave functions.
By Gauss formula, these level sets have the same lower curvature bound. 
In the proof, we use the Bochner-type formula together with DC-calculus developed by Grigory Perelman \cite{PerDC}.
The higher-dimensional case uses the proof of three-dimensional case plus an extra estimate.

\parbf{Acknolegements.}
We wish to thank John Lott; without him the paper would not be written ---
these results were announced more than 15 years ago, and all this time John Lott was kipping to express his interest in a written version.
