\section{Introduction}

The weak convergence and measure-valued tensor used in the following theorem are defined in the next section;
a more precise formulation is given in \ref{main}.

\begin{thm}{Main theorem}
Let $M_1,M_2,\dots$ be a sequence of complete $m$-dimensional Riemannian manifolds with sectional curvature bounded below by~$\kappa$.
Assume that the sequence $M_n$ Gromov--Hausdrorff converges to an Alexandrov space $A$ of the same dimension.
Then the curvature tensors of $M_n$ weakly converge to a measured-valued tensor on $A$.
\end{thm}

Note that from the theorem we get that the limit tensor of the sequence depends only on $A$ and does not depend on the choice of the sequence $M_n$.
Indeed, suppose another sequence $M_n'$ satisfies the assumptions of the theorem.
If the limit tensor is different, 
then a contradiction would occur for the alternated sequence $M_1,M_1',M_2,M_2',\dots$ 
In~particular, if the limit space is Riemannian, then the limit curvature tensor is the curvature tensor of the limit space.
The latter statement was announced by the second author \cite{petrunin-poly}.

Analogous statements about metric tensor and Levi-Civita connection were essentially proved by Perelman \cite{PerDC},
we only had to tie his argument with an appropriate convergence.
This part is discussed in Section~\ref{sec:DC}. 
It provides a technique that could be useful elsewhere as well.
For curvature tensor (which has a higher order of derivative), this argument cannot be extended directly;
we found a way around applying Bochner-type formulas as in \cite{petrunin-SC}.

The following statement looks like a direct corollary of the main theorem, 
and indeed, it follows from its proof but strictly speaking, it cannot be deduced directly from the main theorem alone.
We will denote by $\Sc$ the scalar curvature and $\vol^m$ the $m$-dimensional volume; that is, $m$-dimensional Hausdorff measure calibrated so that the unit $m$-dimensional cube has unit measure.

\begin{thm}{Corollary}\label{cor:Sc}
In the assumption of the main theorem,
the measures $\Sc\cdot \vol^m$ on $M_n$ weakly converge to a locally finite signed measure $\mathfrak m$  on $A$.
\end{thm}

The following subcorollary requires no new definitions.

\begin{thm}{Subcorollary}\label{cor:cor:Sc}
In the assumption of the main theorem, suppose $A$ is compact.
Then the sequence
\[s_n=\int_{M_n}\Sc \cdot\vol^m\]
converges.
\end{thm}

The main theorem in \cite{petrunin-SC} implies that if a sequence of complete $m$-dimensional Riemannian  manifolds $M_n$ has uniformly bounded diameter and uniform lower curvature bound, then 
the corresponding sequence $s_n$ is bounded;
in particular, it has a converging subsequence.
However, if $M_n$ is collapsing, then this sequence may not converge.
For example, an alternating sequence of flat 2-toruses and round 2-spheres might collapse to the one-point space; in this case, the sequence $s_n$ is $0,4\cdot\pi,0,4\cdot\pi,\dots$

From the main theorem (and the definition of weak convergence) we get the following.

\begin{thm}{Corollary}
Let $\mathfrak{K}$ be a convex closed subset of curvature tensors on $\RR^m$ 
such that 
all sectional curvatures of tensors in $\mathfrak{K}$ are at least $-1$.
Assume that $\mathfrak{K}$ is invariant with respect to the rotations of $\RR^m$.
(For example, one can take as $\mathfrak{K}$ the set of all curvature tensors with nonnegative curvature operator.)

Suppose $M_n$ is a sequence of complete $m$-dimensional Riemannian manifolds that converges to a Riemannian manifold $M$ of the same dimension.
Assume that for any $n$, all curvature tensors of $M_n$ belong to $\mathfrak{K}$, 
then the same holds for the curvature tensors of $M$.
\end{thm}



\paragraph{Remarks.}
The limit measure $\mathfrak m$ in \ref{cor:Sc} has some specific properties;
let us describe a couple of them:
\begin{itemize}
\item The measure $\mathfrak m$ vanishes on any subset of $A$ with a vanishing $(m\zz-2)$-dimensional Hausdorff measure.
In particular, $\mathfrak m$ vanishes on the set of singularities of codimension 3.
This is an easy corollary of \cite{petrunin-SC}.

\item The measure can be explicitly described on the set of singularities
of codimension~$2$.
Namely, suppose $A'\subset A$
denotes the set of all points $x$ with tangent space
$\T_xA\zz=\R^{m-2}\zz\times \Cone(\theta)$,
where $\Cone(\theta)$ is a $2$-dimensional cone
with the total angle $\theta\zz=\theta(x)\zz<2\cdot\pi$.
Then 
$$\mathfrak m|_{A'}=(2\cdot \pi-\theta)\cdot \vol^{m-2}.$$
This statement follows from \ref{prop:3parts:codim2+}.
\end{itemize}

The geometric meaning of our curvature tensor is not quite clear. 
In particular, we do not see a solution to the following problem;
compare to \cite[Conjecture~1.1]{G}.

\begin{thm}{Problem}
Suppose that the limit curvature tensor of Alexandrov space $A$ as in the main theorem has sectional curvature bounded below by $K>\kappa$.
Show that $A$ is an Alexandrov space with curvature bounded below by~$K$.
\end{thm}

The theorem makes it possible to define a curvature tensor for every \textit{smoothable} Alexandrov space.
It is expected that the same can be done for general Alexandrov space; so the following problem has to have a solution:

\begin{thm}{Problem}\label{prob:curvature}
Extend the definition of measure-valued curvature tensor to general Alexandrov spaces.
\end{thm}

If this is the case, then one may expect to have a generalization of the Gauss formula for the curvature of a convex hypersurface, which in turn might lead to a solution of the following open problems in Alexandrov geometry.
This conjecture is open even for convex sets in \textit{smoothable} Alexandrov space.

\begin{thm}{Conjecture}
The boundary of an Alexandrov space equipped with its intrinsic metric is an Alexandrov space with the same lower curvature bound.
\end{thm}

More importantly, a solution to \ref{prob:curvature} might provide nontrivial ways to deform Alexandrov space; see \cite[Section 9]{petrunin-conc}. 

\parbf{Related results.}
The result of the main theorem in dimension 2 is well known \cite[VII \S13]{AZ}.

The construction of harmonic coordinates at regular points of $\rcd$ space (in particular, Alexandrov space) given by Elia Bruè, Aaron Naber, and Daniele Semola \cite{BNS} might help to solve \ref{prob:curvature}.

The problem of introducing Ricci tensor
was studied in far more general settings \cite{G1,St,H,L}.
Curvature tensor for $\rcd$ spaces was defined by Nicola Gigli \cite{G};
it works for a more general class of spaces, but this approach does not see the curvature of singularities.
It is expected that our definitions agree on the regular locus.

\paragraph{About the proof.}
As it was stated, the 2-dimensional case is proved in \cite[VII \S13]{AZ}.
The 3-dimensional case is the main step in the proof;
the higher-dimensional case requires only minor modifications.

We subdivide the limit space $A$ into
three subsets: $A^\circ$ --- the subset of regular 
points, $A'$ --- points with singularities of codimension 2,
$A''$ --- singularities of higher codimension.
These sets are treated independently.

First, we show that limit curvature vanishes on $A''$; 
this part is an easy application of the main result in \cite{petrunin-SC}.

The $A'$-case is reduced to its partial case when the limit is isometric to the product of the real line and a two-dimensional cone.
The proof uses a Bochner-type formula (\ref{thm:bochner-formula}) and Theorem \ref{thm:extimage-of-G-and-H} which is a more exact version of the following problem from \cite{petrunin-PIGTIKAL}.

{

\begin{wrapfigure}{r}{35 mm}
\vskip-7mm
\centering
\includegraphics{mppics/pic-25}
\vskip0mm
\end{wrapfigure}

\begin{thm}{Convex-lens problem}
Let $D$ and $D'$ be two smooth discs with a common boundary that bound a convex set (a lens) $L$ in a positively-curved 3-dimensional Riemannian manifold $M$.
Assume that the discs meet at a small angle.
Show that the integral $\int_{D}k_1\cdot k_2$
is small; here $k_1$ and $k_2$ denote the principal curvatures of $D$.
\end{thm}


The $A^\circ$-case is proved by induction.
The base is the 2-dimensional case.
Further, we apply the induction hypothesis to level sets of special concave functions.
By the Gauss formula, these level sets have the same lower curvature bound. 
In the proof, we use the Bochner-type formula together with the DC-calculus developed in \cite{PerDC}.
The first step in the induction is slightly simpler.

}

As a rule, the calculus is done in the approximating sequence of Riemannian manifolds.

\parbf{Acknowledgments.}
We wish to thank Sergei Ivanov for pointing out a gap in a preliminary version of this paper,
John Lott for expressing his interest in a written version for many years,
and Alexander Lytchak for helping us to write this paper in a more readable way.
Our very special thanks to an anonymous referee, who suggested several dozens of refinements. 

The first author was partially supported by the Russian Foundation for Basic Research grant 20-01-00070, 
the second author was partially supported by the National Science Foundation grant DMS-2005279
and the Ministry of Education and Science of the Russian Federation, grant 075-15-2022-289.
