

\section{Blow-up at $m-2$ singularity}\label{sec:blow}

Assume $M_n\smooths{} A$ is a smoothing of Alexandrov space $A$.
Note that given $p\in A$,
one can always find a sequence $\lambda_n$
which converge to infinity so slow that
$(\lambda_n M_n,p_n)\smooths{} \T_p A$;
that is, $\lambda_n M_n$ is a smoothing of $\T_pA$.
This new smoothing will be called a \emph{blow-up smoothing} at $p$.

We will use blow-up smoothing to study behavior of curvature near points $p\in A'$.
In this case $\T_p\iso\Cone\times \RR^{m-2}$.






\begin{thm}{Construction}\label{constr}
Fix $\lambda_n\to\infty$. There is $\varepsilon_n\to 0$ so that

$$\GHdist(h_{\lambda_n}B^A_{R_n}(p), B^{\T_p A}_{\lambda_n R_n}(p))\le\varepsilon_n/2$$

There is a monotonic function
 $k(n)\:N\to N$
so that for every $l\ge k(n)$
$$\GHdist(h_{\lambda_n}B^{M_l}_{1}(p), B^{\T_p A}_{\lambda_n}(p))\le\varepsilon_n/2.$$

Then for every  $k'(n)\ge k(n)$ we have
$$\lambda_n M_{k'(n)}\smooths{} (\T_p A, p)=(\RR^{m-2}\times \Cone_x^2,x).$$
\end{thm}
For every $\lambda>0$ we supply manifold $\lambda M_n$
with vector fields
$\bar v_n^i=1/\lambda(d h_{\lambda}v_n^i)$.
We will write shortly
$R_{\lambda M_n}=R_{\lambda M_n}(\bar v_n^1\wedge\dots\wedge\bar v_n^{m-2})$

\begin{thm}{Lemma}\label{l:convVCone}
Let vector field $v_n$ on $M_n$
converges with respect to Gromov--Hausdorff
convergence $M_n\GHto A$.
Fix $\lambda_n\to\infty$.
There is a monotonic function
 $k(n)\:N\to N$ so that  for every  $k'(n)\ge k(n)$
 we have that
$$\lambda_n M_{k'(n)}\GHto (\T_p A, p)=(\RR^{m-2}\times \Cone_x^2,x)$$
and
vertical parts of $\bar v_n$ converge.
\end{thm}

\begin{thm}{Claim}\label{cl:convVCone}
Let vector field $v_n$ on $M_n$
converges with respect to Gromov--Hausdorff
convergence $M_n\GHto A$.
Fix $\lambda_n\to\infty$.
There is a monotonic function
 $k(n)\:N\to N$ so that  for every  $k'(n)\ge k(n)$
 we have that
$$\lambda_n M_{k'(n)}\GHto (\T_p A, p)=(\RR^{m-2}\times \Cone_x^2,x)$$
and
$R_{\lambda_n M_{k'(n)}}$
weakly converges to $R^{\mathring{v}^1\wedge \mathring{v}^2\wedge\dots\wedge \mathring{v}^{m-2}}_p$,
where $\mathring{v}^i$ are limits of vertical parts of $\bar v_n$.
\end{thm}

\parit{Proof.} By Lemma~\ref{l:convVCone} we can apply claim \ref{cl:vfcone}.
\qeds




\subsection{Proof of \ref{l:convVCone}}

Let sequence of Alexandrov spaces converges
$(N_n,p_n)\GHto (A,p)$.
 Let $A$ contains a line $\RR_a\subset A$
 with direction
$a\in \T_p A$.
We say that a sequence of points  $x_n\in N_n$
converges $x_n\GHto\RR_a^{+\infty}$ if
there are sequences $R_n\to\infty,\varepsilon_n\to 0$
so that
$$\GHdist(B^{N_n}_{R_n}(p_n), B^{A}_{R_n}(p))\le\varepsilon_n$$
and there is a sequence $t_n\to+\infty$,
so that
$t_n a\le R_n+2$ and
 for $a_n=t_n a\in \RR_a\subset A$
we have
$\GHdist(x_n,a_n)\le\varepsilon_n$.
We say that a sequence of points  $y_n\in N_n$
converges in weak sense $y_n\GHwto\RR_a^{+\infty}$ if
$|p_n y_n|\to \infty$ and
for some sequence
$x_n\GHto\RR_a^{+\infty}$ we have $\angle (x_n p_n y_n)\to 0$.

\begin{thm}{Lemma}\label{lem:angle}
Let we have two sequences $x_n, y_n\GHwto\RR_a^{+\infty}$,
and  sequence $q_n\in N_n$ is uniformly bounded, i.e. $|p_n q_n|<c$,
than $\angle (x_n q_n y_n)\to 0$.
\end{thm}




\begin{thm}{Lemma} \label{lem:vconvergLoc}
Let $M_n\GHto A$ and vector fields $v_n$ converges.
Let sequence $\lambda_n\to\infty$.
Let vector $a\in \T_p A$
and for sequences $q_s, q_s'$
we have $\angle (a,\dir{p}{q_s})<\varepsilon_s$,
 $\angle (-a,\dir{p}{q'_s})<\varepsilon_s$.
There is a  limit product $\<v,a\>$
with the following properties:
there is a constant $c>0$,
a function $k\: N\to N$,
 sequences
$M_i\ni q_{n,i}\to q_n$, $\varepsilon_n\to 0$
and
so that for   all $n$
$$\<v,a\>- c\cdot \varepsilon_n
\le
\inf_{i\ge k(n)}\inf_{x\in B(p_i,1/\lambda_n)_{M_i}} \< v_i,\dir{x}{q_{n,i}}\>
\le
\sup_{i\ge k(n)}\sup_{x\in B(p_i,1/\lambda_n)_{M_i}}  \< v_i,\dir{x}{q_{n,i}}\>\le
\<v,a\>+
 c\cdot \varepsilon_n$$
and for any function $s(n)>k(n)$
$$h_{\lambda_n}( q_{n,s(n)})\GHwto\RR_a^{+\infty}.$$

\end{thm}

\parit{Proof.}
We know, that $(\lambda_n A, p)\GHto (\T_p A, p)=(\RR^{m-2}\times \Cone_x^2,x)$.
There is a sequence $\tilde q_s, \tilde q_s' \in B(p,1)_A$,
 sequence $\tilde \varepsilon_s\to 0$, constant $c$ so that the following holds:
%$$h_{\lambda_n}(\tilde q_n)\GHwto\RR_a^{+\infty}$$
%$$h_{\lambda_n}(\tilde q_n')\GHwto\RR_{-a}^{+\infty}$$
$$\angle(a,\dir{\tilde q_s}{p})\ge\pi- \varepsilon_s/2$$
 $$\angle(\tilde q_s p\tilde q_s')\ge\pi- \varepsilon_s/2$$
and shortest paths $p\tilde q_s$, $p\tilde q_s'$ are unique.
%For any such sequences holds: for any$p^0\in B(p,1/\lambda_n)$ we have
We can find monotonic sequence $r_s\to 0$ so that $|p\tilde q_s|/r_s\to\infty$
for any $p^0\in B(p,r_s)$
and for any approximating sequences
$M_i\ni q_{s,i}\to \tilde q_s$, $p^0_i\to p^0$ we have by property~\ref{convvf}
$$\limsup_{i\to\infty}  \< v_i,\dir{p_i}{\tilde q_{s,i}}\> -c\varepsilon_k\le
\liminf_{i\to\infty}  \<v_i,\dir{p^0_i}{\tilde q_{k,i}}\>\le\limsup_{i\to\infty}  \< v_i,\dir{p^0_i}{\tilde q_{k,i}}\>
\le\liminf_{i\to\infty}  \<v_i,\dir{p_i}{\tilde q_{k,i}}\> +c\varepsilon_k .
$$
We know that $$\angle(\tilde q_{t_1}p\tilde q_{t_2})\le\varepsilon_s$$ for $t_1, t_2>s$.
Hence (taking other sequences $\varepsilon_k$ and costant $c$)
we can find ``limit constant'' $\<v,a\>$ so that
 $$\<v,a\>-c\varepsilon_k\le
\liminf_{i\to\infty}
\<v_i,\dir{p^0_i}{\tilde q_{k,i}}\>\le\limsup_{i\to\infty}  \< v_i,\dir{p^0_i}{\tilde q_{k,i}}\>
\le\<v,a\> +c\varepsilon_k(*).$$

Let
$$k(n)=\max\set{t\in N}{\forall s\ge n\quad 1/\lambda_s<r_t}$$
We set $q_n:=\tilde q_{k(n)}$.
By construction $h_{\lambda_n}( q_n)\GHwto\RR_a^{+\infty}$
and the estimate $(*)$ implies that we can find appropriate function $k(n)$.
\qeds


\parit{Proof of \ref{l:convVCone}.}
We take $k(n)=\max$ of that from \ref{constr}
and lemma~\ref{lem:vconvergLoc},
sequences
$M_i\ni q_{n,i}\to q_n$ from  lemma~\ref{lem:vconvergLoc}.
Let $k'(n)\ge k(n)$.
We regard convergence
$$\lambda_n M_{k'(n)}\GHto (\T_p A, p)=(\RR^{m-2}\times \Cone_x^2,x)$$
then $y_n=h_{\lambda_n}(q_{n, k'(n)})\GHto \R_a^{+\infty} $.
Let sequence $\lambda_n M_{k'(n)}\ni x_n\to x\in \RR^{m-2}\times \Cone_x^2$.
By definition we know that
 $\<v_n,\dir{x}{y}\>=\<\bar v_n, \dir{h_{\lambda_n}(x)}{h_{\lambda_n}(y )}\>$
for any $x,y\in M_n$
hence lemma~\ref{lem:vconvergLoc} implies that
$$\<v,a\>-c\varepsilon_n\le\<\bar v_n,\dir{x_n}{y_n}\>\le\<v,a\>+c\varepsilon_n.$$
Let also regard some
$a_n\GHto \R_a^{+\infty} $ from construction of $f_a$ in \ref{???}.
By lemma \ref{lem:angle}
$\angle (a_n p_n y_n)\to 0$.
Hence
$$\<\bar v_n,\dir{x_n}{a_n}\>\to\<v,a\>.$$
\qeds

\section{Proof of claim~\ref{A'}}\label{sec:codim2}
\subsection{Locally two partitial limits approximately coinside.}
\begin{thm}{Claim}\label{cl:convLocCodim2}
Let sequence of manifolds $M_n\GHto A$,
$R_n$ weakly converges to $R^o$,
point $p\in A^{m-2}$. Then
for every  $r_n\to 0$

$(1/r_n)^{m-2}R^i(B(p,r_n))\to R_p^{\mathring{v}^1\wedge \mathring{v}^2\wedge\dots\wedge \mathring{v}^{m-2}}(B(p,1))$

\end{thm}

Before the prove we give the following general lemma:

\begin{thm}{Lemma}
Let $p $ be the measure on $K$.
Let we have convergence of Alexandrov spaces $K_i\GHto K$
  and for  any space of the sequence there is smoothing $N_{in}\to  K_i$.
Let for every $i$ measures $p_{in}$ on $N_{in}$ weakly
converge to measure $p_i$ on $K_i$.
Suppose there is a function $k(i)$ so that
for every $k'\ge k$ we
have $N_{i k'(i)}\GHto K$ and
$p_{i k'(i)}$ weakly converges to $p$ (*).
Then $p_i$ weakly converges to $p$.

\end{thm}

\parit{Proof.} Standart argument - diagonal subsequence. \qeds

 \parit{Proof of \ref{cl:convLocCodim2}.}
In our case: set $\lambda_n=1/r_n$,
for homothety $h_\lambda\:A\to \lambda A$ and measure
$p$ on $A$ we denote $p^\lambda$ the pullback (pushforward?) on  $\lambda A$.


$K={\T_p A}$, $p=R_p^{\mathring{v}^1\wedge \mathring{v}^2\wedge\dots\wedge
\mathring{v}^{m-2}}$,  $K_i={\lambda_i A}$,
$N_{in}=\lambda_i M_n$. The function $k(i)$ is
from \ref{cl:convVCone} above.
$p_{in}={\lambda_i}^{m-2}(R_n)^{\lambda_i}$
$p_i={\lambda_i}^{m-2}(R^0)^{\lambda_i}$.
Because of \ref{cl:convVCone} condition (*) of this lemma foolfilled.

By previous lemma we obtain that
${\lambda_i}^{m-2}(R^0)^{\lambda_i}$ weakly converges to measure
$R_p^{\mathring{v}^1\wedge \mathring{v}^2\wedge\dots\wedge
\mathring{v}^{m-2}}$.
Because measure $R_p^{\mathring{v}^1\wedge \mathring{v}^2\wedge\dots\wedge
\mathring{v}^{m-2}}$ is special? as it is
we have ${\lambda_i}^{m-2}(R^0)^{\lambda_i}(B^{\lambda_i}(p,1))\to R_p^{\mathring{v}^1\wedge \mathring{v}^2\wedge\dots\wedge \mathring{v}^{m-2}}(B(p,1))$.
Since $(R^0)^{\lambda_i}(B^{\lambda_i}(p,1))=R^0(B(p,1/\lambda_i)_{M_i})$
the claim follows.
\qeds

Let $R^1, R^2$ be two weak limits
of  $R(v^1_n\wedge\dots\wedge v^{m-2}_n )$.

\begin{thm}{Claim}\label{cl:R12loc}


1) Suppose $p\in A^{m-2}$ be so that $R_p^{\mathring{v}^1\wedge \mathring{v}^2\wedge\dots\wedge \mathring{v}^{m-2}}\neq 0$, i.e.
$\mathring{v}^1\wedge \mathring{v}^2\wedge\dots\wedge \mathring{v}^{m-2}\neq 0$
Then
$$R^1(B(p,r))/R^2(B(p,r))\xrightarrow{r\to 0} 1$$
and
$$\frac{(r_2)^{m-2}R^i(B(p,r_1))}{(r_1)^{m-2}R^i(B(p,r_2))}\xrightarrow{r_1, r_2\to 0} 1, \qquad i=1, 2,$$
in particular for sufficiently small $r>0$
we have
$$ R^i(B(p,2r))<2\cdot 2^{m-2}(R^i(B(p,r)),\qquad i=1, 2.$$

2) Suppose $p\in A'$ be so that $R_p^{\mathring{v}^1\wedge \mathring{v}^2\wedge\dots\wedge \mathring{v}^{m-2}}= 0$.
Then for every $\epsilon>0$ there is $r_0>0$ so that
for $r<r_0$
$$ R_i(B(p,r))<\epsilon\cdot r^{m-2},\qquad i=1, 2.$$
\end{thm}

\parit{Proof.} Simple consequence of \ref{cl:convLocCodim2} \qeds
