\section{Formulation}
  
\begin{thm}{Main theorem}\label{main}
Let $M_n\smooths{} A$ be a smoothing of Alexandrov space $A$.
Then there is a measure-valued tensor $\mathfrak{q}$ on $A$ such that $\mathfrak{q}_n\rightharpoonup \mathfrak{q}$, where $\mathfrak{q}_n$ denotes the dual measure-valued curvature tensor of $M_n$.
\end{thm}

\section{Plan of the proof}

Let $M$ be an $m$-dimensional Riemannian manifold
Set
\[K_{max}(x)=\sup |K_\sigma|\]
where $\sigma$ runs along all sectional directions at $x\in M$.
The following statement is a direct corollary of the main result in \cite{petrunin-SC}:

\begin{thm}{Corollary}\label{cor:Kmax}
Given an integer $m\ge 0$, there is a constant $\const(m)$ such that the following holds.

Let $M$ be an $m$-dimensional Riemannian manifold with sectional curvature bounded below by $-1$.
If for some $r<1$ the closed ball $\bar B(p,2\cdot r)_M$ is compact,
then 
$$\int_{B(p,r)_M} K_{max}\le \const(m)\cdot r^{m-2}.$$

\end{thm}

Observe that there is another constant $\const(m)$ such that 
\[
\begin{aligned}
|q_n(X_1,&\dots,X_{m-2},Y_1,\dots,Y_{m-2})|
\\
&\le 
\const(m)\cdot K_{max}\cdot|X_1\wedge\dots\wedge X_{m-2}|\cdot |Y_1\wedge\dots\wedge Y_{m-2}|
\end{aligned}
\eqlbl{q=<K}
\]
Therefore \ref{cor:Kmax} and \ref{q=<K} imply the following:

\begin{thm}{Claim}\label{clm:weak-partial-limit}
Given a smoothing $M_n\smooths{} A$ and a sequence of test function $f_{i,n},g_{i,n}\:M_n\to \RR$ such that 
$f_{i,n}\testto f_i\:A\to \RR$ and $g_{i,n}\testto g_i\:A\to \RR$ the sequence of measures 
$\mathfrak{q}_n(f_{1,n},\dots f_{m-2,n},g_{1,n},\dots, g_{m-2,n})$ has a weakly converging subsequence.
\end{thm}

By Perelman's stability theorem, the limit space $A$ is a topological manifold.
In particular, $A$ has no boundary.
Together with Claim~\ref{clm:weak-partial-limit} it implies that the main theorem \ref{main} follows from the following statement.

\begin{thm}{Proposition}\label{prop:3parts}
Let $M_n\smooths{} A$ and $\dim M_n=m$ for any $n$.
Suppose $h_{1,n},\dots, h_{m-2,n}\:M_n\to\RR$ is a sequence of test functions
such that $h_{i,n}\zz\testto  h_i\:A\to \RR$ for each $i$.
Let $\mathfrak m_1$ and $\mathfrak m_2$ be two measures on $A$ that are a weak partial limits of the sequence of measures $\mathfrak{q}_n(\nabla h_{1,n},\zz\dots, \nabla h_{m,n},\nabla h_{1,n},\zz\dots, \nabla h_{m,n})$ on $M_n$.
Then the following statements hold:
\begin{enumerate}[(i)]
\item\label{prop:3parts:codim3} Measures $\mathfrak m_1$ and $\mathfrak m_2$ vanish on $A''$; that is, 
\[\mathfrak m_i|_{A''}=0\]
for $i=1,2$.

\item\label{prop:3parts:codim2} Measures $\mathfrak m_1$ and $\mathfrak m_2$ satisfy the following equation:
\[\mathfrak m_i|_{A'}=\omega\cdot |d h_{1,n}\wedge\dots\wedge d h_{m,n}(u)|^2\]
for $i=1,2$,
where $u=u(p)$ is a choice of unit $(m-2)$-vector at any point $p\in A'$ in the vertical direction of $\T_p=\Cone\times \RR^{m-2}$;

\item\label{prop:3parts:reg} $\mathfrak m_1|_{A^\circ}=\mathfrak m_2|_{A^\circ}$.
\end{enumerate}
\end{thm}

The three parts of the proposition will be proved in sections \ref{sec:codmi=3}, \ref{sec:codmi=2}, and \ref{sec:ref} respectively.
