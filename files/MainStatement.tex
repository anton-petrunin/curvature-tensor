

  


\section{Formulation}
  
\begin{thm}{Main theorem}\label{main}
Let $M_n\smooths{} A$ be a smoothing of Alexandrov space $A$.
Then $\mathfrak{q}_n\rightharpoonup \mathfrak{q}$, where $\mathfrak{q}_n$ denoted the dual measure-valued curvature tensor of $M_n$
and $\mathfrak{q}$ is some for some measure-valued tensor on $A$.
\end{thm}

\section{Plan of the proof}

Let $M$ be an $m$-dimensional Riemannian manifold
Set
\[K_{max}(x)=\sup |K_\sigma|\]
where $\sigma$ runs along all sectional directions at $x\in M$.
The following statement is a direct corollary of the main result in \cite{petrunin-SC}:

\begin{thm}{Corollary}\label{cor:Kmax}
Let $M$ be an $m$-dimensional Riemannian manifold with sectional curvature bounded below by $-1$.
Then there a constant $\const(m)$ such that
$$\int_{B(p,r)_M} K_{max}\le \const(m)\cdot r^{m-2}.$$
for any $r<1$.
\end{thm}

Observe that there is a constant $\const(m)$ such that 
\[
\begin{aligned}
|q_n(\nabla f_1,&\dots,\nabla f_{m-2},\nabla g_1,\dots,\nabla g_{m-2})|
\\
&\le 
\const(m)\cdot K_{max}\cdot|\nabla f_1\wedge\dots\wedge\nabla f_{m-2}|\cdot |\nabla g_1\wedge\dots\wedge\nabla g_{m-2}|
\end{aligned}
\eqlbl{q=<K}
\]
Therefore \ref{cor:Kmax} and \ref{q=<K} imply the following:

\begin{thm}{Claim}\label{clm:weak-partial-limit}
Given a smoothing $M_n\smooths{} A$ and a sequence of test function $f_{i,n},g_{i,n}\:M_n\to \RR$ such that 
$f_{i,n}\testto f_i\:A\to \RR$ and $g_{i,n}\testto g_i\:A\to \RR$ the sequence of measures 
$\mathfrak{q}_n(f_{1,n},\dots f_{m-2,n},g_{1,n},\dots, g_{m-2,n})$ has a weakly converging subsequence.
\end{thm}

By Perelman's stability theorem $A$ is a topological manifold.
In particular, $A$ has no boundary.
Together with Claim~\ref{clm:weak-partial-limit} it implies that the theorem follows from the following more exact formulation.

\begin{thm}{Proposition}\label{prop:3parts}
Let $M_n$ be a sequence of $3$-dimensional Riemannian manifolds such that $M_n\zz\to A$ without collapsing.
Suppose $h_n\:M_n\to\RR$ be a sequence of $\lambda$-concave $L$-Lipscitz smooth functions that converges to a function $h\:A\to R$.
Let $\mu_1$ and $\mu_2$ be a weak partial limits of the sequence of measures $q_{M_n}(\nabla h_n,\nabla h_n)$ on $A$.
Then the following statements hold:
\begin{enumerate}
\item\label{prop:3parts:codim3} Measures $\mu_1$ and $\mu_2$ vanish on $A''$; that is, 
\[\mu_i|_{A''}=0\]
for $i=1,2$.

\item\label{prop:3parts:codim2} Measures $\mu_1$ and $\mu_2$ satisfy the following equation:
\[\mu_i|_{A'}=\omega\cdot |dh(u)|^2\]
for $i=1,2$.
where $u=u(p)$ is a choice of vertical unit vectors at any point $p\in A'$;

\item\label{prop:3parts:reg} $\mu_1|_{A^\circ}=\mu_2|_{A^\circ}$.
\end{enumerate}
\end{thm}

The three parts of the proposition will be proved in sections \ref{sec:codmi=3}, \ref{sec:codmi=2}, and \ref{sec:ref} respectively.
