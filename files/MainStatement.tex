\subsection{Formulation and plan}
  
\begin{thm}{Main theorem}\label{main}
Consider a smoothing $M_n\smooths{} A$.
Denote by $\qm_n$ the dual measure-valued curvature tensor on $M_n$.
Then there is a measure-valued tensor $\qm$ on $A$ such that $\qm_n\rightharpoonup \qm$.
\end{thm}

Let $A$ be an $m$-dimensional Alexandrov space without boundary. 
Let us partition $A$ into three subsets $A^\circ$, $A'$ and $A''$:
\begin{itemize}
\item $A^\circ$ is the set of regular points in $A$; that is, the set of points with tangent cone isometric to the Euclidean space.
\item $A'$ --- the set of points in $A\backslash A^\circ$ with an isometric copy of $\RR^{m-2}$ in their tangent space;
in other words, for any $p\in A'$, the tangent space $\T_p$ is isometric to the product $\Cone(\theta)\zz\times\RR^{m-2}$ where $\Cone(\theta)$ denotes a two-dimensional cone with the total angle $\theta\zz=\theta(p)<2\cdot \pi$.
\item $A''$ --- the remaining set; this is the set of points with tangent space that does \emph{not} contain an isometric copy of $\RR^{m-2}$.
\end{itemize}
According to \cite{li-naber}, $A'$ is countably $(m-2)$-rectifiable and $A''$ is a countably $(m-3)$-rectifiable. 

Observe that the set of regular points $A^\circ$ can be presented as
$$A^{\circ}=\bigcap_{\delta>0} A^\delta,$$
where $A^\delta$ denotes the set of $\delta$-strained points of $A$.

Let $M$ be an $m$-dimensional Riemannian manifold.
Denote by $K_{max}(x)$ the maximal sectional curvature at $x\in M$.
The following statement is a direct corollary of the main result in \cite{petrunin-SC}:

\begin{thm}{Corollary}\label{cor:Kmax}
Given an integer $m\ge 0$, there is a constant $\const(m)$ such that the following holds:

Let $M$ be an $m$-dimensional Riemannian manifold 
(possibly noncomplete)
with sectional curvature bounded below by $-1$.
If for some $r<1$ the closed ball $\bar B(p,2\cdot r)_M$ is compact,
then 
$$\int_{B(p,r)_M} K_{max}\le \const(m)\cdot r^{m-2}.$$

\end{thm}

\begin{thm}{Observation}\label{q=<K}
There is another constant $\const'(m)$ such that 
\begin{align*}
|\Qm(X_1,&\dots,X_{m-2},Y_1,\dots,Y_{m-2})|\le
\\
&\le 
\const'(m)\cdot K_{max}\cdot|X_1\wedge\dots\wedge X_{m-2}|\cdot |Y_1\wedge\dots\wedge Y_{m-2}|
\end{align*}

\end{thm}

Note that \ref{cor:Kmax} and \ref{q=<K} imply the following:

\begin{thm}{Claim}\label{clm:weak-partial-limit}
Given a smoothing $M_n\smooths{} A$,
test functions $f_i\:A\zz\to \RR$,
and sequences of $C^1$-smooth functions $f_{i,n}\:M_n\to \RR$ such that 
$f_{i,n}\zz\smoothto f_i$ the sequence of measures 
$\qm_n(f_{1,n},\zz\dots f_{2\cdot m-4,n})$ has a weakly converging subsequence.

Moreover, the subsequence can be chosen simultaneously for several choices of function arrays so that it meets the chain rule.
More precisely, choose $i$; fix all functions $f_{1,n},\zz\dots, f_{2\cdot m-4,n}$ except $f_i$;
suppose $\hat\qm_{i,n}(f_{i,n})=\qm_n(f_{1,n},\zz\dots, f_{2\cdot m-4,n})$.
Assume $h_{j,n}\:M_n\to \RR$ are $C^1$-smooth functions such that 
$h_{j,n}\zz\smoothto h_j$ and 
\[h_{0,n}=\phi(h_{1,n},\dots,h_{k,n})\]
for a fixed $C^1$-function $\phi\:\RR^k\to\RR$.
Then the sequence of measure arrays $\hat\qm_{i,n}(h_{0,n}),\zz\dots,\hat\qm_{i,n}(h_{k,n})$
has a partial limit $\hat\qm_{i}(h_0),\zz\dots,\hat\qm_{i}(h_k)$, and
\[\hat\qm_{i}(h_0)=\sum_{j=1}^k (\partial_j\phi)(f_1,\dots,f_n)\cdot\hat\qm_{i}(h_j).\]

\end{thm}


By Perelman's stability theorem, the space $A$ in the claim is a topological manifold.
In particular, $A$ has no boundary;
in other words, the singular set in $A$ has codimension at least~2.
Together with the claim, it implies that the main theorem (\ref{main}) follows from the next statement:

\begin{thm}{Proposition}\label{prop:3parts}
Let $M_n\smooths{} A$ and $\dim A=m$.
Suppose $h_{1},\zz\dots, h_{m-2}$ are 
test functions on $A$ and
$h_{1,n},\dots, h_{m-2,n}$ are 
$C^1$-smooth functions on $M_n$ 
such that $h_{i,n}\zz\smoothto  h_i$ for each $i$.
Let $\mathfrak m_1$ and $\mathfrak m_2$ be two measures on $A$ that are weak partial limits of the sequence of measures $\qm_n( h_{1,n},\zz\dots,  h_{m,n}, h_{1,n},\zz\dots,  h_{m,n})$ on $M_n$.
Then the following statements hold:
\begin{enumerate}[(i)]
\item\label{prop:3parts:codim3} $\mathfrak m_1|_{A''}=\mathfrak m_2|_{A''}=0$;

\item\label{prop:3parts:codim2} $\mathfrak m_1|_{A'}=\mathfrak m_2|_{A'}$;

\item\label{prop:3parts:reg} $\mathfrak m_1|_{A^\circ}=\mathfrak m_2|_{A^\circ}$.
\end{enumerate}
\end{thm}

The three parts of the proposition will be proved below in sections \ref{sec:codmi=3}, \ref{sec:codmi=2}, and \ref{sec:ref} respectively.
