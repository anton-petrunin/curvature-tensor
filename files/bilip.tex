\section{Bi-Lipschitz covering}\label{sec:bilip}

In this section we will prove Lemma~\ref{lem:A-prime-Q}.
A more general version of the lemma can be proved along the same lines as Lemma~11.1 in \cite{simon}.

Note that the lemma follows from the next proposition.


\begin{thm}{Proposition}\label{prop:Q-covering}
Let $A$ be an $m$-dimensional Alexandrov space with curvature at least $-1$ and $p\in A'$.
Then there is a compact set $Q$ such that 
\begin{enumerate}[(i)]
 \item $Q$ admits a bi-Lipschitz embedding into $\RR^{m-2}$ and
 \item there is a neighborhood $U\ni p$ and $\eps>0$ such that $q\in Q$ for any point $q\in U\cap A'$ such that 
 \[\theta(q)<\theta(p)+\eps.\]
\end{enumerate}
\end{thm}

Let $x$ be a point in an Alexandrov space $A$ with curvature at least $-1$.
Recall that Bishop--Gromov inequality implies that 
\[\frac{\vol^m B(x,R)_A}{\vol^m B(\tilde x,R)_{\HH^m}}
\le
\frac{\vol^{m-1} \Sigma_x}{\vol^{m-1} \SS^{m-1}}\]
for any $R>0$; here $\HH^m$ denotes the $m$-dimensional hyperbolic space.
The following lemma makes this inequality more precise. 

\begin{thm}{Lemma}
Let $x$ be a point in an $m$-dimensional Alexandrov space $A$ with curvature at least $-1$.
Suppose $y\in A$ is a point such that $|x-y|<R$ and $\measuredangle\hinge yxz<\pi-\eps$ for any point $z$.
Then
\[\frac{\vol^m B(x,R)_A}{\vol^m B(R)_{\HH^m}}
\le
(1-\delta)\cdot\frac{\vol^{m-1} \Sigma_x}{\vol^{m-1} \SS^{m-1}},\]
where $\delta$ is a positive number that depends on $m$, $|x-y|$, $R$ and $\eps$.
 
\end{thm}

\parit{Proof.}
To simplify the presentation we will assume that $A$ is nonnegatively curved;
it is straightforward to adapt the proof to the general case.
In this case, we need to show that 
\[\frac{\vol^m B(x,R)_A}{\vol^m B(R)_{\RR^m}}
\le 
(1-\delta)\cdot\frac{\vol^{m-1} \Sigma_x}{\vol^{m-1} \SS^{m-1}},\]

Let us denote by $\tilde p$ a vector in $\T_x$ that is tangent to a geodesic path $\gamma\:[0,1]\zz\to A$ from $x$ to~$p$.
By comparison, the map $p\mapsto \tilde p$ is a distance-noncontracting map.

Since $\measuredangle\hinge yxz<\pi-\eps$ for any $z$, the image of the map $p\mapsto \tilde p$ does not include points in a cone $C$ behind $\tilde y$ of angle $\eps$.
It follows that 
\[\vol^m(B(0,R)_{\T_x}\setminus C)\ge \vol^m(B(x,R)_A)\]
for any $R>0$.

\begin{wrapfigure}{r}{35 mm}
\vskip-0mm
\centering
\includegraphics{mppics/pic-200}
\vskip0mm
\end{wrapfigure}

Since $R>|x-y|$, the intersection $C\zz\cap B(0,R)_{\T_x}$ includes a ball of a certain radius $r>0$ that can be found in terms of $|x-y|$, $R$ and $\eps$.
By Bishop--Gromov inequality, we get $\delta=\delta( m, |x-y|, R, \eps)>0$ such that 
\[\frac{\vol^m(C\cap B(0,R)_{\T_x})}{\vol^m(B(0,R)_{\T_x})}>\delta.\]


Further, observe that 
\[\frac{\vol^m(B(0,R)_{\T_x})}{\vol^m(B(0,R)_{\RR^m})}
=
\frac{\vol^{m-1} \Sigma_x}{\vol^{m-1} \SS^{m-1}}.
\]
--- whence the lemma.
\qeds







\parit{Proof of \ref{prop:Q-covering}.}
Since the tangent cone at $p$ has $\RR^{m-2}$-factor, 
we can choose points $a_1,\zz\dots,a_{m-2},b_1,\zz\dots,b_{m-2}$ that are $\delta$-strainers of $p$ for arbitrary $\delta>0$.
The corresponding distance map $s\:x\zz\mapsto (|a_1-x|,\zz\dots,|x-a_{m-2}|)$ is an almost submersion of a neighborhood $U\ni p$ to $\RR^{m-2}$.
Choose small $\eps>0$ and set 
\[Q'=\set{x\in U\cap A'}{\theta(x)<\theta(p)+\eps}.\]
Let us show that $s|_{Q'}$ is bi-Lipschitz.
Once it is done, passing to the closure $Q=\bar Q'$ gives the required set. 

Note that for some $R>0$ the ball $B(p,10\cdot R)_A$ is almost isometric to the ball $B(0,10\cdot R)_{\T_p}$
and we can assume that $U\subset B(p,R)_A$.
By the volume convergence (see \cite[10.8]{BGP}) and Bishop--Gromov inequality, we can assume that 
\[\vol^m B(x,R)_A>\tfrac{\theta(p)-\eps}{2\cdot \pi}\cdot\vol^m B(0,R)_{\HH^m}\]
for any $x\in U$; here $\HH^m$ denotes the $m$-dimensional hyperbolic space. 

Assume $x$ and $y$ in $Q'$.
Since $\eps$ is small, the lemma implies that there is $z\in A$ such that $\measuredangle\hinge yxz$ is near $\pi$.
It follows that $\dir yx$ lies very close to the $\RR^{m-2}$-factor in~$\T_y$.
The same way we can show that $\dir xy$ lies very close to the $\RR^{m-2}$-factor in~$\T_x$.
In other words $[xy]$ lies nearly horizontally with respect to almost submetry $s$.
In particular,  
\[|s(x)-s(y)|_{\RR^{m-2}} \lessgtr \lambda^{\pm1}\cdot |x-y|_{A}\]
some constant $\lambda>1$.
(In fact, we can take $\lambda$ arbitrarily close to $1$, but we do not need it.)
\qeds
