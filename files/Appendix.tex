\section{Bochner formula}\label{sec:bochner}


We will need few integral formulas based on the Bochner formula.
We made these calculations based on \cite[Chapter II]{lawson-michelsohn}.
First let us state the Bochner formula for function with Dirichlet boundary condition.

\begin{thm}{Proposition}\label{prop:bochner-dirichle-old}
Assume $\Omega$ is a compact domain with smooth boundary $\partial \Omega$ in a Riemannian manifold
and $f$ is a smooth function that vanish on $\partial \Omega$.
Then
\[\int\limits_\Omega |\Delta f|^2
-|\mathrm{Hess}f|^2
-\langle\mathrm{Ric}(\nabla f),\nabla f\rangle
=\int\limits_{\partial\Omega}
H\cdot|\nabla f|^2,\]
where $H$ denotes mean curvature of $\partial \Omega$.
\end{thm}
 
\begin{thm}{Corollary}
Assume $\Omega$ is a compact domain with smooth boundary $\partial \Omega$ in a 2-dimensional Riemannian manifold with nonnegaive curvature
and $f$ is a smooth concave function that vanish on $\partial \Omega$.
Then
\[\int\limits_\Omega 
\det(\mathrm{Hess}f)
\le\pi\cdot\sup_{x\in\partial\Omega}|\nabla_x f|^2.\]

\end{thm}




In this section we give necessary versions of the Bochner formula.
The calculations are based on \cite[Chapter II]{lawson-michelsohn}.
We use Riemannian metric to identify differential forms and multivector fields on $M$.
Therefore the statement about differential forms can be also formulated in terms of multivector fields and the other way around.


Let $M$ be a Riemannian manifold.
Denote by $\nabla$ the Levi-Cevitta connection on $M$.
The bundle $\LT M$ of multivectors over $M$ is equipped with Clifford product, denoted by $\,\bullet \,$.
We will denote by $e_i$ is an orthonormal frame at a point; the following definitions will not depend on its choice.

\parbf{Laplasians.}
The Dirac operator on differential forms forms will be denoted by $D$;
it is defined as
\[D=\sum_i e_i\bullet \nabla_{e_i}.\]
Its square 
\[D^2=\sum_i e_i\bullet e_j\bullet \nabla^2_{e_i,e_j}\]
is called Hodge laplacian.

The Dirac operator if \emph{formally self-adjoint}, in particular,
\[\int_M \langle D^2\phi,\psi\rangle=\int_M \langle D\phi,D\psi\rangle\]
for any two vector fields $\phi$ and $\psi$ with compact support.

Further, define the connection laplacian
\[\nabla^*\nabla\phi =-\sum_i\nabla^2_{e_i,e_i}\phi\]
and the gradient
\[\nabla \phi=\sum e_i\otimes \nabla_{e_i}\phi.\]

For the connection laplacian we also have the identity
\[\int_M \langle \nabla^*\nabla\phi,\psi\rangle
=
\int_M \langle \nabla\phi,\nabla\psi\rangle.\]
If $\phi$ and $\psi$ have support in the domain where the frame is defined, then the right handside can be written as 
\[\int_M \langle \nabla\phi,\nabla\psi\rangle=\sum_i\int_M\langle \nabla_{e_i}\phi,\nabla_{e_i}\psi\rangle;\]
using the a partition of unity, one can use the latter expression to redefine the left hand side. 

\parbf{Bochner formula.}
The difference $D^2-\nabla^*\nabla$ between two laplasians described above is a 0-order differential operator which can be written in terms of curvature.
For a vector field $v$, the formula is 
\[D^2v-\nabla^*\nabla v=\Ric(v).\]

Using the identities above, the formula can be written in an integral form
\[\int_M \langle Dv,Dw\rangle -\langle \nabla v,\nabla w\rangle=\langle \Ric(v),w\rangle\]
In particular, if $w=\phi \cdot v$ for a smooth function $\phi$;
we get
\[\int_M \langle Dv,D(\phi\cdot v)\rangle -\langle \nabla v,\nabla(\phi\cdot v) \rangle=\phi\cdot\langle \Ric(v), v\rangle,\]
or, equivalently
\[\int_M \phi\cdot(\langle Dv,D v\rangle -\langle \nabla v,\nabla v \rangle)
+
\int_M (\langle Dv,\nabla \phi \bullet v\rangle -\langle \nabla v,\nabla v \rangle)
=
\phi\cdot\langle \Ric(v), v\rangle,\]
\parbf{Relative formulas.}
For a domain $\Omega$ with boundary $\partial \Omega$, the formula above takes form
\[\int_\Omega (\langle D^2\phi,\psi\rangle- \langle D\phi,D\psi\rangle)
=
\int_{\partial \Omega}\langle \nu\bullet D\phi,\psi\rangle,\]
where $\nu$ is the outer normal field on $\partial \Omega$.

The square $D^2$ of the Dirac operator is called Hodge laplacian.


For a domain $\Omega$ with boundary $\partial \Omega$, the formula above takes form
\[\int_\Omega (\langle \nabla^*\nabla\phi,\psi\rangle-\langle \nabla\phi,\nabla\psi\rangle)
=
\int_{\partial \Omega}\langle \nabla_\nu \phi,\psi\rangle,\]
where $\nu$ is the outer normal field on $\partial \Omega$.

The Bochner's identity for a vector field $u$ can be written as
\[D^2 u-\nabla^*\nabla u=\Ric(u).\]
relates the Dirac's laplasian $D^2$ and the connection laplasian $\nabla^*\nabla$

In this section we will write an integral version of the Bochner formula \cite[8.3]{lawson-michelsohn}
\[D^2-\nabla^*\nabla=\Ric\]
in such a way that each term has geometric meaning.


Further $\nabla^*\nabla=-\sum_i\nabla^2_{e_i,e_i}$ is the connection Laplacian.



Let $M$ be Riemannian $m$-manifold and $f\:M\to\R$ be a smooth function without critical points on an open domain $\Omega\i M$.
Assume $\phi\:\Omega\to\R$ be a smooth function with compact support.
Set $u=\nabla f/|\nabla f|$.
Let us define $\Int_f(x)$ (or just $\Int$) to be scalar curvature of the level set $L_x=f^{-1}(f(x))$ at $x\in L_x\i M$.
Set
\begin{enumerate}
 \item $\kappa_1(x)\le\kappa_2(x)\le\dots\le\kappa_{m-1}(x)$ the principle curvatures of $L_x$ at $x$;
 \item $H_f=H_f(x)=\kappa_1+\kappa_2+\dots+\kappa_{m-1}$ is mean curvature of $L_x$ at $x$
\item $G_f=G_f(x)=2\sum_{i<j}\kappa_i\cdot\kappa_j$ is the extrinsic term
 in the Gauss formula for $\Int_f(x)$. 
\end{enumerate}

Let us define the strange curvature as
\[\Str(u)=\Sc-\<\Ric(u),u\>,\]
where $\Sc$ and $\Ric$ denote scalar and Ricci curvature correspondingly.

\begin{thm}{Bochner's formula}\label{thm:bochner-formula}
Let $M$ be an $m$-dimensional Riemannian manifold,
$f\:M\to\R$ be a smooth function without critical points on an open domain $\Omega\subset M$ and $u=\nabla f/|\nabla f|$.
Assume $\phi\:\Omega\to\R$ is a smooth function with compact support.
Then 
$$\int\limits_\Omega \phi\cdot \Str(u)
=
\int\limits_\Omega \l[H\cdot\<u,\nabla\phi\>- \<\nabla\phi,\nabla_u u\> \r]+
\int\limits_\Omega \phi\cdot \Int_f.
\eqlbl{Bochner}$$
\end{thm}

In the proof of the main result we will 


\parit{Proof.}
Assume $b_1,\dots, b_m$ is an orthonormal frame such that $b_m=u$, 
then 
\[\Sc-2\cdot \<\Ric(u),u\>=2\cdot \sum_{i<j<m} \sec(b_i\wedge b_j).\] 
Therefore the Gauss formula can be written as
\[
\begin{aligned}
\Int_f&=G_f+\Sc-2\cdot \<\Ric(u),u\>=
\\
&=G_f+\Str(u)+ \<\Ric(u),u\>.
\end{aligned}
\eqlbl{eq:gauss}
\]

We can choose the frame $b_i$ so that $b_m=u$ and such that $b_i$ points in the principle directions of the level set $L_x$ for $i<m$.
Note that $\<\nabla_u u,u\>=0$, therefore
\begin{align*}
Du&=\sum_{i} b_i\bullet  \nabla_{b_i}u=
\\
&=\sum_{i<m}\kappa_i\cdot  b_i\bullet  b_i+u\bullet  \nabla_{u}u=
\\
&=
\sum_{i=1}^{m}\kappa_i+u\wedge\nabla_{u}u=H+u\wedge\nabla_{u}u,
\end{align*}
here ``$\,\bullet \,$'' denotes the Clifford multiplication.
Applying again that $\<\nabla_u u,u\>=0$, we get that
$$ \langle Du,Du \rangle=
\l(\sum_{i<m}\kappa_i\r)^2+|\nabla_{u}u|^2=H_f^2+|\nabla_{u}u|^2.$$
On the other hand
$$\nabla u=\sum_{i<m}\kappa_i\cdot b_i\otimes b_i+\nabla_u u\otimes u,$$
hence
$$\langle\nabla u,\nabla u\rangle =
\sum_{i<m}\kappa_i^2+|\nabla_{u}u|^2.$$

Therefore
$$\langle D u,D u\rangle-\langle \nabla u,\nabla u \rangle =2\cdot\sum_{i<j}\kappa_i\cdot\kappa_j=G_f.$$

Further,
\begin{align*}
\int\limits_\Omega\phi\cdot\l[\<D u,D u\>-\<D^2 u, u\>\r]
&=
\int\limits_\Omega\<\nabla\phi\bullet u,D u\>
=
\\
&=
-\int\limits_\Omega\l[H\cdot\<\nabla\phi,u\>- \<\nabla\phi,\nabla_u u\> \r],
\end{align*}
here ``$\bullet $'' denotes Kliford's multiplication.

Since $| u|\equiv 1$, we have $\<\nabla_{\nabla\phi}  u, u\>=0$.
Therefore
$$\int\limits_\Omega\phi\cdot\l[\<\nabla u,\nabla u\>-\<\nabla^*\nabla u, u\>\r]
=
\int\limits_\Omega\<\nabla_{\nabla\phi}  u, u\>=0.$$

Let us write Bochner formula \cite[8.3]{lawson-michelsohn} for field $u$:
$$D^2u-\nabla^*\nabla u=\Ric(u);$$
in particular, 
$$\phi\cdot \<D^2u,u\>-\phi\cdot \<\nabla^*\nabla u,u\>=\phi\cdot \<\Ric(u),u\>.\eqlbl{eq:prebochner}$$
Integrating \ref{eq:prebochner}, using the given calculations we get that
\begin{align*}
\int\limits_\Omega \phi \cdot G_f
&=\int\limits_\Omega\phi\cdot(\<D u,D u\>-\<\nabla u,\nabla u\>)
=
\\
&=\int\limits_\Omega \phi\cdot \Ric(u,u) 
-
\int\limits_\Omega H_f\cdot\<u,\nabla\phi\>- \<\nabla\phi,\nabla_u u\> .
\end{align*}

It remains to apply the Gauss formula \ref{eq:gauss}.
\qeds




 
 
 
 
\subsection{ Convergent vector fields}

\begin{thm}{Proposition}
Let $f_n$ be a sequence of $C^1$-smooth $\lambda$-concave functions
and
$f_n\to f$.
Suppose that for any compact $K$ there is $c>0$ such that
$d_x(\xi)+d_x(\eta)\le c(\pi-\angle(\xi,\eta))$
for $x\in K$ and $\xi, \eta\in \Sigma_x$.
Then $f_n$ roughly $C^1$-converge to $f$. 
\end{thm}

Fix some convergent directions $\xi_n\to\xi_0$ and
corresponding sequences of points
$x_n\to x_0$, $y_n\to y_0$
such that  $\xi_i=\uparrow_{x_i}^{y_i}$ for $i\in\mathbb N\cup\{0\}$.
There exists $z_0 \in U$ such that for $\eta=\uparrow_x^z$ 
we have
$\angle(\xi,\eta)\ge \pi-3\delta(\xi)/2$.
We choose some sequence 
$U_n\ni z_n\to z_0$
and denote
$\eta_n=\uparrow_{x_n}^{z_n}$,
then for large $n$ we have $\angle(\xi_n,\eta_n)\ge \pi-2\delta(\xi)$.
Let us denote by $V_x$ be the set of points that can be joint with $x$
by a unique geodesic. For $i\in\mathbb N\cup\{0\}$ and
$q\in V_{x_i}$ we have $|d_{x_i} \dist_q(\xi_i)+d_{x_i} \dist_q(\eta_i)|\le 4\de$ by first
variation formula.
Since Hausdorff dimension of $A\setminus V_x\le n-1$ we also obtain
that
$|d_{x_i} \widetilde\dist_{p_i,r}(\xi_i)+d_{x_i} \widetilde\dist_{p_i,r}(\eta_i)|\le 4\de(\xi) (*)$.
Since functions
$\widetilde\dist_{p_i,r}$ are uniformly semiconcave converging to
$\widetilde\dist_{p,r}$ we have
$\liminf d_{x_n}\widetilde\dist_{p_n,r}(\xi_n)\ge d_{x_0}\widetilde\dist_{p_0,r}(\xi_0)$ and $\liminf d_{x_n}\widetilde\dist_{p_n,r}(\eta_n)\ge d_{x_0}\widetilde\dist_{p_0,r}(\eta_0)$.
Together with $(*)$ this gives
\begin{align*}
 d_{x_0}\widetilde\dist_{p_0,r}(\xi_0)+8\de\ge
\limsup d_{x_n}\widetilde\dist_{p_n,r}(\xi_n)\ge\\
\ge
\liminf d_{x_n}\widetilde\dist_{p_n,r}(\xi_n)\ge d_{x_0}\widetilde\dist_{p_0,r}(\xi_0)-8\de.
\end{align*}





\begin{thm}{Proposition}
Any test sequence roughly $C^1$-converges.
\end{thm}
It is sufficient to show  that functions
$\widetilde{\dist}_{p_n,r}(t) =\oint_{B(p_n,r)} \dist_{s}(t)\cdot\vol(s)$
roughly $C^1$-converge to $\widetilde{\dist}_{p_0,r}(t)$.

Fix some convergent directions $\xi_n\to\xi_0$ and
corresponding sequences of points
$x_n\to x_0$, $y_n\to y_0$
such that  $\xi_i=\uparrow_{x_i}^{y_i}$ for $i\in\mathbb N\cup\{0\}$.
There exists $z_0 \in U$ such that for $\eta=\uparrow_x^z$ 
we have
$\angle(\xi,\eta)\ge \pi-3\delta(\xi)/2$.
We choose some sequence 
$U_n\ni z_n\to z_0$
and denote
$\eta_n=\uparrow_{x_n}^{z_n}$,
then for large $n$ we have $\angle(\xi_n,\eta_n)\ge \pi-2\delta(\xi)$.
Let us denote by $V_x$ be the set of points that can be joint with $x$
by a unique geodesic. For $i\in\mathbb N\cup\{0\}$ and
$q\in V_{x_i}$ we have $|d_{x_i} \dist_q(\xi_i)+d_{x_i} \dist_q(\eta_i)|\le 4\de$ by first
variation formula.
Since Hausdorff dimension of $A\setminus V_x\le n-1$ we also obtain
that
$|d_{x_i} \widetilde\dist_{p_i,r}(\xi_i)+d_{x_i} \widetilde\dist_{p_i,r}(\eta_i)|\le 4\de(\xi) (*)$.
Since functions
$\widetilde\dist_{p_i,r}$ are uniformly semiconcave converging to
$\widetilde\dist_{p,r}$ we have
$\liminf d_{x_n}\widetilde\dist_{p_n,r}(\xi_n)\ge d_{x_0}\widetilde\dist_{p_0,r}(\xi_0)$ and $\liminf d_{x_n}\widetilde\dist_{p_n,r}(\eta_n)\ge d_{x_0}\widetilde\dist_{p_0,r}(\eta_0)$.
Together with $(*)$ this gives
\begin{align*}
 d_{x_0}\widetilde\dist_{p_0,r}(\xi_0)+8\de\ge
\limsup d_{x_n}\widetilde\dist_{p_n,r}(\xi_n)\ge\\
\ge
\liminf d_{x_n}\widetilde\dist_{p_n,r}(\xi_n)\ge d_{x_0}\widetilde\dist_{p_0,r}(\xi_0)-8\de.
\end{align*}

\begin{thm}{Definition}
We say that sequence of vector fields $v_n$ on $U_n$
converges if for any compact $K$ there is $c>0$ such that
for any $x\in K$ and convergent sequence of directions $\xi_n\to\xi_0\in \Sigma_x$ we have
$\limsup \<v_n,\xi_n\>-\liminf \<v_n,\xi_n\>\le c\delta(\xi)$.

\end{thm}

\begin{thm}{Remark}
A sequence $f_n$ of $C^1$ functions roughly $C^1$-converges
iff $\nabla f_n$ converges.

\end{thm}


\begin{thm}{Claim}\label{lem:scalprod}
Let vector fields $v_n, w_n$ converge, then for every compact
set $K\subset A$ there is a constant $c>0$ so that
$\<v_n, w_n\>$ delta-converges on $A^\de\cap K$.
\end{thm}

\parit{Proof}
For $x\in A^\delta$ fix $\delta$-orthogonal basis





In particular we have
\begin{thm}{Corollary}\label{cor:cdeltacoeff}
Let vector fields $v_n^1,\dots, v_n^m, v_n$ be
uniformly bounded, converging
 and $m$-sets of vectors
$v_n^1,\dots, v_n^m$ be uniformly linearly independent;
that is, $\det(\<v_n^i, v_n^j\>)>c_0$
for some constant $c_0>0$.
Then for coefficient of decomposition
of $v_n=\alpha_n^1 v_n^1+\dots+\alpha_n^m v_n^m$
 and for every compact
set $K\subset A$ there is a constant $c>0$ so that
$\alpha_n^i$ delta-converges on $A^\de\cap K$.
\end{thm}

\section{DC calculus.}\label{sec:DC}

This section is based on the paper of Perelman \cite{PerDC}.
We define DC-functions, their relatives, and related convergences.
(DC stands for \emph{difference of convex} functions.)

\subsection{Functions}

Let $\Omega$ be an open bounded domain in $\RR^m$
and $S_\Omega\subset\Omega\subset\R^m $ subset of
zero $(m-1)$-Hausdorff measure; that is, $\haus_{m-1}(S_\Omega)=0$.
%The set $S$  will be referred as a subset of {\it singularities}.

We denote by  $\op{DC_0}(\Omega,S_\Omega)$ a class of
 $\DC$-function, such  that are continuously differentiable on
$\Omega\setminus S_\Omega$.

We denote by $\op{C_0}(\Omega,S_\Omega)$ a set of bounded functions
$f\:\Omega\to\RR$, which are continuous on $\Omega\setminus S_\Omega$.

Recall that a measurable function $f\:\RR^m\to \RR$ has bounded variation
 (briefly $f\in\op{BV}(\Omega)$) if there is  $c>0$,
such that  the inequality
$$\int\limits_\Omega f\cdot D_i\phi
\le
 c\cdot\sup_{x\in\Omega}|\phi(x)|$$
holds
for any $i$ and any smooth function $\phi\:\Omega\to \RR$ with compact support.

Let $\op{BV_0}(\Omega,S_\Omega)\subset \op{BV}(\Omega)$ be the subset of bounded functions that are continuous on $\Omega\setminus S_\Omega$.


We say that a signed Radon measure $\mathfrak m$ is in
$\aleph_0(\Omega,S_\Omega)$ if $\mathfrak m(S_\Omega)=0$.



\subsection{Sequences}
Let $h_n:\Omega\to\R$ be a sequence of continuous functions that
converges to the function $h_0\in DC(\Omega)$.
We will write 
$$(h_n,h_0)\in \op{DC^{seq}}(\Omega)$$
if
there are sequences of concave functions $f_n,g_n$ converging to
$f_0,g_0$, such that 
$h_0=f_0-g_0$.

Suppose in addition that 
$f_n,g_n$ are  continuously differentiable in
$\Omega\setminus S_\Omega$ for $n=0,1,2,\dots$.
In this case we will write 
$$(h_n,h)\in \op{DC_0^{seq}}(\Omega,S_\Omega).$$



Let $h_n:\Omega\to\R$ be a sequence of
uniformly bounded continuous 
functions
% with uniformly bounded supports 
that
converges  to the function $h\in \op{C_0}(\Omega,S_\Omega)$
 on
$\Omega\setminus S_\Omega$.
In this case we will write 
$$(h_n,h)\in \op{C_0^{seq}}(\Omega,S_\Omega).$$

Let $h_n:\Omega\to\R$ be a sequence of uniformly bounded functions
with uniformly bounded variations
that
almost everywhere converges  to a function $h_0\in\op{BV}(\Omega)$.
 Suppose that for any $i$ there are
 almost everywhere converging  sequences
 $f_n, g_n$ with uniformly bounded variations
  such that 
 $h_n=f_n- g_n$ and
 $\partial_i f_n$,
 $\partial_i  g_n$ are
 nonnegative Radon measures.
  In this case we will write 
$$(h_n,h)\in \op{BV^{seq}}(\Omega,S_\Omega).$$

If in addition $f_n, g_n\in\op{C_0}(\Omega,S_\Omega)$ for
$n\ge 0$ we
will write 
$$(h_n,h_0)\in \op{BV_0^{seq}}(\Omega,S_\Omega).$$




%Let $h_n:\Omega\to\R$ be a sequence of $C^0$-functions with uniformly bounded variations and  $h:\Omega\to\R$ be a  function so that that $h_n$C-converges  to the function $h$on the set where $h$ is continuous.  In this case we will write $$(h_n,h)\in \op{CBV^{seq}}(\Omega).$$

Let $\mathfrak m_n$ be
  a sequence of uniformly bounded
signed Radon measure  such that
$\mathfrak m_n=\mathfrak m_n^+-\mathfrak m_n^-$
for some
sequences of nonnegative Radon measures
that converge weakly to measures $\mathfrak m^+,
\mathfrak m^-\in\aleph_0(\Omega,S_\Omega)$.
Let $\mathfrak m=\mathfrak m^+-\mathfrak m^-$,
we will write
$$(\mathfrak m_n,\mathfrak m)\in\aleph_0^{seq}(\Omega,S_\Omega).$$


\subsection{Operations}

It is known that
the first partial derivative of a $DC$ function has bounded variation
and the second partial derivative are signed Radon measures.
The corresponding properties for sequences are given in the following lemma:

\begin{thm}{Lemma}\label{thm-D}

(1) If $( f_n, f)\in\op{DC_0^{seq}}(\Omega,S_\Omega)$,
then for every $i$ the partial derivative (defined almost everywhere)
we have
$$\biggl(\frac{\partial f_n}{\partial x_n^i}, \frac{\partial f}{\partial x^i}\biggr)\in\op{BV_0^{seq}}(\Omega,S_\Omega).$$

(2) 
 If     $( f_n, f)\in\op{BV_0^{seq}}(\Omega,S_\Omega)$, 
then for every $i$ for the partial derivative (defined as measures)
we have
$$\biggl(\frac{\partial f_n}{\partial x_n^i}, \frac{\partial f}{\partial x^i}\biggr)\in\aleph_0^{seq}(\Omega,S_\Omega).$$

\end{thm}

We can  multiply $C_0$ functions by $\aleph_0$ measures and
obtain again $\aleph_0$ measures, the corresponding statement
for sequences is the following:

\begin{thm}{Lemma}\label{thm-CM}
Let $f_n, f\in\op{C_0^{seq}}(\Omega,S_\Omega)$,
measures $g_n, g\in\aleph_0(\Omega,S_\Omega)$,
then
$$(f_n\cdot g_n,f\cdot g)\in\aleph_0^{seq}(\Omega,S_\Omega).$$
\end{thm}

The class of $\op{BV_0}$ and $C_0$ functions and sequences is closed under analytic expression as
says the following theorem:
\begin{thm}{Lemma}\label{thm-A}
Let sequences $(f^1_n,f^1),\dots,(f^k_n, f^k) \in \op{BV_0^{seq}(C_0^{seq})}(\Omega,S_\Omega)$,
and $A$ be  $C^\infty$-function well defined in a small neighborhood of
$(f^1,\dots,f^k)(\Omega)$.
Then $$(A(f^1_n,\dots,f^k_n), A(f^1,\dots,f^k))\in\op{BV_0^{seq}(C_0^{seq})}(\Omega,S_\Omega).$$
\end{thm}


\section{ Functions }
 \begin{thm}{Proposition}\label{NiceFunctions}
 Let   
	$M_n\in\M_{\ge -1}^n$,
	$M_n\GHto A$ , $ x\in A^\circ$, $M_n\ni x_n\to x$
	and $U\ni x$
	be a good domain for this convergence.
 For any
  $v\in \T_xA$,
 $\varepsilon>0$
 there is  $r>0$
 and a
  sequence of smooth strictly concave functions
 $f_n: B(x_n,r)\to \R$, roughly $C^1$-convergent to a function 
 $ f:B(x,r)\to\R$,
such that $d_xf$  is arbitrary close to $ <v,\cdot>$.
 \end{thm}
  
 
\section{DC-coordinates for Alexandrov space}

\subsection{Existense of common chart}\label{NiceChartProof}
Here should be the proof of 
Proposition~\ref{Prop:chart}.
\subsection{Metric tensor  in coordinates}\label{sec:metr-con}

\begin{thm}{Theorem (Metric tensor in a common chart)}\label{metricBV}
Let
$\mathfrak X_n:U_n\to\Omega$,
$\mathfrak X:U\to\Omega$ be a common chart.
Denote by $g_{ij,n}$ coordinates of metric tensor in this chart
and by $g^{ij}_n$ coordinates of the inverse matrix. 
Then
$g_{ij,n}\in BV_0^{seq}(\Omega)$
and
$g^{ij}_n\in BV_0^{seq}(\Omega)$.
Moreover, $\det(g_{ij,n})$ and $\det(g^{ij}_n)$ are bounded and bounded away from 0.

\end{thm}




For the proof of this theorem we define a class of $DC$-convergent sequences with some additional restriction,
that allow to take limits of derivatives.
 this will be applied to sequences of distance functions
in coordinates.
Let $h_n:\Omega\to\R$ be a sequence of $C^0$ functions that
converges to the function $h_0\in DC(\Omega)$.
We shall write
$(h_n, h_0)\in DC_+^{seq}(\Omega)$
if
there are sequences of concave functions $f_n,g_n\in C^0(\Omega)$ converging to
$f_0,g_0$, such that 
$h_0=f_0-g_0$ and in addition $f_n$ and $g_n$
are  differentiable at all points
from $\Omega\setminus S_\Omega$ where $h_n$ is
differentiable for $n=0,1,2\dots$.

 
 
\begin{thm}{Lemma}\label{DC+}
Let
$(h_n, h_0)\in DC_+^{seq}(\Omega)$
and $h_n$ be
differentiable at
$x_n$ for $n=0,1,2\dots$.
Then $d_{x_n}h_n\to d_{x_0}h_0$. 
\end{thm}

{\it Proof of the Theorem~\ref{metricBV} }
As it is shown in \cite{PerDC} Section 4.2
that if $\Omega$ is sufficiently small chart around a regular point,
for appropriate $p_i\in U$ in general position, $N=m(m+1)/2$ and rational function $Q$ components of metric tensor can be expressed as:

$$g_{ij}=Q\left( \frac{\partial (d_{p_1}\circ \X^{-1})}{\partial x_1},\dots,
\frac{\partial (d_{p_1}\circ \X^{-1})}{\partial x_m},\dots,
\frac{\partial (d_{p_N}\circ \X^{-1})}{\partial x_1},\dots,
\frac{\partial (d_{p_N}\circ \X^{-1})}{\partial x_m}\right)
$$

$$g_{ij}^n=Q\left( \frac{\partial (d_{p_1^n}\circ \X_n^{-1})}{\partial x_1},\dots,
\frac{\partial (d_{p_1^n}\circ \X_n^{-1})}{\partial x_m},\dots,
\frac{\partial (d_{p_N^n}\circ \X_n^{-1})}{\partial x_1},\dots,
\frac{\partial (d_{p_N^n}\circ \X_n^{-1})}{\partial x_m}\right).
$$

It can be obtained using arguments similar to \cite{PerDC} Section 3, that 
$(d_{p_i^n}\circ \X_n^{-1}, d_{p_i}\circ \X^{-1})\in DC_+^{seq}(\Omega)$,
then in particular
$(g^{ij}_n, g^{ij})\in BV^{seq}(\Omega)$.
Now it suffices to show that for any sequence of points
 $ q_n\to q\in \Omega\setminus S_\Omega $ the sequence 
$g_{ij}^n(q_n)\to g_{ij}(q)$. 
For a point $p$ in $U$ or $U_n$ we denote by $V_p$ points
with only one minimizing geodesic to $p$.
Let note that
$d_{p}\circ \X^{-1}$ is $C^1$ on $\X(V_p)$
in a weak sense of \cite{OS} (Section 1.5) and
$d_{p}\circ \X_n^{-1}$ is $C^1$ on $\X_n(V_p)$.



Then by Lemma~\ref{DC+} for $U_n\ni p^n\to p\in U$ and
$q_n\in\X(V_{p_n})$ such that $q_n\to q\in \X(V_p)$
we have
$\frac{\partial (d_{p^n}\circ \X_n^{-1})}{\partial x_i}(q_n)\to
\frac{\partial (d_{p}\circ \X^{-1})}{\partial x_i}(q)$.

Now for a given sequence of points $ q_n\to q\in \Omega\setminus S_\Omega $ 
we can choose points $p_i^n$ and $p_i$
in the expression for metric tensor
 in such a way 
that
$q_n\in\X(V_{p_i^n})$ and $q\in \X(V_{p_i})$.

\qeds






