\subsection{Strange curvature}

Suppose $M$ is a 3-dimensional Riemannian manifold.
\emph{Strange curvature tensor} $\Str$ on $M$ is a bilinear that is uniquely defined by
$$\Str(w,w)=\Sc\cdot |w|^2-\Ric(w,w)$$
for $w\in \T M$.
Note that $\Str$ completely describes $\Ric$;
further, since $M$ is 3-dimensional, $\Str$ completely describes its curvature tensor $\Rm$.

In Riemannian manifold we can (and will) use the metric tensor to identify tangent and cotangent bundle.
Therefore the tensor $\Str$ can be applied equally to vector fields and forms, in particular,
\[\Str(df,df)=\Str(\nabla f,\nabla f).\]

\begin{thm} {Proposition}\label{strconvergence}
Let $M_n\smooths{} A$ and $\dim A=3$; choose a common char with range $\Omega\subset \RR^3$.
Suppose that $f$ is a convex combination of coordinate functions of the chart.
Then for sufficiently small $r_0>0$ the measures 
\[\frak m_n=\Str_n(df,df) \cdot \vol^m_n\] are weakly delta-converging in $\Omega$.

\end{thm}

The strange curvature tensor appears in the following integral expression from \ref{thm:bochner-formula}:
$$\int\limits_\Omega \phi\cdot \Str(u, u)
=\int\limits_\Omega \phi\cdot \Int+
\int\limits_\Omega \l[H\cdot\<u,\nabla\phi\>- \<\nabla\phi,\nabla_{u} u\> \r],
\eqlbl{Bochner}$$
where 
\begin{itemize}
\item $H(x)$ --- the mean curvature of the level set $f^{-1}(f(x))$,
\item $\Int(x)$ --- the scalar curvature of  $f^{-1}(f(x))$.
\end{itemize}
This formula is the main tool in the proof of the proposition.
It also motivates the definition of $\Str$.



Applying this  formula, we can reduce the proposition to the following two lemmas;
each lemma provides the convergence of an integral term in the right-hand-side of \ref{Bochner}.

\begin{thm} {Lemma}\label{Int}
In condition of Proposition~\ref{strconvergence}, $\Int_n\cdot \vol^3_n$ is a delta-converging sequence of measures on $\Omega$.
\end{thm}

The proof of this lemma uses 
convergence of curvature measures
$\Int_n\cdot \vol^2$
on  the 2-dimension level sets of concave functions $f_n$ and the coarea formula.
The factor $|\nabla f_n|$ appears in the coarea formula.
Since the sequence $|\nabla f_n|$ is only delta-convergent (see \ref{lem:test-delta}(\ref{lem:test-delta|nabla|})), we get that only delta-convergence of $\Int_n\cdot \vol^3_n$.

\parit{Proof.}
Any point  in an Alexandrov space
has a convex neighborhood \cite{petrunin-conc} and by
construction  it can be lifted  to a smoothing sequence.
So let 
$V\subset A$ be an open  convex
neighborhood of $x$ and
$ V_n\subset M_n$
be open convex sets such that
$V_n  \GHto   V$.

Set
\begin{align*}
L_{t,n}&=f^{-1}(t)\cap V_n,&
C_{t,n}&=f^{-1}[t,\infty)\cap \bar V_n,
\\
L_{t}&=f^{-1}(t)\cap V,&
C_{t}&=f^{-1}[t,\infty)\cap \bar V,
\end{align*}


For every $t$ and $n$, the set $C_{t,n}$ is a convex subset in Alexandrov space 
 and hence is an Alexandrov space 
 with curvature $\ge -1$.
Note that
$C_{t,n} \GHto C_{t}$. 
By \cite[Theorem 1.2]{petrunin-QG}
the boundaries $\partial C_{t, n}  $ equipped with inner metric
 $\rho^{\partial C_{t,n}}$ converge to
 $\partial C_{t}  $ with inner metric
 $\rho^{\partial C_{t}}$.
 
 
It follows from \cite{AKP-buyalo} that
$(\partial C_{t,n}, \rho^{\partial C_{t_n,n}}) $ is
an Alexandrov space 
with curvature $\ge -1$.
Hence so is the limit
$(\partial C_{t}, \rho^{\partial C_{t}}) $ is
an Alexandrov space 
 with curvature $\ge -1$.

Let $\rho_{t,n}$ be induced inner metric on $L_{t,n}\subset M_n$ and $\rho_{t}$ be induced inner metric on $L_{t}\subset A$.
Note that $\rho_{t,n}$ coincides locally coincide with the induced length-metric in 
$\partial C_{t,n}$.

Note that since 
$\partial C_{t}$
is an extremal subset for
$C_{t}$
 the inner metric
$ \rho^{\partial C_{t}} $ is bi-Lipschitz to %???REF
the metric restricted from $A$ to
$\partial C_{t}$.
It follows that
we can take $r$ sufficiently small
such that for all $t$ and
$U_{t,n}=L_{t,n}\cap B(x_n,r)$
we will have
$\diam_{\rho_{t,n}} U_{t,n}\le \tfrac 1{10} \dist   (U_{t,n},\partial C_{t,n}\setminus L_{t,n})$.
Then the local 2-dimensional case of the main theorem can be applied to $U_{t,n}$ and it implies weak convergence of measures  $\Int_n\cdot \vol^2_n$ on $L_{t,n}$.

Choose a smooth
$\phi\:\Omega\to\RR$ with a compact support in $A^\delta$.
Applying the coarea formula, we get
$$\int\limits_{s\in \Omega}\Int_n(s)\cdot\phi(s)\cdot\vol^3_n
=\int\limits_{-h}^{h} d t\cdot 
\int\limits_{s\in U_{t,n}}
 \frac{\phi_n(s)}{|\nabla_n f(s)|}\cdot\Int_n(s)\cdot \vol^2.\eqlbl{eq:Int-coarea}$$
 
Note that $\nabla_n f$ is bounded away from zero.
By \ref{lem:test-delta}(\ref{lem:test-delta|nabla|}), $\frac{1}{|\nabla_n f(s)|}$ is weakly delta-converging.
Since $\Int_n\cdot \vol^2$ is weakly converging measures on $L_n$, we get 
that $\Int_n\cdot\vol^3_n$ is weakly delta-converging measures on $M_n$.
\qeds

The following lemma is related to the convergence of the second integral in \ref{Bochner}, the proof uses
$DC$-calculus in a common chart. 

\begin{thm}{Lemma}\label{HnablaU}
In condition of Proposition~\ref{strconvergence}, 
suppose $\psi\:\Omega\to\RR$ is a smooth function with compact support. Then
\[\int\limits_\Omega \l[H_n\cdot\<u_n,\nabla_n\phi\>_n- \<\nabla_{u_n} u_n,\nabla_n\phi\>_n \r]
\eqlbl{Bochner}\]
converges, where $H_n$ as in  \ref{Bochner} and $u_n=\nabla_n f/|\nabla_n f|$.
\end{thm}

\parit{Proof.}
Note that $H_n=\div u_n$.
Let us rewrite the integral in $\Omega$:
$$\int\limits_{\Omega}
\Bigl(\sum_{i=1}^3u_n^i \tfrac{\partial \phi}{\partial x_i}\Bigr)
\sum_{i=1}^3
\Bigl(\tfrac{\partial u_n^i}{\partial x^i} +u^i_n\tfrac{\partial \log \sqrt {\det (g_{ij,n})}}{\partial x^i}\Bigr)
\sqrt {\det (g_{ij,n})}dx^1\wedge dx^2\wedge dx^3$$

By \ref{Prop:chart}(\ref{metric}), we have
$g_{i j,n}, g^{ij}_n\in  \op{BV_0^{seq}}(\Omega,S_\Omega,\RR)$
and
$\det g_{ij,n}$ are bounded away from $0$.
We also have
$$u_n^i=\frac{g^{ij}\frac{\partial f}{ \partial x_j}}
{\sqrt{g^{jk}\frac{\partial f}{\partial x_j }\frac{\partial f}{\partial x_k}}}.$$
By Proposition~\ref{Prop:chart}.\ref{funktioninchart}
 $f\circ\bm{x}_n^{-1}\in DC_0^{seq}(\Omega, S_\Omega)$
hence 
${u^i_n}\in  \op{BV_0^{seq}}(\Omega,S_\Omega,\RR)$.

Then applying lemmas~\ref{thm-D}, \ref{thm-CM}, and \ref{thm-A} we get

\begin{itemize}

\item $\partial \log \sqrt {\det (g_{ij,n})}/\partial x^i\in \aleph_0^{seq}(\Omega,S_\Omega)$

\item $\frac{\partial u_n^i}{\partial x_i}\in \aleph_0^{seq}(\Omega,S_\Omega)$
 
\end{itemize} 
 
 and by above
 
  \begin{itemize}
  
  \item ${u^i_n}\in  \op{BV_0^{seq}}(\Omega,S_\Omega,\RR)\subset\op{C_0^{seq}}(\Omega,S_\Omega,\RR)$ 
 
 \item $g_{i j,n}\in   \op{BV_0^{seq}}(\Omega,S_\Omega,\RR)\subset \op{C_0^{seq}}(\Omega,S_\Omega,\RR)$
 
  \end{itemize}
  
Then applying Lemma~\ref{thm-CM} we get under integral the sum of elements from $\aleph_0^{seq}(\Omega,S_\Omega)$ multiplied by $\frac{\partial \phi}{\partial x_i}$ --- smooth functions with compact support.
This gives convergence of the integral by definition of  $\aleph_0^{seq}(\Omega,S_\Omega)$.
              

Further, for the second integral % (see claim~\ref{cl:funDC})
we have
\begin{align*}
\int\limits_{M_n}&\<\nabla\psi_n,\nabla_{u_n} u_n\>\cdot\vol^3_n=\\
&=\int\limits_{\Omega}  \sum_{i,j,k}u^i_n \frac{\partial \phi}{\partial x_k}
\biggl(\frac{\partial u^k_n}{\partial x_i} +\tfrac{1}{2}u^j_n\sum_s
\biggl(\frac{\partial g_{ js,n}}{\partial x_i}+\frac{\partial g_{si,n}}{\partial x_j}-\frac{\partial g_{i j,n}}{\partial x_s}\biggr) g^{ks}_n\biggr)\cdot 
\\
&\quad\cdot\sqrt{\op{det}(g_{i j,n})}\cdot dx^1\wedge dx^2\wedge dx^3.
\end{align*}


For the first integral,
 applying
lemmas~\ref{thm-D}, \ref{thm-CM}, and \ref{thm-A} we get under the integral sequence from $\aleph_0^{seq}(\Omega,S_\Omega)$ multiplied by smooth function with compact support $\frac{\partial \phi}{\partial x_i}$. Then the integral converges and the lemma follows.
\qeds

\parit{Proof of Proposition~\ref{strconvergence}.}
By \ref{Bochner}, \ref{Int}, and \ref{HnablaU} we get that $\Str_n(u_n,u_n)\cdot \vol^3_n$ is a weakly delta-converging sequence of measures.
It remains to apply \ref{lem:test-delta}(\ref{lem:test-delta|nabla|}) and \ref{obs:delta-weak-uniform}.
\qeds
