\subsection{Strange curvature}

Suppose $M$ is a 3-dimensional Riemannian manifold.
\emph{Strange curvature tensor} $\Str$ on $M$ is a bilinear form that is uniquely defined by
$$\Str(w,w)=\Sc\cdot |w|^2-\Ric(w,w)$$
for $w\in \T M$.
Note that $\Str$ completely describes the Ricci curvature tensor $\Ric$.
Further, since $M$ is 3-dimensional, $\Str$ completely describes the curvature tensor $\Rm$ of $M$.

In Riemannian manifolds, we can (and will) use the metric tensor to identify tangent and cotangent bundles.
Therefore the tensor $\Str$ can be applied to vector fields and forms;
in particular, for any smooth function $f$ we have
\[\Str(df,df)
=
\Str(\nabla f,\nabla f).\]

\begin{thm} {Proposition}\label{strconvergence}
Let $M_n\smooths{} A$ and $\dim A=3$; choose a common chart with range $\Omega\subset \RR^3$.
Suppose that $f$ is a convex combination of coordinate functions of the chart.
Then the measures 
\[\mathfrak m_n=\Str_n(df,df) \cdot \vol^3_n\] are weakly delta-converging in $\Omega$.

\end{thm}

The definition of strange curvature tensor is motivated by the following integral expression from \ref{thm:bochner-formula}:
$$\int\limits_\Omega \phi\cdot \Str(u, u)
=\int\limits_\Omega \phi\cdot \Int+
\int\limits_\Omega \l[H\cdot\<u,\nabla\phi\>- \<\nabla\phi,\nabla_{u} u\> \r],\eqlbl{eq:Bochner2}$$
where 
\begin{itemize}
\item $u=\nabla f/|\nabla f|$,
\item $H(x)$ --- the mean curvature of the level set $f^{-1}(f(x))$,
\item $\Int(x)$ --- the scalar curvature of  $f^{-1}(f(x))$.
\end{itemize}
This formula is the main tool in the proof of the proposition.
It reduces the proposition to the following two lemmas;
each lemma provides the convergence of an integral term in the right-hand side of \ref{eq:Bochner2}.

\begin{thm} {Lemma}\label{Int}
In the assumptions of Proposition~\ref{strconvergence}, $\Int_n\cdot \vol^3_n$ is a delta-converging sequence of measures on $\Omega$.
\end{thm}

The proof of this lemma uses 
convergence of curvature measures
$\Int_n\cdot \vol^2$
on  the 2-dimension level sets of concave functions $f$ and the coarea formula.
Recall that the sequence $|\nabla_n f|$ is only weakly delta-converge (see \ref{lem:test-delta}(\ref{lem:test-delta|nabla|})).
Since the factor $|\nabla_n f|$ appears in the coarea formula,
we get that only weak delta-convergence of $\Int_n\cdot \vol^3_n$.

\parit{Proof.}
Recall that any point  in an Alexandrov space $A$ has a convex neighborhood \cite{petrunin-conc}.
This construction can be lifted  to the smoothing sequence~$(M_n)$.
Let 
$V\subset A$ be an open  convex
neighborhood of $x$ and
$ V_n\subset M_n$
be open convex sets such that
$\bar V_n  \GHto   \bar V$.

Set
\begin{align*}
L_{t,n}&=f^{-1}(t)\cap V_n,&
C_{t,n}&=f^{-1}[t,\infty)\cap \bar V_n,
\\
L_{t}&=f^{-1}(t)\cap V,&
C_{t}&=f^{-1}[t,\infty)\cap \bar V.
\end{align*}


For every $t$ and $n$, the set $C_{t,n}$ is a convex subset in Alexandrov space 
 and hence is an Alexandrov space 
 with curvature $\ge -1$.
Note that
$C_{t,n} \GHto C_{t}$.
Let us equip the boundaries  $\partial C_{t,n}$ and
 $\partial C_{t}$ with the induced inner metrics.
By \cite[Theorem 1.2]{petrunin-QG}, $\partial C_{t, n}$ converges to $\partial C_{t}$ as $n\to\infty$.
 
 
By \cite{AKP-buyalo},
$\partial C_{t,n}$ is
an Alexandrov space 
with curvature $\ge -1$;
hence, so is the limit
$\partial C_{t}$.

Note that $L_{t,n}$ with induced inner metric is isometric to its image in $\partial C_{t,n}$.
Since 
$\partial C_{t}$
is an extremal subset of
$C_{t}$, the inner metric of
$\partial C_{t} $ is bi-Lipschitz to %???REF
the metric restricted from $A$.
It follows that
we can take $r$ sufficiently small
such that for all $t$ and
$U_{t,n}=L_{t,n}\cap B(x_n,r)$
we will have
\[h_n\le \tfrac 1{10}\cdot \dist   (U_{t,n},\partial C_{t,n}\setminus L_{t,n}),\]
where $h_n$ denotes the intrinsic diameter of $U_{t,n}$.
Then the local version of 2-dimensional case of the main theorem can be applied to $U_{t,n}$; it implies weak convergence of measures  $\Int_n\cdot \vol^2_n$ on $L_{t,n}$.

Choose a smooth function
$\phi\:\Omega\to\RR$ with a compact support in $A^\delta_\Omega$.
Applying the coarea formula, we get
$$\int\limits_{s\in \Omega}\Int_n(s)\cdot\phi(s)\cdot\vol^3_n
=\int\limits_{-h}^{h} d t\cdot 
\int\limits_{s\in U_{t,n}}
 \frac{\phi_n(s)}{|\nabla_n f(s)|}\cdot\Int_n(s)\cdot \vol^2.\eqlbl{eq:Int-coarea}$$
 
Note that $\nabla_n f$ is bounded away from zero.
By \ref{lem:test-delta}(\ref{lem:test-delta|nabla|}), $\frac{1}{|\nabla_n f(s)|}$ is weakly delta-converging.
Since $\Int_n\cdot \vol^2$ is weakly converging measures
{\color{blue} ???REF}
on $L_n$, \ref{obs:delta-weak-uniform} implies
that $\Int_n\cdot\vol^3_n$ is weakly delta-converging measures on $M_n$.
\qeds

The following lemma is related to the convergence of the second integral in \ref{eq:Bochner2}, the proof uses
DC-calculus in a common chart; see Section~\ref{sec:DC}.

\begin{thm}{Lemma}\label{HnablaU}
In the assumptions of Proposition~\ref{strconvergence}, 
suppose $\psi\:\Omega\to\RR$ is a smooth function with compact support.
Then
\[\int\limits_\Omega \l[H_n\cdot\<u_n,\nabla_n\phi\>_n- \<\nabla_{u_n} u_n,\nabla_n\phi\>_n \r]\cdot\vol^3_n\]
converges, where $H_n$ and $u_n$ as in  \ref{eq:Bochner2}.
\end{thm}

\parit{Proof.}
Note that $H_n=\div u_n$.
Let us rewrite the first term of \ref{eq:Bochner2} in coordinates:
$$\int\limits_{\Omega}
\biggl[
\sum_{i}
\Bigl(\partial_i u_n^i
+
\tfrac{u^i_n}{2}\cdot\partial_i \log \det g_{n}\Bigr)\biggr]
\cdot
\biggl[\sum_{i,j} g_{ij,n} \cdot u_n^i\cdot \partial_j \phi\biggr]
\cdot\sqrt {\det g_n}
\cdot dx^1 dx^2 dx^3$$
We also have
$$u_n^i
=
\frac{\sum_j g^{ij}\cdot\partial_j f}
{\sqrt{\sum_{j,k}g^{jk}\cdot\partial_j f\cdot\partial_k f}}.$$

Taking the derivatives, we see under the integral a sum of products the following two types of expressions:
the first a partial derivative
$\partial_kg_{ij,n}$,
$\partial_kg^{ij}_n$,
or $\partial_i\partial_j f$,
and the second is an expression made from
$g_{ij,n}$,
$g^{ij}_n$,
$\partial_i f$,
$\partial_i \phi$.
Applying \ref{lem:test-delta}, we get that 
the integral converges.

Further, for the second term in \ref{eq:Bochner2}
we have
\begin{align*}
\int\limits_{M_n}&\<\nabla\psi_n,\nabla_{u_n} u_n\>\cdot\vol^3_n=\\
&=
\int\limits_{\Omega}
\sum_{i,j,k}
u^i_n \partial_k \phi
\left(
\partial_i u^k_n
+
\tfrac{u^j_n}2
\cdot
\sum_s
\left(
\partial_i g_{ js,n}
+
\partial_j g_{si,n}
-
\partial_s g_{i j,n}\right)
\cdot
g^{ks}_n
\right)
\cdot 
\\
&\quad\cdot\sqrt{\op{det}g_{n}}\cdot dx^1 dx^2 dx^3.
\end{align*}
The convergence follows by the same argument.
\qeds

\parit{Proof of Proposition~\ref{strconvergence}.}
By \ref{eq:Bochner2}, \ref{Int}, and \ref{HnablaU} we get that $\Str_n(u_n,u_n)\cdot \vol^3_n$ is a weakly delta-converging sequence of measures.
It remains to apply \ref{lem:test-delta}(\ref{lem:test-delta|nabla|}),
\ref{obs:delta-weak-uniform},
and the fact that $\Rm$ can be expressed from $\Str$.
\qeds
