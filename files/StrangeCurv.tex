\section{Strange curvature convergence}

Let $M$ be a Riemannian manifold.
We define a $(0,2)$ tensor  $ \Str$  by
$$\Str(w,w)=\Sc\cdot |w|^2-\Ric(w,w), \quad w\in TM$$
and call it {\it Strange curvature} tensor or just Strange curvature.
 
 In this section we prove a 
convergence of Strange curvature 
in $3$-dimensional case
in the following sense.

\begin{thm} {Proposition}\label{strconvergence}
Let $M_n\in\M_{\ge -1}^3$,
	$M_n\GHto A$ , $x\in A^\delta$
	and $B_r (x)$
	be a good domain for this convergence.
	
Let $M_n\ni x_n \to x\in A^0$ and
 $f_n\in C^\infty(B_r(x_n))$ be a
 sequence of concave functions such that
$f_n\ccto f $, suppose in addition that 
$1/c\le|\nabla f_n|\le c$ and set $u_n=\nabla f_n/|\nabla f_n|$.
Then for sufficiently small $r_0>0$
and some constant $ C>0$  the following holds.

% Let us denote $u_n=\nabla f/|\nabla f|$.
Consider
a sequence of measures 
$m_n=
\Str\left(
u_n, u_n\right) \cdot \vol_n$
on $B_{r_0}(x_n)$ and
suppose $\tau_1, \tau_2\in \mathfrak M(B_{r_0}(x))$ be two partial
weak limits of $m_n$.
Then we have
$|\tau_1(S)-\tau_2(S)| \le
C\delta $ for every $S\subset A^\delta \cap B_{r_0}(x)$.

\end{thm}


The proof relies on \ref{thm:bochner-formula},
it provides the following integral expression
for Strange curvature
$$\int\limits_\Omega \phi\cdot \Str(u_n, u_n)
=\int\limits_\Omega \phi\cdot \Int_n+
\int\limits_\Omega \l[H_n\cdot\<u_n,\nabla\phi\>- \<\nabla\phi,\nabla_{u_n} u_n\> \r],
\eqlbl{Bochner}$$
where 
\begin{itemize}
\item $H_n(x)$ --- the mean curvature of the level set $f_n^{-1}(f_n(x))$,
\item $\Int_n(x)$ --- the scalar curvature of  $f_n^{-1}(f_n(x))$.
\end{itemize}
 
This formula reduces the Proposition~\ref{strconvergence} to the two lemmas below,
each lemma provides the convergence an integral term in the right-hand-side of \ref{Bochner}.
We prove the convergence of the first integral in a sense of $C\delta$-convergence of measures:


\begin{thm} {Lemma}\label{Int}
In condition of Proposition~\ref{strconvergence}
for sufficiently small $r_0>0$
and some constant $ C>0$  the following holds.

% Let us denote $u_n=\nabla f/|\nabla f|$.
For any two partial
weak limits $\tau_1, \tau_2\in \mathfrak M(B_{r_0}(x))$  of 
measures $\Int_n\cdot \vol_n$ on $B_{r_0}(x_n)$
we have
$|\tau_1(S)-\tau_2(S)| \le
C\delta $ for every $S\subset A^\delta \cap B_{r_0}(x)$.

\end{thm}

The proof of this lemma
 uses 
convergence of curvature measures
$\Int_n\cdot \vol_n$
on $2$-dimensional 
smoothing sequence (level  sets of concave functions$f_n$ are
locally in $\DM^3$).
The error $C\delta$ arises 
because  sequence of functions $|\nabla f_n|$ is 
only
 $C\delta$-convergent   on $A^\delta$.




 
The next lemma is related to the convergence of the second integral, the proof uses
$DC$-calculus in a Nice common chart. 


\begin{thm}{Lemma}\label{HnablaU}
In condition of Proposition~\ref{strconvergence}
let $\mathfrak X_n:U_n\to\Omega$
and
$\mathfrak X:U_x\to\Omega$
be a Nice common chart around $x$. 
We fix some smooth function with compact support
$\psi\in C^\infty_0(\Omega)$ and denote by
$\phi_n=\psi\circ\mathfrak X_n$. Then

$$
\int\limits_{U_n} \l[H_{f_n}\cdot\<u_n,\nabla\phi_n\>- \<\nabla\phi_n,\nabla_{u_n} u_n\> \r]
\eqlbl{Bochner}$$
converges.

\end{thm}


\parit{Proof of Lemma~\ref{HnablaU}.}
We rewrite
$$\int\limits_{V_n} \<u_n,\nabla\psi_n\> H_nd\vol_n
=
\int\limits_{V_n} \<u_n,\nabla\psi_n\> \div u_nd \vol_n
$$

Let us rewrite the integral in $\Omega$:
$$\int\limits_{\Omega}
\Bigl(\sum_{i=1}^3u_n^i \tfrac{\partial \phi}{\partial x_i}\Bigr)
\sum_{i=1}^3
\Bigl(\tfrac{\partial u_n^i}{\partial x^i} +u^i_n\tfrac{\partial \log \sqrt {\det (g_{ij,n})}}{\partial x^i}\Bigr)
\sqrt {\det (g_{ij,n})}dx^1\wedge dx^2\wedge dx^3$$

Now by Proposition~\ref{Prop:chart}.\ref{metric}
$g_{i j,n}, g^{ij}_n\in  \op{BV_0^{seq}}(\Omega,S_\Omega,\RR)$
and
$\det g_{ij,n}$ are bounded away from $0$.
We also have
$$u_n^i=\frac{g^{ij}\frac{\partial (f\circ \X_n^{-1})}{ \partial x_j}}
{\sqrt{g^{jk}\frac{\partial (f\circ \X_n^{-1})}{\partial x_j }\frac{\partial (f\circ \X_n^{-1})}{\partial x_k}}}.$$
By Proposition~\ref{Prop:chart}.\ref{funktioninchart}
 $f_n\circ\X_n^{-1}\in DC_0^{seq}(\Omega, S_\Omega)$
hence 
${u^i_n}\in  \op{BV_0^{seq}}(\Omega,S_\Omega,\RR)$.

Then applying lemmas~\ref{thm-D}, \ref{thm-CM}, and \ref{thm-A} we get

\begin{itemize}

\item $\partial \log \sqrt {\det (g_{ij,n})}/\partial x^i\in \aleph_0^{seq}(\Omega,S_\Omega)$

\item $\frac{\partial u_n^i}{\partial x_i}\in \aleph_0^{seq}(\Omega,S_\Omega)$
 
\end{itemize} 
 
 and by above
 
  \begin{itemize}
  
  \item ${u^i_n}\in  \op{BV_0^{seq}}(\Omega,S_\Omega,\RR)\subset\op{C_0^{seq}}(\Omega,S_\Omega,\RR)$ 
 
 \item $g_{i j,n}\in   \op{BV_0^{seq}}(\Omega,S_\Omega,\RR)\subset \op{C_0^{seq}}(\Omega,S_\Omega,\RR)$
 
  \end{itemize}
  
  Then applying Lemma~\ref{thm-CM}
  we get under integral the sum of elements from $\aleph_0^{seq}(\Omega,S_\Omega)$
  multiplied by $\frac{\partial \phi}{\partial x_i}$ --- smooth functions with compact support.
       This gives convergence of the integral by definition of  $\aleph_0^{seq}(\Omega,S_\Omega)$.
              

Further, for the second integral % (see claim~\ref{cl:funDC})
we have
$$\int\limits_{M_n}\<\nabla\psi_n,\nabla_{u_n} u_n\>d\vol_n=$$
 $$\int\limits_{\Omega}  \sum_{i,j,k}u^i_n \frac{\partial \phi}{\partial x_k}
 \biggl(\frac{\partial u^k_n}{\partial x_i} +\frac{1}{2}u^j_n\sum_s
 \biggl(\frac{\partial g_{ js,n}}{\partial x_i}+\frac{\partial g_{si,n}}{\partial x_j}-\frac{\partial g_{i j,n}}{\partial x_s}\biggr) g^{ks}_n\biggr)\cdot \sqrt{\op{det}(g_{i j,n})}dx^1\wedge dx^2\wedge dx^3$$ 
 
For the first integral,
 applying
lemmas~\ref{thm-D}, \ref{thm-CM}, and \ref{thm-A} we get under the integral sequence from $\aleph_0^{seq}(\Omega,S_\Omega)$ multiplied by smooth function with compact support $\frac{\partial \phi}{\partial x_i}$. Then the integral converges and the lemma follows.
\qeds


\parit{Proof of Lemma~\ref{Int}.}
Any point  in an Alexandrov space
has a convex neighborhood (\cite{convexity}) and by
construction  it can be lifted  to a smoothing sequence.
So let 
$V\subset A$ be an open  convex
neighborhood of $x$ and
$ V_n\subset M_n$
be open convex sets such that
$V_n  \dto{GH}   V$.

We set $f_0:=f$, $h:=\sup |f_n|<\infty$
and denote 
$$L_{t,n}=f_n^{-1}(t)\cap V_n,\qquad
C_{t,n}=f_n^{-1}[t,h]\cap \bar V_n.$$
 Now we want to show that local version of
Main theorem  for dimension 2 can be applied to small subsets of $L_{t,n}$ with some uniform estimates.



For every $t$ and $n\ge 0$, the set $C_{t,n}$ is a convex subset in Alexandrov space 
 and hence is an Alexandrov space 
 with curvature $\ge -1$.
For any $t_n\to t^*$ we have that
$C_{t_n,n}     \dto{GH}    C_{t^*,0} $. 
Then boundaries  $\partial C_{t_n,n}  $
  converge to $\partial C_{t^*,0}  $ in
Gromov--Hausdorff sense and then by  
  \cite{petrunin-QG} (Theorem 1.2)
 $\partial C_{t_n, n}  $ equipped with inner metric
 $\rho^{\partial C_{t_n,n}}$ converge to
 $\partial C_{t^*,0}  $ with inner metric
 $\rho^{\partial C_{t^*,0}}$.
 
 
 It follows from \cite{AKP} that
 for $n\ge 1$
  (since $C_{t,n}$ are
 convex subsets in a Riemannian manifolds with curvature $\ge -1$)
  the boundary
$(\partial C_{t,n}, \rho^{\partial C_{t_n,n}}) $ is
an Alexandrov space 
 with curvature $\ge -1$. Hence the limit
$(\partial C_{t^*,0}, \rho^{\partial C_{t^*,0}}) $ is
an Alexandrov space 
 with curvature $\ge -1$.

Let $\rho_{t,n}$ be induced inner metric on $L_{t,n}$,
then $\rho_{t,n}$ locally coincide with
$\rho^{\partial C_{t,n}}$ and hence 
for any $t^*$ and $t_n\to t_*$
$(L_{t_n,n}, \rho_{t_n,n})\dto{GH} (L_{t_*,0},\rho_{t^*,0} )$.
Note that since $(L_{t,n}, \rho_{t,n})\in\M_{\ge -1}^m$
we obtained
local smoothing sequences. It remains to find good inner
subsets for these smoothings.

Let  note that since 
$\partial C_{t^*,0}$
is an extremal subset for
$C_{t^*,0}$
 the inner metric
$ \rho^{\partial C_{t^*,0}} $ is bi-Lipschitz to
the metric restricted from $A$ to
$\partial C_{t^*,0}$.
It follows that
we can take $r$ sufficiently small
such that for all $t$ and
$U_{t,n}=L_{t,n}\cap B_{r}(x_n)$
we will have
$\diam_{\rho_{t,n}} U_{t,n}\le 1/10 \dist   (U_{t,n},\partial C_{t,n}\setminus L_{t,n})$. Then for
all $y\in U_{t,n}$ closed balls
$\overline {B^{L_{t,n}}_{5\diam_{\rho_{t,n}} U_{t,n}}}(y) $ are compact.
Hence every $U_{t,0}$ is a good inner subset for 
smoothing  $L_{t,n}\to L_{t,0}$
and local version of Main theorem for dimension 2
can be applied to this set.

Now we fix some
$\phi\in C^0_c(B_r(x)\cap A^\delta) $ and
$\phi_n\in C^0_c(B_r(x_n)) $,
such that
$\phi_n\cccto\phi$.


$$ \int\limits_{B_r(x_n)}\Int_n(s)\phi_n(s)d\vol_n(s)
=\int\limits_{-h}^{h} d t\int\limits_{ U_{t,n}}
 \frac{\phi_n(s)}{|\nabla f_n(s)|}\Int_n(s)d S_n^t(s),$$
 where $ S_n^t$ is the $2$-volume on $ L_{t,n}$.
 For any $t\in[-h,h]$ we have
 $L_{t,0}\in\DAl^m$,
$L_{t,n}\in\DM^m$,
$L_{t,n}\dto{GH} L_{t,0}$
 and $U_{t,0}$ is good inner set for this smoothing.
Then local version of Main theorem  for dimension $2$ can be applied.
 Hence $\Int_n(x)dS_n^t(x)$ weakly converges to some measure on $U_{t,n}$.
 From 
compactness  $\sup_{t} \diam U_{t,n}<\infty$.
 Then by  Corollary~\ref{Kbound} 
$\int\limits_{ L_t^n}
 |\Int_n(x)|dS_n^t(x)\le c_1$ for some $c_1$ independent on $t$. We know that
 $\operatorname{supp}(\phi/{|\nabla  f_n|})\in C^0_c(B_r(x)) $
and the sequence
  ${\phi_n}/{|\nabla  f_n|}$
$c_2\de$-converges on $A^\delta$ 
for some $c_2>0$
(see claim~\ref{lem:scalprod}).
Then for all $t$ we have
$$\limsup_{n\to\infty}\int\limits_{U_{t,n}} \frac{\phi_n(x)}
{|\nabla f_n(x)|}\Int_n(x)dS_n^t(x)-
\liminf_{n\to\infty}\int\limits_{U_{t,n}} \frac{\phi_n(x)}{|\nabla f_n(x)|}\Int_n(x)dS_n^t(x)
\le c_1c_2\delta.$$

Hence

$$
\limsup_{n\to\infty}
 \int\limits_{B_r(x_n)}\Int_n(s)\phi_n(s)d\vol_n(s)-
\liminf_{n\to\infty}
 \int\limits_{B_r(x_n)}\Int_n(s)\phi_n(s)d\vol_n(s)
\le 2hc_1c_2\delta.$$
\qeds

\parit{Proof of Proposition~\ref{strconvergence}.}
We take a Nice common chart 
$\mathfrak{X}_n:U_n\to\Omega$ and
$\mathfrak{X}:U\to\Omega$ around $x$.
We can assume 
that $U\subset B_{r_0}(x)$ where $r_0$ is 
from conclusion of Lemma~\ref{Int}.


Let us define sequences 
$L_n, L_n^1, L_n^2:C^\infty_0(\X(U\cap A^\delta))\to \R$
of continuous linear operators

$$L_n(\psi)=
\int\limits_{U_n}(\psi\circ\mathfrak X_n )\cdot \Str(u_n, u_n)=
\int\limits_{U_n} \phi_n\cdot \Str(u_n, u_n)
$$

$$L_n^1(\psi)=
\int\limits_{U_n} \phi_n\cdot \Int_n,
\qquad L_n^2(\psi)=
\int\limits_{U_n}
 \l[H_n\cdot\<u_n,\nabla\phi_n\>- \<\nabla\phi_n,\nabla_{u_n} u_n\> \r].$$

Then \ref{Bochner} implies that $L_n=L_n^1+L_n^2$.
It follows from
Lemma~\ref{Int} that 
$L^1_n(\psi)$ $c\delta$-converges for any $\psi\in C^\infty_0(\X(U\cap A^\delta))$.
From Lemma~\ref{HnablaU} we have that
$L^2_n(\psi)$ converges for any $\psi\in C^\infty_0(\X(U\cap A^\delta))$.
It follows from Corollary~\ref{Kbound}  that operators $L_n$
are uniformly bounded with respect to the uniform norm and by above
$L_n(\psi)$ $c\delta$-converges for any $\psi\in C^\infty_0(\X(U\cap A^\delta))$.
Then the sequence $L_n$ regarded as a sequence 
of measures weakly $c\delta$ converges to some 
measure.
Proposition~\ref{strconvergence} follows.
\qeds
