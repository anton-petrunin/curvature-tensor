\subsection{Strange curvature}

Suppose $M$ is a 3-dimensional Riemannian manifold.
\emph{Strange curvature tensor} $\Str$ on $M$ is a bilinear that is uniquely defined by
$$\Str(w,w)=\Sc\cdot |w|^2-\Ric(w,w)$$
for $w\in \T M$.
Note that strange curvature tensor completely describes Ricci curvature tensor.
Further, since $M$ is 3-dimensional, $\Str$ completely describes its curvature tensor $\Rm$.

The strange curvature tensor appears in the following integral expression from \ref{thm:bochner-formula}:
$$\int\limits_\Omega \phi\cdot \Str(u_n, u_n)
=\int\limits_\Omega \phi\cdot \Int_n+
\int\limits_\Omega \l[H_n\cdot\<u_n,\nabla\phi\>- \<\nabla\phi,\nabla_{u_n} u_n\> \r],
\eqlbl{Bochner}$$
where 
\begin{itemize}
\item $H_n(x)$ --- the mean curvature of the level set $f_n^{-1}(f_n(x))$,
\item $\Int_n(x)$ --- the scalar curvature of  $f_n^{-1}(f_n(x))$.
\end{itemize}
This formula is the main tool in the proof of the following proposition.

\begin{thm} {Proposition}\label{strconvergence}
Let $M_n\smooths{} A$ be a smoothing of a 3-dimensional Alexandrov space $A$.
Suppose $M_n\ni x_n \to x\in A^\circ$ and $f_n\in C^\infty(B(x_n,r))$ is a sequence of concave functions such that
$f_n\testto f\:A\to \RR$ and $1/c\le|\nabla f_n|\le c$ for some constant $c$.
Set $u_n=\nabla f_n/|\nabla f_n|$.
Then for sufficiently small $r_0>0$ the measures 
$m_n=
\Str\left(
u_n, u_n\right) \cdot \vol_n$ are weakly delta-converging in $B(x,r_0)$.

\end{thm}

Applying the Bochner-type formula \ref{thm:bochner-formula} we can reduce the proposition to the following two lemmas;
each lemma provides the convergence of an integral term in the right-hand-side of \ref{Bochner}.

\begin{thm} {Lemma}\label{Int}
In condition of Proposition~\ref{strconvergence}
for sufficiently small $r_0>0$
and some constant $ C>0$  the following holds the mearures $\Int_n\cdot \vol_n$ is delta-converging in $B(x,r_0)$.
\end{thm}

The proof of this lemma uses 
convergence of curvature measures
$\Int_n\cdot \vol_2$
on  the 2-dimension level sets of concave functions $f_n$ and coarea formula.
The factor $|\nabla f_n|$ appears in the coarea formula;
since the sequence $|\nabla f_n|$ is only delta-convergent, we get that only delta-convergence of $\Int_n\cdot \vol_n$.

\parit{Proof.}
Any point  in an Alexandrov space
has a convex neighborhood \cite{petrunin-conc} and by
construction  it can be lifted  to a smoothing sequence.
So let 
$V\subset A$ be an open  convex
neighborhood of $x$ and
$ V_n\subset M_n$
be open convex sets such that
$V_n  \dto{GH}   V$.

We set $f_0:=f$, $h:=\sup |f_n|<\infty$
and denote 
$$L_{t,n}=f_n^{-1}(t)\cap V_n,\qquad
C_{t,n}=f_n^{-1}[t,h]\cap \bar V_n.$$
Now we want to show that the local version of the main theorem (\ref{mainloc}) in dimension 2 can be applied to small subsets of $L_{t,n}$ with some uniform estimates.

For every $t$ and $n\ge 0$, the set $C_{t,n}$ is a convex subset in Alexandrov space 
 and hence is an Alexandrov space 
 with curvature $\ge -1$.
For any $t_n\to t^*$ we have that
$C_{t_n,n}     \dto{GH}    C_{t^*,0} $. 
Then boundaries  $\partial C_{t_n,n}  $
  converge to $\partial C_{t^*,0}  $ in
Gromov--Hausdorff sense and then by  
  \cite{petrunin-QG} (Theorem 1.2)
 $\partial C_{t_n, n}  $ equipped with inner metric
 $\rho^{\partial C_{t_n,n}}$ converge to
 $\partial C_{t^*,0}  $ with inner metric
 $\rho^{\partial C_{t^*,0}}$.
 
 
 It follows from \cite{AKP} that
 for $n\ge 1$
  (since $C_{t,n}$ are
 convex subsets in a Riemannian manifolds with curvature $\ge -1$)
  the boundary
$(\partial C_{t,n}, \rho^{\partial C_{t_n,n}}) $ is
an Alexandrov space 
 with curvature $\ge -1$. Hence the limit
$(\partial C_{t^*,0}, \rho^{\partial C_{t^*,0}}) $ is
an Alexandrov space 
 with curvature $\ge -1$.

Let $\rho_{t,n}$ be induced inner metric on $L_{t,n}\subset M_n$ and $\rho_{t}$ be induced inner metric on $L_{t}\subset A$
Note that $\rho_{t,n}$ coincides locally coincide with the induced length-metric in 
$\partial C_{t,n}$.
If $t_n\to t_\infty$, then 
$(L_{t_n,n}, \rho_{t_n,n})\dto{GH} (L_{t_\infty},\rho_{t_\infty})$.
Since $(L_{t,n}, \rho_{t,n})$ is a 2-dimensional manifold with curvature at least $-1$,
we obtained local smoothing sequences where local version of the main theorem is applicable.

Note that since 
$\partial C_{t^*,0}$
is an extremal subset for
$C_{t^*,0}$
 the inner metric
$ \rho^{\partial C_{t^*,0}} $ is bi-Lipschitz to
the metric restricted from $A$ to
$\partial C_{t^*,0}$.
It follows that
we can take $r$ sufficiently small
such that for all $t$ and
$U_{t,n}=L_{t,n}\cap B(x_n,r)$
we will have
$\diam_{\rho_{t,n}} U_{t,n}\le \tfrac 1{10} \dist   (U_{t,n},\partial C_{t,n}\setminus L_{t,n})$.
Then for any point $y\in U_{t,n}$ the closed ball
$\overline{B}(y,5\diam_{\rho_{t,n}})_{L_{t,n}}$ is compact.
Hence every $U_{t,0}$ is a good inner subset for 
smoothing  $L_{t,n}\to L_{t,0}$
and local version of Main theorem for dimension 2
can be applied to this set.

Choose a test function
$\phi$ with a compact support in $A^\delta$
suppose 
$\phi_n$ is a test function on $M_n$ 
such that
$\phi_n\testto\phi$.
Applying the coarea formula, we get
$$\int\limits_{s\in B(x_n,r)}\Int_n(s)\cdot\phi_n(s)\cdot\vol_3
=\int\limits_{-h}^{h} d t\cdot 
\int\limits_{s\in U_{t,n}}
 \frac{\phi_n(s)}{|\nabla f_n(s)|}\cdot\Int_n(s)\cdot \vol_2.\eqlbl{eq:Int-coarea}$$
For any $t\in[-h,h]$ we have that
$L_{t}$ is locally Alexandrov space,
$L_{t,n}$ is a Riemannian manifold with lower curfature bound $-1$,
and
$L_{t,n}\smooths{} L_{t}$.
Then local version of Main theorem  for dimension $2$ can be applied.
Hence $\Int_n(x)\cdot \vol_2$ weakly converges to some measure on $U_{t}$.
From compactness  $\sup_{t} \diam U_{t,n}<\infty$.
Then by Corollary~\ref{Kbound} 
\[\int\limits_{ L_{t,n}}
 |\Int_n(x)|\cdot \vol_2\le c_1\] 
for some $c_1$ independent on $t$. 
The function $\phi/{|\nabla  f_n|}$ has support in 
 $B(x,r)$
and the sequence
  ${\phi_n}/{|\nabla  f_n|}$
is uniformly delta-converges on $A^\delta$ 
(see claim~\ref{lem:scalprod}).
Therefore for all $t$ we have $\frac{\phi_n(x)}
{|\nabla f_n(x)|}\Int_n(x)\cdot \vol_2$ weakly delta-converges in $U_{t}$ for any $t$.
Therefore, by \ref{eq:Int-coarea}, $\Int_n\phi_n(s)\cdot \vol_3$ weakly delta-converges in $B(x,r)$.
\qeds

The following lemma is related to the convergence of the second integral in \ref{Bochner}, the proof uses
$DC$-calculus in a common chart. 

\begin{thm}{Lemma}\label{HnablaU}
In condition of Proposition~\ref{strconvergence}
let $\mathfrak X_n:U_n\to\Omega$
and
$\mathfrak X:U_x\to\Omega$
be a common chart around $x$. 
We fix some smooth function with compact support
$\psi\in C^\infty_0(\Omega)$ and denote by
$\phi_n=\psi\circ\mathfrak X_n$. Then

$$
\int\limits_{U_n} \l[H_n\cdot\<u_n,\nabla\phi_n\>- \<\nabla\phi_n,\nabla_{u_n} u_n\> \r]
\eqlbl{Bochner}$$
converges, where $H_n$ as in  \ref{Bochner}.

\end{thm}

\parit{Proof.}
Note that $H_n=\div u_n$.
Let us rewrite the integral in $\Omega$:
$$\int\limits_{\Omega}
\Bigl(\sum_{i=1}^3u_n^i \tfrac{\partial \phi}{\partial x_i}\Bigr)
\sum_{i=1}^3
\Bigl(\tfrac{\partial u_n^i}{\partial x^i} +u^i_n\tfrac{\partial \log \sqrt {\det (g_{ij,n})}}{\partial x^i}\Bigr)
\sqrt {\det (g_{ij,n})}dx^1\wedge dx^2\wedge dx^3$$

By \ref{Prop:chart}(\ref{metric}), we have
$g_{i j,n}, g^{ij}_n\in  \op{BV_0^{seq}}(\Omega,S_\Omega,\RR)$
and
$\det g_{ij,n}$ are bounded away from $0$.
We also have
$$u_n^i=\frac{g^{ij}\frac{\partial (f\circ \X_n^{-1})}{ \partial x_j}}
{\sqrt{g^{jk}\frac{\partial (f\circ \X_n^{-1})}{\partial x_j }\frac{\partial (f\circ \X_n^{-1})}{\partial x_k}}}.$$
By Proposition~\ref{Prop:chart}.\ref{funktioninchart}
 $f_n\circ\X_n^{-1}\in DC_0^{seq}(\Omega, S_\Omega)$
hence 
${u^i_n}\in  \op{BV_0^{seq}}(\Omega,S_\Omega,\RR)$.

Then applying lemmas~\ref{thm-D}, \ref{thm-CM}, and \ref{thm-A} we get

\begin{itemize}

\item $\partial \log \sqrt {\det (g_{ij,n})}/\partial x^i\in \aleph_0^{seq}(\Omega,S_\Omega)$

\item $\frac{\partial u_n^i}{\partial x_i}\in \aleph_0^{seq}(\Omega,S_\Omega)$
 
\end{itemize} 
 
 and by above
 
  \begin{itemize}
  
  \item ${u^i_n}\in  \op{BV_0^{seq}}(\Omega,S_\Omega,\RR)\subset\op{C_0^{seq}}(\Omega,S_\Omega,\RR)$ 
 
 \item $g_{i j,n}\in   \op{BV_0^{seq}}(\Omega,S_\Omega,\RR)\subset \op{C_0^{seq}}(\Omega,S_\Omega,\RR)$
 
  \end{itemize}
  
Then applying Lemma~\ref{thm-CM} we get under integral the sum of elements from $\aleph_0^{seq}(\Omega,S_\Omega)$ multiplied by $\frac{\partial \phi}{\partial x_i}$ --- smooth functions with compact support.
This gives convergence of the integral by definition of  $\aleph_0^{seq}(\Omega,S_\Omega)$.
              

Further, for the second integral % (see claim~\ref{cl:funDC})
we have
\begin{align*}
\int\limits_{M_n}&\<\nabla\psi_n,\nabla_{u_n} u_n\>\cdot\vol_n=\\
&=\int\limits_{\Omega}  \sum_{i,j,k}u^i_n \frac{\partial \phi}{\partial x_k}
\biggl(\frac{\partial u^k_n}{\partial x_i} +\tfrac{1}{2}u^j_n\sum_s
\biggl(\frac{\partial g_{ js,n}}{\partial x_i}+\frac{\partial g_{si,n}}{\partial x_j}-\frac{\partial g_{i j,n}}{\partial x_s}\biggr) g^{ks}_n\biggr)\cdot 
\\
&\quad\cdot\sqrt{\op{det}(g_{i j,n})}\cdot dx^1\wedge dx^2\wedge dx^3.
\end{align*}


For the first integral,
 applying
lemmas~\ref{thm-D}, \ref{thm-CM}, and \ref{thm-A} we get under the integral sequence from $\aleph_0^{seq}(\Omega,S_\Omega)$ multiplied by smooth function with compact support $\frac{\partial \phi}{\partial x_i}$. Then the integral converges and the lemma follows.
\qeds

\parit{Proof of Proposition~\ref{strconvergence}.}
We take a common chart 
$\mathfrak{X}_n:U_n\to\Omega$ and
$\mathfrak{X}:U\to\Omega$ around $x$.
We can assume 
that $U\subset B(x,r_0)$ where $r_0$ is 
from conclusion of Lemma~\ref{Int}.


Let us define sequences 
$L_n, L_n^1, L_n^2:C^\infty_0(\X(U\cap A^\delta))\to \R$
of continuous linear operators

$$L_n(\psi)=
\int\limits_{U_n}(\psi\circ\mathfrak X_n )\cdot \Str(u_n, u_n)=
\int\limits_{U_n} \phi_n\cdot \Str(u_n, u_n)
$$

$$L_n^1(\psi)=
\int\limits_{U_n} \phi_n\cdot \Int_n,
\qquad L_n^2(\psi)=
\int\limits_{U_n}
 \l[H_n\cdot\<u_n,\nabla\phi_n\>- \<\nabla\phi_n,\nabla_{u_n} u_n\> \r].$$

Then \ref{Bochner} implies that $L_n=L_n^1+L_n^2$.
It follows from
Lemma~\ref{Int} that 
$L^1_n(\psi)$ delta-converges for any $\psi\in C^\infty_0(\X(U\cap A^\delta))$.
From Lemma~\ref{HnablaU} we have that
$L^2_n(\psi)$ converges for any $\psi\in C^\infty_0(\X(U\cap A^\delta))$.
It follows from Corollary~\ref{Kbound}  that operators $L_n$
are uniformly bounded with respect to the uniform norm and by above the sequence 
$L_n(\psi)$ is weakly  delta-converging for any $\psi\in C^\infty_0(\X(U\cap A^\delta))$.
Then the sequence $L_n$ regarded as a sequence 
of measures weakly delta-converges to some 
measure.
Proposition~\ref{strconvergence} follows.
\qeds
