\section{Singularities of codimension 3}

Let $M$ be an $m$-dimensional Riemannian manifold
Set
\[K_{max}(x)=\sup |K_\sigma|\]
where $\sigma$ runs along all sectional directions at $x\in M$.
The following statement is a direct corollary of the main result in \cite{petrunin-SC}:

\begin{thm}{Corollary}\label{cor:Kmax}
Let $M$ be an $m$-dimensional Riemannian manifold with sectional curvature bounded below by $-1$.
Then there a constant $\const(m)$ such that
$$\int_{B(p,r)_M} K_{max}\le \const(m)\cdot r^{m-2}.$$
for any $r<1$.
\end{thm}

\parit{Proof of part \ref{prop:3parts:codim3} in Proposition~\ref{prop:3parts}.}
Observe that there is a constant $\const(m)$ such that 
\[
\begin{aligned}
|q_n(\nabla f_1,&\dots,\nabla f_{m-2},\nabla g_1,\dots,\nabla g_{m-2})|
\\
&\le 
\const(m)\cdot K_{max}\cdot|\nabla f_1\wedge\dots\wedge\nabla f_{m-2}|\cdot |\nabla g_1\wedge\dots\wedge\nabla g_{m-2}|
\end{aligned}
\eqlbl{q=<K}
\]

According to \cite[10.6]{BGP}, $A''$ has vanishing $(m-2)$-dimensional Hausdorff measure;
that is, $A''$ can be covered by a countable family of balls $B(x_i,r_i)$ such that $\sum r_i^{m-2}$ is arbitrary small.
Therefore \ref{cor:Kmax} and \ref{q=<K} imply the statement.
\qeds
