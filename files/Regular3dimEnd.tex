\subsection{Three-dimensional case}\label{sec:3D-smooth}

In this section we prove Lemma~\ref{A^0} in the 3-dimensional case.

Vectors $w_1,\dots,w_{m(m-1)/2}\in\R^m$ are said to be \emph{in general position}
if the vectors $w_i\otimes w_i$ form a basis in $\RR^{m\cdot(m-1)/2}$ --- the symmetric square of $\RR^m$.
In this case, any quadratic form $Q$ on $\R^m$
can be computed from the $m(m-1)/$ values 
\[Q(w_1, w_1),\ \dots,\ Q(w_{m(m-1)/2}, w_{m(m-1)/2}).\] 
More precisely, there are rational functions 
$s_1,\dots,s_{m(m-1)/2}$ that take $\tfrac{m(m-1)}2$ vectors in $\R^m$ and returns a quadratic form on $\R^m$
such that
$$Q=\sum_{k=1}^{m(m-1)/2}s_k(w_1,\dots,w_{m(m-1)/2})\cdot Q(w_k,w_k).
\eqlbl{Qij}$$

Note that the vectors $w_1,\dots,w_{m(m-1)/2}\in\R^m$  are  in general position if and only if 
$s_k(w_1,\dots,w_{m(m-1)/2})$ are finite for all $k$.
Since $s_k$ are rational functions, we get the following:

\begin{thm}{Observation.}\label{obs:genpos}
Suppose that vectors $w_1,\dots,w_{m(m-1)/2}\in\R^m$ are in general position.
Then the functions $s_1,\dots,s_{m(m-1)/2}$ are Lipschitz in a neighborhood of $(w_1,\dots,w_{m(m-1)/2})\in (\R^m)^{m(m-1)/2}$.
\end{thm}


\parit{Proof of the 3-dimensional case of \ref{A^0}.} 
Choose a common chart
\[M_n\supset U_n\to\Omega,
\quad\text{and}\quad
A_n\supset U\to\Omega.\]
Let us use it to identify tangent spaces of $M_n$ and $A$ with $\RR^3$.
 
We can choose 6 sequences of convex combinations of coordinate functions $f_1,\dots,  f_6$, such that
$\nabla f_1,\zz\dots,\nabla f_6$ are in general position at $p\in \Omega$.
We can assume that $\Omega$ is a small neighborhood of $p$, so by Proposition~\ref{strconvergence}
the measures $\Str_n(\nabla_n f_k, \nabla_n f_k)\cdot\vol^3_n$ weakly delta-converges
on $A^\delta_\Omega$ for $k=1,\dots,6$.

By \ref{obs:genpos}, the functions $s_i$ are Lipschitz in a neighborhood of $(\nabla f_1,\zz\dots,\nabla f_6)\zz\in (\R^3)^6$.
Applying \ref{Qij}, we get that 
\begin{align*}
\Str
&=
\sum_{k=1}^6 s_k(\nabla_n f_1,\dots, \nabla_n f_6)\cdot\Str_n(\nabla_n f_k,\nabla_n f_k).
\end{align*}
Hence the measure $\Str_n(dx_i,dx_j)\cdot\vol^3_n$ are weakly delta-converging for all $i$ and $j$,
where $x_1,x_2,x_3$ is the standard coordinates in $\RR^3$.

By Lemma~\ref{lem:test-delta}, the sequence metric tensors $g_n$ of $M_n$ on $\Omega$ is uniformly delta-converging.
Since the following equality
\[\operatorname{Tr}\Str_n=\sum_{i,j}g^{ij}_n\Str_n(e_{i},e_{j})\cdot \sqrt {\det g_n}\]
holds almost everywhere, we get that the sequence of measures $\operatorname{Tr}\Str_n\cdot\vol^3_n$ is weakly delta-converging.

Note, that for $3$-dimensional manifold we have
$$\Qm_n(V,V)=\Str_n(V,V)-\tfrac14\cdot |V|^2\cdot \Tr \Str_n.\eqlbl{Q}$$
Hence the measures $\Qm_n(d x_i, d x_j)\cdot\vol^3_n$ are weakly delta-converging for all $i$ and~$j$.

Finally, according to \ref{lem:test-delta}(\ref{lem:test-delta-partial}) the components $\alpha_{ik,n}$ of $\nabla_n f_k$ are uniformly delta-converging.
The result follows since
\[\qm(f_k, f_k)=
\sum_{i,j} \alpha_{ik,n}\cdot \alpha_{jk,n}\cdot \qm(x_{i}, x_{j}).\]
\qeds

\subsection{Higher-dimensional case}


\begin{thm}{Observation}\label{obs:nested-convex}
Choose a common chart with the range $\Omega\subset\RR^m$ for a smoothing $M_n\smooths{} A$.
Consider the sequence of coordinate level sets $\Omega\zz=L_m\supset L_{m-1}\supset\dots\supset L_0$, 
where $L_i=L_i(c_{i+1},\dots,c_m)$ is defined by setting the last $m-i$ coordinates to be $c_{i+1},\dots,c_m$ respectively.
Then the level sets $L_i$ is a smooth convex hypersurface in $L_{i+1}$ in each $M_n$;
in particular, each $L_i$ has sectional curvature bounded below by $-1$.

Moreover, the same holds after applying any linear transformation to $\Omega$ that is close to the identity.  
\end{thm}



\parit{Proof of the key lemma (\ref{A^0}).}
Let us use notations as in the observation.
By the key lemma (\ref{A^0}) in dimensions $2$ and $3$,  we get weak delta-convergence of curvature tenors on $L_2$ and $L_3$.
(Again, we apply the local version of these statements as described in Section~\ref{sec:local}.)
In particular, applying the coarea formula, we get convergence of sectional curvature of $L_3$ in the directions of $L_2$ as well as 
the sectional curvature of $L_2$ 
for all values $c_3,\dots,c_m$.
The difference between these curvatures is the Gauss curvature $G_n$ of $L_2$ as submanifold in $L_3$.
Therefore, $G_n$ is weakly delta-converging as well.

Consider a linear transformation of $\Omega$ that preserves the direction of $L_2$.
By the last statement in \ref{obs:nested-convex},
the above argument shows weak delta-convergence of $G_n(w)$, where the direction $w$ of $L_3$ on $L_2$ can be chosen in an open set of $\RR^{m-2}$ --- the space transversal to $L_2$.
In particular, we may choose directions $w_1,\dots, w_{(m-2)\cdot(m-3)/2}$ in $\RR^{m-2}$ that form a generic set (see the  definition in Subsection~\ref{sec:3D-smooth}).

Denote by $G_n^+$ the term in Gauss formula for $L_2$ in $M_n$;
that is, $G_n^+$ is the difference between curvature of $L_2$ and the sectional curvature of $M_n$ in the same direction.
Denote by $g_n$ the Riemannian metric of $M_n$ in $\Omega$.
Note that 
\[G_n^+=\sum\alpha_{k,n}\cdot G_n(w_k),\]
where the coefficients $\alpha_{k,n}$ depend continuously on the components of $g_n$ and $w_1,\zz\dots, w_{(m-2)\cdot(m-3)/2}$. 
It follows that weak delta-convergence of $G_n(w_k)$ implies weak delta-convergence of $G_n^+$ as $n\to\infty$.
Since the curvature of $L_2$ is weakly delta-converging, it implies weak delta-convergence of sectional curvature in the direction of $L_2$.

By the second statement in the observation,
the above argument can be repeated after applying a linear transformation of $\Omega$ that changes the direction of $L_2$ slightly.
It follows that sectional curvatures converge for a generic array of simple bivectors in $\RR^m$.
Note that the curvature tensor can be expressed from these sectional curvatures and the metric tensor.
Hence, the weak delta-convergence of components of curvature tensor and therefore dual curvature tensor follows.
\qeds

