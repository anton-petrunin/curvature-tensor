\subsection{3-dimensional case}
In this section we prove Lemma~\ref{A^0} for 3-dimensional case.

Vectors $w_1,\dots,w_{m(m-1)/2}\in\R^m$ are said to be \emph{in general position}
if the vectors $w_i\otimes w_i$ form a basis in $\RR^{m\cdot(m-1)/2}$ --- the symmetric square of $\RR^m$.
In this case any quadratic form $Q$ on $\R^m$
can be computed from the values 
\[Q(w_1, w_1),\dots,Q(w_{m(m-1)/2}, w_{m(m-1)/2}).\] 
More precisely, there are rational functions 
$s_1,\dots,s_{m(m-1)/2}$ that take $\tfrac{m(m-1)}2$ vectors in $\R^m$ and returns a quadratic form on $\R^m$
such that
$$Q=\sum_{k=1}^{m(m-1)/2}s_k(w_1,\dots,w_{m(m-1)/2})\cdot Q(w_k,w_k).
\eqlbl{Qij}$$

Note that the vectors $w_1,\dots,w_{m(m-1)/2}\in\R^m$  are  in general position if and only if 
$s_k(w_1,\dots,w_{m(m-1)/2})$ are finite for all $k$.
Since 
$s_k$ are rational functions we get the following:

\begin{thm}{Observation.}\label{obs:genpos}
Suppose that vectors $w_1,\dots,w_{m(m-1)/2}\in\R^m$ are in general position.
Then the functions $s_1,\dots,s_{m(m-1)/2}$ are Lipschitz in a neighborhood of $(w_1,\dots,w_{m(m-1)/2})\in (\R^m)^{m(m-1)/2}$.
\end{thm}


\parit{Proof of the 3-dimensional case of \ref{A^0}.} 
Choose a common chart in
\[M_n\supset U_n\to\Omega,
\quad\text{and}\quad
A_n\supset U\to\Omega.\]
Let us use it to identify tangent spaces of $M_n$ and $A$ with $\RR^3$.
 
Choose vectors $w_1,\dots,w_6\in \R^3$  in general position.
We can choose 6 sequences of test concave functions $f^1_n\testto f^1,\dots, f^6_n\testto f^6$, such that
$\nabla f^1,\zz\dots,\nabla f^6$ are sufficiently close to
$w_1,\dots,w_6$.
We can assume $U$ to be sufficiently small, so by Proposition~\ref{strconvergence}
the measures $\Str(\nabla f^k_n, \nabla f^k_n)\cdot\vol_n$ weakly delta-converges
on $U\cap A^\delta$ for $k=1,\dots,6$.

By \ref{obs:genpos}, the functions $s_i$ are Lipschitz in a neighborhood of $(w_1,\dots,w_6)\zz\in (\R^3)^6$.
Applying \ref{Qij}, we get that 
\begin{align*}
\Str
&=
\sum_{k=1}^6 s_k(\nabla f^1_n,\dots \nabla f^6_n)\cdot\Str(\nabla f^k_n,\nabla f^k_n).
\end{align*}
Hence the measure $\Str_n(e_{i},e_{j})\cdot\vol_n$ are weakly delta-converging for all $i$ and $j$,
where $e_1,\dots,e_3$ is the standard basis in $\RR^3$.

By Lemma~\ref{lem:scalprod}, the sequence metric tensors $g_n$ of $M_n$ on $\Omega$ is uniforly delta-converging.
Since the following equality
\[\operatorname{Tr}\Str_n=g^{ij}_n\Str_n(e_{i},e_{j})\cdot \sqrt {\det (g_{ij,n})}\]
holds almost everywhere, we get that the sequence of measures $\operatorname{Tr}\Str_n\cdot\vol_n$ is weakly delta-converging.

Note, that for $3$-dimensional manifold we have
$$q_n(v,v)=\Str_n(v,v)-|v|^2\Tr \Str_n/4.\eqlbl{Q}$$
Hence the measures $q_n(e_i, e_j)\cdot\vol_3$ are weakly delta-converging for all $i$ and $j$.

Finally, according to ??? the components $\alpha^k_{n,i}$ of $\nabla f^k_n$ are uniformly delta-converging.
The result follows since
\[q(\nabla f^k_n, \nabla f^k_n)=
\sum \alpha^k_{n,i} \alpha^k_{n,j} q(e_{i}, e_{j})\]
almost everywhere, here $e_1,\dots,e_3$ denotes the standard basis in $\RR^3$.
\qeds

\subsection{Higher-dimensional case}


Choose sufficiently small positive $\delta$.
Let $p\in A$ be a $\delta$-strained point;
choose a common chart around $p$ and use it to identify points in $M_n$ and $A$ near $p$ with an open domain $\Omega\subset\RR^m$.

The following observation plays a key role in the proof.

\begin{thm}{Observation}
Consider the sequence of coordinate level sets $\Omega=L_m\supset L_{m-1}\supset\dots\supset L_0$, 
where $L_i=L_i(c_{i+1},\dots,c_m)$ is defined by setting the last $m-i$ coordinates to be $c_{i+1},\dots,c_m$ respectively.
Then the level sets $L_i$ is a smooth convex hypersurface in $L_{i+1}$ in each $M_n$;
in particular each $L_i$ has sectional curvature bounded below by $-1$.

Moreover, the same holds after applying any linear transformation to $\Omega$ that is close to the identity.  
\end{thm}



\parit{Proof of \ref{A^0}.}
By the lemma in dimensionals $2$ and $3$,  we get delta-convergence of curvature tenors on $L_2$ and $L_3$.
In particular, applying the coarea formula, we get convergence of sectional curvature of $L_3$ in the directions of $L_2$ as well as 
the sectional curvature of $L_2$ 
for all values $c_3,\dots,c_m$.
The difference of these curvatures is the Gauss curvature $G_n$ of $L_2$ as submanifold in $L_3$.
Therefore $G_n$ is delta-converging as well.

Consider a liner transformations of $\Omega$ that preserves the direction of $L_2$.
By the last statement in the observation,
the above argument shows delta-convergence of $G_n(w)$, where the direction $w$ of $L_3$ on $L_2$ can be chosen in an open set of $\RR^{m-2}$ --- the space transversal to $L_2$.
In particular, we may choose directions $w_1,\dots, w_{(m-2)\cdot(m-3)/2}$ in $\RR^{m-2}$ that form a generic set.

Denote by $G_n^+$ the term in Gauss formula for $L_2$ in $M_n$;
that is, $G_n^+$ is the difference of curvature of $L_2$ and the sectional curvature of $M_n$ in the same direction.
Denote by $g_n$ the Riemannian metric of $M_n$ in $\Omega$.
Note that 
\[G_n^+=\sum\alpha_{k,n}\cdot G_n(w_k),\]
where the coefficients $\alpha_{k,n}$ depend continuously on the components of $g_n$ and $w_1,\zz\dots, w_{(m-2)\cdot(m-3)/2}$. 
It follows that delta-convergence of $G_n(w_k)$ implies delta-convergence of $G_n^+$ as $n\to\infty$.
Since the curvature of $L_2$ is delta-converging, it implies delta-convergence of sectional curvature in the direction of $L_2$.

The above argument can be repeated after applying a linear transformation of $\Omega$ that changes the direction of $L_2$ slightly and keeps the properties of common chart.
It follows that sectional curvatures converge to a generic array of simple bivectors in $\RR^m$.
Note the curvature tensor can be expressed from these sectional curvatures and metric tensor.
Hence the delta-convergence of components of curvature tensor and therefore dual curvature tensor follows.
\qeds

