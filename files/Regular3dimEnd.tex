\section{Completion of the proof near regular
set in dimension 3}
In this section we 
 prove Lemma~\ref{A^0} for 3-dimensional case.

\subsection{Vectors in general position}

We say that  vectors 
 $w_1,\dots,w_{N(N-1)/2}\in\R^N$ are 
  \emph{in general position}
if any quadratic form $Q:\R^N\times \R^N\to\R$
can be computed from the values $Q(w_k, w_k), k=1,\dots,N.$ 
More formally,
there are rational functions $s_{ij}^k:(R^{N(N-1)})^N\to [-\infty,\infty]$
such that
 if we fix basis $e_1,\dots, e_N\in \R^N$
 and take vectors $w_s=w_s^ie_t$ for $s=1,\dots,\frac{N(N-1)}{2}$,
 then for any quadratic form $Q$ we can formally express its coordinates as:
$$Q(e_i,e_j)=\sum_{k=1}^{N(N-1)}s_{ij}^k(w_s^t)Q(w_k,w_k). \eqlbl{Qij}$$
Vectors are  in general position iff 
$s_{ij}^k(w_s^t)<\infty$ for all $i,j\in \{1,\dots, N\}$
and
$k\in \{1,\dots, N(N-1)/2\}$.

We need further the following observation:
since 
$s_{ij}^k$ are rational functions
any array of vectors in general position has a small
neighbourhood where $s_{ij}^k$ are Lipschitz.

\subsection{Proof} 

Let us fix Nice common chart sequence
\[\mathfrak X_n =(\mathfrak x^1_n, \mathfrak x^2_n,\mathfrak x^3_n):U_n\to\Omega,\quad
\mathfrak X =(\mathfrak x^1, \mathfrak x^2,\mathfrak x^3):U\to\Omega.\]
Set
\begin{align*}
e_{1n}&=\nabla \mathfrak x^1_n,
&e_{2n}&=\nabla \mathfrak x^2_n,
&e_{3n}&=\nabla \mathfrak x^3_n
\\
e_{1}&=\nabla \mathfrak x^1,
&e_{2}&=\nabla \mathfrak x^2,
&e_{3}&=\nabla \mathfrak x^3.
\end{align*}
 
Let
${w^*}_1,\dots,{w^*}_6\in T_pA=\R^3$ 
be vectors in general position, with
coordinate expressions ${w^*}_k=\sum_i{w^*}_k^ie_i$.
By Proposition~\ref{NiceFunctions}
we can choose 6 smooth sequences of concave functions $f^1_n\ccto f^1,\dots, f^6_n\ccto f^6$ around $p$, such that
$\nabla f^1,\dots,\nabla f^6$ are sufficiently close to
${w^*}_1,\dots,{w^*}_6$
and we can assume $U$ to be sufficiently small such
that
measures $\Str(\nabla f^k_n, \nabla f^k_n)d\vol_n$ weakly $c\delta$-converges
on $U\cap A^\delta$ for $k=1,\dots,6$ (by Proposition~\ref{strconvergence}).
 Let us denote
$w_{k,n}=\nabla_nf^k_n$ and expression
in the frame
$w_{k,n}= w_{k,n}^i e_{i,n}$.
Then
taking $U, U_n$ sufficiently
small  so that for all $y\in U_n$ coordinates
$w_{i, n}^k(y)\in \R^{3\times 6}$ are in a neighborhood 
of $ {w^*}_i^k\in \R^{3\times 6}$, where
corresponding functions $s_{ij}^k$ in \ref{Qij} are Lipschitz.
Then applying \ref{Qij} we get that 
$$\Str(e_{i,n},e_{j,n})=\sum_{k=1}^6 s_{ij}^k(w_{s,n}^t)\Str(w_{k,n}, w_{k,n})=\sum_{k=1}^6 s_{ij}^k(w_{s,n}^t)\Str(\nabla f^k_n, \nabla f^k_n).   $$
We know that $w_{s,n}^t$ are 
$c\delta$-converging
on $U\cap A^\delta$ (Corollary~\ref{cor:cdeltacoeff}) and $s_{ij}^k$ are Lipschitz,
hence components of Strange curvature
$\Str_n(e_{i,n},e_{j,n})d\vol_n$
 weakly $c\delta$-converges
on $U\cap A^\delta$. 

Let note, that for $3$-dimensional manifold 
curvature tensor $q$ can be expressed via Strange curvature and
for $M_n$ we have:
 $$q_{M_n}(v,v)=\Str_n(v,v)-|v|^2\Tr \Str_n/4.\eqlbl{Q}$$
Hence
$$q((w_{k,n}, w_{k,n})=
\Str_n(w_{k,n}, w_{k,n})-(\sum_{i,j=1}^3 g_{ij,n}w^i_{k,n} w^j_{k,n})\operatorname{Tr}\Str_n$$

We know that $g^{ij}_n$ are
$c\delta$ converging on $U\cap A^\delta$ (Lemma~\ref{lem:scalprod}) then measures
$\operatorname{Tr}\Str_nd\vol_n=g^{ij}_n\Str_n(e_{i,n},e_{j,n})d\vol_n$
and  hence
 $q((w_{k,n}, w_{k,n})d\vol_n$
weakly $c\delta$-converges
on $U\cap A^\delta$.
 

Then applying again \ref{Qij} we get that 
$$q(e_{i,n},e_{j,n})=\sum_{k=1}^6 s_{ij}^k(w_{s,n}^t)q(w_{kn}, w_{k,n})=\sum_{k=1}^6  s_{ij}^k(w_{s,n}^t)q(\nabla f^k_n, \nabla f^k_n) .   $$
Then as above we can obtain that
$q((e_{i,n}, e_{j,n})d\vol_n$
  weakly $c\delta$-converges
on $U\cap A^\delta$.

Now let $f_n\ccto f$ be arbitrary roughly $C^1$-converging sequence. For
decomposition $\nabla f_n =\sum \alpha^i_ne_{i,n}$,
we have that $ \alpha^i_n$
$c\delta$-converges
on $U\cap A^\delta$. Then measures
$q(\nabla f_n, \nabla f_n)d\vol_n=
\sum \alpha^i_n \alpha^j_n q((e_{i,n}, e_{j,n})d\vol_n$
weakly $c\delta$-converges
on $U\cap A^\delta$.

\qeds