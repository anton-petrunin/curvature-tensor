\subsection{3D case of \ref{prop:3parts}(\ref{prop:3parts:reg})}
In this section we prove Lemma~\ref{A^0} for 3-dimensional case.

Vectors $w_1,\dots,w_{m(m-1)/2}\in\R^m$ are \emph{in general position}
if the vectors $w_i\otimes w_i$ form a basis in $\RR^{m\cdot(m-1)/2}$ --- the symmetric square of $\RR^m$.
In this case any quadratic form $Q$ on $\R^m$
can be computed from the values 
\[Q(w_1, w_1),\dots,Q(w_{m(m-1)/2}, w_{m(m-1)/2}).\] 
More precisely, there are rational functions 
$s_1,\dots,s_{m(m-1)/2}$ that take $\tfrac{m(m-1)}2$ vectors in $\R^m$ and returns a quadratic form on $\R^m$
such that
$$Q=\sum_{k=1}^{m(m-1)/2}s_k(w_1,\dots,w_{m(m-1)/2})\cdot Q(w_k,w_k).
\eqlbl{Qij}$$

Note that the vectors $w_1,\dots,w_{m(m-1)/2}\in\R^m$  are  in general position if and only if 
$s_k(w_1,\dots,w_{m(m-1)/2})$ are finite for all $k$.
Since 
$s_k$ are rational functions we get the following:

\begin{thm}{Observation.}\label{obs:genpos}
Suppose that vectors $w_1,\dots,w_{m(m-1)/2}\in\R^m$ are in general position.
Then the functions $s_1,\dots,s_{m(m-1)/2}$ are Lipschitz in a neighborhood of $(w_1,\dots,w_{m(m-1)/2})\in (\R^m)^{m(m-1)/2}$.
\end{thm}


\parit{Proof of \ref{A^0}.} 
Choose a nice common chart in
\[M_n\supset U_n\to\Omega,
\quad\text{and}\quad
A_n\supset U\to\Omega.\]
Let us use it to identify tangent spaces of $M_n$ and $A$ with $\RR^3$.
 
Choose vectors $w_1,\dots,w_6\in \R^3$  in general position.
We can choose 6 sequences of test concave functions $f^1_n\testto f^1,\dots, f^6_n\testto f^6$, such that
$\nabla f^1,\zz\dots,\nabla f^6$ are sufficiently close to
$w_1,\dots,w_6$.
We can assume $U$ to be sufficiently small, so by Proposition~\ref{strconvergence}
the measures $\Str(\nabla f^k_n, \nabla f^k_n)\cdot\vol_n$ weakly $\delta$-converges
on $U\cap A^\delta$ for $k=1,\dots,6$.

By \ref{obs:genpos}, the functions $s_i$ are Lipschitz in a neighborhood of $(w_1,\dots,w_6)\zz\in (\R^3)^6$.
Applying \ref{Qij}, we get that 
\begin{align*}
\Str
&=
\sum_{k=1}^6 s_k(\nabla f^1_n,\dots \nabla f^6_n)\cdot\Str(\nabla f^k_n,\nabla f^k_n).
\end{align*}
Hence the measure $\Str_n(e_{i},e_{j})\cdot\vol_n$ are weakly $\delta$-converging for all $i$ and $j$,
where $e_1,\dots,e_3$ is the standard basis in $\RR^3$.

By Lemma~\ref{lem:scalprod}, the sequence metric tensors $g_n$ of $M_n$ on $\Omega$ is uniforly $\delta$-converging.
Since the following equality
\[\operatorname{Tr}\Str_n=g^{ij}_n\Str_n(e_{i},e_{j})\cdot \sqrt {\det (g_{ij,n})}\]
holds almost everywhere, we get that the sequence of measures $\operatorname{Tr}\Str_n\cdot\vol_n$ is weakly $\delta$-converging.

Note, that for $3$-dimensional manifold we have
$$q_n(v,v)=\Str_n(v,v)-|v|^2\Tr \Str_n/4.\eqlbl{Q}$$
Hence the measures $q_n(e_i, e_j)\cdot\vol_3$ are weakly $\delta$-converging for all $i$ and $j$.

Finally, according to ??? the components $\alpha^k_{n,i}$ of $\nabla f^k_n$ are uniformly $\delta$-converging.
The result follows since
\[q(\nabla f^k_n, \nabla f^k_n)=
\sum \alpha^k_{n,i} \alpha^k_{n,j} q(e_{i}, e_{j})\]
almost everywhere, here $e_1,\dots,e_3$ denotes the standard basis in $\RR^3$.
\qeds

\subsection{Sketch of the proof of Lemma~\ref{A^0} for higher dimensions}
We prove the $\delta$-convergence of dual curvature tensor in a small neighborhood 
for a number of $(m-2)$-collections of test sequences,
so that a curvature tensor can be reconstructed 
as a linear combination.
So we prove firstly the convergence
for a given $(m-2)$-collection of test sequences.

%Let a point $x_0\in A^0$ and a neighborhood $U\ni x_0 $ be sufficiently small.
We fix a tight collection of
Nice sequences of functions $g_n^1,\dots, g_n^{m-2}$    on
small converging neighborhoods $U_n$          
and regard а "foliation"  of $U^n$ into $2$-dimensional manifolds, that are
 intersections of the level sets,
 for every $t=(t_1,\dots,t_{m-2})\in (g_n^1,\dots, g_n^{m-2})(U_n)\subset\R^{m-2}$    
  we denote these manifolds by
 $K^t_n=(g_n^1)^{-1}(t_1)\cap\dots\cap (g_n^{m-2})^{-1}(t_{m-2})$
 and by $Int_n^t$ theirs intrinsic curvature. ({\bf?} 
 or sometimes  we use  notation $K_n^x$ for intersection of the level sets containing point $x\in U_n$).
 
 
 
 We know that
 $\sec K^t_n \ge\kappa$
  (see \ref{convexinconvex}) and $K_n^t \GHto K^t$(for any fixed $t$), hence 
  $K^t$ is an Alexandrov space with curvature bounded below by 
  $\kappa$ and measures with densities $Int^t_n$ w.r.t. areas
  weakly converge by Gauss-Bonnet formula to some measure on $K^t$.
  So to prove weak $\delta$-convergence of sectional curvature 
  along $K^t_n$
   it is sufficient  
   to estimate weak $\delta$-convergence of the
 full Gauss curvature of $K^t_n$ in $U_n$ (because of Gauss theorem).
  
  We prove this convergence in two steps:  
  
 1) We prove firstly 
 weak $\delta$-convergence of
 determinant second  fundamental form of $K^t_n$ w.r.t. some 
 normal field.
 We denote the $3$-dimensional submanifolds
 $N^{t_2,\dots,t_{m-2}}_n=g_2^{-1}(t_2)\cap\dots\cap g_{m-2}^{-1}(t_{m-2})$, then
 by \ref{convexinconvex}  the restrictions of $g_n^1$ onto $N_n^x$ are concave   functions
 with level sets
 $K_n^x$. 
 We have by \ref{convexinconvex} 
  $\sec N^{t}_n\ge\kappa$ and
 $N_n^t\GHto N^t$, sequence $g^\#_n=g^1_n|_{N_n}$ is
 Nice {\bf (?!!)},   hence Main theorem can be applied for this sequence.
 So measures with densities  
 $R^{N_n}(\nabla g^\#_n)$ w.r.t $3$-dimensional volume
 weakly converge.
We also  know that measures with densities $Int^t_n$ w.r.t. 
$3$-dimensional volume
  weakly $\delta$-converge (see Claim~\ref{Int-alm-conv}), hence
   applying   Gauss theorem
    we obtain  weak $\delta$-convergence 
    for measures with densities $G^{K\subset N}$
    w.r.t.       $3$-dimensional volume.
    Let note that
    $\cdot G^{K\subset N}=\frac{1}{|w|^2}G^{K\subset U,w}$, where $w$
    is the projection of $\nabla g^1_n$ onto tangent  
    space of $N^t_n$, that is the orthogonal complement to
    $\nabla g^2_n,\dots,\nabla g^{m-2}_n$. 
    So we have weak $\delta$-convergence 
of         the determinant of the 
second fundamental form w.r.t. normal fields $w_n$, which depends only on
gradients $\nabla g^1_n,\dots,\nabla g^{m-2}_n$. 
       
  2)We use Remark~\ref{concavelinear}
and change  linearly  $(m-2)$-collection of sequences of functions.
 By the Remark, the
    $2$-dimensional intersection sets  $K_n$
    don't change, but gradients and consequently 
     normal fields $w_n$ change linearly.
   In this way  we can prove
    $\delta$-convergence 
of         the determinant of the 
second fundamental form for $K_n$ w.r.t.  different normal fields.     
 Finely we note that for sufficiently many (namely (m-2)(m-3)/2 {\bf?})
  normal fields in general position
  we can express the     
    full Gauss curvature of $2$-dimensional surface 
      in correspondent "relative Gauss curvatures".



\begin{thm}{Remark} \label{concavelinear}
Let 
$g_1,\dots,g_l$ be a collection of concave functions
and coefficients 
$\alpha_1,\dots,\alpha_l \ge0$, then the function
$\alpha_1g_1+\dots+\alpha_lg_l$ is concave.
Let $(\alpha_{ij})_{i,j=1}^l$ be a nondegenerate
matrix , then the intersection of all ($l$) level sets
for the collection $f_1,\dots,f_l$
coincides  
 (for appropriate parameters) with  the intersection of $l$ level sets
for the collection
 $\alpha_{11}g_1+\dots+\alpha_{1l}g_l$,\dots,  $\alpha_{l1}g_1+\dots+\alpha_{ll}g_l$.
 \end{thm}
\begin{thm}{Lemma} \label{convexinconvex}

Let 
$g_1,\dots,g_l$ be a tight collection of smooth concave functions on a Riemannian manifold $M$.
Let denote the intersection of level sets
$L_s=(g^1)^{-1}(t_1)\cap\dots\cap (g_n^{s})^{-1}(t_{s})$ for $s=1,\dots, l$,
then the restriction of
$g^{s+1}$ onto $L_s$
is a smooth concave function.  In particular
$L_{s+1}$ is a smooth 
locally convex
submanifold in $L_s$ for $s=1,\dots, l$. 
Therefore if we have lower curvature bound
$\sec M \ge\kappa$
then
$\sec L_s \ge\kappa$
for
$s=1,\dots, m-2$.
\end{thm}
