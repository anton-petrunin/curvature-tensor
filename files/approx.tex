\subsection{Three-dimensional case}


\parit{Proof of the 3-dimensional case in \ref{prop:3parts:codim2+}.}
Suppose $M_n\smooths{}A$ and $A$ is 3-dimensional.
Choose a set $Q\subset A$ that admits a bi-Lipschitz embedding into $\RR$. 

Let us split $\mathfrak m$ into negative and positive part $\mathfrak m=\mathfrak m^+-\mathfrak m^-$; that is,
\[\mathfrak m^\pm(X)\df\sup\set{\pm\mathfrak m(Y)}{Y\subset X}.\]
Since the sectional curvature of $M_n$ is bounded below, we get that $\mathfrak m^-$ has bounded density; in other words, $\mathfrak m^-$ is a regular measure on $A$.
Since $Q$ has zero volume, we get $\mathfrak m^-(Q)=0$.

Set $\mathfrak n=\mathfrak m|_{Q}$; from above we have $\mathfrak n\ge 0$.
By \cite{petrunin-SC}, $\mathfrak n$ is a regular with respect to $\vol^1$ on $Q$.
Therefore it is sufficient to show that 
\[(2\cdot\pi-\theta)\cdot (\jac(\bm{h}|_Q))^2\]
is the $\vol^1$-density of $\mathfrak n$
at $\vol^1$-almost all $p\in Q$.

Chosse a bi-Lipschitz embedding $s\:Q\to \RR$;
set $K\zz=s(Q)$.
Since $s^{-1}$ and $h\circ s^{-1}$ are Lipschitz, 
by Rademacher's theorem, we can assume that $s^{-1}$ and $h\circ s^{-1}$ are differentiable at almost all $x\in K$.
Moreover, we can assume that $d_xs^{-1}(y)=(\lambda\cdot y,0)\in \RR\times \Cone(\theta)=\T_p$ and the $\vol^1$-density of $\mathfrak n$ at $p=s^{-1}(x)$ is defined.

Shifting and scaling the interval $K$, we may assume that $x=0$ and $\lambda=1$.
In this case, $|\jac_p(h|_Q)|=|d_0(h\circ s^{-1})|$.

Note that we can choose a sequence of points $p_n\in M_n$ and a sequence of factors $c_n\to \infty$ such that $(c_n\cdot M_n,p_n)$ converges to the tangent space~$\T_pA$.
Since $p\in A'$, the space $\T_p$ is a product space $\Cone(\theta)\times\RR$, where $\Cone(\theta)$ is a 2-dimensional cone with the total angle $\theta(p)\le2\cdot\pi$.

Let $f_n\:c_n\cdot M_n\to \RR$ be as in \ref{thm:HG-converge} and $u_n=\nabla f_n/|\nabla f_n|$.
Consider the sequence of functions $\hat h_n\:c_n\cdot M_n\to \RR$ defined by 
\[\hat h_n(x)=c_n\cdot(h_n(x)-h_n(p_n)).\]
Since $\lambda=1$, we have that $|\langle\nabla \hat h_n,u_n\rangle|$ uniformly converges to $|d_0(h\zz\circ s^{-1})|$.
By \ref{cor:Ricci}(\ref{cor:Ricci:vw}), the sequence of measures $\qm(\hat h_n, \hat h_n)$ on $c_n\cdot M_n\to \RR$ weakly converges to $\omega_{\Cone(\theta)\times\RR}$.
Recall that $\vol^1$-density of $\omega_{\Cone(\theta)\times\RR}$ on the singular line is $2\cdot \pi-\theta(p)$.

Observe that 
\[\qm(\hat h_n, \hat h_n)[B(p_n,1)_{c_n\cdot M_n}]
=
c_n\cdot \qm(h_n, h_n)[B(p_n,\tfrac1{c_n})_{M_n}]\]
Whence $2\cdot \pi-\theta(p)$ is the $\vol^1$-density of $\mathfrak n$ at $p$ as required.
\qeds

\subsection{Higher-dimensional case}


Suppose $M_n\smooths{} \Cone(\theta)\times \RR^{m-2}$, where $\Cone(\theta)$ denotes a two-dimensional cone with the total angle $\theta<2\cdot\pi$.
First, we will show that the curvatures of $M_n$ in the vertical sectional directions of $\Cone(\theta)\times \RR^{m-2}$ weakly converge to zero;
an exact statement is given in the following proposition.
By combining this result with the 3-dimensional case we get \ref{prop:3parts:codim2+} in all dimensions.

For $X,Y\in \T_p$, denote by $K(X\wedge Y)$ the curvature in the sectional direction of $X\wedge Y$.
A function $f\:\Cone(\theta)\times\RR^{m-2}$ will be called a \emph{vertical affine function} if $f$ can be obtained as a composition of the projection to $\RR^{m-2}$ and an affine function on $\RR^{m-2}$.

\begin{thm}{Proposition}\label{prop:vert-vert}
Let $M_n\smooths{} \Cone(\theta)\times\RR^{m-2}$
and $f_n,h_n\: M_n\zz\to \RR$ be sequences of strongly concave functions.
Suppose that $\sec M_n\ge -\tfrac1n$ for each $n$,
and $f_n\to f$, 
$h_n\to h$ where $f$ and $h$ are vertical affine functions on $\RR^{m-2}\times \Cone(\theta)$.
Then 
\[K(\nabla f_n\wedge\nabla h_n)\cdot\vol^m \rightharpoonup0.\]

\end{thm}

Let $\Sigma$ be a convex surface in a Riemannian manifold $M$.
Suppose $x$ is a smooth point of $\Sigma$; that is, the tangent hyperplane $H_x$ of $\Sigma$ is defined at $x$;
denote by $e_1,\dots,e_{m-1}$ an orthonormal basis of $H_p$.
Set 
\[\Zc_\Sigma(x)= \sum_{i,j} K(e_i\wedge e_j).\]
In other words, 
\[\Zc_\Sigma=\Sc-2\cdot\Ric(n_\Sigma,n_\Sigma),\]
where $n_\Sigma$ is the unit normal vector to $\Sigma$.

\begin{thm}{Lemma}\label{lem:nonsmooth-convex}
Let $\Sigma$ be a strongly convex hypersurface in a Riemannian manifold with curvature $\ge -1$.
Suppose that for some point $p\in \Sigma$ and $r<1$ the closed ball $\bar B(p,2\cdot r)$ in the intrinsic metric of $\Sigma$ is compact.

Then, 
\[\int_{x\in B(p,r)}\Zc_\Sigma(x)\cdot \vol^{m-1}\le (m-1)\cdot(m-2)\cdot\const(m-1)\cdot r^{m-3},\]
where $\const(m-1)$ is the constant in \ref{cor:Kmax}.
\end{thm}

\parit{Proof.} If $\Sigma$ is smooth, then the inequality follows from \ref{cor:Kmax}, and the fact that curvature cannot decrease when we pass to a convex hypersurface.

In the general case, the surface $\Sigma$ can be approximated by a smooth convex surface;
this can be done by applying the Green--Wu smoothing procedure; compare to \cite{AKP-buyalo}.
It remains to pass to the limit.
\qeds

\begin{wrapfigure}{r}{35 mm}
\vskip-0mm
\centering
\includegraphics{mppics/pic-100}
\vskip0mm
\end{wrapfigure}

\parit{Proof of \ref{prop:vert-vert}.}
Let $p$ be a singular point on $\Cone(\theta)\times \RR^{m-2}$;
let us denote its liftings by $p_n\in M_n$.
We can assume that $p$ corresponds to the origin of $\RR^{m-2}$.
Choose points $a_{1,n},\zz\dots,a_{m-2,n}$, $b_{1,n},\zz\dots,b_{m-2,n}$ in $M_n$ such that the functions $f_{i,n}\zz=\dist_{a_{i,n}}\zz-|a_{i,n}-p|$ and $-h_{i,n}\zz=-\dist_{b_{i,n}}+|b_{i,n}-p|$ converge to $i^{\text{th}}$ vertical coordinate function on $\Cone(\theta)\zz\times \RR^{m-2}$.
Further, choose points $c_{1,n},c_{2,n},c_{3,n}$ so that the functions $g_{i,n}\zz=\dist_{c_{i,n}}-|c_{i,n}-p|$ converge to Busemann functions for different horizontal rays in $\Cone(\theta)\times \RR^{m-2}$ emerging from $p$.
Note that the latter implies that the angles $\angk{p_n}{c_{i,n}}{c_{j,n}}$ are bounded away from zero for all large~$n$.

By \cite[Lemma 7.2.1]{petrunin-conc}, there is an increasing concave function $\phi$ defined in a neighborhood of zero in $\RR$ such that $\phi'$ is close to 1 and for any $\eps>0$ and $i\ne j$ the function 
\[s_{ij,n}
=
\phi\circ g_{i,n}
+
\phi\circ g_{j,n}
+
\sum_i\left[\phi\left(\tfrac{f_{i,n}}\eps\right)+\phi\left(\tfrac{h_{i,n}}\eps\right)\right]\]
is strongly concave in $B(p_n,R)$ for fixed $R>0$ and every large $n$.

Denote by $s_{ij}\:\Cone(\theta)\zz\times \RR^{m-2}\zz\to \RR$ the limits of $s_{ij,n}$.
Note that given $w>0$, we can take small $\eps>0$ so that for all $i\ne j$ for the set $s_{ij}^{-1}[-w,w]$ covers the singular locus in $B(p,R)$. 

Note that we can choose $\eps_0>0$ so that for almost all points $x\zz\in B(p_n,R)$ 
the differential $d_xs_{ij,n}$ is linear and $|d_xs_{ij,n}|>\eps_0>0$ for some $i$ and $j$.
Indeed, these differentials are linear outside cutlocuses of $a_{i,n}$, $b_{i,n}$, and $c_{i,n}$;
in particular, they are linear at almost any point $x_n\zz\in M_n$.
Further, if the differential $d_{x_n}s_{12,n}$ is very close to zero,
then the directions of $[x_n,c_{1,n}]$ and $[x_n,c_{2,n}]$ are nearly opposite.
Since $x_n$ is close to $p_n$, we get that for large $n$ the angles $\measuredangle\hinge{x_n}{c_{i,n}}{c_{j,n}}$ is bounded away from zero, we get $|d_{x_n}s_{13,n}|$ is bounded away from zero as well.

Since $f_n$ and $h_n$ are converging to vertical affine functions, we get that for large $n$ their gradients are nearly orthogonal to $\nabla_{x_n} g_{i,n}$ at almost all $x_n\in M_n$.
Suppose $d_xs_{ij,n}$ is linear and $|d_xs_{ij,n}|>\eps_0>0$.
Then gradients $\nabla_{x_n}f_n$ and $\nabla_{x_n}h_n$ are nearly orthogonal to $\nabla_x s_{ij,n}$.

Set $\Sigma_{ij,n}=\Sigma_{ij,n}(c)=\set{x\in M_n}{s_{ij,n}(x)=c}$.
The argument above implies that for almost all points $x\in B(p_n,R)$ one of the sectional directions of the tangent directions of $\Sigma_{ij,n}$ is close to the sectional direction $\nabla f_n\wedge\nabla h_n$.

By \ref{lem:nonsmooth-convex} and the coarea formula, the sum integral curvatures of $M_n$ in the directions of $\Sigma_{ij,n}$ at $x_n$ at the subsets where $|d_{x_n}s_{ij,n}|>\eps_0$ is bounded by $\const\cdot w$.
Therefore, the same holds for the integral of $K(\nabla f_n\zz\wedge\nabla h_n)$.
The proposition follows since $w$ can be taken arbitrarily small.
\qeds


\parit{Proof of the general case of \ref{prop:3parts:codim2+}.}
Choose $m-2$ sequences of strongly concave functions $f_{1,n},\zz\dots, f_{m-2,n}\: M_n\to \RR$ that converge to vertical affine functions $f_1,\zz\dots,f_n$ on  $\Cone(\theta)\times \RR^{m-2}$ with orthonormal gradients.

Note that the fields $e_{1,n}=\nabla f_1,\dots,e_{m-2,n}=\nabla f_{m-2}$ are nearly orthonormal;
in particular, they are linear independent for all large $n$.
Let us add two fields $e_{m-1,n}$ and $e_{m,n}$  so that $e_{1,n},\dots, e_{m,n}$ form a nearly orthonormal frame in~$M_n$;
that is, $\langle e_{i,n}, e_{j,n}\rangle\to 0$ for $i\ne j$ and $\langle e_{i,n}, e_{i,n}\rangle\to 1$ for any $i$ as $n\to\infty$.

Observe that \ref{prop:vert-vert} implies that $K(e_i\wedge e_j)\cdot\vol\rightharpoonup 0$ if $i,j\zz\le m-2$.

Let us show that $K(e_i\wedge e_j)\cdot\vol\rightharpoonup 0$ for $i\le m-2$ and $j\zz\ge m\zz-1$.
The $3$-dimensional case is already proved; it is used as a base of induction.
Let us apply the induction hypothesis to the level surfaces of $f_{1,n}$.
(Formally speaking, we apply the local version of the induction hypothesis described in Section~\ref{sec:local}.)
Since the curvature of convex hypersurfaces is larger than the curvature of the ambient manifold in the same direction, we get the statement for $i\ne 1$.
Applying the same argument for the level surfaces of $f_{2,n}$, we get the claim.

Now let us show that $K(e_{m-1}\wedge e_m)\cdot\vol^m\rightharpoonup \omega_{\Cone(\theta)\times \RR^{m-2}}$.
Consider the level sets $L_n$ defined by 
\[f_{1,n}=c_1,\ \dots,\ f_{m-2,n}=c_{m-2}.\eqlbl{eq:fi=ci}\]
Note that $L_n\smooths{} \Cone(\theta)$.
Applying the 2-dimensional case to $L_n$ and the coarea formula, we get that curvatures of $L_n$ weakly converge to $\omega_{\Cone(\theta)\times \RR^{m-2}}$.
It remains to show that the extra term in the Gauss formula for the curvature of $L_n$ weakly converges to zero; 
in other words, the difference between the curvature of $L_n$ and sectional curvature of $M_n$ in the direction tangent to $L_n$ weakly converges to zero.

The 3-dimensional case is already proved.
To prove the general case, we apply the 3-dimensional case to the 3-dimensional level sets defined by $m-3$ equations from the following $m-2$ equations in \ref{eq:fi=ci}.
(The same argument is used in the proof of \ref{A^0}, and it is written with more details.)

Note that for $\theta=0$, the last argument implies the following:

\begin{thm}{Claim}
Let $M_n\smooths{} A$.
If $A$ has a flat open set $U$,
then $|K_{\max}|\zz\cdot\vol_n\zz\rightharpoonup0$ on $U$. 
\end{thm}

In particular, the weak limit of dual curvature tensor has support on the singularity of $\Cone(\theta)\times \RR^{m-2}$.

The same argument as in \ref{cor:Ricci} shows that $\langle\Rm(e_i,e_j)e_q,e_r\rangle\zz\cdot\vol^m\zz\rightharpoonup0$ is at least one of the indices $i,j,q,r$ is at most $m-2$.
The latter statement implies the result.
\qeds
