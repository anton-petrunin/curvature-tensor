                                                                            |%\RequirePackage{ifluatex}
%\let\ifluatex\relax
%\documentclass{sesamanuel}
\documentclass[a4paper,10pt]{article}
\UseRawInputEncoding

\usepackage{kusochek+}
\usepackage{eucal}
\usepackage{xcolor}
\usepackage{bm}
\usepackage{enumitem}
\usepackage{calligra}
\DeclareMathAlphabet{\mathcalligra}{T1}{calligra}{m}{n}

%\usepackage{dipole}
%\usepackage{kubik}
\begin{document}
\title{Curvature tensor of smoothable Alexandrov spaces. 
\date{{\it Very preliminary announcement} \\ \today}
}
\author{Nina Lebedeva, Anton Petrunin}
\maketitle

\tableofcontents

%\newcommand{\M}{\mathcal M} 
%\newcommand{\cto}{\stackrel{C}{\longrightarrow}}
\newcommand{\Test}{\operatorname{Test}}
\newcommand{\Al}{\mathcal{A}lex} 
\newcommand{\All}{\mathcal{A}lex_{loc}} 
\newcommand{\DAl}{\mathcal{DA}lex}
\newcommand{\DM}{\mathcal{DM}}
%\newcommand{\M}{\mathcal{M}}
\newcommand{\rcd}{\operatorname{RCD}}
%\newcommand{\SMA}{\mathcal{SMA}} 
%\newcommand{\SAA}{\mathcal{SAA}} 
\newcommand{\dccto}{\stackrel{DC_{Alex}}{\longrightarrow}}
%\newcommand{\testto}{\stackrel{test}{\longrightarrow}}
\newcommand{\cccto}{\stackrel{C^0_c}{\longrightarrow}}
%\newcommand{\cto}{\stackrel{C}{\longrightarrow}}
%\newcommand{\x}{\mathfrak x}
%\newcommand{\X}{\mathfrak X}
\newcommand{\C}{\mathfrak{C}}
\newcommand{\vv}{\mathfrak{v}}
\newcommand{\ep}{\varepsilon}

\section{Introduction}
%$(\kern-4pt\mathcalligra{C_m}\kern1pt)$

%$(\mathcal{C}_m)$
The meaning of weak convergence and measure valued tensor used in the following theorem are introduced in the next section.

\begin{thm}{Main theorem}
Let $\kappa$ be a real number and 
$M_n$ be a sequence of $m$-dimensional Riemannian manifolds with sectional curvature bounded below by~$\kappa$.
Assume that $M_n$ converges to an Alexandrov space $A$ of the same dimension.
Then the curvature tensors of $M_n$ converge weakly to a measured-valued tensor on $A$.
\end{thm}

The result is new and nontrivial even
in the case if the limit space is
a Riemannian manifold.

Let us denote by $\Sc$ the scalar curvature of Riemannian manifold.
Unlike the main theorem, the following corollary requires no new definitions.


\begin{thm}{Corollary}
In the assumption of the main theorem,
the measures $\Sc\cdot \vol_m$ on $M_n$ weakly converges to a locally finite signed measure  on $A$.

In particular, if $A$ is compact, then the sequence
\[s_n=\int_{M_n}\Sc d\vol_m\]
converges.
\end{thm}

The limit measure $\mu$ has some specific properties,
we describe some of them.

1) For any $X\subset A$ with Hausdorff dimension
$Hdim(X)< m-2$ we have
$\mu(X)=0$, in particular the measure of the set of singularities
of codimension $\ge 3$ is $0$.

2) We can explicitly  describe the measure 
on the set $A'$ of singularities
of codimension $2$. That is the set $A'\subset A$
contains all points $x$ with tangent space
$T_xA=\R^{m-2}\times \C$
where $\C$ is a $2$-dimensional cone
with the total angle $2\pi-\omega(x)$.
Then for any subset $X\subset A'$
 $$\int_Xd\mu=\int_X
\omega(x)d h_{m-2}, $$ where $h_{m-2}$ denotes $(m-2)$-dimensional Hausdorff measure.


Rigorously we can not deduce convergence of scalar curvature
directly from the main theorem, because generally we can not multiply
 measure (sectional curvature) by a noncontinuous 
 function (function  depending on the metric tensor). The proof
 of this convergence will follow the same lines as the proof
 of the main theorem.

Note that the limit tensor of the sequence depends only on $A$ and does not depend on the choice of the sequence $M_n$.
Indeed, if for another sequence $M_n'$ satisfying the assumptions of the corollary and converging to the same Alexandrov space $A$ we would get a different limits, 
then a contradiction would occur for the alternated sequence $M_1,M_1',M_2,M_2',\dots$. This allows to define a curvature
tensor for every smoothable Alexandrov space. 

Also let us mention that the main theorem in \cite{petrunin-SC} states in particular that if a sequence of complete $m$-dimensional Riemannian  manifolds $M_n$ has uniformly bounded diameter and uniform lower curvature bound, then 
the corresponding sequence $s_n$ is bounded; in particular it has a converging subsequence.
However if $M_n$ is collapsing this sequence may not converge
without the non-collapsing assumption.
For example an alternating sequence of flat 2-toruses and round 2-spheres might collapse to the one-point space, in this case the sequence $s_n$ is $0,4\cdot\pi,0,4\cdot\pi,\dots$.

\subsection{Related results}
The result of the main theorem in dimension 2 is well
 known (Theorem 8.4.2 \cite{Resh}, \cite{AZ} ).

The problem of introducing Ricci tensor
was 
studied in far more general settings of 
 $\rcd$-spaces (\cite{G1}, \cite{St}, \cite{H})
 and
for some singular  spaces in \cite{L}.
In \cite{G}  curvature tensor for  $\rcd$-spaces was defined by
 N.Gigli. It would be very interesting to know 
the relation between these and our definitions
of curvature tensors.
%In particular, it is interesting if curvature tensor or Ricci tensor in above mentioned works are stable with respect to Gromov--Hausdorff convergence in noncollapsing case.

The precise geometric meaning of our curvature tensor is not quite clear. 
We also have a question, similar to imposed in \cite[Conjecture~1.1]{G}.

Suppose in conditions of main theorem we get
a limit curvature tensor, with sectional curvature bounded
below by $K>\kappa$.
Does it mean that $A$ is an Alexandrov space with curvature
bounded below by $K$?


\section{Preliminaries}

\medskip
In this section we give definitions necessary for a precise formulation 
of the main theorem.
For simplicity  we will always assume that the lower
curvature bound in the main theorem is  $-1$. 

We denote by
$\Al^m$ the class of m-dimensional Alexandrov's spaces
with curvature $\ge -1$.

Suppose a sequence $A_n\in \Al^m$ Gromov--Hausdorff converges without collapse to some
$A\in \Al^m$ via
approximations $a_n\:A_n\to A$.
It follows from Perelman stability theorem (see \cite{PerStab}, \cite{KapStab})
that for sufficiently large $n$  we can take approximations
to be homeomorphisms $a_n^{hom}: A_n\to A$.
Everywhere further by GH-convergence we mean 
Gromov--Hausdorff converges without collapse with a given
homeomorphism
approximations and write $A_n\GHto A$.


We say that $A\in \Al^m$ is \emph{ smoothable }
if it can be presented as a Gromov--Hausdorff limit by a non-collapsing sequence of Riemannian manifolds $M_n$ with $\sec M_n\ge-1$.
Given a smoothable Alexandrov space $A$,
a sequence of Riemannian manifolds $M_n$ as above
together with a sequence of homeomorphism approximations $a_n\:M_n\to A$
will be called \emph{smoothing} of $A$
(briefly $M_n\smooths{} A$, or $M_n\smooths{a_n} A$).
It follows from the stability theorem that any smoothable Alexandrov space is a topological manifold without boundary.



%Everywhere in this section $M_n, A, a_n)$

%$A_n\GHto A$ is a converging sequence of Alexandrov spaces
%with a fixed approximation sequence $a_n:A_n\to A$. 


%Let $A$ be a finite dimensional Alexandrov space with curvature bounded below.
 
 
 
\subsection{Weak convergence of measures}
In this section we give formal definition of weak convergence of measures
that we use.
For more detailed definitions and terminology we refer for example
\cite{GMS}.
For a Hausdorff topological space $X$ let
$\mathfrak M(X)$ be the space of signed Radon measures on $X$.
%and $C(X)$ space of continuous functions on $X$
%with a uniform norm. If $X$ is compact  then 
%$\mathfrak M(X)$ is the dual space for $C(X)$
% (by Riesz-Markov-Kakutani representation theorem).
%Now  definition given below
% of weak convergence
%of measures for not necessary compact Hausdorff space
%(that rigorously is a kind of local weak convergence.)
We denote by $C_c(X)\subset C(X)$  the space of continuous functions on $X$
with a compact support. 
We  denote by $<m|f>$ the value of
$m\in\mathfrak M(X)$
 on $f\in C_c(X)$.
 We say that measures $m_n\in \mathfrak M(X)$ 
\emph{weakly converge} to
to $m\in \mathfrak M(X)$ and write
$m_n\rightharpoonup m$
 if for
any
$f\in C_c(A)$ the sequence $<m_n|f>\to <m|f>$.

For a given  sequence $A_n\GHto A$
we define weak convergence of  measures from
$\mathfrak M(A_n)$
by identification of all spaces via homeomorphism
approximations $a_n:A_n\to A$.
For verification of weak convergence it is convenient to 
give an equivalent definition.
We say that  a sequence $f_n\in C_c(A_n)$
uniformly converges to $f\in C_c(A)$
if $| f_n   \circ a_n^{-1}-f|\to 0$.
Then  $m_n\rightharpoonup m$
if for any sequence $f_n\in C_c(A_n)$
with uniformly bounded supports and
uniformly converging to $f\in C_c(A)$
 the sequence $<m_n|f_n>\to <m|f>$.

%We will denote by $<h|f>$ the value of $h\in C^*(A)$ on $f\in C(A)$.

\subsection{ Test functions}
In this section we introduce a class of test functions
which we use
to define  a notion of measure valued tensor.
To simplify definitions we take as test  functions
 expressed via smoothed distance functions.
 In fact the proofs  work 
  also for wider class of test
 sequences, it is possible to define general notion of
 rough $C^1$-convergence 
 for Alexandrov spaces, for this we refer section~\ref{sec:rC}. 
However  functions
 expressed via smoothed distance functions
 is almost a unique known source of
 examples of roughly smooth functions and
we shell limit the formulation of our main result to such a test functions.



Let $A\in \Al^m$, distance between $x,y\in A$ is denoted by $|xy|$,
distance function  by $d_x(y)=|xy|$.



Given $A_n, A\in \Al^m$,  $A_n\GHto A$,
we can naturally lift a distance function $d_p$ to $A_n$
if we choose a convergent sequence $p_n\to p$ and take the
sequence $d_{p_n}$.
Let us fix small $r>0$,
$p\in A$
and define smoothed distance function as $$\widetilde{d}_{p,r} =\oint_{B_r(p)} d_{x}dx.$$ 
We can approximate this function by
$\widetilde{d}_{p_n,r}:A_n\to [0,\infty)  $
 choosing
  some  sequence $A_n\ni p_n\to p\in A$.

We say that $f$ is a test function
and write $f\in \Test(A)$ if it can be expressed by the formula
$$f=\varphi( \widetilde {d}_{p_1,r}, \widetilde{d}_{p_2,r},\dots,   \widetilde{d}_{p_N,r}),$$
where $\varphi:[0,\infty)^N\to\R$ is  $C^1$ function.
If for some sequences of points $A_n\ni p_{k,n}\to p_k\in A$
and $C^1$ functions   
$\varphi_n\cdto \varphi$ we have
$$f_n=\varphi_n( \widetilde {d}_{p_1,r}, \widetilde{d}_{p_2,r},\dots,   \widetilde{d}_{p_N,r}),$$
we will say that $f_n$ is a test sequence converging to $f$ and write
$(f_n,f)\in \Test(A_n,A)$ or $f_n\testto f$.

Let note that
test functions form an algebra and a
partition of unity can be formed by test functions.
On a smooth Riemannian manifold test functions
are exactly $C^1$-functions.
Indeed, around any point one can take a
smoothed distance 
coordinate chart, express $C^1$-function in these 
coordinates and then apply partition of unity. 
 
 
\subsection{Tensors}


\begin{rdef} {Definition}\label{def:mestens}
Let $A\in \A$ and $\mathfrak M(A)$
denotes the set of signed Radon measures on $A$.
A \emph{vector measure } $\mathfrak{v}$  on $A$
is a  continuous (with respect to test convergence) linear map
$\mathfrak{v}\:\Test(A)\to \mathfrak M(A)$ which satisfies a
chain rule;
i.e. for any collection  $f_1,f_2,\dots,f_\kay \in \Test(A)$
and a smooth function $\phi\:\RR^\kay\to\RR$ we have
$$\mathfrak{v}(\phi(f_1,f_2,\dots,f_n))
=
\sum_{i=1}^n (\partial_i\phi)(f_1,f_2,\dots,f_n)\cdot\mathfrak{v}(f_i)$$
where $\partial_i$ denotes partial derivatives in $\RR^n$.

Analogously, we define (contravariant) measure tensor field.
A \emph{measure tensor } $\mathfrak{t}$ of valence $\kay$ on $A$ is a multilinear map of $\kay$ variables in $\Test(A)$ with values in $\mathfrak M(A)$ which satisfies chain rule in each of the arguments.


\end{rdef}

Let us mention some basic properties that follow 
from the definition.
Firstly, if $f_i=c=\operatorname{const}$  for some $i\in\{1,\dots,\kay\}$ 
then $\mathfrak{v}(f_1,\dots,f_\kay)=0.$
Indeed, assume for simplicity  $\kay=1$, then
chain rule implies in particular the
product rule:
$\mathfrak{v}(fg)=\mathfrak{v}(f)g+f\mathfrak{v}(g)$.
Then $\vv(1)=\vv(1^2)=2\vv(1)=0$ and hence
$\vv(c\cdot 1)=c\cdot\vv(1)=0$.



Also the notion is local,
i.e. if $f_i|_{U}=\operatorname{const}$
for some open $U\subset A$ and some $i\in\{1,\dots,\kay\}$ 
 then $\mathfrak{v}(f_1,\dots,f_\kay)|_U=0.$
By above it suffices to show
for the case when $\operatorname{const}=0$.
Assume for simplicity  $\kay=1$. 
For arbitrary compact $K\subset U$  let
us take test function $g:A\to\R$ such that
$\supp g\subset U$ and $g|_K=1$. Then applying product rule
 $0=\vv(fg)=f\vv(g)+g\vv(f)$ yields 
$\vv(f)|_K=0$ hence 
 the result.
 
For $\kay=1$ in the case when $A$ is an $m$-dimensional Riemannian manifold 
 vector measure $\mathfrak{v}$  in coordinates
 $x_1,\dots,x_m$
is defined by $m$ components, these are measures $(\mu_1,\dots, \mu_{m})=(\vv(x_1),\dots,\vv(x_m))$;
these components transform by contravariant rule under change
 of coordinates. The map from the definition of vector measure is
 $\mathfrak{v}(f)=\sum_{i=1}^m\frac{\partial f}{\partial x_i}\cdot \mu_i$.
 Similarly for arbitrary $\kay$
 measure tensor of valence $\kay$ 
 is defined by $m^\kay$ components 
 $\mu_{{i_1},\dots,{i_\kay}}=\vv(x_{i_1},\dots,x_{i_\kay})$
 and
 $\mathfrak{v}(f)=\sum
 \frac{\partial f}{\partial x_{i_1}}\frac{\partial f}{\partial x_{i_2}}\dots\frac{\partial f}{\partial x_{i_k}}
 \cdot \mu_{{i_1},\dots,{i_\kay}}$.
 The smooth contravariant tensor field
 can be presented as a measure tensor
 if we multiply each component by the volume measure.



Now we define a weak convergence of measure tensors.

\begin{rdef} {Definition}
	Let $A\in \A$ be smoothable and $M_n\to A$ be a
	smoothing sequence.
Assume that $\mathfrak{t}_n$ is a sequence of %uniformly bounded
 measure tensors on $M_n$  and $\mathfrak{t}$ is a
measure tensor on $A$,
all of the same valence $\kay$.
We say that $\mathfrak{t}_n$ weakly converges to  $\mathfrak{t}$
(briefly $\mathfrak{t}_n\rightharpoonup\mathfrak{t}$) if for arbitrary $\kay$
test sequences 
$f_{i,n}\testto f_i, i=1,\dots,\kay$, the measures $\mathfrak{t}_n(f_{1,n},f_{2,n},\dots,f_{\kay,n})$ weakly converge to $\mathfrak{t}(f_{1},f_{2},\dots,f_{\kay})$.
\end{rdef}

% \begin{thm}{Corollary}\label{limitTensor}

\section {Curvature measure tensor and the main theorem}

The curvature of a smoothable $m$-dimensional Alexandrov space $A$
will be described using a tensor distribution $\mathfrak{q}=\mathfrak{q}_A$ of valence $2\cdot(m-2)$.
Our main theorem claims the weak convergence
of this tensors if we have a smoothing $M_n\GHto A$.
Let us start with a two dimensional case, 
in this case the statement of the main theorem is a well known fact and
we use it as a base step of a kind of induction in our proof. 

 \subsection{Two dimensional case}\label{2dim}

In particular case, if $m=2$ the valence of $\mathfrak{q}$ is $0$;
in this case $\mathfrak{q}$ coincides
 with the curvature measure --- the standard way to describe curvature of surfaces, 
(see, for example \cite{Resh}).
For a smooth surface a density of this curvature measure with respect to the area measure
is   the Gauss  curvature.
The statement of our main theorem in this case, i.e. 
stability of these measures under Gromov--Hausdorff convergence
is well  known (Theorem 8.4.2 \cite{Resh}, \cite{AZ} ).   


\subsection{Higher dimensional case}
\parbf{Curvature  measure tensor.}
For an $m$-dimensional Riemannian manifold $M$
its Riemannian curvature tensor % $\mathfrak{q}_M$
 is a smooth tensor field.
It can be described using the field of curvature operator
 $\Rm_M\:(\bigwedge^2\T)M\to (\bigwedge^2\T)M$.

We define curvature measure tensor
$\mathfrak{q}_M$ (in the sense of Definition~\ref{def:mestens}) as follows.
 For smooth functions
 $f_i,g_i:M\to\R, i=1,\dots, m-2$
with compact supports (i.e. test functions)
 we define $\mathfrak{q}_M(f_1,f_2,\dots,f_{m-2},g_1,g_2,\dots,g_{m-2})$
to be  the measure with a density
 ${q}_M(\nabla f_1,\nabla f_2,\dots,\nabla f_{m-2},\nabla g_1,\nabla g_2,\dots,\nabla g_{m-2}): M\to\R$
with respect to $d \vol$, which is given by:
\begin{multline*}
{q}_M(\nabla f_1,\nabla f_2,\dots,\nabla f_{m-2},\nabla g_1,\nabla g_2,\dots,\nabla g_{m-2})
=
\\
= d \vol (
\Rm_M(*(\nabla f_1\wedge \nabla f_2\wedge\dots\wedge \nabla f_{m-2}))\wedge\nabla g_1\wedge \nabla g_2\wedge\dots\wedge \nabla g_{m-2}),
\end{multline*}
where ${*}\:(\bigwedge^{m-2}\T)M\to(\bigwedge^2\T)M$ is the  Hodge star operator.

\begin{rdef}{Remark}\label{K=q}
Let note that in the case if $f_i=g_i$, for $i=1,\dots,m-2$
we have:
% can express $q_M$ using co-sectional curvature: 
$$q_M(\nabla f_1,\nabla f_2,\dots,f_{m-2},\nabla f_1,\nabla f_2,\dots,\nabla f_{m-2})=|\nabla f_{1,n}(x)\wedge\dots \wedge\nabla f_{m-2,n}(x)|^2\cdot K^M_\sigma, $$
where $K^M_\sigma $ is the sectional curvature at $x\in M_n$
on a plane orthogonal to 
$\nabla f_{1,n}(x)\wedge\dots \wedge\nabla f_{m-2,n}(x)$.
Hence
sectional curvatures of $M$ can  be computed using the tensor
$\mathfrak{q}_M$.
Also the density $q_M$ is defined by the sectional curvature,
because of the symmetry:
$${q}_M(\nabla f_1,\dots,\nabla f_{m-2},\nabla g_1,\dots,\nabla g_{m-2})=
{q}_M(\nabla g_1,\dots,\nabla g_{m-2},\dots,\nabla f_1,\dots,\nabla f_{m-2}).$$
Therefore measure tensor $\mathfrak{q}_M$
gives an equivalent description of curvature of Riemannian manifold.
\end{rdef}
Now we can formulate our main result:
  
\begin{thm}{Main theorem}\label{main}
Let  $A$  be a smoothable Alexandrov space with approximations
$M_n\smooths{a_n} A$ then $\mathfrak{q}_{M_n}\rightharpoonup \mathfrak{q}_A$ 
for some measure tensor $\mathfrak{q}_A$ on $A$.
\end{thm}


\section{Plan of the proof}
For the proof we subdivide the limit Alexandrov space into
three subsets: subset of regular 
points, points with singularities of codimension 2 and points with higher dimension 
singularities. We treat the proof for this subsets almost independently.

The most substantial case is for dimension $3$, higher dimensional case is reduced to it.

The part of the proof for regular points is a sort of induction.
The base is $2$-dimensional case; in the prove
 for dimension $3$ we apply the main theorem to
 level sets of special concave functions.
 These level sets have
the same lower curvature bound 
provided by Gauss theorem. 
  Multidimensional case
 is reduced to dimension $3$, using special $3$-dimensional intersections
 of concave functions.
In this way we need formally more strong, local version of 
 the main theorem, because level sets of concave functions
have structure of Alexandrov spaces only locally.
By technical reasons
we also need a more general class of functions and convergence,
one of the reason is that restrictions of our test
function to some neighbourhood might not to be a "local test function" and we cannot
reduce main theorem to its local version in a very direct way.

 All things necessary for these local settings are 
 formally described in the next three subsections.


\subsection{Locally defined Alexadrov space structure }

\begin{rdef}{Definition}
Let $A$ be a metric space. 
We say that an open  $U\subset A$
is a \emph{strongly inner domain} if
for some 
$R>0$ we have
$U\subset B_R(x)$
and $\overline{B_{10R}(x_n)}$ is compact.

\end{rdef}

\begin{rdef}{Definition}
We say that
 a locally compact inner metric space $A$
is an \emph{ Alexandrov region}
any point has a neighborhood where Alexandrov
comparison for curvature $\ge -1$ holds.
We say that $U\subset A$
is an \emph{ Alexandrov domain}
 if $U$ is 
 a strongly inner domain.

\end{rdef}

It is possible to show  that 
Toponogov comparison holds for sets with local structure of Alexandrov space and most of arguments and constructions for Alexandorov spaces could be applied to Alexandrov domain.
In particular we need main result from
 \cite{petrunin-SC}, where complete manifold can be replaced by 
strongly inner domain in a possibly open manifold. 
 
 
 
 Next we describe smoothing for Alexandrov domains.
 We denote by
$\M_{\ge -1}^m$ a class of $m$-dimensional Riemannian 
manifolds without boundary, but possibly non-complete, with sectional curvature bounded
from below by $-1$.

\begin{rdef}{Definition}
Let sequence
$M_n\in\M_{\ge -1}^m$ (with corresponding intrinsic metric)
converges in Gromov--Hausdorff sense to some metric space $A$.
We say that an open $U\subset A$ is \emph{ good} with respect to this convergence
 if
$\dim U=m$ and for some 
$R>0$, $M_n\ni x_n\to x\in A$ we have
$U\subset B_R(x)$
and ${B_{R}(x_n)}$ is a strongly inner domain in
$ M_n$
. 
\end{rdef}
 Let note that in this case $B_{10R}(x)$ is Alexandrov region
 and $U$ is an Alexandrov domain.
 
 We will say that open $U_n\subset M_n$
approximates good subset $U$ and write $U_n\to U$
if $U_n\GHto U$ and for any $p\in U$ there is $M_n\ni p_n\to p$
and $r>0$ such that $B_r(p_n)\subset U_n$.
 
 \subsection{Rough $C^1$-convergence }\label{sec:rC}
 Let   
 $M_n\in\M_{\ge -1}^m$,
 $M_n\GHto A$ 
 and  $U\subset A$ be a good domain for this convergence.
 

\begin{thm}{Definition}
	Let $p_n\to p$.
 We say that a \emph {sequence of directions} $\xi_n\in \Sigma_{p_n}$
 \emph{strongly converges} to $\xi\in\Sigma_{p}$ if there exist points
 $x,y\in U$ that can be joint by a unique minimizing geodesic
 with initial direction
 $\uparrow_{x}^{y}=\xi$ and sequences $M_n\ni x_n\to x$, 
 $M_n\ni y_n\to y$ such that
 $\xi_n=\uparrow_{x_n}^{y_n}$.
 \end{thm}
 
\begin{thm}{Definition}
Let $p\in A$, $\xi\in T_pA$
we say directions $\xi_n\in T_pM_n$
converge  to $\xi$
if for any $\ep>0$ there is $\xi^\ep\in T_pA$
with $\angle (\xi^\ep, \xi)<\ep$ and a sequence $\xi_n^\ep\in T_{p_n}M_n$
strongly converging to $\xi^\ep$ and with $\angle (\xi_n^\ep, \xi_n)<\ep$.

\end{thm}

 
 
 
 For $\xi\in\Sigma_p$ define $\delta(\xi)=2\pi-\sup\{\angle(\xi,\xi^*)|\xi^*\in\Sigma_p\}$
 
 \begin{thm}{Definition}
 	Let $U_n\to U$.
 We say that a sequence of 
 $C^1$-functions $f_n:U_n\to\R$ \emph{roughly
 $C^1$-converges}  to $f:U\to\R$
 and write $f_n\ccto f$
 if $f_n\cto f$ and
 the following holds.
% For any $x\in A$
  There is   $c>0$
 such that for
 any %$M_n\ni p_n\to p\in U$ and
  convergent
 sequence of directions
 $\xi_n\to \xi$
 $$\limsup_{n\to\infty} d_{p_n}f_n(\xi_n)-
 \liminf_{n\to\infty} d_{p_n}f_n(\xi_n)\le
 c\delta(\xi).
 $$
 \end{thm}
 
 \begin{thm}{Remark}
 It is straightforward to show that 
 applying the definition
   it suffices to verify the convergence
 only for strongly convergent sequences of directions  $\xi_n\to \xi$.
 \end{thm}
 
 
Informally we can say that $\delta(\xi)$ measures
 how nonlinear is tangent space in direction $\xi$.
 And for roughly $C^1$-convergent sequence
limit jump of differential in  direction $\xi$ is linearly bounded by
 this number. Let also note that in the case when the limit
 space is a smooth manifold rough
 $C^1$-convergence is equivalent to $C^1$-convergence 
 (in particular rough
 $C^1$-convergence implies that the limit function
 is $C^1$).
 
 
 
\begin{thm}{Lemma}\label{testrC}
	Any test sequence roughly
	$C^1$-converges.
	
\end{thm}

This lemma is proved in section???
 
 
 
 
 \subsection{Local version of the main theorem}\label{sec:loc}
 
 
\begin{thm}{Main theorem (local)}\label{mainloc}
Let   
$M_n\in\M_{\ge -1}^m$,
$M_n\GHto A$, 
  $U\subset A$ be a good domain for this convergence
  and $U_n\to U$.
  Let sequences $f_{1,n},\dots,f_{m-2,n},\\ g_{1,n},\dots,g_{m-2,n}\in C^1(U_n)$ 
  roughly $C^1$-converge to
   $f_1,\dots, f_{m-2}, g_1,\dots, g_{m-2}\in C(U)$ correspondingly.
Then measures 
$${q}_M(\nabla f_{1,n},\nabla f_{2,n},\dots,\nabla f_{m-2,n},
\nabla g_{1,n},\nabla g_{2,n},\dots,\nabla g_{m-2,n})d\vol_m$$ weakly converges to some
measure on $U$.
\end{thm}

Theorem~\ref{main} could be reduced to the local version
in view of Lemma~\ref{testrC}.


\subsection{Measures are uniformly bounded}
Our proof strongly relies on the following result from
\cite{petrunin-SC}.
\begin{thm}{Theorem}\label{scPet}
Let $M$ be a complete Riemannian m-manifold with sectional
curvature $ \ge -1$.
Then for any $p \in M$, $r<1$
$$\int_{B_r(p)} \Sc\le \const(m)r^{m-2},$$
where $B_r(p)$ denotes the ball of radius $r$ centered at $p\in M$ and $\Sc$ is a scalar curvature
of M.
\end{thm}

This result can be proved in the same way for
strongly inner subsets in $M\in\M_{\ge -1}^m$.
We denote by $K_{max}(x)=\sup_{\sigma\subset T_xM} |K_\sigma|$
for any point $x\in M$.
The straightforward consequence that we need is the following:

\begin{thm}{Corollary}\label{Kbound}
Let $M\in\M_{\ge -1}^m$ ,
$U\subset M$ is a strongly inner domain and
 $r<\min\{\diam U, 1\}$.
 Then for any $p\in U$,

$$\int_{B_r^M(p)} K_{max}\le \const(m)r^{m-2}.$$


\end{thm}



\subsection{Partition into three subsets}
 Let   
$M_n\in\M_{\ge -1}^m$,
$M_n\GHto A$ 
and  $U\subset A$ is a good domain for this convergence.
Because of the symmetry 
 of curvature tensor it is sufficient
to  prove  the weak convergence 
of measures 
$\mathfrak{q}_{M_n}(f_{1,n},\dots,f_{m-2,n},f_{1,n},\dots,f_{m-2,n})$
with densities
$$\omega_n
=
{q}_{M_n}(f_{1,n},\dots,f_{m-2,n},f_{1,n},\dots,f_{m-2,n})
\:M_n\to\RR$$
for any collection of  sequences
$f_{i,n}\in C^1(M_n)$,  $i\in\{1,2,\dots,m-2\}$,
such that
$f_{i,n}\ccto f_i  \in rC^1(A) $. Let us fix such a collection
and keep the notation $\omega_n$ for densities above.
Recall that
$${q}_{M_n}(f_{1,n},\dots,f_{m-2,n},f_{1,n},\dots,f_{m-2,n})(x)=
|\nabla f_{1,n}(x)\wedge\dots \wedge\nabla f_{m-2,n}(x)|^2\cdot K_\sigma, $$
where $K_\sigma $ is the sectional curvature at $x\in M_n$
on a plane orthogonal to 
$\nabla f_{1,n}(x)\wedge\dots \wedge\nabla f_{m-2,n}(x)$.
Since gradients are bounded the
Corollary~\ref{Kbound} 
implies that
 for any point $p\in M_n$ and $r<1$,
$$\int\limits_{B_r(p)}|\omega_n|d\vol_n\le \const\cdot r^{m-2},\eqlbl{sc-pet}$$
i.e.  locally measures  $\omega_nd\vol_n$ are uniformly bounded.
This boundness of measures
is the crucial point for our proof.
Firstly this makes possible
 passing to a subsequence of $M_n$ such that
measures  $\omega_nd\vol_n$  weakly converge.
Now
let $\mathfrak{r}_1, \mathfrak{r}_2\in\mathfrak M(A)$ be two weak partial limits, we want to show that $\mathfrak{r}_1=\mathfrak{r}_2$.
To do this
we partition $A$ into three subsets $A^\circ$, $A'$ and $A''$ and separately prove that $\mathfrak{r}_1$ and $\mathfrak{r}_2$ coincide on each of these subsets (claims~\ref{A''}, \ref{A'} and \ref{A^0}).
The partition is constructed as follows
\begin{enumerate}
\item $A^\circ=\set{x\in A}{\T_x\ \text{is isometric to}\  \RR^m}$ --- the set of regular points.
\item $A'=\set{x\in A\backslash A^{\circ}}{\T_x\ \text{is isometric to}\  \RR^{m-2}\times \Cone^2}$ the set of singular points of codimension two.
\item $A''$ --- the rest, i.e. $A''=A\backslash (A^{\circ}\cup A')$ the set of singular points of codimension three and larger.
\end{enumerate}


 Proofs for $A'$ and $A''$ (for exact formulation see \ref{A''}, \ref{A'})
 are self-contained and don't differ  for different dimensions.
  For the set  $A^\circ$  we use a sort of induction (for plan see \ref{sec:A0} ).
  
  Passing to partial limits
 $\mathfrak{r}_1$, $\mathfrak{r}_2$
is convenient for formulations and on the last step of the proof. In the proof we don't pass to
convergent subsequences, but verify weak convergence of
measures directly.
While there is still
the point (in the proof of Lemma~\ref{A^0}) where we strongly rely on uniform boundness 
of measures: we prove convergence of integrals 
not for all continuous functions as a test functions, but
only for
smooth functions, i.e. we prove convergence in the sense 
of distributions. Then we  obtain weak
convergence  provided by the uniform boundedness of measures.   




\subsection{The set $A''$ of  singularities of higher codimension }
According to \cite[10.6]{BGP}, the $(m-2)$-hausdorff measure of $A''$ vanishes.
Thus, from \ref{sc-pet}, we get

\begin{thm}{Claim}\label{A''}
$\mathfrak{r}_1|_{A''}\equiv\mathfrak{r}_2|_{A''}\equiv0$.
\end{thm}

\subsection{The set $A'$ of codimension 2 singularities  }
Consider function $\omega\:A\to\RR$ defined as
$$\omega(p)=2\cdot\pi\cdot\l(1-\frac{\vol\Sigma_p}{\vol\SS^{m-1}}\r).$$
%Clearly $\omega|_{A^\circ}\equiv0$, further 
According to \cite[7.14]{BGP}, the
function $\omega\:A\to\RR$ is lower-semicontinuous.
If $p\in A'$, i.e. $\T_p=\RR^{m-2}\times \Cone^2_p$, then the total angle of $\Cone^2_x$ is equal to $2\pi-\omega(x)$.

Remind that we have a  collection of 
sequences $f_1^n,f_2^n,\dots,f_{m-2}^n$,  $C^1$-converging in Alexandrov sense.
For $x\in A''$ the tangent cone splits as
$\T_xA=\RR^{m-2}\times \Cone$.  
$C^1$-convergence implies in particular 
that 
in some sense there exists  a "limit projection" of $\nabla f_i^n$ onto $\RR^{m-2}$-factor for all $i\in \{1, 2, \dots, m-2\}$
(for exact statement see ??). We denote these projections by
$(\nabla f_i)^\bot(x)\in\RR^{m-2}\times \{O\}\subset T_xA$
and define $\theta_x(f_1,f_2,\dots,f_{m-2})=
|(\nabla f_1)^\bot \wedge (\nabla f_2)^\bot\wedge\dots
  \wedge (\nabla f_{m-2})^\bot|^2$. 

We denote by $h_\alpha$ the $\alpha$-dimensional Hausdorff measure. Measures
$\mathfrak{r}_1$ and
$\mathfrak{r}_2$ are
 absolutely continuous with respect to
 $h_{m-2}$ (this follows from \ref{sc-pet}).
Moreover,
densities of
 $\mathfrak{r}_1$ and
$\mathfrak{r}_2$
coincide on $A'$ and equal 
$ \omega(x)\cdot\theta_x(f_1,f_2,\dots,f_{m-2})$:
\begin{thm}{Claim}\label{A'}
 
For any subset $X\subset A'$
 
 $$\int_Xd\mathfrak{r}_1=\int_Xd\mathfrak{r}_2=\int_X
\omega(x)\cdot\theta_x(f_1,f_2,\dots,f_{m-2}) d h_{m-2} $$
 
  
 %$\mathfrak{r}_1|_{A'}=\mathfrak{r}_2|_{A'}=\omega(x)\cdot\theta_x(f_1,f_2,\dots,f_{m-2})\cdot h_{m-2}|_{A'}$.
\end{thm}

Firstly we prove this claim in the particular case, when
the limit Alexandrov space is a cone of the form $A=\Cone\times\RR^{m-2}$. Then for general $A$ we blow up the 
neighborhood of  $x\in A'$ and apply result for a cone.

\subsection{The set $A^\circ$  of regular points }\label{sec:A0}
This part of the proof is the most technical.
We recall that the set of regular points can be presented as
$$A^{\circ}=\bigcap_{\delta>0} A^\delta,$$
where $A^\delta$ denotes the set of $\delta$-strained points of $A$.
Then to show that
$\mathfrak{r}_1|_{A^o}\equiv\mathfrak{r}_2|_{A^o}$
it is enough to show the following

\begin{thm}{Lemma}\label{A^0}
Let $\nu=|\mathfrak{r}_1-\mathfrak{r}_2|$.
Then for any $x\in A^\circ$, there is a neighbourhood $U_x\ni x$ such that
$$\nu({A^\delta\cap U_x})\le \const\cdot\delta$$
for some fixed $\const\in\RR$, independent of $\delta$.
\end{thm}


We prove firstly Lemma~\ref{A^0} for dimension 3. In the proof
we use the main theorem (local) for dimension 2. 
 Then we go through the proof and obtain Main theorem (local) for dimension 3.
 Then we reduce higher dimensional case of Lemma~\ref{A^0} to 
   the main theorem (local) for dimension 3. 
    
 
%Let note that the lemma asserts 
%weak $ c\delta$-convergence 
%of measures with densities $r_n$ on $A^\delta\cap U_x$ in the sense
%of definition~\ref{deltaconv2},  but  in the proof we don't pass to patial limits and verify convergence for smooth function as a test function (Lemma~\ref{smoothmeasure}).
 
 

  Now we sketch the proof of this lemma in dimension 3. 
 For the proof
 we introduce 
 a new tensor 
 for Riemannian manifold $M$ and call it
 Strange curvature:  $Str(w,w)=\Sc\cdot |w|^2-Ric(w,w)$, $w\in TM$.
 In 3-dimensional case
 Strange curvature
  completely defines curvature tensor
  and we reduce the proof of convergence of curvature tensor
  on $A^\delta$ to the convergence of Strange curvature.
 For the proof of this convergence we use 
 expression for Strange curvature for Riemannian 
 manifold,
namely
 a Bochner-type formula \ref{Bochner}.

To prove convergence of
integrals in this
 formula  we use
special charts with common domain $\Omega$ for all the sequence and the limit space
 around a  point $x\in A^0$.
These charts are special
diffeomorphisms
$\mathfrak X_n:U_n\to\Omega\subset\R^3$, $U_n\subset M_n$ and
homeomorphism
$\mathfrak X:U_x\to\Omega\subset\R^3$, $U_x\subset A$.
Here we  generalize methods introduced in
\cite{PerDC}.  In  this paper G. Perelman 
constructed $DC$-atlas ($DC$=difference concave)
for Alexandrov space and developed calculus for
$DC$-functions. We adjust these for a sequence
of manifolds converging to Alexandrov space.



 
The proof of Lemma~\ref{A^0} for higher dimensions
relies on the main theorem (local version) for dimension 3.



\section{Common chart and metric tensor}

The main tool for the proof
of Lemma~\ref{A^0} for dimension 3
is DC-calculus in  charts near regular point
with common domain for the smoothing sequence and the
limit space. For DC-calculus in domain in $\R^n$
we refer Appendix (Section~\ref{sec:DC}).
In the next Proposition we 
state the existence of a common chart with 
properties that allow to apply these calculus.
The proof is given in \ref{NiceChartProof}.

\begin{thm}{ Proposition}\label{Prop:chart}
	There exists $\delta_0>0$, such that for $\delta<\delta_0$
	the following hold.	
	Let   
	$M_n\in\M_{\ge -1}^m$,
	$M_n\GHto A$ , $ p\in A^\delta$
	and $B_r (p)$
	be a good domain for this convergence.
	For some open
	$\Omega\subset \R^m$, open 
	neighborhood $U\subset B_r (p)$ of $p$
	and $U_n\GHto U$ 
	there exist a
	sequence of
	diffeomorphisms $\mathfrak X_n=(\x_{1,n},\dots,\x_{m,n}):U_n\to\Omega$
	converging to a homeomorphism 	$\mathfrak X=(\x_{1},\dots,\x_{m}):U\to\Omega$ with 
	the following properties.
	
	\begin{enumerate}[label=\alph*. ]
	
	%\addtocounter{enumi}{2}
	\item\label{obtuse}

 Every coordinate sequence $\x_{i,n}$ is a sequence of smooth concave functions roughly $C^1$-converging to $\x_i$. Gradients
	$\nabla\x_{i,n}$ and $\nabla\x_{j,n}$ form strictly obtuse angles,
	bounded from $\pi$ and $\pi/2$ for
	$i\neq j$. 
	\item $\mathfrak X_n, \mathfrak X$ are bi-Lipschitz 
	with uniform  bi-Lipschitz 
	constant
	
\item\label{metric} 
	There exists a continuous Riemannian metric
	$g$ on $\Omega\setminus S_\Omega$ which locally 
	realizes distance on $U$, where
	$S_\Omega=\mathfrak X(U\cap A\setminus A^0)$.
	This metric tensor 
	(defined almost everywhere on $\Omega$)
	is of bounded variation
	on $\Omega$. 
	
\item\label{metricseq}
	Let $g_{ij,n}$ be coordinates of metric tensor of $M_n$ 
	in the chart $\X_n$. 
	Then
	$(g_{ij,n}, g_{ij})\in BV_0^{seq}(\Omega, S_\Omega)$.
	Moreover, $\det(g_{ij,n})$ is bounded and bounded away from 0.
	
	
	\item\label{funktioninchart}
	For any
	sequence
	of concave smooth functions
	$f_n:B_{r_n}(p_n)\to\R$ 
	which roughly $C^1$-converges to  $f:B_r(p )\to\R$ we have for its coordinate
	expression
	$(f_n\circ\X_n^{-1}, f\circ\X^{-1})\in DC_0^{seq}(\Omega, S_\Omega)$
	
	\end{enumerate}
\end{thm}

\begin{thm}{Definition}
	We call the sequence of
	diffeomorphisms $\mathfrak X_n:U_n\to\Omega$
	together with a homeomorphism 	$\mathfrak X:U\to\Omega$
	defined in
Proposition~\ref{Prop:chart}
a {\emph Nice common chart} around $p$.

\end{thm}

\section{Strange curvature convergence}
\subsection{Statement and outline of the proof  }

Let $M$ be a Riemannian manifold.
We define a $(0,2)$ tensor  $ \Str$  by
$$\Str(w,w)=\Sc\cdot |w|^2-\Ric(w,w), \quad w\in TM$$
and call it {\it Strange curvature} tensor or just Strange curvature.
 
 In this section we prove a 
convergence of Strange curvature 
in $3$-dimensional case
in the following sense.

\begin{thm} {Proposition}\label{strconvergence}
Let   
	$M_n\in\M_{\ge -1}^3$,
	$M_n\GHto A$ , $ x\in A^\delta$
	and $B_r (x)$
	be a good domain for this convergence.
	
Let $M_n\ni x_n \to x\in A^0$ and
 $f_n\in C^\infty(B_r(x_n))$ be a
 sequence of concave functions such that
$f_n\ccto f $, suppose in addition that 
$1/c\le|\nabla f_n|\le c$ and set $u_n=\nabla f/|\nabla f|$.
Then for sufficiently small $r_0>0$
and some constant $ C>0$  the following holds.

% Let us denote $u_n=\nabla f/|\nabla f|$.
Consider
a sequence of measures 
$m_n=
\Str\left(
u_n, u_n\right) \cdot \vol_n$
on $B_{r_0}(x_n)$ and
suppose $\tau_1, \tau_2\in \mathfrak M(B_{r_0}(x))$ be two partial
weak limits of $m_n$.
Then we have
$|\tau_1(S)-\tau_2(S)| \le
C\delta $ for every $S\subset A^\delta \cap B_{r_0}(x)$.

\end{thm}


For the proof we use some integral expression
for Strange curvature via some formula of Bochner type for Riemannian manifold.
 To present this formula
 let us introduce some notation for a smooth function 
 $f$ without critical points defined on an open domain
 $U\subset M$.
For $x\in U$ we set

\noindent
 $u_f=\nabla f/ |\nabla f|$ and
denote by

\noindent
$H_f(x)$ -- the mean curvature of the level set $f^{-1}(f(x))$ and
by

\noindent
 $\Int_f$ -- the scalar curvature of  $f^{-1}(f(x))$.
 
 \noindent
 Then for any smooth function with compact support
  $\phi\:\Omega\to\R$ we have the following expression.
 (The proof is given in \ref{}.)
$$\int\limits_\Omega \phi\cdot \Str(u_f, u_f)
=\int\limits_\Omega \phi\cdot \Int_f+
\int\limits_\Omega \l[H_f\cdot\<u_f,\nabla\phi\>- \<\nabla\phi,\nabla_{u_f} u_f\> \r].
\eqlbl{Bochner}$$
 
We apply
this formula 
to functions $f_n$
 and
reduce the Proposition~\ref{strconvergence} to the two lemmas below (proved
in the next subsection),
each lemma is related 
  to the convergence 
of two integral components in the above formula.
We prove the convergence of the first integral
in a sense of $C\delta$-convergence of measures:


\begin{thm} {Lemma}\label{Int}
In condition of Proposition~\ref{strconvergence}
for sufficiently small $r_0>0$
and some constant $ C>0$  the following holds.

% Let us denote $u_n=\nabla f/|\nabla f|$.
For any two partial
weak limits $\tau_1, \tau_2\in \mathfrak M(B_{r_0}(x))$  of 
measures $\Int_{f_n}\cdot \vol_n$ on $B_{r_0}(x_n)$
we have
$|\tau_1(S)-\tau_2(S)| \le
C\delta $ for every $S\subset A^\delta \cap B_{r_0}(x)$.


\end{thm}

 The proof of this lemma
 uses 
convergence of curvature measures
$\Int_{f_n}\cdot \vol_n$
on $2$-dimensional 
smoothing sequence (level  sets of concave functions$f_n$ are
locally in $\DM^3$).
The error $C\delta$ arises 
because  sequence of functions $|\nabla f_n|$ is 
only
 $C\delta$-convergent   on $A^\delta$.




 
 The next lemma is related to
  the convergence of the second integral, the proof 
 uses
$DC$-calculus in Nice common chart. 


\begin{thm}{Lemma}\label{HnablaU}
In condition of Proposition~\ref{strconvergence}
let $\mathfrak X_n:U_n\to\Omega$
and
$\mathfrak X:U_x\to\Omega$
be a Nice common chart around $x$. 
We fix some smooth function with compact support
$\psi\in C^\infty_0(\Omega)$ and denote by
$\phi_n=\psi\circ\mathfrak X_n$. Then

$$
\int\limits_{U_n} \l[H_{f_n}\cdot\<u_{f_n},\nabla\phi_n\>- \<\nabla\phi_n,\nabla_{u_{f_n}} u_{f_n}\> \r]
\eqlbl{Bochner}$$
converges.

\end{thm}



Let us deduce Proposition~\ref{strconvergence} from Lemmas.
We take a Nice common chart 
$\mathfrak{X}_n:U_n\to\Omega$ and
$\mathfrak{X}:U\to\Omega$ around $x$.
We can assume 
that $U\subset B_{r_0}(x)$ where $r_0$ is 
from conclusion of Lemma~\ref{Int}.


Let us define sequences 
$L_n, L_n^1, L_n^2:C^\infty_0(\X(U\cap A^\delta))\to \R$
of continuous linear operators

$$L_n(\psi)=
\int\limits_{U_n}(\psi\circ\mathfrak X_n )\cdot \Str(u_{f_n}, u_{f_n})=
\int\limits_{U_n} \phi_n\cdot \Str(u_{f_n}, u_{f_n})
$$

$$L_n^1(\psi)=
\int\limits_{U_n} \phi_n\cdot \Int_{f_n},
\qquad L_n^2(\psi)=
\int\limits_{U_n}
 \l[H_{f_n}\cdot\<u_{f_n},\nabla\phi_n\>- \<\nabla\phi_n,\nabla_{u_{f_n}} u_{f_n}\> \r].$$

Then \ref{Bochner} implies that $L_n=L_n^1+L_n^2$.
It follows from
Lemma~\ref{Int} that 
$L^1_n(\psi)$ $c\delta$-converges for any $\psi\in C^\infty_0(\X(U\cap A^\delta))$. From
Lemma~\ref{HnablaU} we have that
$L^2_n(\psi)$ converges for any $\psi\in C^\infty_0(\X(U\cap A^\delta))$.
It follows from Corollary~\ref{Kbound}  that operators $L_n$
are uniformly bounded with respect to the uniform norm and by above
$L_n(\psi)$ $c\delta$-converges for any $\psi\in C^\infty_0(\X(U\cap A^\delta))$.
Then  the sequence $L_n$ regarded as a sequence 
of measures weakly $c\delta$ converges to some 
measure. Proposition~\ref{strconvergence} follows.

\subsection{Proof of Lemma~\ref{HnablaU}
and Lemma~\ref{Int}}
\parit{Proof of Lemma~\ref{HnablaU}.}

We rewrite
$$\int\limits_{V_n} \<u_n,\nabla\psi_n\> H_nd\vol_n
=
\int\limits_{V_n} \<u_n,\nabla\psi_n\> \div u_nd\vol_n
$$

Let us
 rewrite the integral in $\Omega$:

$$\int\limits_{\Omega} \Bigl(\sum_{i=1}^3u_n^i \frac{\partial \phi}{\partial x_i}\Bigr)
\sum_{i=1}^3\Bigl(\partial u_n^i/ \partial x^i +u^i_n\partial \log \sqrt {\det (g_{ij,n})}/\partial x^i\Bigr)
\sqrt {\det (g_{ij,n})}dx^1\wedge dx^2\wedge dx^3$$

Now by Proposition~\ref{Prop:chart}.\ref{metric}
$g_{i j,n}, g^{ij}_n\in  \op{BV_0^{seq}}(\Omega,S_\Omega,\RR)$
and
$\det g_{ij,n}$ are bounded away from $0$.
We also have
$u_n^i=\frac{g^{ij}\frac{\partial (f\circ \X_n^{-1})}{ \partial x_j}}
{\sqrt{g^{jk}\frac{\partial (f\circ \X_n^{-1})}{\partial x_j }\frac{\partial (f\circ \X_n^{-1})}{\partial x_k}}}$,
 by Proposition~\ref{Prop:chart}.\ref{funktioninchart}
 $f_n\circ\X_n^{-1}\in DC_0^{seq}(\Omega, S_\Omega)$
hence 
${u^i_n}\in  \op{BV_0^{seq}}(\Omega,S_\Omega,\RR)$.

Then applying
Lemma~\ref{thm-D}, Lemma~\ref{thm-CM} and
Lemma~\ref{thm-A} we get

\begin{itemize}

\item $\partial \log \sqrt {\det (g_{ij,n})}/\partial x^i\in \aleph_0^{seq}(\Omega,S_\Omega)$

 \item $\frac{\partial u_n^i}{\partial x_i}\in \aleph_0^{seq}(\Omega,S_\Omega)$
 
 \end{itemize} 
 
 and by above
 
  \begin{itemize}
  
  \item ${u^i_n}\in  \op{BV_0^{seq}}(\Omega,S_\Omega,\RR)\subset\op{C_0^{seq}}(\Omega,S_\Omega,\RR)$ 
 
 \item $g_{i j,n}\in   \op{BV_0^{seq}}(\Omega,S_\Omega,\RR)\subset \op{C_0^{seq}}(\Omega,S_\Omega,\RR)$
 
  \end{itemize}
  
  Then applying Lemma~\ref{thm-CM}
  we get under integral the sum of elements from $\aleph_0^{seq}(\Omega,S_\Omega)$
  multiplied by $\frac{\partial \phi}{\partial x_i}$ - smooth functions with compact support.
       This gives convergence of the integral by definition of  $\aleph_0^{seq}(\Omega,S_\Omega)$.
              

Further for the second integral % (see claim~\ref{cl:funDC})
we have
$$\int\limits_{M_n}\<\nabla\psi_n,\nabla_{u_n} u_n\>d\vol_n=$$
 $$\int\limits_{\Omega}  \sum_{i,j,k}u^i_n \frac{\partial \phi}{\partial x_k}
 \biggl(\frac{\partial u^k_n}{\partial x_i} +\frac{1}{2}u^j_n\sum_s
 \biggl(\frac{\partial g_{ js,n}}{\partial x_i}+\frac{\partial g_{si,n}}{\partial x_j}-\frac{\partial g_{i j,n}}{\partial x_s}\biggr) g^{ks}_n\biggr)\cdot \sqrt{\op{det}(g_{i j,n})}dx^1\wedge dx^2\wedge dx^3$$ 
 
 Then as above for the first integral
 applying
Lemma~\ref{thm-D}, Lemma~\ref{thm-CM} and
Lemma~\ref{thm-A}  
 we  get under the integral sequence from $\aleph_0^{seq}(\Omega,S_\Omega)$
  multiplied by smooth function with compact support $\frac{\partial \phi}{\partial x_i}$. Then the integral converges and the lemma follows.
  
  \qeds


\parit{Proof of Lemma~\ref{Int}.}


Any point  in an Alexandrov space
has a convex neighborhood (\cite{convexity}) and by
construction  it can be lifted  to a smoothing sequence.
So let 
$V\subset A$ be an open  convex
neighborhood of $x$ and
$ V_n\subset M_n$
be open convex sets such that
$V_n  \dto{GH}   V$.

We set $f_0:=f$, $h:=\sup |f_n|<\infty$
and denote 
$$L_{t,n}=f_n^{-1}(t)\cap V_n,\qquad
C_{t,n}=f_n^{-1}[t,h]\cap \overline{V_n}.$$
Now we want to show that local version of the main theorem  for dimension 2 can be applied to small subsets of $L_{t,n}$ with some uniform estimates.



For every $t$ and $n\ge 0$ the set
$C_{t,n}$ is 
 a
 convex subset in
  Alexandrov space 
 and hence is an Alexandrov space 
 with curvature $\ge -1$.
For any $t_n\to t^*$ we have that
$C_{t_n,n}     \dto{GH}    C_{t^*,0} $. 
Then boundaries  $\partial C_{t_n,n}  $
  converge to $\partial C_{t^*,0}  $ in
Gromov Hausdorff sense and then by  
  \cite{petrunin-QG} (Theorem 1.2)
 $\partial C_{t_n, n}  $ equipped with inner metric
 $\rho^{\partial C_{t_n,n}}$ converge to
 $\partial C_{t^*,0}  $ with inner metric
 $\rho^{\partial C_{t^*,0}}$.
 
 
 It follows from \cite{AKP} that
 for $n\ge 1$
  (since $C_{t,n}$ are
 convex subsets in a Riemannian manifolds with curvature $\ge -1$)
  the boundary
$(\partial C_{t,n}, \rho^{\partial C_{t_n,n}}) $ is
an Alexandrov space 
 with curvature $\ge -1$. Hence the limit
$(\partial C_{t^*,0}, \rho^{\partial C_{t^*,0}}) $ is
an Alexandrov space 
 with curvature $\ge -1$.

Let $\rho_{t,n}$ be induced inner metric on $L_{t,n}$,
then $\rho_{t,n}$ locally coincide with
$\rho^{\partial C_{t,n}}$ and hence 
for any $t^*$ and $t_n\to t_*$
$(L_{t_n,n}, \rho_{t_n,n})\dto{GH} (L_{t_*,0},\rho_{t^*,0} )$.
Note that since $(L_{t,n}, \rho_{t,n})\in\M_{\ge -1}^m$
we obtained
local smoothing sequences. It remains to find good inner
subsets for these smoothings.

Let  note that since 
$\partial C_{t^*,0}$
is an extremal subset for
$C_{t^*,0}$
 the inner metric
$ \rho^{\partial C_{t^*,0}} $ is bi-Lipschitz to
the metric restricted from $A$ to
$\partial C_{t^*,0}$.
It follows that
we can take $r$ sufficiently small
such that for all $t$ and
$U_{t,n}=L_{t,n}\cap B_{r}(x_n)$
we will have
$\diam_{\rho_{t,n}} U_{t,n}\le 1/10 \dist   (U_{t,n},\partial C_{t,n}\setminus L_{t,n})$. Then for
all $y\in U_{t,n}$ closed balls
$\overline {B^{L_{t,n}}_{5\diam_{\rho_{t,n}} U_{t,n}}}(y) $ are compact.
Hence every $U_{t,0}$ is a good inner subset for 
smoothing  $L_{t,n}\to L_{t,0}$
and local version of the main theorem for dimension 2
can be applied to this set.

Now we fix some
$\phi\in C^0_c(B_r(x)\cap A^\delta) $ and
$\phi_n\in C^0_c(B_r(x_n)) $,
such that
$\phi_n\cccto\phi$.


$$ \int\limits_{B_r(x_n)}\Int_n(s)\phi_n(s)d\vol_n(s)
=\int\limits_{-h}^{h} d t\int\limits_{ U_{t,n}}
 \frac{\phi_n(s)}{|\nabla f_n(s)|}\Int_n(s)d S_n^t(s),$$
 where $ S_n^t$ is the $2$-volume on $ L_{t,n}$.
 For any $t\in[-h,h]$ we have
 $L_{t,0}\in\DAl^m$,
$L_{t,n}\in\DM^m$,
$L_{t,n}\dto{GH} L_{t,0}$
 and $U_{t,0}$ is good inner set for this smoothing.
Then local version of Main theorem  for dimension $2$ can be applied.
 Hence $\Int_n(x)dS_n^t(x)$ weakly converges to some measure on $U_{t,n}$.
 From 
compactness  $\sup_{t} \diam U_{t,n}<\infty$.
 Then by Corollary~\ref{Kbound} 
$\int\limits_{ L_t^n}
 |\Int_n(x)|dS_n^t(x)\le c_1$ for some $c_1$ independent on $t$. We know that
 $\operatorname{supp}(\phi/{|\nabla  f_n|})\in C^0_c(B_r(x)) $
and the sequence
  ${\phi_n}/{|\nabla  f_n|}$
$c_2\de$-converges on $A^\delta$ 
for some $c_2>0$
(see claim~\ref{lem:scalprod}).
Then for all $t$ we have
$$\limsup_{n\to\infty}\int\limits_{U_{t,n}} \frac{\phi_n(x)}
{|\nabla f_n(x)|}\Int_n(x)dS_n^t(x)-
\liminf_{n\to\infty}\int\limits_{U_{t,n}} \frac{\phi_n(x)}{|\nabla f_n(x)|}\Int_n(x)dS_n^t(x)
\le c_1c_2\delta.$$

Hence

$$
\limsup_{n\to\infty}
 \int\limits_{B_r(x_n)}\Int_n(s)\phi_n(s)d\vol_n(s)-
\liminf_{n\to\infty}
 \int\limits_{B_r(x_n)}\Int_n(s)\phi_n(s)d\vol_n(s)
\le 2hc_1c_2\delta.$$
\qeds

\section{Completion of the proof near regular
set in dimension 3}
In this section we 
 prove Lemma~\ref{A^0} for 3-dimensional case.

 \subsection{Vectors in general position}

We say that  vectors 
 $w_1,\dots,w_{N(N-1)/2}\in\R^N$ are 
  \emph{in general position}
if any quadratic form $Q:\R^N\times \R^N\to\R$
can be computed from the values $Q(w_k, w_k), k=1,\dots,N.$ 
More formally,
there are rational functions $s_{ij}^k:(R^{N(N-1)})^N\to [-\infty,\infty]$
such that
 if we fix basis $e_1,\dots, e_N\in \R^N$
 and take vectors $w_s=w_s^ie_t$ for $s=1,\dots,\frac{N(N-1)}{2}$,
 then for any quadratic form $Q$ we can formally express its coordinates as:
$$Q(e_i,e_j)=\sum_{k=1}^{N(N-1)}s_{ij}^k(w_s^t)Q(w_k,w_k). \eqlbl{Qij}$$
Vectors are  in general position iff 
$s_{ij}^k(w_s^t)<\infty$ for all $i,j\in \{1,\dots, N\}$
and
$k\in \{1,\dots, N(N-1)/2\}$. 
We need further the following observation:
since 
$s_{ij}^k$ are rational functions
any array of vectors in general position has a small
neighbourhood where $s_{ij}^k$ are Lipschitz.
\subsection{Proof} 





Let us fix Nice common chart sequence
$\mathfrak X_n =(\mathfrak x^1_n, \mathfrak x^2_n,\mathfrak x^3_n):U_n\to\Omega$,
$\mathfrak X =(\mathfrak x^1, \mathfrak x^2,\mathfrak x^3):U\to\Omega$.
Let us denote coordinate frame
$e_{1n}=\nabla \mathfrak x^1_n, e_{2n}=\nabla \mathfrak x^2_n,e_{3n}=\nabla \mathfrak x^3_n$
 and $e_{1}=\nabla \mathfrak x^1, e_{2}=\nabla \mathfrak x^2,e_{3}=\nabla \mathfrak x^3$. Let
${w^*}_1,\dots,{w^*}_6\in T_pA=\R^3$ 
be vectors in general position, with
coordinate expressions ${w^*}_k={w^*}_k^ie_i$.
By Proposition~\ref{NiceFunctions}
we can choose 6 smooth sequences of concave functions $f^1_n\ccto f^1,\dots, f^6_n\ccto f^6$ around $p$, such that
$\nabla f^1,\dots,\nabla f^6$ are sufficiently close to
${w^*}_1,\dots,{w^*}_6$
and we can assume $U$ to be sufficiently small such
that
measures $\Str(\nabla f^k_n, \nabla f^k_n)d\vol_n$ weakly $c\delta$-converges
on $U\cap A^\delta$ for $k=1,\dots,6$ (by Proposition~\ref{strconvergence}).
 Let us denote
$w_{k,n}=\nabla_nf^k_n$ and expression
in the frame
$w_{k,n}= w_{k,n}^i e_{i,n}$.
Then
taking $U, U_n$ sufficiently
small  so that for all $y\in U_n$ coordinates
$w_{i, n}^k(y)\in \R^{3\times 6}$ are in a neighborhood 
of $ {w^*}_i^k\in \R^{3\times 6}$, where
corresponding functions $s_{ij}^k$ in \ref{Qij} are Lipschitz.
Then applying \ref{Qij} we get that 
$$\Str(e_{i,n},e_{j,n})=\sum_{k=1}^6 s_{ij}^k(w_{s,n}^t)\Str(w_{k,n}, w_{k,n})=\sum_{k=1}^6 s_{ij}^k(w_{s,n}^t)\Str(\nabla f^k_n, \nabla f^k_n).   $$
We know that $w_{s,n}^t$ are 
$c\delta$-converging
on $U\cap A^\delta$ (Corollary~\ref{cor:cdeltacoeff}) and $s_{ij}^k$ are Lipschitz,
hence components of Strange curvature
$\Str_n(e_{i,n},e_{j,n})d\vol_n$
 weakly $c\delta$-converges
on $U\cap A^\delta$. 

Let note, that for $3$-dimensional manifold 
curvature tensor $q$ can be expressed via Strange curvature and
for $M_n$ we have:
 $$q_{M_n}(v,v)=\Str_n(v,v)-|v|^2\Tr \Str_n/4.\eqlbl{Q}$$
Hence
$$q((w_{k,n}, w_{k,n})=
\Str_n(w_{k,n}, w_{k,n})-(\sum_{i,j=1}^3 g_{ij,n}w^i_{k,n} w^j_{k,n})\operatorname{Tr}\Str_n$$

We know that $g^{ij}_n$ are
$c\delta$ converging on $U\cap A^\delta$ (Lemma~\ref{lem:scalprod}) then measures
$\operatorname{Tr}\Str_nd\vol_n=g^{ij}_n\Str_n(e_{i,n},e_{j,n})d\vol_n$
and  hence
 $q((w_{k,n}, w_{k,n})d\vol_n$
weakly $c\delta$-converges
on $U\cap A^\delta$.
 

Then applying again \ref{Qij} we get that 
$$q(e_{i,n},e_{j,n})=\sum_{k=1}^6 s_{ij}^k(w_{s,n}^t)q(w_{kn}, w_{k,n})=\sum_{k=1}^6  s_{ij}^k(w_{s,n}^t)q(\nabla f^k_n, \nabla f^k_n) .   $$
Then as above we can obtain that
$q((e_{i,n}, e_{j,n})d\vol_n$
  weakly $c\delta$-converges
on $U\cap A^\delta$.

Now let $f_n\ccto f$ be arbitrary roughly $C^1$-converging sequence. For
decomposition $\nabla f_n =\sum \alpha^i_ne_{i,n}$,
we have that $ \alpha^i_n$
$c\delta$-converges
on $U\cap A^\delta$. Then measures
$q(\nabla f_n, \nabla f_n)d\vol_n=
\sum \alpha^i_n \alpha^j_n q((e_{i,n}, e_{j,n})d\vol_n$
weakly $c\delta$-converges
on $U\cap A^\delta$.

\qeds

\section{Cone}

In this section we regard the case when the limit Alexandrov space
is a 3-dimensional cone splitting over a line.

Further $\C$ denotes the cone over a circle of length  $\omega<2\pi$
with a tip $p$.
Let $M_n\in\M_{\ge -1}^3$ converges in pointed
Gromov Hausdorff topology to $\C\times\R$.

\begin{thm}{Definition}
For a space of the form  $\C\times\R$
we call the
coordinate function   
the \emph{horizontal function}. 
\end{thm}

\begin{thm}{Proposition}
Let $M_n\in\M_{\ge -1}^3, p_n\in M_n$ and $(M_n,p_n)$ converges in pointed
Gromov Hausdorff topology to $(\C\times\R, p)$
and $\operatorname{sec}M_n\ge\frac{-1}{n}$.
Let $r_n\to\infty$ and $f_n:B_{r_n}^{M_n}(p_n)\to \R$ be a sequence of concave
functions 
converging to a horizontal function $f:\C\times\R\to\R$.
Then measures $q_n(\nabla f_n, \nabla f_n)d\vol_n$
weakly converge to the measure $\omega\cdot\delta(p)\times h_1$.

\end{thm}

\begin{thm}{Proposition}
Let $M_n\in\M_{\ge -1}^3, p_n\in M_n$ and $(M_n,p_n)$ converges in pointed
Gromov Hausdorff topology to $(\C\times\R, p)$.
Let $r_n\to\infty$ and $f_n:B_{r_n}^{M_n}(p_n)\to \R$ be a sequence of concave
functions 
converging to horizontal function $f:\C\times\R\to\R$.
Then measures $\Ric_n(\nabla f_n, \nabla f_n)d\vol_n$
weakly converge to zero measure.

\end{thm}



\begin{thm}{Definition}

We say that \emph {a sequence of vector fields $v_n$ on $M_n$ roughly converges }
if for any compact $K\subset A$ there is $c>0$ such that
for any converging sequence of directions $\xi_n\to\xi$
we have
$$\limsup_{n\to\infty} \<v_n,\xi_n\>-
 \liminf_{n\to\infty} \<v_n,\xi_n\>\le
 c\delta(\xi).
 $$
\end{thm}

\begin{thm}{Remark}
sequence of $C^1$ functions roughly $C^1$-converges
iff the gradient vector fields
roughly converges.

\end{thm}


\begin{thm}{Definition}

We say that a direction $\xi\in \Sigma_xA$ is \emph {vertical }
if it is a velocity vector of vertical line $\{p\}\times \R$.
\end{thm}

\begin{thm}{Definition}

We say that \emph {a sequence of vector fields $v_n$ on $M_n$  has vertical limit
(with projection $v_0$) } if for any vertical direction $\xi$ and 
$\xi_n\to\xi$
we have $|v_n-v_0\xi_n|\to 0 $.


We say that \emph {a sequence of vector fields $v_n$ on $M_n$  has horizontal limit}
if for any vertical direction $\xi$ and 
$\xi_n\to\xi$
$\<v_n,\xi_n\>\to 0$.

We say that \emph {a sequence of tensor fields $v_n$ on $\Lambda M_n$  has vertical limit
(with projection $v_0$)} if for any vertical directions $\xi_1,\dots,\xi_{m-2}$ and 
$\xi_n\to\xi$
we have $|v_n-v_0\xi_n|\to 0 $.


\end{thm}




\section{Blow-up at $m-2$ singularity}\label{sec:blow}

Assume $M_n$ is a smoothing of Alexandrov space $A$;
i.e. $M_n$ is a non-collapsing sequence of Riemannian manifolds
with curvature $\ge\kappa$ for some $\kappa\in\RR$ and
$M_n\GHto A$.

Note that given $p\in A$,
one can allways find a sequence $\lambda_n$
which converge to infinity so slow that
$(\lambda_n M_n,p_n)\GHto \T_p A$.
In other words, $\lambda_n M_n$ is a smoothing of $\T_pA$;
further this new smoothing will be called a \emph{blow-up smoothing} at $p$.

We will use blow-up smoothg to invesigate behavior of curvature near points $p\in A'$.
In this case $\T_p\iso\Cone\times \RR^{m-2}$.






\begin{thm}{Construction}\label{constr}
Fix $\lambda_n\to\infty$. There is $\varepsilon_n\to 0$ so that

$$\GHdist(h_{\lambda_n}B^A_{R_n}(p), B^{\T_p A}_{\lambda_n R_n}(p))\le\varepsilon_n/2$$

There is a monotonic function
 $k(n)\:N\to N$
so that for every $l\ge k(n)$
$$\GHdist(h_{\lambda_n}B^{M_l}_{1}(p), B^{\T_p A}_{\lambda_n}(p))\le\varepsilon_n/2.$$

Then for every  $k'(n)\ge k(n)$ we have
$$\lambda_n M_{k'(n)}\GHto (\T_p A, p)=(\RR^{m-2}\times \Cone_x^2,x).$$
\end{thm}
For every $\lambda>0$ we supply manifold $\lambda M_n$
with vector fields
$\bar v_n^i=1/\lambda(d h_{\lambda}v_n^i)$. We will write shortly
$R_{\lambda M_n}=R_{\lambda M_n}(\bar v_n^1\wedge\dots\wedge\bar v_n^{m-2})$

\begin{thm}{Lemma}\label{l:convVCone}
Let vector field $v_n$ on $M_n$
converges with respect to Gromov--Hausdorff
convergence $M_n\GHto A$.
Fix $\lambda_n\to\infty$.
There is a monotonic function
 $k(n)\:N\to N$ so that  for every  $k'(n)\ge k(n)$
 we have that
$$\lambda_n M_{k'(n)}\GHto (\T_p A, p)=(\RR^{m-2}\times \Cone_x^2,x)$$
and
vertical parts of $\bar v_n$ converge.
\end{thm}

\begin{thm}{Claim}\label{cl:convVCone}
Let vector field $v_n$ on $M_n$
converges with respect to Gromov--Hausdorff
convergence $M_n\GHto A$.
Fix $\lambda_n\to\infty$.
There is a monotonic function
 $k(n)\:N\to N$ so that  for every  $k'(n)\ge k(n)$
 we have that
$$\lambda_n M_{k'(n)}\GHto (\T_p A, p)=(\RR^{m-2}\times \Cone_x^2,x)$$
and
$R_{\lambda_n M_{k'(n)}}$
weakly converges to $R^{\mathring{v}^1\wedge \mathring{v}^2\wedge\dots\wedge \mathring{v}^{m-2}}_p$,
where $\mathring{v}^i$ are limits of vertical parts of $\bar v_n$.
\end{thm}

\parit{Proof.} By the lemma~\ref{l:convVCone} we can apply claim \ref{cl:vfcone}.
\qeds




\subsection{Proof of \ref{l:convVCone}}

Let sequence of Alexandrov spaces converges
$(N_n,p_n)\GHto (A,p)$.
 Let $A$ contains a line $\RR_a\subset A$
 with direction
$a\in \T_p A$.
We say that a sequence of points  $x_n\in N_n$
converges $x_n\GHto\RR_a^{+\infty}$ if
there are sequences $R_n\to\infty,\varepsilon_n\to 0$
so that
$$\GHdist(B^{N_n}_{R_n}(p_n), B^{A}_{R_n}(p))\le\varepsilon_n$$
and there is a sequence $t_n\to+\infty$,
so that
$t_n a\le R_n+2$ and
 for $a_n=t_n a\in \RR_a\subset A$
we have
$\GHdist(x_n,a_n)\le\varepsilon_n$.
We say that a sequence of points  $y_n\in N_n$
converges in weak sense $y_n\GHwto\RR_a^{+\infty}$ if
$|p_n y_n|\to \infty$ and
for some sequence
$x_n\GHto\RR_a^{+\infty}$ we have $\angle (x_n p_n y_n)\to 0$.

\begin{thm}{Lemma}\label{lem:angle}
Let we have two sequences $x_n, y_n\GHwto\RR_a^{+\infty}$,
and  sequence $q_n\in N_n$ is uniformly bounded, i.e. $|p_n q_n|<c$,
than $\angle (x_n q_n y_n)\to 0$.
\end{thm}




\begin{thm}{Lemma} \label{lem:vconvergLoc}
Let $M_n\GHto A$ and vector fields $v_n$ converges.
Let sequence $\lambda_n\to\infty$.
Let vector $a\in \T_p A$
and for sequences $q_s, q_s'$
we have $\angle (a,\dir{p}{q_s})<\varepsilon_s$,
 $\angle (-a,\dir{p}{q'_s})<\varepsilon_s$.
There is a  limit product $\<v,a\>$
with the following properties:
there is a constant $c>0$,
a function $k\: N\to N$,
 sequences
$M_i\ni q_{n,i}\to q_n$, $\varepsilon_n\to 0$
and
so that for   all $n$
$$\<v,a\>- c\cdot \varepsilon_n
\le
\inf_{i\ge k(n)}\inf_{x\in B_{1/\lambda_n}^{M_i}(p_i)} \< v_i,\dir{x}{q_{n,i}}\>
\le
\sup_{i\ge k(n)}\sup_{x\in B_{1/\lambda_n}^{M_i}(p_i)}  \< v_i,\dir{x}{q_{n,i}}\>\le
\<v,a\>+
 c\cdot \varepsilon_n$$
and for any function $s(n)>k(n)$
$$h_{\lambda_n}( q_{n,s(n)})\GHwto\RR_a^{+\infty}.$$

\end{thm}

\parit{Proof.}
We know, that $(\lambda_n A, p)\GHto (\T_p A, p)=(\RR^{m-2}\times \Cone_x^2,x)$.
There is a sequence $\tilde q_s, \tilde q_s' \in B_1^A(p)$,
 sequence $\tilde \varepsilon_s\to 0$, constant $c$ so that the following holds:
%$$h_{\lambda_n}(\tilde q_n)\GHwto\RR_a^{+\infty}$$
%$$h_{\lambda_n}(\tilde q_n')\GHwto\RR_{-a}^{+\infty}$$
$$\angle(a,\dir{\tilde q_s}{p})\ge\pi- \varepsilon_s/2$$
 $$\angle(\tilde q_s p\tilde q_s')\ge\pi- \varepsilon_s/2$$
and shortest paths $p\tilde q_s$, $p\tilde q_s'$ are unique.
%For any such sequences holds: for any$p^0\in B_{1/\lambda_n}(p)$we have
We can find monotonic sequence $r_s\to 0$ so that $|p\tilde q_s|/r_s\to\infty$
for any $p^0\in B_{r_s}(p)$
and for any approximating sequences
$M_i\ni q_{s,i}\to \tilde q_s$, $p^0_i\to p^0$ we have by property~\ref{convvf}
$$\limsup_{i\to\infty}  \< v_i,\dir{p_i}{\tilde q_{s,i}}\> -c\varepsilon_k\le
\liminf_{i\to\infty}  \<v_i,\dir{p^0_i}{\tilde q_{k,i}}\>\le\limsup_{i\to\infty}  \< v_i,\dir{p^0_i}{\tilde q_{k,i}}\>
\le\liminf_{i\to\infty}  \<v_i,\dir{p_i}{\tilde q_{k,i}}\> +c\varepsilon_k .
$$
We know that $$\angle(\tilde q_{t_1}p\tilde q_{t_2})\le\varepsilon_s$$ for $t_1, t_2>s$.
Hence (taking other sequences $\varepsilon_k$ and costant $c$)
we can find ``limit constant'' $\<v,a\>$ so that
 $$\<v,a\>-c\varepsilon_k\le
\liminf_{i\to\infty}
\<v_i,\dir{p^0_i}{\tilde q_{k,i}}\>\le\limsup_{i\to\infty}  \< v_i,\dir{p^0_i}{\tilde q_{k,i}}\>
\le\<v,a\> +c\varepsilon_k(*).$$

Let
$$k(n)=\max\set{t\in N}{\forall s\ge n\quad 1/\lambda_s<r_t}$$
We set $q_n:=\tilde q_{k(n)}$.
By construction $h_{\lambda_n}( q_n)\GHwto\RR_a^{+\infty}$
and the estimate $(*)$ implies that we can find appropriate function $k(n)$.
\qeds


\parit{Proof of \ref{l:convVCone}.}
We take $k(n)=\max$ of that from \ref{constr}
and lemma~\ref{lem:vconvergLoc},
sequences
$M_i\ni q_{n,i}\to q_n$ from  lemma~\ref{lem:vconvergLoc}.
Let $k'(n)\ge k(n)$.
We regard convergence
$$\lambda_n M_{k'(n)}\GHto (\T_p A, p)=(\RR^{m-2}\times \Cone_x^2,x)$$
then $y_n=h_{\lambda_n}(q_{n, k'(n)})\GHto \R_a^{+\infty} $.
Let sequence $\lambda_n M_{k'(n)}\ni x_n\to x\in \RR^{m-2}\times \Cone_x^2$.
By definition we know that
 $\<v_n,\dir{x}{y}\>=\<\bar v_n, \dir{h_{\lambda_n}(x)}{h_{\lambda_n}(y )}\>$
for any $x,y\in M_n$
hence lemma~\ref{lem:vconvergLoc} implies that
$$\<v,a\>-c\varepsilon_n\le\<\bar v_n,\dir{x_n}{y_n}\>\le\<v,a\>+c\varepsilon_n.$$
Let also regard some
$a_n\GHto \R_a^{+\infty} $ from construction of $f_a$ in \ref{???}.
By lemma \ref{lem:angle}
$\angle (a_n p_n y_n)\to 0$.
Hence
$$\<\bar v_n,\dir{x_n}{a_n}\>\to\<v,a\>.$$
\qeds

\section{Proof of claim~\ref{A'}}\label{sec:codim2}
\subsection{Locally two partitial limits approximately coinside.}
\begin{thm}{Claim}\label{cl:convLocCodim2}
Let sequence of manifolds $M_n\GHto A$,
$R_n$ weakly converges to $R^o$,
point $p\in A^{m-2}$. Then
for every  $r_n\to 0$

$(1/r_n)^{m-2}R^i(B_{r_n}(p))\to R_p^{\mathring{v}^1\wedge \mathring{v}^2\wedge\dots\wedge \mathring{v}^{m-2}}(B_{1}(p))$

\end{thm}

Before the prove we give the following general lemma:

\begin{thm}{Lemma}
Let $p $ be the measure on $K$.
Let we have convergence of Alexandrov spaces $K_i\GHto K$
  and for  any space of the sequence there is smoothing $N_{in}\to  K_i$.
Let for every $i$ measures $p_{in}$ on $N_{in}$ weakly
converge to measure $p_i$ on $K_i$.
Suppose there is a function $k(i)$ so that
for every $k'\ge k$ we
have $N_{i k'(i)}\GHto K$ and
$p_{i k'(i)}$ weakly converges to $p$ (*).
Then $p_i$ weakly converges to $p$.

\end{thm}

\parit{Proof.} Standart argument - diagonal subsequence. \qeds

 \parit{Proof of \ref{cl:convLocCodim2}.}
In our case: set $\lambda_n=1/r_n$,
for homothety $h_\lambda\:A\to \lambda A$ and measure
$p$ on $A$ we denote $p^\lambda$ the pullback (pushforward?) on  $\lambda A$.


$K={\T_p A}$, $p=R_p^{\mathring{v}^1\wedge \mathring{v}^2\wedge\dots\wedge
\mathring{v}^{m-2}}$,  $K_i={\lambda_i A}$,
$N_{in}=\lambda_i M_n$. The function $k(i)$ is
from \ref{cl:convVCone} above.
$p_{in}={\lambda_i}^{m-2}(R_n)^{\lambda_i}$
$p_i={\lambda_i}^{m-2}(R^0)^{\lambda_i}$.
Because of \ref{cl:convVCone} condition (*) of this lemma foolfilled.

By previous lemma we obtain that
${\lambda_i}^{m-2}(R^0)^{\lambda_i}$ weakly converges to measure
$R_p^{\mathring{v}^1\wedge \mathring{v}^2\wedge\dots\wedge
\mathring{v}^{m-2}}$.
Because measure $R_p^{\mathring{v}^1\wedge \mathring{v}^2\wedge\dots\wedge
\mathring{v}^{m-2}}$ is special? as  it is
we have ${\lambda_i}^{m-2}(R^0)^{\lambda_i}(B_1^{\lambda_i}(p))\to R_p^{\mathring{v}^1\wedge \mathring{v}^2\wedge\dots\wedge \mathring{v}^{m-2}}(B_{1}(p))$.
Since $(R^0)^{\lambda_i}(B_1^{\lambda_i}(p))=R^0(B_{1/\lambda_i}^{M_i}(p)$
the claim follows.
\qeds

Let $R^1, R^2$ be two weak limits
of  $R(v^1_n\wedge\dots\wedge v^{m-2}_n )$.

\begin{thm}{Claim}\label{cl:R12loc}


1) Suppose $p\in A^{m-2}$ be so that $R_p^{\mathring{v}^1\wedge \mathring{v}^2\wedge\dots\wedge \mathring{v}^{m-2}}\neq 0$, i.e.
$\mathring{v}^1\wedge \mathring{v}^2\wedge\dots\wedge \mathring{v}^{m-2}\neq 0$
Then
$$R^1(B_{r}(p))/R^2(B_{r}(p))\xrightarrow{r\to 0} 1$$
and
$$\frac{(r_2)^{m-2}R^i(B_{r_1}(p))}{(r_1)^{m-2}R^i(B_{r_2}(p))}\xrightarrow{r_1, r_2\to 0} 1, \qquad i=1, 2,$$
in particular for sufficiently small $r>0$
we have
$$ R^i(B_{2r}(p))<2\cdot 2^{m-2}(R^i(B_r(p)),\qquad i=1, 2.$$

2) Suppose $p\in A'$ be so that $R_p^{\mathring{v}^1\wedge \mathring{v}^2\wedge\dots\wedge \mathring{v}^{m-2}}= 0$.
Then for every $\epsilon>0$ there is $r_0>0$ so that
for $r<r_0$
$$ R_i(B_{r}(p))<\epsilon\cdot r^{m-2},\qquad i=1, 2.$$
\end{thm}

\parit{Proof.} Simple consequence of \ref{cl:convLocCodim2} \qeds

\subsection{Completion of the proof.}
The set $A'$ has $\sigma$-finite $(m-2)$ Hausdorff measure,
hence in what follows we can assume all subsets of $A'$ to
be of finite $(m-2)$ Hausdorff measure.
We subdivide $A'=A'^1\cup A'^0$, here
$$A'^1=\set{x\in A'}{R_p^{\mathring{v}^1\wedge \mathring{v}^2\wedge\dots\wedge
\mathring{v}^{m-2}}(x)\neq 0},
\quad
A'^0=\set{x\in A'}{R_p^{\mathring{v}^1\wedge \mathring{v}^2\wedge\dots\wedge
\mathring{v}^{m-2}}(x)= 0}$$
and prove that
$R^1|_{A'^1}=R^2|_{A'^1}$ and $R^i(A'^0)=0$.

Let $K\subset A'^1$ be a measurable set.
For every $\varepsilon>0$
 by \cite[2.2.2]{federer} there exists open set $W\supset K$ with
$R^i(W)\le R^i(A)+\varepsilon$.
For every $x\in A'^1$  by claim~\ref{cl:R12loc} (1)
   we can choose $r_0(x)$ sufficiently
small
so that for $r<r_0(x)$ we have
$R_1(B_r(x))/R_2(B_r(x))=1\pm \epsilon$ and
$ R^i(B_{2r}(p))<2\cdot 2^{m-2}(R_i(B_r(p)),\qquad i=1, 2$.
The set $F=B_r(x)$ for $x\in A'^1, r<r_0(x), B_r(x)\subset W$ is
$R^i$-adequate by  \cite[2.8.7]{federer}.
So we can choose countable (disjoint?!!) subfamily $G\subset F$, such that
$$R^i(W\setminus\cup G)=0.  $$
Then obviously
we have
$$|R^1(K)-R^2(K)|\le (R^1(K)+R^2(K)+1)\epsilon,$$
this proves that $R^1|_{A'^1}=R^2|_{A'^1}$.

Let $K\subset A'^0$ be a measurable set. We know that
$R^i_-(A')=0$ than for every $\varepsilon>0$
 by \cite[2.2.2]{federer} we find open set $W\supset K$ with
$R^i_-(W)\le \varepsilon$.
For every $x\in A'^0$  by claim~\ref{cl:R12loc} (2)
   we can choose $r_0(x)$ sufficiently
small
so that for $r<r_0(x)$ we have
$ R^i(B_{r}(p))<\epsilon\cdot r^{m-2}.$
We set
$$K^{r_*}
=
\set{x\in K}{r_0(x)\le r_*\ \text{and}\  dist(x,A\setminus W)>r_* },$$
obviously
$K=\cup_{r_*}K^{r_*}$. Let $h^{m-2}(K^{r_*})=c<\infty$.
We can choose approximating covering for Hausdorff measure of $K^{r_*}$:
$$\cup_k B_{r_k}(x_k)\supset K^{r_*}, r_i<r_*\  \text{and}\
\sum_k r_k^{m-2}< c+\varepsilon,$$
then $$\sum_k (R^i B_{r_k}(x_k))<\epsilon\cdot(c+1).$$
Since
$\cup_k B_{r_k}(x_k)\subset W$ we have
$R^i(\cup_k B_{r_k}(x_k))<\epsilon\cdot(c+2)$.
Hence $R^i(K^{r_*})<\epsilon\cdot(c+3)$.

\section{Proof of claim~\ref{A''}}\label{sec:codim3}

\begin{thm} {Lemma}
There is a constant $c$, so that
$$R^i(K)\le c\cdot h_{m-2}(K)$$ for every
measurable set $K$ with $h_{m-2}(K)<\infty$.
\end{thm}

\parit{Proof.}
Follows from \cite[???]{petrunin-SC}.
\qeds

In \cite{BGP} it is proved the following
\begin{thm}{Claim}
$dim_H(A\setminus A(m-2,\de) )\le n-3$
\end{thm}
Hence $h_{m-2}(A'')=0$, the claim~\ref{A''} follows.

\section{APPENDIX}
\subsection{Bochner formula}


We will need few integral formulas based on the Bochner formula.
We made these calculations based on \cite[Chapter II]{lawson-michelsohn}.
First let us state the Bochner formula for function with Dirichlet boundary condition.

\begin{thm}{Proposition}\label{prop:bochner-dirichle-old}
Assume $\Omega$ is a compact domain with smooth boundary $\partial \Omega$ in a Riemannian manifold
and $f$ is a smooth function that vanish on $\partial \Omega$.
Then
\[\int\limits_\Omega |\Delta f|^2
-|\mathrm{Hess}f|^2
-\langle\mathrm{Ric}(\nabla f),\nabla f\rangle
=\int\limits_{\partial\Omega}
H\cdot|\nabla f|^2,\]
where $H$ denotes mean curvature of $\partial \Omega$.
\end{thm}
 
\begin{thm}{Corollary}
Assume $\Omega$ is a compact domain with smooth boundary $\partial \Omega$ in a 2-dimensional Riemannian manifold with nonnegaive curvature
and $f$ is a smooth concave function that vanish on $\partial \Omega$.
Then
\[\int\limits_\Omega 
\det(\mathrm{Hess}f)
\le\pi\cdot\sup_{x\in\partial\Omega}|\nabla_x f|^2.\]

\end{thm}




In this section we give necessary versions of the Bochner formula.
The calculations are based on \cite[Chapter II]{lawson-michelsohn}.
We use Riemannian metric to identify differential forms and multivector fields on $M$.
Therefore the statement about differential forms can be also formulated in terms of multivector fields and the other way around.


Let $M$ be a Riemannian manifold.
Denote by $\nabla$ the Levi-Cevitta connection on $M$.
The bundle $\LT M$ of multivectors over $M$ is equipped with Clifford product, denoted by $\,\bullet \,$.
We will denote by $e_i$ is an orthonormal frame at a point; the following definitions will not depend on its choice.

\parbf{Laplasians.}
The Dirac operator on differential forms forms will be denoted by $D$;
it is defined as
\[D=\sum_i e_i\bullet \nabla_{e_i}.\]
Its square 
\[D^2=\sum_i e_i\bullet e_j\bullet \nabla^2_{e_i,e_j}\]
is called Hodge laplacian.

The Dirac operator if \emph{formally self-adjoint}, in particular,
\[\int_M \langle D^2\phi,\psi\rangle=\int_M \langle D\phi,D\psi\rangle\]
for any two vector fields $\phi$ and $\psi$ with compact support.

Further, define the connection laplacian
\[\nabla^*\nabla\phi =-\sum_i\nabla^2_{e_i,e_i}\phi\]
and the gradient
\[\nabla \phi=\sum e_i\otimes \nabla_{e_i}\phi.\]

For the connection laplacian we also have the identity
\[\int_M \langle \nabla^*\nabla\phi,\psi\rangle
=
\int_M \langle \nabla\phi,\nabla\psi\rangle.\]
If $\phi$ anad $\psi$ have support in the domain where the frame is defined, then the right handside can be written as 
\[\int_M \langle \nabla\phi,\nabla\psi\rangle=\sum_i\int_M\langle \nabla_{e_i}\phi,\nabla_{e_i}\psi\rangle;\]
using the a partition of unity, one can use the latter expression to redefine the left hand side. 

\parbf{Bochner formula.}
The difference $D^2-\nabla^*\nabla$ between two laplasians described above is a 0-order differential operator which can be written in terms of curvature.
For a vector field $v$, the formula is 
\[D^2v-\nabla^*\nabla v=\Ric(v).\]

Using the identities above, the formula can be written in an integral form
\[\int_M \langle Dv,Dw\rangle -\langle \nabla v,\nabla w\rangle=\langle \Ric(v),w\rangle\]
In particular, if $w=\phi \cdot v$ for a smooth function $\phi$;
we get
\[\int_M \langle Dv,D(\phi\cdot v)\rangle -\langle \nabla v,\nabla(\phi\cdot v) \rangle=\phi\cdot\langle \Ric(v), v\rangle,\]
or, equivalently
\[\int_M \phi\cdot(\langle Dv,D v\rangle -\langle \nabla v,\nabla v \rangle)
+
\int_M (\langle Dv,\nabla \phi \bullet v\rangle -\langle \nabla v,\nabla v \rangle)
=
\phi\cdot\langle \Ric(v), v\rangle,\]
\parbf{Relative formulas.}
For a domain $\Omega$ with boundary $\partial \Omega$, the formula above takes form
\[\int_\Omega (\langle D^2\phi,\psi\rangle- \langle D\phi,D\psi\rangle)
=
\int_{\partial \Omega}\langle \nu\bullet D\phi,\psi\rangle,\]
where $\nu$ is the outer normal field on $\partial \Omega$.

The square $D^2$ of the Dirac operator is called Hodge laplacian.


For a domain $\Omega$ with boundary $\partial \Omega$, the formula above takes form
\[\int_\Omega (\langle \nabla^*\nabla\phi,\psi\rangle-\langle \nabla\phi,\nabla\psi\rangle)
=
\int_{\partial \Omega}\langle \nabla_\nu \phi,\psi\rangle,\]
where $\nu$ is the outer normal field on $\partial \Omega$.

The Bochner's identity for a vector field $u$ can be written as
\[D^2 u-\nabla^*\nabla u=\Ric(u).\]
relates the Dirac's laplasian $D^2$ and the connection laplasian $\nabla^*\nabla$

In this section we will write an integral version of the Bochner formula \cite[8.3]{lawson-michelsohn}
\[D^2-\nabla^*\nabla=\Ric\]
in such a way that each term has geometric meaning.


Further $\nabla^*\nabla=-\sum_i\nabla^2_{e_i,e_i}$ is the connection Laplacian.



Let $M$ be Riemannian $m$-manifold and $f\:M\to\R$ be a smooth function without critical points on an open domain $\Omega\i M$.
Assume $\phi\:\Omega\to\R$ be a smooth function with compact support.
Set $u=\nabla f/|\nabla f|$.
Let us define $\Int_f(x)$ (or just $\Int$) to be scalar curvature of the level set $L_x=f^{-1}(f(x))$ at $x\in L_x\i M$.
Set
\begin{enumerate}
 \item $\kappa_1(x)\le\kappa_2(x)\le\dots\le\kappa_{m-1}(x)$ the principle curvatures of $L_x$ at $x$;
 \item $H_f=H_f(x)=\kappa_1+\kappa_2+\dots+\kappa_{m-1}$ is mean curvature of $L_x$ at $x$
\item $G_f=G_f(x)=2\sum_{i<j}\kappa_i\cdot\kappa_j$ is the extrinsic term
 in the Gauss formula for $\Int_f(x)$. 
\end{enumerate}

Let us define the strange curvature as
\[\Str(u)=\Sc-\<\Ric(u),u\>,\]
where $\Sc$ and $\Ric$ denote scalar and Ricci curvature correspondingly.

\begin{thm}{Bochner's formula}\label{thm:bochner-formula}
Let $M$ be an $m$-dimensional Riemannian manifold,
$f\:M\to\R$ be a smooth function without critical points on an open domain $\Omega\subset M$ and $u=\nabla f/|\nabla f|$.
Assume $\phi\:\Omega\to\R$ is a smooth function with compact support.
Then 
$$\int\limits_\Omega \phi\cdot \Str(u)
=
\int\limits_\Omega \l[H\cdot\<u,\nabla\phi\>- \<\nabla\phi,\nabla_u u\> \r]+
\int\limits_\Omega \phi\cdot \Int_f.
\eqlbl{Bochner}$$
\end{thm}

In the proof of the main result we will 


\parit{Proof.}
Assume $b_1,\dots, b_m$ is an orthonormal frame such that $b_m=u$, 
then 
\[\Sc-2\cdot \<\Ric(u),u\>=2\cdot \sum_{i<j<m} \sec(b_i\wedge b_j).\] 
Therefore the Gauss formula can be written as
\[
\begin{aligned}
\Int_f&=G_f+\Sc-2\cdot \<\Ric(u),u\>=
\\
&=G_f+\Str(u)+ \<\Ric(u),u\>.
\end{aligned}
\eqlbl{eq:gauss}
\]

We can choose the frame $b_i$ so that $b_m=u$ and such that $b_i$ points in the principle directions of the level set $L_x$ for $i<m$.
Note that $\<\nabla_u u,u\>=0$, therefore
\begin{align*}
Du&=\sum_{i} b_i\bullet  \nabla_{b_i}u=
\\
&=\sum_{i<m}\kappa_i\cdot  b_i\bullet  b_i+u\bullet  \nabla_{u}u=
\\
&=
\sum_{i=1}^{m}\kappa_i+u\wedge\nabla_{u}u=H+u\wedge\nabla_{u}u,
\end{align*}
here ``$\,\bullet \,$'' denotes the Clifford multiplication.
Applying again that $\<\nabla_u u,u\>=0$, we get that
$$ \langle Du,Du \rangle=
\l(\sum_{i<m}\kappa_i\r)^2+|\nabla_{u}u|^2=H_f^2+|\nabla_{u}u|^2.$$
On the other hand
$$\nabla u=\sum_{i<m}\kappa_i\cdot b_i\otimes b_i+\nabla_u u\otimes u,$$
hence
$$\langle\nabla u,\nabla u\rangle =
\sum_{i<m}\kappa_i^2+|\nabla_{u}u|^2.$$

Therefore
$$\langle D u,D u\rangle-\langle \nabla u,\nabla u \rangle =2\cdot\sum_{i<j}\kappa_i\cdot\kappa_j=G_f.$$

Further,
\begin{align*}
\int\limits_\Omega\phi\cdot\l[\<D u,D u\>-\<D^2 u, u\>\r]
&=
\int\limits_\Omega\<\nabla\phi\bullet u,D u\>
=
\\
&=
-\int\limits_\Omega\l[H\cdot\<\nabla\phi,u\>- \<\nabla\phi,\nabla_u u\> \r],
\end{align*}
here ``$\bullet $'' denotes Kliford's multiplication.

Since $| u|\equiv 1$, we have $\<\nabla_{\nabla\phi}  u, u\>=0$.
Therefore
$$\int\limits_\Omega\phi\cdot\l[\<\nabla u,\nabla u\>-\<\nabla^*\nabla u, u\>\r]
=
\int\limits_\Omega\<\nabla_{\nabla\phi}  u, u\>=0.$$

Let us write Bochner formula \cite[8.3]{lawson-michelsohn} for field $u$:
$$D^2u-\nabla^*\nabla u=\Ric(u);$$
in particular, 
$$\phi\cdot \<D^2u,u\>-\phi\cdot \<\nabla^*\nabla u,u\>=\phi\cdot \<\Ric(u),u\>.\eqlbl{eq:prebochner}$$
Integrating \ref{eq:prebochner}, using the given calculations we get that
\begin{align*}
\int\limits_\Omega \phi \cdot G_f
&=\int\limits_\Omega\phi\cdot(\<D u,D u\>-\<\nabla u,\nabla u\>)
=
\\
&=\int\limits_\Omega \phi\cdot \Ric(u,u) 
-
\int\limits_\Omega H_f\cdot\<u,\nabla\phi\>- \<\nabla\phi,\nabla_u u\> .
\end{align*}

It remains to apply the Gauss formula \ref{eq:gauss}.
\qeds

 
 
 
 
\subsection{ Convergent vector fields}

\begin{thm}{Proposition}
Let $f_n$ be a sequence of $C^1$-smooth $\lambda$-concave functions
and
$f_n\to f$.
Suppose that for any compact $K$ there is $c>0$ such that
$d_x(\xi)+d_x(\eta)\le c(\pi-\angle(\xi,\eta))$
for $x\in K$ and $\xi, \eta\in \Sigma_x$.
Then $f_n$ roughly $C^1$-converge to $f$. 
\end{thm}

Fix some convergent directions $\xi_n\to\xi_0$ and
corresponding sequences of points
$x_n\to x_0$, $y_n\to y_0$
such that  $\xi_i=\uparrow_{x_i}^{y_i}$ for $i\in\mathbb N\cup\{0\}$.
There exists $z_0 \in U$ such that for $\eta=\uparrow_x^z$ 
we have
$\angle(\xi,\eta)\ge \pi-3\delta(\xi)/2$.
We choose some sequence 
$U_n\ni z_n\to z_0$
and denote
$\eta_n=\uparrow_{x_n}^{z_n}$,
then for large $n$ we have $\angle(\xi_n,\eta_n)\ge \pi-2\delta(\xi)$.
Let us denote by $V_x$ be the set of points that can be joint with $x$
by a unique geodesic. For $i\in\mathbb N\cup\{0\}$ and
$q\in V_{x_i}$ we have $|d_{x_i} \dist_q(\xi_i)+d_{x_i} \dist_q(\eta_i)|\le 4\de$ by first
variation formula.
Since Hausdorff dimension of $A\setminus V_x\le n-1$ we also obtain
that
$|d_{x_i} \widetilde\dist_{p_i,r}(\xi_i)+d_{x_i} \widetilde\dist_{p_i,r}(\eta_i)|\le 4\de(\xi) (*)$.
Since functions
$\widetilde\dist_{p_i,r}$ are uniformly semiconcave converging to
$\widetilde\dist_{p,r}$ we have
$\liminf d_{x_n}\widetilde\dist_{p_n,r}(\xi_n)\ge d_{x_0}\widetilde\dist_{p_0,r}(\xi_0)$ and $\liminf d_{x_n}\widetilde\dist_{p_n,r}(\eta_n)\ge d_{x_0}\widetilde\dist_{p_0,r}(\eta_0)$.
Together with $(*)$ this gives
\begin{align*}
 d_{x_0}\widetilde\dist_{p_0,r}(\xi_0)+8\de\ge
\limsup d_{x_n}\widetilde\dist_{p_n,r}(\xi_n)\ge\\
\ge
\liminf d_{x_n}\widetilde\dist_{p_n,r}(\xi_n)\ge d_{x_0}\widetilde\dist_{p_0,r}(\xi_0)-8\de.
\end{align*}





\begin{thm}{Proposition}
Any test sequence roughly $C^1$-converges.
\end{thm}
It is sufficient to show  that functions
$\widetilde{\dist}_{p_n,r}(t) =\oint_{B_r(p_n)} \dist_{s}(t)d\vol(s)$
roughly $C^1$-converge to $\widetilde{\dist}_{p_0,r}(t)$.

Fix some convergent directions $\xi_n\to\xi_0$ and
corresponding sequences of points
$x_n\to x_0$, $y_n\to y_0$
such that  $\xi_i=\uparrow_{x_i}^{y_i}$ for $i\in\mathbb N\cup\{0\}$.
There exists $z_0 \in U$ such that for $\eta=\uparrow_x^z$ 
we have
$\angle(\xi,\eta)\ge \pi-3\delta(\xi)/2$.
We choose some sequence 
$U_n\ni z_n\to z_0$
and denote
$\eta_n=\uparrow_{x_n}^{z_n}$,
then for large $n$ we have $\angle(\xi_n,\eta_n)\ge \pi-2\delta(\xi)$.
Let us denote by $V_x$ be the set of points that can be joint with $x$
by a unique geodesic. For $i\in\mathbb N\cup\{0\}$ and
$q\in V_{x_i}$ we have $|d_{x_i} \dist_q(\xi_i)+d_{x_i} \dist_q(\eta_i)|\le 4\de$ by first
variation formula.
Since Hausdorff dimension of $A\setminus V_x\le n-1$ we also obtain
that
$|d_{x_i} \widetilde\dist_{p_i,r}(\xi_i)+d_{x_i} \widetilde\dist_{p_i,r}(\eta_i)|\le 4\de(\xi) (*)$.
Since functions
$\widetilde\dist_{p_i,r}$ are uniformly semiconcave converging to
$\widetilde\dist_{p,r}$ we have
$\liminf d_{x_n}\widetilde\dist_{p_n,r}(\xi_n)\ge d_{x_0}\widetilde\dist_{p_0,r}(\xi_0)$ and $\liminf d_{x_n}\widetilde\dist_{p_n,r}(\eta_n)\ge d_{x_0}\widetilde\dist_{p_0,r}(\eta_0)$.
Together with $(*)$ this gives
\begin{align*}
 d_{x_0}\widetilde\dist_{p_0,r}(\xi_0)+8\de\ge
\limsup d_{x_n}\widetilde\dist_{p_n,r}(\xi_n)\ge\\
\ge
\liminf d_{x_n}\widetilde\dist_{p_n,r}(\xi_n)\ge d_{x_0}\widetilde\dist_{p_0,r}(\xi_0)-8\de.
\end{align*}

\begin{thm}{Definition}
We say that sequence of vector fields $v_n$ on $U_n$
converges if for any compact $K$ there is $c>0$ such that
for any $x\in K$ and convergent sequence of directions $\xi_n\to\xi_0\in \Sigma_x$ we have
$\limsup \<v_n,\xi_n\>-\liminf \<v_n,\xi_n\>\le c\delta(\xi)$.



\end{thm}

\begin{thm}{Remark}
A sequence $f_n$ of $C^1$ functions roughly $C^1$-converges
iff $\nabla f_n$ converges.

\end{thm}


\begin{thm}{Claim}\label{lem:scalprod}
Let vector fields $v_n, w_n$ converge, then for every compact
set $K\subset A$ there is a constant $c>0$ so that
$\<v_n, w_n\>$ $c\de$-converges on $A^\de\cap K$.
\end{thm}

{\parit proof}
For $x\in A^\delta$ fix $delta$ orthogonal basis





In particular we have
\begin{thm}{Corollary}\label{cor:cdeltacoeff}
Let vector fields $v_n^1,\dots, v_n^m, v_n$ be
uniformly bounded, converging
 and $m$-sets of vectors
$v_n^1,\dots, v_n^m$ be uniformly linearly independent,
i.e. $\det(\<v_n^i, v_n^j\>)>c_0$
for some constant $c_0$.
Then for coefficient of decomposition
of $v_n=\alpha_n^1 v_n^1+\dots+\alpha_n^m v_n^m$
 and for every compact
set $K\subset A$ there is a constant $c>0$ so that
$\alpha_n^i$ $c\de$-converges on $A^\de\cap K$.
\end{thm}

\section{DC in domain of $\R^n$.}\label{sec:DC}

In this section we give simplest calculations for  a class of function sequences on domain in $\RR^m$ 
 that arise as coordinate expressions of  convergent sequences
 of concave functions.
 We follow Perelman generalizing theory (\cite{}) for sequences
 of $DC$-functions.
 
Let fix $\Omega\i \RR^m$  an open bounded domain
and
 $S_\Omega\subset\Omega\subset\R^m $ subset of
zero $(m-1)$-Hausdorff measure $h_{m-1}(S_\Omega)=0$.
%, this will be referred as a subset of {\it singularities}.


\subsubsection{Functions}
We denote by  $\op{DC_0}(\Omega,S_\Omega)$ a class of
 $\DC$-function, such  that are continuously differentiable on
$\Omega\setminus S_\Omega$.


We denote by $\op{C_0}(\Omega,S_\Omega)$ a set of bounded functions
$f\:\Omega\to\RR$, which are continuous on $\Omega\setminus S_\Omega$.


We recall that a measurable function $f\:\RR^m\to \RR$ is of
bounded variation
 ($f\in\op{BV}(\Omega,\RR)$) if there is  $c>0$,
such that  the inequality
$$\int\limits_\Omega f\cdot D_i\phi
\le
 c\cdot\sup_{x\in\Omega}|\phi(x)|$$
holds
for any $i$ and any smooth function $\phi\:\Omega\to \RR$ with compact support.



We denote by $\op{BV_0}(\Omega,S_\Omega)$ a set of bounded functions
$f\:\Omega\to\RR$, which are continuous on $\Omega\setminus S_\Omega$.


We say that a signed Radon measure $\mu$ is in
$\aleph_0(\Omega,S_\Omega)$ if $\mu(S_\Omega)=0$.



\subsubsection{Sequences}
Let $h_n:\Omega\to\R$ be a sequence of continuous functions that
converges to the function $h_0\in DC(\Omega)$.
We will write 
$$(h_n,h_0)\in \op{DC^{seq}}(\Omega)$$
if
there are sequences of concave functions $f_n,g_n$ converging to
$f_0,g_0$, such that 
$h_0=f_0-g_0$.

Suppose in addition that 
$f_n,g_n$ are  continuously differentiable in
$\Omega\setminus S_\Omega$ for $n=0,1,2,\dots$.
In this case we will write 
$$(h_n,h)\in \op{DC_0^{seq}}(\Omega,S_\Omega).$$



Let $h_n:\Omega\to\R$ be a sequence of
uniformly bounded continuous 
functions
% with uniformly bounded supports 
that
converges  to the function $h$
 on
$\Omega\setminus S_\Omega$.
 and  $h\in \op{C_0}(\Omega,S_\Omega)$. 
In this case we will write 
$$(h_n,h)\in \op{C_0^{seq}}(\Omega,S_\Omega).$$

Let $h_n:\Omega\to\R$ be a sequence of uniformly bounded functions
with uniformly bounded variations
that
almost everywhere converges  to a function $h_0\in\op{BV}(\Omega)$.
 Suppose that for any $i$ there are
 almost everywhere converging  sequences
 $f_n, g_n$ with uniformly bounded variations
  such that 
 $h_n=f_n- g_n$ and
 $\partial_i f_n$,
 $\partial_i  g_n$ are
 nonnegative Radon measures.
  In this case we will write 
$$(h_n,h)\in \op{BV^{seq}}(\Omega,S_\Omega).$$

If in addition $f_n, g_n\in\op{C_0}(\Omega,S_\Omega)$ for
$n\ge 0$ we
will write 
$$(h_n,h_0)\in \op{BV_0^{seq}}(\Omega,S_\Omega).$$




%Let $h_n:\Omega\to\R$ be a sequence of $C^0$-functions with uniformly bounded variations and  $h:\Omega\to\R$ be a  function so that that $h_n$C-converges  to the function $h$on the set where $h$ is continuous.  In this case we will write $$(h_n,h)\in \op{CBV^{seq}}(\Omega).$$

Let $\mu_n$ be
  a sequence of uniformly bounded
signed Radon measure  such that
$\mu_n=\mu_n^+-\mu_n^-$
for some
sequences of nonnegative Radon measures
that converge weakly to measures $\mu^+,
\mu^-\in\aleph_0(\Omega,S_\Omega)$.
Let $\mu=\mu^+-\mu^-$,
we will write
$$(\mu_n,\mu)\in\aleph_0^{seq}(\Omega,S_\Omega).$$


\subsubsection{Operations}
It is known that
partial derivative of $DC$ functions are of bounded variation
and the second are signed Radon measures, 
the corresponding properties for  sequences 
are given in the following lemma.
\begin{thm}{Lemma}\label{thm-D}
(1) If     $( f_n, f)\in\op{DC_0^{seq}}(\Omega,S_\Omega)$ 
then for every $i$ the partial derivative (defined almost everywhere)
we have
$$\biggl(\frac{\partial f_n}{\partial x_n^i}, \frac{\partial f}{\partial x^i}\biggr)\in\op{BV_0^{seq}}(\Omega,S_\Omega).$$
(2) 
 If     $( f_n, f)\in\op{BV_0^{seq}}(\Omega,S_\Omega)$ 
then for every $i$ for the partial derivative (defined as measures)
we have
$$\biggl(\frac{\partial f_n}{\partial x_n^i}, \frac{\partial f}{\partial x^i}\biggr)\in\aleph_0^{seq}(\Omega,S_\Omega).$$

\end{thm}
We can  multiply $C_0$ functions by $\aleph_0$ measures and
obtain again $\aleph_0$ measures, the corresponding statement
for sequences is the following:
\begin{thm}{Lemma}\label{thm-CM}
Let $f_n, f\in\op{C_0^{seq}}(\Omega,S_\Omega)$,
measures $g_n, g\in\aleph_0(\Omega,S_\Omega)$,
then
$$(f_n\cdot g_n,f\cdot g)\in\aleph_0^{seq}(\Omega,S_\Omega).$$
\end{thm}
The class of $BV_0$ and $C_0$ functions and sequences is closed under analytic expression as
says the following theorem:
\begin{thm}{Lemma}\label{thm-A}
Let sequences $(f^1_n,f^1),\dots,(f^k_n, f^k) \in \op{BV_0^{seq}(C_0^{seq})}(\Omega,S_\Omega)$,
and $A$ be  $C^\infty$-function well defined in a small neighborhood of
$(f^1,\dots,f^k)(\Omega)$.
Then $$(A(f^1_n,\dots,f^k_n), A(f^1,\dots,f^k))\in\op{BV_0^{seq}(C_0^{seq})}(\Omega,S_\Omega).$$
\end{thm}


\section{ Functions }
 \begin{thm}{Proposition}\label{NiceFunctions}
 Let   
	$M_n\in\M_{\ge -1}^n$,
	$M_n\GHto A$ , $ x\in A^0$, $M_n\ni x_n\to x$
	and $U\ni x$
	be a good domain for this convergence.
 For any
  $v\in T_xA$,
 $\varepsilon>0$
 there is  $r>0$
 and a
  sequence of smooth strictly concave functions
 $f_n: B_r(x_n)\to \R$, roughly $C^1$-convergent to a function 
 $ f:B_r(x)\to\R$,
such that $d_xf$  is arbitrary close to $ <v,\cdot>$.
 \end{thm}
  
 
\section{DC-coordinates for Alexandrov space}

\subsection{Existense of Nice common chart}\label{NiceChartProof}
Here should be the proof of 
Proposition~\ref{Prop:chart}.
\subsection{Metric tensor  in coordinates}\label{sec:metr-con}

\begin{thm}{Theorem (Metric tensor in Nice common chart)}\label{metricBV}
Let
$\mathfrak X_n:U_n\to\Omega$,
$\mathfrak X:U\to\Omega$ be a Nice common chart.
Denote by $g_{ij,n}$ coordinates of metric tensor in this chart
and by $g^{ij}_n$ coordinates of the inverse matrix. 
Then
$g_{ij,n}\in BV_0^{seq}(\Omega)$
and
$g^{ij}_n\in BV_0^{seq}(\Omega)$.
Moreover, $\det(g_{ij,n})$ and $\det(g^{ij}_n)$ are bounded and bounded away from 0.

\end{thm}




For the proof of this theorem we define a class of $DC$-convergent sequences with some additional restriction,
that allow to take limits of derivatives.
 this will be applied to sequences of distance functions
in coordinates.
Let $h_n:\Omega\to\R$ be a sequence of $C^0$ functions that
converges to the function $h_0\in DC(\Omega)$.
We shall write
$(h_n, h_0)\in DC_+^{seq}(\Omega)$
if
there are sequences of concave functions $f_n,g_n\in C^0(\Omega)$ converging to
$f_0,g_0$, such that 
$h_0=f_0-g_0$ and in addition $f_n$ and $g_n$
are  differentiable at all points
from $\Omega\setminus S_\Omega$ where $h_n$ is
differentiable for $n=0,1,2\dots$.

 
 
\begin{thm}{Lemma}\label{DC+}
	Let
$(h_n, h_0)\in DC_+^{seq}(\Omega)$
and $h_n$ be
differentiable at
$x_n$ for $n=0,1,2\dots$.
Then $d_{x_n}h_n\to d_{x_0}h_0$. 


\end{thm}

{\it Proof of the Theorem~\ref{metricBV} }
As it is shown in \cite{PerDC} Section 4.2
that if $\Omega$ is sufficiently small chart around a regular point,
for appropriate $p_i\in U$ in general position, $N=m(m+1)/2$ and rational function $Q$ components of metric tensor can be expressed as:

$$g_{ij}=Q\left( \frac{\partial (d_{p_1}\circ \X^{-1})}{\partial x_1},\dots,
\frac{\partial (d_{p_1}\circ \X^{-1})}{\partial x_m},\dots,
\frac{\partial (d_{p_N}\circ \X^{-1})}{\partial x_1},\dots,
\frac{\partial (d_{p_N}\circ \X^{-1})}{\partial x_m}\right)
$$

$$g_{ij}^n=Q\left( \frac{\partial (d_{p_1^n}\circ \X_n^{-1})}{\partial x_1},\dots,
\frac{\partial (d_{p_1^n}\circ \X_n^{-1})}{\partial x_m},\dots,
\frac{\partial (d_{p_N^n}\circ \X_n^{-1})}{\partial x_1},\dots,
\frac{\partial (d_{p_N^n}\circ \X_n^{-1})}{\partial x_m}\right).
$$

It can be obtained using arguments similar to \cite{PerDC} Section 3, that 
$(d_{p_i^n}\circ \X_n^{-1}, d_{p_i}\circ \X^{-1})\in DC_+^{seq}(\Omega)$,
then in particular
$(g^{ij}_n, g^{ij})\in BV^{seq}(\Omega)$.
Now it suffices to show that for any sequence of points
 $ q_n\to q\in \Omega\setminus S_\Omega $ the sequence 
$g_{ij}^n(q_n)\to g_{ij}(q)$. 
For a point $p$ in $U$ or $U_n$ we denote by $V_p$ points
with only one minimizing geodesic to $p$.
Let note that
$d_{p}\circ \X^{-1}$ is $C^1$ on $\X(V_p)$
in a weak sense of \cite{OS} (Section 1.5) and
$d_{p}\circ \X_n^{-1}$ is $C^1$ on $\X_n(V_p)$.



Then by Lemma~\ref{DC+} for $U_n\ni p^n\to p\in U$ and
$q_n\in\X(V_{p_n})$ such that $q_n\to q\in \X(V_p)$
we have
$\frac{\partial (d_{p^n}\circ \X_n^{-1})}{\partial x_i}(q_n)\to
\frac{\partial (d_{p}\circ \X^{-1})}{\partial x_i}(q)$.

Now for a given sequence of points $ q_n\to q\in \Omega\setminus S_\Omega $ 
we can choose points $p_i^n$ and $p_i$
in the expression for metric tensor
 in such a way 
that
$q_n\in\X(V_{p_i^n})$ and $q\in \X(V_{p_i})$.

\qeds





\begin{thebibliography}{52}
\bibitem[AKP]{AKP} Alexander, S.; Kapovitch, V.; Petrunin A.
An optimal lower curvature bound for convex hypersurfaces in Riemannian manifolds.


\bibitem[AZ]{AZ}Aleksandrov, A.D. and Zalgaller, V.A. [1962]: Two-dimensional manifolds of bounded curvature.
Tr. Mat. Inst. Steklova 63. English transl.: Intrinsic geometry of surfaces. Trans!. Math. Monographs
15, Am. Math. Soc. (1967), Zb1.l22,170


\bibitem[BN]{BN}Berestovskij, V.N.; Nikolaev, I.G.
Multidimensional generalized Riemannian spaces. (English. Russian original)
Geometry. IV. Non-regular Riemannian geometry. Encycl. Math. Sci. 70, 168-243 (1993); translation from Itogi Nauki Tekh., Ser. Sovrem. Probl. Mat., Fundam. Napravleniya 70, 190-272 (1992).


\bibitem[BGP]{BGP} Burago, Yu.; Gromov, M.; Perelman, G., \textit{A. D. Aleksandrov spaces
with curvatures bounded below.} (Russian)  Uspekhi Mat. Nauk  47  (1992),  no.
2(284), 3--51, 222;   translation in  Russian Math. Surveys  47  (1992),  no. 2, 1--58
\bibitem[Federer]{federer}
Federer, Herbert (1969), \textit{Geometric measure theory}, series Die Grundlehren der mathematischen Wissenschaften, Band 153, New York: Springer-Verlag New York Inc., pp. xiv+676, ISBN 978-3-540-60656-7, MR 0257325

\bibitem[G]{G}
Nicola Gigli,
Riemann curvature tensor on RCD spaces and possible applications https://arxiv.org/pdf/1902.02282.pdf

\bibitem[G1]{G1}
Nicola Gigli,
Non-smooth differential geometry, to appear on Mem. Amer. Math. Soc..
Preprint, arXiv:1407.0809, 2014.

\bibitem[GMS]{GMS}
Nicola Gigli ? Andrea Mondino ? Giuseppe Savar´e ?
Convergence of pointed non-compact metric measure spaces
and stability of Ricci curvature bounds and heat flows



\bibitem[H]{H} Bang-Xian Han, 
Ricci tensor on $RCD^*(K, N)$ spaces,



\bibitem[OS] {OS}   Otsu, Y.,  Shioya, T. (1994)
The Riemannian structure of Alexandrov spaces. Journal of Differential Geometry, 39(3), 629-658

\bibitem[LawMich]{lawson-michelsohn}Lawson, H. and Michelsohn, M.
\textit{Spin geometry}

\bibitem[L] {L} John Lott,\textit{
Ricci measure for some singular Riemannian metrics}
Math. Ann. (2016) 365:449-471


\bibitem[Per91]{PerStab} G. Perelman. \textit{Alexandrov spaces with curvatures bounded from below II. preprint}, (1991).
\bibitem[PerDC]{PerDC} G. Perelman. \textit{DC structure on Alexandrov space}

\bibitem[Pet-QG]{petrunin-QG} Petrunin, A., \textit{Applications of quasigeodesics
and gradient curves.}

\bibitem[Pet-SC]{petrunin-SC} Petrunin, A., \textit{An upper bound for curvature integral.} April 2009St Petersburg Mathematical Journal 20(2):255-265
\bibitem[Pet-Conc]{petrunin-conc} Petrunin, A., \textit{Semiconcave functions in Alexandrov’s geometry}




\bibitem[R]{R} Reshetnyak \textit{ Two-dimensional manifolds of bounded curvature} // Geometry-4: Non-Regular Riemannian Geometry. — Berlin, 1993. — P. 3-163, 245-250. — (Encyclopaedia of Math. Sci.; 70). 

\bibitem[St]{St} Karl-Theodor Sturm
Ricci Tensor for Diffusion Operators and Curvature-Dimension Inequalities under Conformal Transformations and Time Changes
https://arxiv.org/abs/1401.0687
 


\bibitem[K]{KapStab}Vitali Kapovitch \textit{Perelman's Stability Theorem.}
\end{thebibliography}
 \end{document}

Ðåøåòíÿê Þ. Ã., Äâóìåðíûå ìíîãîîáðàçèÿ îãðàíè÷åííîé êðèâèçíû, Èòîãè íàóêè è òåõíèêè. Ñîâðåìåííûå ïðîáëåìû ìàòåìàòèêè. Ôóíäàìåíòàëüíûå íàïðàâëåíèÿ. Ò. 70 (1989), 7-189.

Íàó÷íàÿ áèáëèîòåêà äèññåðòàöèé è àâòîðåôåðàòîâ disserCat http://www.dissercat.com/content/skleivanie-rimanovykh-mnogoobrazii-s-kraem#ixzz3PZY6zKdR

 
