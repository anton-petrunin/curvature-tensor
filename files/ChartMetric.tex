
\section{Common chart and metric tensor}

The main tool for the proof
of Lemma~\ref{A^0} for dimension 3
is DC-calculus in  charts near regular point
with common domain for the smoothing sequence and the
limit space. For DC-calculus in domain in $\R^n$
we refer Appendix (Section~\ref{sec:DC}).
In the next Proposition we 
state the existence of a common chart with 
properties that allow to apply these calculus.
The proof is given in \ref{NiceChartProof}.

\begin{thm}{ Proposition}\label{Prop:chart}
	There exists $\delta_0>0$, such that for $\delta<\delta_0$
	the following hold.	
	Let   
	$M_n\in\M_{\ge -1}^m$,
	$M_n\GHto A$ , $ p\in A^\delta$
	and $B_r (p)$
	be a good domain for this convergence.
	For some open
	$\Omega\subset \R^m$, open 
	neighborhood $U\subset B_r (p)$ of $p$
	and $U_n\GHto U$ 
	there exist a
	sequence of
	diffeomorphisms $\mathfrak X_n=(\x_{1,n},\dots,\x_{m,n}):U_n\to\Omega$
	converging to a homeomorphism 	$\mathfrak X=(\x_{1},\dots,\x_{m}):U\to\Omega$ with 
	the following properties.
	
	\begin{enumerate}[label=\alph*. ]
	
	%\addtocounter{enumi}{2}
	\item\label{obtuse}

 Every coordinate sequence $\x_{i,n}$ is a sequence of smooth concave functions roughly $C^1$-converging to $\x_i$. Gradients
	$\nabla\x_{i,n}$ and $\nabla\x_{j,n}$ form strictly obtuse angles,
	bounded from $\pi$ and $\pi/2$ for
	$i\neq j$. 
	\item $\mathfrak X_n, \mathfrak X$ are bi-Lipschitz 
	with uniform  bi-Lipschitz 
	constant
	
\item\label{metric} 
	There exists a continuous Riemannian metric
	$g$ on $\Omega\setminus S_\Omega$ which locally 
	realizes distance on $U$, where
	$S_\Omega=\mathfrak X(U\cap A\setminus A^0)$.
	This metric tensor 
	(defined almost everywhere on $\Omega$)
	is of bounded variation
	on $\Omega$. 
	
\item\label{metricseq}
	Let $g_{ij,n}$ be coordinates of metric tensor of $M_n$ 
	in the chart $\X_n$. 
	Then
	$(g_{ij,n}, g_{ij})\in BV_0^{seq}(\Omega, S_\Omega)$.
	Moreover, $\det(g_{ij,n})$ is bounded and bounded away from 0.
	
	
	\item\label{funktioninchart}
	For any
	sequence
	of concave smooth functions
	$f_n:B_{r_n}(p_n)\to\R$ 
	which roughly $C^1$-converges to  $f:B_r(p )\to\R$ we have for its coordinate
	expression
	$(f_n\circ\X_n^{-1}, f\circ\X^{-1})\in DC_0^{seq}(\Omega, S_\Omega)$
	
	\end{enumerate}
\end{thm}

\begin{thm}{Definition}
	We call the sequence of
	diffeomorphisms $\mathfrak X_n:U_n\to\Omega$
	together with a homeomorphism 	$\mathfrak X:U\to\Omega$
	defined in
Proposition~\ref{Prop:chart}
a {\emph Nice common chart} around $p$.

\end{thm}
