\part{Formulations}

In this part we give definitions necessary for a precise formulation 
of the main theorem.
For simplicity  we will always assume that the lower
curvature bound in main theorem is  $-1$;
applying rescaling, we can get the general curvature bound.

We denote by
$\Al^m$ the class of m-dimensional Alexandrov's spaces
with curvature $\ge -1$.

Suppose a sequence $A_n\in \Al^m$ Gromov--Hausdorff converges without collapse to some
$A\in \Al^m$.
Denote by $a_n\:A_n\to A$ the Hausdorff approximations.
By Perelman stability theorem (see \cite{PerStab}, \cite{KapStab}) we can assume that $a_n$ is a homeomorphism for all sufficiently large $n$.
Everywhere further by GH-convergence we mean 
Gromov--Hausdorff converges without collapse and the Hausdorff approximations are assumed to be homeomorphisms.


We say that $A\in \Al^m$ is \emph{smoothable}
if it can be presented as a Gromov--Hausdorff limit by a non-collapsing sequence of Riemannian manifolds $M_n$ with $\sec M_n\ge-1$.
Given a smoothable Alexandrov space $A$,
a sequence of Riemannian manifolds $M_n$ as above
together with a sequence of homeomorphism approximations $a_n\:M_n\to A$
will be called \emph{smoothing} of $A$
(briefly $M_n\smooths{} A$, or $M_n\smooths{a_n} A$).
By the stability theorem that any smoothable Alexandrov space is a topological manifold without boundary.



%Everywhere in this section $M_n, A, a_n)$

%$A_n\GHto A$ is a converging sequence of Alexandrov spaces
%with a fixed approximation sequence $a_n:A_n\to A$. 


%Let $A$ be a finite dimensional Alexandrov space with curvature bounded below.
 
 
 
\section{Weak convergence of measures}

In this section we give formal definition of weak convergence of measures
that we use.
For more detailed definitions and terminology we refer to
\cite{GMS}.
For a Hausdorff topological space $X$ let
$\mathfrak M(X)$ be the space of signed Radon measures on $X$.
%and $C(X)$ space of continuous functions on $X$
%with a uniform norm. If $X$ is compact  then 
%$\mathfrak M(X)$ is the dual space for $C(X)$
% (by Riesz-Markov-Kakutani representation theorem).
%Now  definition given below
% of weak convergence
%of measures for not necessary compact Hausdorff space
%(that rigorously is a kind of local weak convergence.)
We denote by $C_c(X)\subset C(X)$  the space of continuous functions on $X$
with a compact support. 

We  denote by $\langle m|f\rangle $ the value of
$m\in\mathfrak M(X)$
 on $f\in C_c(X)$.
 We say that measures $m_n\in \mathfrak M(X)$ 
\emph{weakly converge} to
to $m\in \mathfrak M(X)$ (briefly
$m_n\rightharpoonup m$)
if $\langle m_n|f\rangle \to \langle m|f\rangle $ for
any
$f\in C_c(A)$.


For a given  sequence $A_n\GHto A$
we define weak convergence $m_n\rightharpoonup m$ of measures $m_n$ on $A_n$ to a measure $m$ on $A$
by identification of all spaces via homeomorphism
approximations $a_n\:A_n\to A$.
If the condition $\langle m_n|f\rangle \to \langle m|f\rangle $ holds for only for functions $f$ with support in an open subset $\Omega\subset A$, then we say that $m_n$ \emph{weakly converges to $m$ in $\Omega$}.

Equivalently, the weak convergence can be defined using uniform convergence of functions.
We say that  a sequence $f_n\in C_c(A_n)$
\emph{uniformly converges} to $f\in C_c(A)$
if 
\[\sup_x\{\,|f_n\circ a_n^{-1}(x)-f(x)|\,\}\to 0.\]
Then  $m_n\rightharpoonup m$
if for any sequence $f_n\in C_c(A_n)$
with uniformly bounded supports and
uniformly converging to $f\in C_c(A)$
we have $\langle m_n|f_n\rangle \to \langle m|f\rangle $.

%We will denote by $\langle h|f\rangle $ the value of $h\in C^*(A)$ on $f\in C(A)$.

\section{Test functions and their convergence}
In this section we introduce a class of test functions
which we use
to define  a notion of measure valued tensor.

Let $A\in \Al^m$, distance between $x,y\in A$ is denoted by $|x-y|$,
distance function  by $\dist_x(y)=|x-y|$.



Given $A_n, A\in \Al^m$,  $A_n\GHto A$,
we can naturally lift a distance function $\dist_p$ to $A_n$.
We have to choose a convergent sequence $p_n\to p$ and take the
sequence $\dist_{p_n}$.

Choose $r>0$ and $p\in A$.
Let us define \emph{smoothed distance function} as 
$$\widetilde{\dist}_{p,r} =\oint_{B(p,r)} \dist_{x}dx.$$ 
We can approximate this function by
$\widetilde{\dist}_{p_n,r}:A_n\to [0,\infty)  $
 choosing
  some  sequence $A_n\ni p_n\to p\in A$.

We say that $f$ is a test function
and write $f\in \Test(A)$ if it can be expressed by the formula
$$f=\varphi( \widetilde{\dist}_{p_1,r_1}, \widetilde{\dist}_{p_2,r_2},\dots,   \widetilde{\dist}_{p_N,r_N}),$$
where $\varphi:[0,\infty)^N\to\R$ is  $C^1$ function.
If for some sequences of points $A_n\ni p_{k,n}\to p_k\in A$
and $C^1$ functions   
$\varphi_n\cdto \varphi$ we have
$$f_n=\varphi_n( \widetilde{\dist}_{p_1,r_1}, \widetilde{\dist}_{p_2,r_2},\dots,   \widetilde{\dist}_{p_N,r_N}),$$
we will say that $f_n$ is a test sequence converging to $f$ and write
$(f_n,f)\zz\in \Test(A_n,A)$ or $f_n\testto f$.

\parbf{Remarks.}
Test functions form an algebra and a
partition of unity can be formed by test functions.

On a smooth Riemannian manifold test functions
are exactly $C^1$-functions.
Indeed, around any point one can take a
smoothed distance 
coordinate chart, express $C^1$-function in these 
coordinates and then apply partition of unity. 
 
 
\section{Tensors}


\begin{rdef} {Definition}\label{def:mestens}
Let $A\in \Al^m$ and $\mathfrak M(A)$
denotes the set of signed Radon measures on $A$.
A \emph{measure-valued vector field} $\mathfrak{v}$  on $A$
is a  continuous (with respect to test convergence) linear map
$\mathfrak{v}\:\Test(A)\to \mathfrak M(A)$ which satisfies a
chain rule;
i.e. for any collection  $f_1,f_2,\dots,f_\kay \in \Test(A)$
and a smooth function $\phi\:\RR^\kay\to\RR$ we have
$$\mathfrak{v}(\phi(f_1,f_2,\dots,f_n))
=
\sum_{i=1}^n (\partial_i\phi)(f_1,f_2,\dots,f_n)\cdot\mathfrak{v}(f_i)$$
where $\partial_i$ denotes partial derivatives in $\RR^n$.

Analogously, we define (contravariant) measure-valued tensor field.
A \emph{measure-valued tensor field} $\mathfrak{t}$ of valence $\kay$ on $A$ is a multilinear map of $\kay$ variables in $\Test(A)$ with values in $\mathfrak M(A)$ which satisfies chain rule in each of the arguments.
\end{rdef}

Let us note, that smooth vector field on Riemannian manifold can be described this way (for
$\kay=1$) "vector field as derivation".
For $\kay=1$ in the case when $A$ is an $m$-dimensional Riemannian manifold measure-valued vector field $\mathfrak{v}$  in coordinates
 $x_1,\dots,x_m$
is defined by $m$ components, these are measures $(\mu_1,\dots, \mu_{m})=(\vv(x_1),\dots,\vv(x_m))$;
these components transform by contravariant rule under change
 of coordinates. The map from the definition of measure-valued vector field is
 $\mathfrak{v}(f)=\sum_{i=1}^m\frac{\partial f}{\partial x_i}\cdot \mu_i$.
 Similarly for arbitrary $\kay$ measure-valued tensor field of valence $\kay$ 
 is defined by $m^\kay$ components 
 $\mu_{{i_1},\dots,{i_\kay}}=\vv(x_{i_1},\dots,x_{i_\kay})$
 and
 $\mathfrak{v}(f)=\sum
 \frac{\partial f}{\partial x_{i_1}}\frac{\partial f}{\partial x_{i_2}}\dots\frac{\partial f}{\partial x_{i_k}}
 \cdot \mu_{{i_1},\dots,{i_\kay}}$.
A smooth contravariant tensor field can be presented as a measure-valued tensor field if we multiply each component by the volume measure.

Let us mention some basic properties that follow 
from the definition.
Firstly, if $f_i=\const$  for some $i\in\{1,\dots,\kay\}$ 
then $\mathfrak{v}(f_1,\dots,f_\kay)=0.$
Let us show it for $\kay=1$; the general case can be done the same way.
Note that the chain rule implies the
product rule:
$\mathfrak{v}(fg)=\mathfrak{v}(f)g+f\mathfrak{v}(g)$.
Therefore $\vv(1)=\vv(1^2)=2\vv(1)=0$ and hence
$\vv(c\cdot 1)=c\cdot\vv(1)=0$.



Also the notion is local,
i.e. if $f_i|_{U}=\operatorname{const}$
for some open $U\subset A$ and some $i\in\{1,\dots,\kay\}$ 
 then $\mathfrak{v}(f_1,\dots,f_\kay)|_U=0.$
By above it suffices to show
for the case when $\operatorname{const}=0$.
Assume for simplicity  $\kay=1$. 
For arbitrary compact $K\subset U$  let
us take test function $g:A\to\R$ such that
$\supp g\subset U$ and $g|_K=1$. Then applying product rule
 $0=\vv(fg)=f\vv(g)+g\vv(f)$ yields 
$\vv(f)|_K=0$ hence 
 the result.
 




Now we define a weak convergence of measure-valued tensor field.

\begin{rdef} {Definition}
Let $M_n\smooths{} A$ be a smoothing.
Assume that $\mathfrak{t}_n$ is a sequence of %uniformly bounded
 measure-valued tensor fields on $M_n$  and $\mathfrak{t}$ is a
measure-valued tensor field on $A$,
all of the same valence $\kay$.
We say that $\mathfrak{t}_n$ weakly converges to  $\mathfrak{t}$
(briefly $\mathfrak{t}_n\rightharpoonup\mathfrak{t}$) if for arbitrary $\kay$
test sequences 
$f_{i,n}\testto f_i, i=1,\dots,\kay$, the measures $\mathfrak{t}_n(f_{1,n},f_{2,n},\dots,f_{\kay,n})$ weakly converge to $\mathfrak{t}(f_{1},f_{2},\dots,f_{\kay})$.
\end{rdef}

\section{Dual curvature tensor}

The curvature of Riemannian manifold $M$ usually described by tensor of valence 4 that will be denoted by $\Rm$.
We will use \emph{dual curvature tensor} --- 
a curvature tensor written in a dual form that is denoted by $q$;
it is a tensor field of valence $2\cdot(m-2)$ defined the following way.

Set
\begin{multline*}
q(\nabla f_1,\dots,\nabla f_{m-2},\nabla g_1,\dots,\nabla g_{m-2})
=
\\
= 
\Rm(*(\nabla f_1\wedge\dots\wedge \nabla f_{m-2})), *(\nabla g_1\wedge\dots\wedge \nabla g_{m-2})),
\end{multline*}
where ${*}\:(\bigwedge^{m-2}\T)M\to(\bigwedge^2\T)M$ is the  Hodge star operator.
In this definition we may assume that $f_1,\dots,f_{m-2},g_1,\dots,g_{m-2}:M\to\R$ are smooth functions; however, everywhere further it will be used only for test functions.

In addition we will need a measure-valued version of $q$ denoted by $\mathfrak{q}$;
it will be called \emph{dual measure-valued curvature tensor}.
Namely, the measure-valued tensor field with density
\[q(\nabla f_1,\dots,\nabla f_{m-2},\nabla g_1,\dots,\nabla g_{m-2}): M\to\R\]
will be denoted by $\mathfrak{q}(f_1,\dots,f_{m-2},g_1,\dots,g_{m-2})$.


\parbf{Remarks.}
The described dual form of curvature tensor behaves better in the limit.
For example in the 2-dimensional case the valence of $\mathfrak{q}$ is $0$;
in this case $\mathfrak{q}$ coincides with the curvature measure --- the standard way to describe curvature of surfaces, 
(see, for example \cite{R}).
For a smooth surface the density of this curvature measure with respect to the area measure
is its Gauss curvature.
The statement of our main theorem in this case, i.e. 
stability of these measures under smoothing
is well-known \cite[Theorem 8.4.2]{R}, \cite{AZ}.


Note that in the case if $f_i=g_i$, for $i=1,\dots,m-2$
we have:
% can express $q$ using co-sectional curvature: 
$$q(\nabla f_1,\dots,\nabla f_{m-2},\nabla f_1,\dots,\nabla f_{m-2})=|\nabla f_{1,n}(x)\wedge\dots \wedge\nabla f_{m-2,n}(x)|^2\cdot K^M_\sigma, $$
where $K^M_\sigma $ is the sectional curvature at $x\in M_n$
on a plane $\sigma$ orthogonal to $(m-2)$-vector
$\nabla f_{1,n}(x)\wedge\zz\dots \wedge\nabla f_{m-2,n}(x)$.
Hence, the sectional curvatures of $M$ (and therefore its curvature tensor) can  be computed using the tensor
$\mathfrak{q}$.
Also the density $q$ is defined by the sectional curvature,
because of the symmetry:
$${q}(\nabla f_1,\dots,\nabla f_{m-2},\nabla g_1,\dots,\nabla g_{m-2})=
{q}(\nabla g_1,\dots,\nabla g_{m-2},\nabla f_1,\dots,\nabla f_{m-2}).$$
Therefore measure tensor $\mathfrak{q}$
gives an equivalent description of curvature of Riemannian manifold.


\section{Partition into three sets}

Let $A$ be a 3-dimensional Alexandrov space without boundary. 
Let us partition $A$ into three subsets $A^\circ$, $A'$ and $A''$:
\begin{itemize}
\item $A^\circ$ is the set of regular points in $A$; that is, the set of points with tangent cone isometric to the Euclidean space.
\item $A'$ --- the set of points in $A\backslash A^\circ$ with a line in their tangent space; that is, for any $p\in A'$, the tangent space $\T_p$ is isometric to the product of the real line and a two-dimensional cone with total angle $\theta(p)<2\cdot \pi$.
\item $A''$ --- the remaining set; this is the set of points with tangent space that does not contain a line.
\end{itemize}
According to \cite{li-naber}, $A'$ is countably 1-rectifiable and $A''$ is a countable set. 

Observe that the set of regular points $A^\circ$ can be presented as
$$A^{\circ}=\bigcap_{\delta>0} A^\delta,$$
where $A^\delta$ denotes the set of $\delta$-strained points of $A$.

Consider the measure $\omega$ on $A'$ defined by
\[\omega=(2\cdot\pi-\theta)\cdot \haus_1,\eqlbl{eq:omega}\]
where $\haus_\alpha$ denotes $\alpha$-dimensional Hausdorff measure.
Let us extend $\omega$ to the whole space by setting $\omega(S)=\omega(S\cap A')$ for any Borel set $S\subset A$.

The function $\theta$ can be extended to the whole $A$ by setting
\[\theta(p)=2\cdot\pi\cdot \tfrac{\area \Sigma_p}{\area \mathbb{S}^2}\] for any $p\in A'$.
According to \cite[7.14]{BGP}, $\theta\:A\to \RR$ is upper-semicontinuous.
The right-hand side of the last equality vanish in $A^\circ$.
Further since $\haus_1(A'')=0$, the measure $\omega$ could be also defined as a measure on whole $A$ that is defined by~\ref{eq:omega}.

Suppose $p\in A'$.
Choose a unit \emph{vertical} vector $u(p)\in \T_p$;
that is, $u(p)$ in the $\RR$-factor of $\T_p$.
Note that $u$ is uniquely defined up to sign.
