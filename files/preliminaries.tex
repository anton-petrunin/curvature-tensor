\part{Formulations}

In this part we give necessary definitions for a precise formulation of the main theorem.
For simplicity  we will always assume that the lower
curvature bound in main theorem is  $-1$;
applying rescaling, we can get the general case.

We denote by
$\Al^m$ the class of $m$-dimensional Alexandrov's spaces
with curvature $\ge -1$.

Suppose a sequence $A_n\in \Al^m$ that GH-converges without collapse to 
$A\in \Al^m$.
Denote by $a_n\:A_n\to A$ the Hausdorff approximations.
By Perelman's stability theorem (see \cite{PerStab}, \cite{KapStab}) we can (and will) assume that $a_n$ is a homeomorphism for all sufficiently large $n$.

We say that $A\in \Al^m$ is \emph{smoothable}
if it can be presented as a Gromov--Hausdorff limit by a non-collapsing sequence of Riemannian manifolds $M_n$ with $\sec M_n\ge-1$.
Given a smoothable Alexandrov space $A$,
a sequence of Riemannian manifolds $M_n$ as above
together with a sequence of homeomorphism approximations $a_n\:M_n\to A$
will be called \emph{smoothing} of $A$
(briefly $M_n\smooths{} A$, or $M_n\smooths{a_n} A$).
By Perelman's stability theorem that any smoothable Alexandrov space is a topological manifold without boundary.

\section{Weak convergence of measures}

In this section we give formal definition of weak convergence of measures.
For more detailed definitions and terminology we refer to
\cite{GMS}.

Let $X$ be a Hausdorff topological space.
Denote by $\mathfrak M(X)$ the space of signed Radon measures on $X$.
Further, denote by $C_c(X)$  the space of continuous functions on $X$
with a compact support. 

We  denote by $\langle \mathfrak m|f\rangle $ the value of $\mathfrak m\in\mathfrak M(X)$ on $f\in C_c(X)$.
We say that measures $\mathfrak m_n\in \mathfrak M(X)$ \emph{weakly converge} to $\mathfrak m\in \mathfrak M(X)$ (briefly
$\mathfrak m_n\rightharpoonup \mathfrak m$) if $\langle \mathfrak m_n|f\rangle \to \langle \mathfrak m|f\rangle $ for any $f\in C_c(A)$.

Suppose $A_n\GHto A$ with Hausdorff approximations $a_n\:A_n\to A$ and
$\mathfrak m_n$ is a measure on $A_n$.
We say that $\mathfrak m_n$ weakly converge to a measure $\mathfrak m$ on $A$ (briefly $\mathfrak m_n\rightharpoonup \mathfrak m$) if the pushforwards $\mathfrak m_n'$ of $\mathfrak m_n$  by the Hausdorff approximations $a_n\:A_n\to A$ weakly converge to 
$\mathfrak m$.
If the condition $\langle \mathfrak m_n'|f\rangle \to \langle \mathfrak m|f\rangle $ holds for only for functions $f$ with support in an open subset $\Omega\subset A$, then we say that $\mathfrak m_n$ \emph{weakly converges to $\mathfrak m$ in $\Omega$}.

Equivalently, the weak convergence can be defined using uniform convergence of functions.
We say that  a sequence $f_n\in C_c(A_n)$
\emph{uniformly converges} to $f\in C_c(A)$
if 
\[\sup_{x\in A_n}\{\,|f_n(x)-f\circ a_n(x)|\,\}\to 0.\]
Then  $\mathfrak m_n\rightharpoonup \mathfrak m$
if for any sequence $f_n\in C_c(A_n)$
with uniformly bounded supports and
uniformly converging to $f\in C_c(A)$
we have $\langle \mathfrak m_n|f_n\rangle \to \langle \mathfrak m|f\rangle $.

%We will denote by $\langle h|f\rangle $ the value of $h\in C^*(A)$ on $f\in C(A)$.

\section{Test functions and their convergence}

In this section we introduce a class of \emph{test functions} that will be used to define  a notion of measure-valued tensor.

Let $A\in \Al^m$, distance between $x,y\in A$ will be denoted by $|x-y|$;
we will denote by $\dist_x\:A\to \RR$ the distance function $\dist_x\:y\mapsto |x-y|$.

Suppose $A_n, A\in \Al^m$ and  $A_n\GHto A$.
Then any distance function $\dist_p\:A\zz\to\RR$ can be lifted to $A_n$;
we have to choose a convergent sequence $p_n\to p$ and take the
sequence $\dist_{p_n}$.

Choose $r>0$ and $p\in A$.
Let us define \emph{smoothed distance function} as 
$$\widetilde{\dist}_{p,r} =\oint_{B(p,r)} \dist_{x}dx.$$ 
We can lift this function to
$\widetilde{\dist}_{p_n,r}:A_n\to [0,\infty)$
choosing some  sequence $A_n\ni p_n\to p\in A$.

We say that $f$ is a \emph{test function} if it can be expressed by the formula
$$f=\varphi( \widetilde{\dist}_{p_1,r_1}, \dots,   \widetilde{\dist}_{p_N,r_N}),$$
where $\varphi:[0,\infty)^N\to\R$ is a $C^1$-smooth function.
If for some sequences of points $A_n\ni p_{i,n}\to p_i\in A$ and $C^1$-smooth functions $\varphi_n$ that $C^1$-converge to $\varphi$ we have
$$f_n=\varphi_n( \widetilde{\dist}_{p_{1,n},r_1},\dots,   \widetilde{\dist}_{p_{N,n},r_N}),$$
we will say that $f_n$ is a test sequence converging to $f$ and write $f_n\testto f$.

\parbf{Remarks.}
Test functions form an algebra.
Further, a partition of unity for any open cover can be found from test functions.

On a smooth Riemannian manifold test functions
are exactly $C^1$-smooth functions.
Indeed, around any point one can take a
smoothed distance 
coordinate chart, express $C^1$-smooth function in these 
coordinates and then apply partition of unity. 
 
 
\section{Tensors}



Let $A\in \Al^m$ and $\mathfrak M(A)$
denotes the set of signed Radon measures on $A$.
A \emph{measure-valued vector field} $\mathfrak{v}$  on $A$
is a  continuous (with respect to test convergence) linear map
$\mathfrak{v}$ that takes a test function, spits a measure in $\mathfrak M(A)$ and satisfies the
chain rule;
that is, for any collection of test functions $f_1,f_2,\dots,f_k$
and a $C^1$-smooth function $\phi\:\RR^k\to\RR$ we have
$$\mathfrak{v}(\phi(f_1,f_2,\dots,f_n))
=
\sum_{i=1}^n (\partial_i\phi)(f_1,f_2,\dots,f_n)\cdot\mathfrak{v}(f_i)$$
where $\partial_i$ denotes partial derivatives in $\RR^n$.

Analogously, we define (contravariant) measure-valued tensor field.
A \emph{measure-valued tensor field} $\mathfrak{t}$ of valence $k$ on $A$ is a multilinear map that takes $k$ test functions and spits a measure in $\mathfrak M(A)$ and satisfies the chain rule in each of its arguments.


If $A$ is an $m$-dimensional Riemannian manifold measure-valued vector field $\mathfrak{v}$  in coordinates $x_1,\dots,x_m$
is defined by $m$ components, these are measures $(\vv(x_1),\dots,\vv(x_m))$;
these components transform by contravariant rule under change of coordinates.
B the definition of measure-valued vector field, we get
\[\mathfrak{v}(f)=\sum_{i}\frac{\partial f}{\partial x_i}\cdot \vv(x_i).\]
Similarly for arbitrary $k$ measure-valued tensor field of valence $k$ is defined by $m^k$ components 
$\mathfrak{t}(x_{i_1},\dots,x_{i_k})$
and
\[\mathfrak{t}(f_1,\dots,f_k)=\sum_{i_1,\dots,i_k}
\frac{\partial f_1}{\partial x_{i_1}}\cdot 
 \cdots \cdot\frac{\partial f_k}{\partial x_{i_k}}
\cdot \mathfrak{t}(x_{i_1},\dots,x_{i_k}).\]
If one multiplies components of a smooth contravariant tensor field by the volume, we obtain a measure-valued tensor field.

Let us mention some basic properties that follow 
from the definition.
Firstly, if $f_i=\const$  for some $i\in\{1,\dots,k\}$ 
then $\mathfrak{t}(f_1,\dots,f_k)=0.$

Let us show it for $k=1$; the general case can be done the same way.
Note that the chain rule implies the
product rule:
$\mathfrak{v}(fg)=\mathfrak{v}(f)g+f\mathfrak{v}(g)$.
Therefore 
\[\vv(1)=\vv(1^2)=2\vv(1)=0
\quad\text{and hence}\quad
\vv(c\cdot 1)=c\cdot\vv(1)=0.
\]

The notion of vector-valued tensor field is local;
that is, if $f_i|_{U}=\operatorname{const}$
for some open $U\subset A$ and some $i\in\{1,\dots,k\}$ 
 then $\mathfrak{t}(f_1,\dots,f_k)|_U=0.$
By above it suffices to show
for the case when $\operatorname{const}=0$.
Assume for simplicity  $k=1$. 
For arbitrary compact $K\subset U$ let
us take test function $g\:A\to\R$ with support in
$U$ and such that $g|_K=1$.
By the product rule,
 $0=\vv(fg)=f\vv(g)+g\vv(f)$;
 hence 
$\vv(f)|_K=0$.
 




Now we define a weak convergence of measure-valued tensor field.

\begin{rdef} {Definition}
Let $M_n\smooths{} A$ be a smoothing.
Assume that $\mathfrak{t}_n$ is a sequence of %uniformly bounded
 measure-valued tensor fields on $M_n$  and $\mathfrak{t}$ is a
measure-valued tensor field on $A$,
all of the same valence $k$.
We say that $\mathfrak{t}_n$ weakly converges to  $\mathfrak{t}$
(briefly $\mathfrak{t}_n\rightharpoonup\mathfrak{t}$) if for arbitrary $k$
test sequences 
$f_{i,n}\testto f_i, i=1,\dots,k$, the measures $\mathfrak{t}_n(f_{1,n},f_{2,n},\dots,f_{k,n})$ weakly converge to $\mathfrak{t}(f_{1},f_{2},\dots,f_{k})$.
\end{rdef}

\section{Dual curvature tensor}

The curvature of Riemannian manifold $M$ usually described by tensor of valence 4 that will be denoted by $\Rm$.
We will use \emph{dual curvature tensor} --- 
a curvature tensor written in a dual form that is denoted by $\Qm$;
it is a tensor field of valence $2\cdot(m-2)$ defined the following way.

Set
\begin{multline*}
\Qm(X_1,\dots,X_{m-2},Y_1,\dots,Y_{m-2})
= 
\Rm(*(X_1\wedge\dots\wedge X_{m-2}), *(Y_1\wedge\dots\wedge Y_{m-2})),
\end{multline*}
where $X_i,Y_i$ are vector fields on $M$ and  ${*}\:(\bigwedge^{m-2}\T)M\to(\bigwedge^2\T)M$ is the  Hodge star operator.
This definition will be used further mostly for gradient vector fields of test functions.

In addition we will need a measure-valued version of $\Qm$ denoted by $\qm$;
it will be called \emph{dual measure-valued curvature tensor}.
Namely, the measure-valued tensor field with density
\[\Qm(\nabla f_1,\dots,\nabla f_{m-2},\nabla g_1,\dots,\nabla g_{m-2}): M\to\R\]
will be denoted by $\qm(f_1,\dots,f_{m-2},g_1,\dots,g_{m-2})$.


\parbf{Remarks.}
Note that 
$$\Qm(X_1,\dots,X_{m-2},X_1,\dots,X_{m-2})
=
|X_1\wedge\dots \wedge X_{m-2}|^2\cdot K_\sigma, $$
where $K_\sigma$ is the sectional curvature of $M$ 
on a plane $\sigma$ orthogonal to $(m-2)$-vector
$X_1(x)\wedge\zz\dots \wedge X_{m-2}$.
Hence, the sectional curvatures of $M$ and therefore its curvature tensor can  be computed using the tensor
$\qm$.
Also the density $\qm$ is defined by the sectional curvature,
because of the symmetry:
$$\qm(f_1,\dots,f_{m-2},g_1,\dots,g_{m-2})=
\qm(g_1,\dots,g_{m-2},f_1,\dots,f_{m-2}).$$
Therefore measure tensor $\qm$
gives an equivalent description of curvature of Riemannian manifold.

As you will see further, the described dual form of curvature tensor behaves better in the limit;
in particular, it is possible to formulate ??? 

In the 2-dimensional case the valence of $\qm$ is $0$;
in this case $\qm$ coincides with the curvature measure --- the standard way to describe curvature of surfaces, 
(see, for example \cite{R}).
For a smooth surface the density of this curvature measure with respect to the area
is its Gauss curvature.
The statement of our main theorem in this case
is well-known \cite[Theorem 8.4.2]{R}, \cite{AZ};
that is, curvature measures are stable under smoothing.


\section{Partition into three sets}

Let $A$ be a $m$-dimensional Alexandrov space without boundary. 
Let us partition $A$ into three subsets $A^\circ$, $A'$ and $A''$:
\begin{itemize}
\item $A^\circ$ is the set of regular points in $A$; that is, the set of points with tangent cone isometric to the Euclidean space.
\item $A'$ --- the set of points in $A\backslash A^\circ$ with a isometric copy of $\RR^{m-2}$ in their tangent space;
in other words, for any $p\in A'$, the tangent space $\T_p$ is isometric to the product $\Cone(\theta)\times\RR^{m-2}$ where $\Cone(\theta)$ denotes the two-dimensional cone with total angle $\theta=\theta(p)<2\cdot \pi$.
\item $A''$ --- the remaining set; this is the set of points with tangent space that does not an isometric copy of $\RR^{m-2}$.
\end{itemize}
According to \cite{li-naber}, $A'$ is countably $(m-2)$-rectifiable and $A''$ is a countably $(m-3)$-rectifiable. 

Observe that the set of regular points $A^\circ$ can be presented as
$$A^{\circ}=\bigcap_{\delta>0} A^\delta,$$
where $A^\delta$ denotes the set of $\delta$-strained points of $A$.
