\section{Formulations}

In this section, we give the necessary definitions for a precise formulation of the main theorem.
For simplicity  we will always assume that the lower
curvature bound is  $-1$;
applying rescaling, we can get the general case.

We denote by
$\Al^m$ the class of $m$-dimensional Alexandrov's spaces
with curvature $\ge -1$.

Suppose $A,A_1,A_2,\dots{} \in \Al^m$ and $A_n\GHto A$.
That is, $A_n$ converges to 
$A$ in the sense of Gromov--Hausdorff;
since $A\in \Al^m$, we have no collapse.
Denote by $a_n\:A_n\to A$ the Hausdorff approximations.
If $A$ is compact, then
by Perelman's stability theorem \cite{PerStab,KapStab} we can (and will) assume that $a_n$ is a homeomorphism for every sufficiently large $n$.
In the case of noncompact limit, we assume that for any $R$, the restriction of $a_n$ to an $R$-neighborhood of the marked point is a homeomorphism to its image for every sufficiently large $n$.

We say that $A\in \Al^m$ is \emph{smoothable}
if it can be presented as a Gromov--Hausdorff limit of a non-collapsing sequence of Riemannian manifolds $M_n$ with $\sec M_n\ge-1$; here $\sec$ stands for sectional curvature.
Given a smoothable Alexandrov space $A$,
a sequence of complete Riemannian manifolds $M_n$ as above together with a sequence of approximations $a_n\:M_n\to A$
will be called \emph{smoothing} of $A$
(briefly, $M_n\smooths{} A$, or $M_n\smooths{a_n} A$).
By Perelman's stability theorem any smoothable Alexandrov space is a topological manifold without boundary.

\subsection{Weak convergence of measures}

In this subsection we give a formal definition of weak convergence of measures.
For more detailed definitions and terminology, we refer to
\cite{GMS}.

Let $X$ be a Hausdorff topological space.
Denote by $\mathfrak M(X)$ the space of signed Radon measures on $X$.
Further, denote by $C_c(X)$  the space of continuous functions on $X$
with a compact support. 

We  denote by $\langle \mathfrak m|f\rangle $ the value of $\mathfrak m\in\mathfrak M(X)$ on $f\in C_c(X)$.
We say that measures $\mathfrak m_n\in \mathfrak M(X)$ \emph{weakly converge} to $\mathfrak m\in \mathfrak M(X)$ (briefly
$\mathfrak m_n\rightharpoonup \mathfrak m$) if $\langle \mathfrak m_n|f\rangle \to \langle \mathfrak m|f\rangle $ for any $f\in C_c(A)$.

Suppose $A_n\GHto A$ with Hausdorff approximations $a_n\:A_n\zz\to A$ and
$\mathfrak m_n$ is a measure on $A_n$.
We say that $\mathfrak m_n$ \emph{weakly converges} to a measure $\mathfrak m$ on $A$ (briefly $\mathfrak m_n\rightharpoonup \mathfrak m$) if the pushforwards $\mathfrak m_n'$ of $\mathfrak m_n$ to $A$  by the Hausdorff approximations $a_n\:A_n\to A$ weakly converge to 
$\mathfrak m$.
If the condition $\langle \mathfrak m_n'|f\rangle \to \langle \mathfrak m|f\rangle $ holds only for functions $f$ with support in an open subset $\Omega\subset A$, then we say that $\mathfrak m_n$ \emph{weakly converges to $\mathfrak m$ in $\Omega$}.

Equivalently, the weak convergence can be defined using the uniform convergence of functions.
We say that  a sequence $f_n\in C_c(A_n)$
\emph{uniformly converges} to $f\in C_c(A)$
if their supports are uniformly bounded and
\[\sup_{x\in A_n}\{\,|f_n(x)-f\circ a_n(x)|\,\}\to 0.\]
Then  $\mathfrak m_n\rightharpoonup \mathfrak m$
if for any sequence $f_n\in C_c(A_n)$
with uniformly bounded supports and
uniformly converging to $f\zz\in C_c(A)$
we have $\langle \mathfrak m_n|f_n\rangle \to \langle \mathfrak m|f\rangle $.

\subsection{Test functions}\label{sec:test-functions}

In this subsection, we introduce a class of \emph{test functions} and define their convergence.

Test functions form a narrow class of functions defined via a formula.
It is just one possible choice of a class containing sufficiently smooth DC functions; see Section~\ref{sec:DC}.
We could use other definitions as well; see Section~\ref{sec:concept}.

Recall that the distance between points $x,y$ in a metric space is denoted by $|x-y|$;
we will denote by $\dist_x$ the distance function $\dist_x\:y\mapsto |x-y|$.

{\sloppy 

Suppose $A_n, A\in \Al^m$ and  $A_n\GHto A$.
Then any distance function $\dist_p\:A\zz\to\RR$ can be \emph{lifted} to $A_n$;
it means that we can choose a convergent sequence $p_n\to p$ and take the
sequence $\dist_{p_n}$.

}

Choose $r>0$ and $p\in A$.
Let us define \emph{smoothed distance function} as the average:
$$\widetilde{\dist}_{p,r} =\oint_{B(p,r)} \dist_{x}dx.$$ 
We can lift this function to
$\widetilde{\dist}_{p_n,r}\:A_n\to [0,\infty)$
by choosing some  sequence $A_n\ni p_n\to p\in A$.

We say that $f$ is a \emph{test function} if it can be expressed by the formula
$$f=\varphi( \widetilde{\dist}_{p_1,r_1}, \dots,   \widetilde{\dist}_{p_N,r_N}),$$
where $\varphi:(0,\infty)^N\to\R$ is a $C^2$-smooth function with compact support.
If for some sequences of points $A_n\ni p_{i,n}\to p_i\in A$ and $C^2$-smooth functions $\varphi_n$ that $C^2$-converge to $\varphi$ with compact support we have
$$f_n=\varphi_n( \widetilde{\dist}_{p_{1,n},r_1},\dots,   \widetilde{\dist}_{p_{N,n},r_N}),$$
then we say that $f_n$ is \emph{test-converging} to $f$ (briefly, $f_n\zz\testto f$).

\parbf{Remarks.}
Note that test functions form an algebra.

Let $M$ be a Riemannian manifold.
Note that for any open cover of $M$, there is a subordinate partition of unity of test functions.
Further, around any point of $M$ one can take a smoothed distance 
coordinate chart.
One can express any $C^2$-smooth function in these 
coordinates, and then apply partition of unity for a covering by charts.
This way, we get the following:

\begin{thm}{Claim}
On a smooth complete Riemannian manifold, test functions
include all $C^2$-smooth functions with compact support.
\end{thm}


\subsection[\texorpdfstring{$C^1$-delta convergence}{C¹-delta convergence}]%
{$\bm{C^1}$-delta convergence}\label{sec:concept}

Here we introduce $C^1$-delta convergence.
It will be necessary to formulate the main theorem in an invariant way, but, except for \ref{sec:test-convergence}, everywhere in the proofs we will use test convergence instead.
(As it claimed in \ref{clm:test=>smooth} test convergence implies $C^1$-delta convergence.)
By that reason, \textit{it would be wise to skip this section for the first reading}.

The $C^1$-delta convergence will be used together with other delta convergence introduced in \ref{subsec:chart+delta}.

\parbf{Convergence of vectors.}
Let $A$ be an Alexandrov space, we denote by $\T A$ the set of all tangent vectors at all points.
So far $\T A$ is a disjoint union of all tangent cones;
let us define a convergence on it.

We will use gradient exponent $\gexp\: \T A\to A$ which is defined in \cite{AKP}.
Given a vector $V\in \T A$, it defines its radial curve $\gamma_{V}\:t\zz\mapsto \gexp (t\cdot V)$.
We say that a sequence of vectors $V_n\in \T A$ \emph{converges} to $V\in \T A$ (briefly, $V_n\to V$) if $\gamma_{V_n}$ converges to $\gamma_V$ pointwise.
Since the radial curve $\gamma_V$ is $|V|$-Lipschitz, we get that any bounded sequence of vectors with base points in a bounded set has a converging subsequence of $\gamma_{V_n}$.
Further, the pointwise limit of such curves is a radial curve as well.
Therefore, any bounded sequence of tangent vectors with base points in a bounded set has a converging sequence.

In a similar fashion, we can define the convergence of tangent vectors to sequences of Alexandrov spaces $A_n$ that converge to $A$.
That is, if $V_n\in \T A_n$ is a bounded sequence of tangent vectors at points on a bounded distance to the base points, then it has a subsequence that converges to some vector $V\in \T A$.

Note that 
\[|V|\le \liminf_{n\to\infty} |V_n|\]
and the inequality might be strict.

Recall that if $V\in \T_p$ is the unit vector in the direction of $[pq]$, then $\gamma_V$ is a unit-speed parametrization of $[pq]$.
Using this we get the following observation;
it provides a way to apply the convergence.

\begin{thm}{Observation}\label{obs:unique-geod}
Let $M_n\smooths{} A$ be a smoothing, $p_n,q_n\in M_n$, and $p_n\to p$, $q_n\to q$ as $n\to \infty$.
Denote by $V_n\in \T_{p_n}$ and $V\in \T_p$ the directions of geodesics $[p_nq_n]$ and $[pq]$.
Suppose that there is a unique geodesic $[pq]$ in $A$.
Then $V_n\zz\to V$.
\end{thm}



\parbf{$\bm{C^1}$-delta smoothness.}
Given a function $f\:A\to \RR$ and a vector $V\in \T A$, set
\[Vf=(f\circ\gamma_V(t))'|_{t=0}.\]
Note that $Vf$ is defined for all DC functions and, in particular, all test functions.

Two vectors $V,W\in \T_pA$ will be called $\delta$-opposite if
$1-\delta< |V|\le 1$,
$1-\delta< |W|\le 1$,
and $|\langle X,V\rangle +\langle X,W\rangle|<\delta$ for any unit vector $X\in \T_p A$.
We say that $V,W\in \T_pA$ are opposite if they are $\delta$-opposite for any $\delta>0$;
in this case, they are both unit vectors and make angle $\pi$ to each other.

A function $f\:A\to\RR$ is called $C^1$-delta smooth if for any compact set $K\subset A$ and $\eps>0$ there is $\delta>0$ such that any sequence of points $p_n\to p\in K$ and unit vectors $V_n\in \T_{p_n} A$ that converges to a vector $V\in \T_p A$ that has a $\delta$-opposite vector we have
\[|Vf-\lim_{n\to\infty} V_nf|<\eps.\]

Suppose $M_n\smooths{} A$, a sequence of $C^1$-smooth functions $f_n\:M_n\zz\to\RR$ is called \emph{$C^1$-delta converging} to $f\:A\to \RR$ if for any compact set $K\subset A$ and any $\eps>0$ there is $\delta>0$ such that any sequence of unit vectors $V_n\in \T_{p_n} M_n$ that converges to a vector $V\zz\in \T_p A$ and has a $\delta$-opposite vector and such that $p\in K$ we have
\[|Vf-\lim_{n\to\infty} V_nf_n|<\eps.\]

\begin{thm}{Claim}\label{clm:test=>smooth}
Any test function is $C^1$-delta smooth.
Moreover, 
\[f_n\testto f
\qquad\Longrightarrow\qquad
f_n\smoothto f\]
for any smoothing $M_n\smooths{} A$,
sequence of test functions $f_n\:M_n\zz\to\RR$,
and a test function $f\:A\zz\to\RR$.
\end{thm}

\parit{Proof.}
Let $V$ and $W$ be $\delta$-opposite vectors in $\T_pA$.
Note that for almost all points $q\in A$, we have 
\[|V\dist_{q}+W\dist_{q}|<\delta.\]
It follows that 
\[|V\widetilde{\dist}_{q,r}+W\widetilde{\dist}_{q,r}|<\delta\eqlbl{eq:|(V+W)tildedist|<delata}\]
for any $q\in A$ and $r>0$.

Suppose $V_n\to V$; that is, $\gamma_{V_n}\to \gamma_V$ as $n\to\infty$.
By monotonicity of radial curves \cite[16.31]{AKP}, we get 
\[V\dist_{q}\le  V_n\dist_{q_n}\]
if $q_n\to q$.
Integrating, we get 
\[V\widetilde{\dist}_{q,r}\le  V_n\widetilde{\dist}_{q_n,r}.\]

Suppose $V$ has a $\delta$-opposite vector $W$.
We can assume that $W$ is a unit geodesic vector; that is, there is a geodesic $[ps]$ in the direction of $W$.
Choose points $s_n$ and $p_n$ that converge to $s$ and $p$ respectively.
By \ref{obs:unique-geod}, the directions $W_n$ of $[p_ns_n]$ converge to $W$.
Note that $W_n$ is $\delta$-opposite to $V_n$ for all large $n$.

Repeating the above argument we get 
\[W\widetilde{\dist}_{q,r}\le  W_n\widetilde{\dist}_{q_n,r}.\]
Applying \ref{eq:|(V+W)tildedist|<delata} we get $C^1$-delta convergence of $\widetilde{\dist}_{q_n,r}$ and, in particular, $C^1$-delta smoothness of $\widetilde{\dist}_{q,r}$.
Applying the definition of test function, we get the result.
\qeds

Recall (\ref{sec:test-functions}) that for any smoothing $M_n\smooths{} A$ and a test function $f\:A\to\RR$ there are test functions  $f_n\:M_n\to \RR$ such that $f_n\testto f$.

\begin{thm}{Corollary}\label{cor:liftability}
Given a smoothing $M_n\smooths{} A$ and a test function $f\:A\to\RR$,
there is a sequence of $C^1$-smooth functions $f_n\:M_n\to \RR$ such that $f_n\smoothto f$.
\end{thm}

\parbf{Remark.}
The test functions with $C^1$-delta smooth convergence will be used to define the convergence of measure-valued tensors.
Instead of test functions, we could use another class of functions that meets \ref{cor:liftability}.

 
\subsection{Tensors}\label{subsec:tensors}

In this subsection, we define measure-valued tensors on Alexandrov spaces.
Basically, we reuse the derivation approach to vector fields in classical differential geometry.
This definition will be used in Claim \ref{clm:weak-partial-limit} that reduces the main theorem to Proposition~\ref{prop:3parts} and will not show up ever after.

Let $A\in \Al^m$.
Recall that $\mathfrak M(A)$
denotes the space of signed Radon measures on $A$.
A \emph{measure-valued vector field} $\mathfrak{v}$  on $A$
is a linear map
$\mathfrak{v}$ that takes a test function,
spits a measure in $\mathfrak M(A)$,
and satisfies the \emph{chain rule}:
for any collection of test functions $f_1,\dots,f_k$
and a $C^1$-smooth function $\phi\:\RR^k\to\RR$, we have
$$\mathfrak{v}(\phi(f_1,\dots,f_n))
=
\sum_{i=1}^n (\partial_i\phi)(f_1,\dots,f_n)\cdot\mathfrak{v}(f_i).
$$

In the same way, we define (contravariant) measure-valued tensor fields.
Namely, a \emph{measure-valued tensor field} $\mathfrak{t}$ of valence $k$ on $A$ is a multilinear map that takes a $k$-array of test  functions, spits a measure in $\mathfrak M(A)$, and satisfies the chain rule in each of its arguments.

Suppose that $x_1,\dots,x_m$ are local coordinates in an $m$-dimensional Riemannian manifold $M$.
Then a measure-valued vector field $\mathfrak{v}$ on $M$ can be described by $m$ components, these are measures $(\vv(x_1),\zz\dots,\vv(x_m))$;
these components transform by contravariant rule under change of coordinates.

By the definition of measure-valued vector field, we get
\[\mathfrak{v}(f)=\sum_{i}\partial_i f\cdot \vv(x_i).\]
Similarly, for arbitrary $k$, a measure-valued tensor field of valence $k$ is defined by $m^k$ components 
$\mathfrak{t}(x_{i_1},\dots,x_{i_k})$; namely,
\[\mathfrak{t}(f_1,\dots,f_k)
=
\sum_{i_1,\dots,i_k}
\partial_{i_1} f_1 
\cdots 
\partial_{i_k} f_k
\cdot \mathfrak{t}(x_{i_1},\dots,x_{i_k}).\]

Note that if $T$ is a smooth contravariant tensor field then $\mathfrak{t}=T\cdot \vol$ is a measure-valued tensor field.
In other words, usual tensor fields might be considered as a subspace of measure-valued tensor fields.

\begin{rdef}{Definition}
Let $M_n\smooths{} A$ be a smoothing.
Assume that $\mathfrak{t}_n$ is a sequence of %uniformly bounded
 measure-valued tensor fields on $M_n$  and $\mathfrak{t}$ is a measure-valued tensor field on $A$,
all of the same valence $k$.
We say that $\mathfrak{t}_n$ \emph{weakly converges} to  $\mathfrak{t}$
(briefly $\mathfrak{t}_n\rightharpoonup\mathfrak{t}$) if
\[f_{i,n}\smoothto f_i\quad\text{for all\ } i
\quad\Longrightarrow\quad
\mathfrak{t}_n(f_{1,n},\dots,f_{k,n})\zz\rightharpoonup\mathfrak{t}(f_{1},\dots,f_{k})\]
for arbitrary $k$ sequences $f_{1,n},\dots,f_{k,n}$ of $C^1$-smooth functions and test functions $f_1,\dots,f_n\:A\zz\to\RR$.
\end{rdef}

\subsection{Dual curvature tensor}

The curvature of Riemannian manifold $M$ is usually described by a tensor of valence 4 that will be denoted by $\Rm$.
We will use a \emph{dual curvature tensor} --- 
a curvature tensor written in a dual form that will be denoted by $\Qm$;
it is a tensor field of valence $2\cdot(m-2)$ defined the following way:
\begin{align*}
\Qm(X_1,&\dots,X_{m-2},Y_1,\dots,Y_{m-2})
= 
\\
&=
\Rm(*(X_1\wedge\dots\wedge X_{m-2}),*(Y_1\wedge\dots\wedge Y_{m-2})),
\end{align*}
where $X_i,Y_i$ are vector fields on $M$ and  ${*}\:(\bigwedge^{m-2}\T)M\zz\to(\bigwedge^2\T)M$ is the  Hodge star operator.
This definition will be used further mostly for gradient vector fields of semiconcave functions.

In addition, we will need a measure-valued version of $\Qm$ denoted by $\qm$;
it will be called \emph{dual measure-valued curvature tensor}.
Namely, we define 
\[\qm(f_1,\zz\dots,f_{m-2},g_1,\zz\dots,g_{m-2})\]
as the measure with density
\[\Qm(\nabla f_1,\dots,\nabla f_{m-2},\nabla g_1,\dots,\nabla g_{m-2}): M\to\R.\]

\parbf{Remarks.}
Note that 
$$\Qm(X_1,\dots,X_{m-2},X_1,\dots,X_{m-2})
=
|X_1\wedge\dots \wedge X_{m-2}|^2\cdot K_\sigma, $$
where $K_\sigma$ is the sectional curvature of $M$ 
on a plane $\sigma$ orthogonal to $(m-2)$-vector
$X_1\wedge\zz\dots \wedge X_{m-2}$.
Hence, the sectional curvatures of $M$ and therefore its curvature tensor $\Rm$ can  be computed from
$\qm$.
By the symmetry
\begin{align*}
\qm(f_1,&\dots,f_{m-2},g_1,\dots,g_{m-2})
\\
&=
\qm(g_1,\dots,g_{m-2},f_1,\dots,f_{m-2}),
\end{align*}
the density of $\qm$ is defined by the sectional curvature.
Therefore measure-valued tensor $\qm$
gives an equivalent description of curvature of Riemannian manifolds.

As you will see further, the described dual form of curvature tensor behaves better in the limit;
in particular, it makes it possible to formulate \ref{prop:3parts:codim2+}.

In the 2-dimensional case, the valence of $\qm$ is $0$;
in this case, $\qm$ coincides with the curvature measure --- the standard way to describe the curvature of surfaces, 
\cite{Resh,AZ}.
For a smooth surface, the density of this curvature measure with respect to the area
is its Gauss curvature.
In this case, it is known that \textit{curvature measures are stable under smoothing} \cite[VII \S13]{AZ};
in other words, our main theorem is known in the two-dimensional case.
