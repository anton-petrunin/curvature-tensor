\section{Formulations}

In this section, we give necessary definitions for a precise formulation of the main theorem.
For simplicity  we will always assume that the lower
curvature bound is  $-1$;
applying rescaling, we can get the general case.

We denote by
$\Al^m$ the class of $m$-dimensional Alexandrov's spaces
with curvature $\ge -1$.

Suppose $A_n, A\in \Al^m$ and $A_n\GHto A$.
That is, $A_n$ converges to 
$A$ in the sense of Gromov--Hausdorff;
since $A\in \Al^m$, we have no collapse.
Denote by $a_n\:A_n\to A$ the Hausdorff approximations.
If $A$ is compact, then
by Perelman's stability theorem \cite{PerStab,KapStab} we can (and will) assume that $a_n$ is a homeomorphism for every sufficiently large $n$.
In the case of noncompact limit, we assume that for any $R$, the restriction of $a_n$ to an $R$-neighborhood of the marked point is a homeomorphism to its image for every sufficiently large $n$.

We say that $A\in \Al^m$ is \emph{smoothable}
if it can be presented as a Gromov--Hausdorff limit of a non-collapsing sequence of Riemannian manifolds $M_n$ with $\sec M_n\ge-1$; here $\sec$ denotes sectional curvature.
Given a smoothable Alexandrov space $A$,
a sequence of complete Riemannian manifolds $M_n$ as above
together with a sequence of approximations $a_n\:M_n\to A$
will be called \emph{smoothing} of $A$
(briefly, $M_n\smooths{} A$, or $M_n\smooths{a_n} A$).
By Perelman's stability theorem any smoothable Alexandrov space is a topological manifold without boundary.

\subsection{Weak convergence of measures}

In this subsection we give a formal definition of weak convergence of measures.
For more detailed definitions and terminology, we refer to
\cite{GMS}.

Let $X$ be a Hausdorff topological space.
Denote by $\mathfrak M(X)$ the space of signed Radon measures on $X$.
Further, denote by $C_c(X)$  the space of continuous functions on $X$
with a compact support. 

We  denote by $\langle \mathfrak m|f\rangle $ the value of $\mathfrak m\in\mathfrak M(X)$ on $f\in C_c(X)$.
We say that measures $\mathfrak m_n\in \mathfrak M(X)$ \emph{weakly converge} to $\mathfrak m\in \mathfrak M(X)$ (briefly
$\mathfrak m_n\rightharpoonup \mathfrak m$) if $\langle \mathfrak m_n|f\rangle \to \langle \mathfrak m|f\rangle $ for any $f\in C_c(A)$.

Suppose $A_n\GHto A$ with Hausdorff approximations $a_n\:A_n\to A$ and
$\mathfrak m_n$ is a measure on $A_n$.
We say that $\mathfrak m_n$ \emph{weakly converges} to a measure $\mathfrak m$ on $A$ (briefly $\mathfrak m_n\rightharpoonup \mathfrak m$) if the pushforwards $\mathfrak m_n'$ of $\mathfrak m_n$ to $A$  by the Hausdorff approximations $a_n\:A_n\to A$ weakly converge to 
$\mathfrak m$.
If the condition $\langle \mathfrak m_n'|f\rangle \to \langle \mathfrak m|f\rangle $ holds only for functions $f$ with support in an open subset $\Omega\subset A$, then we say that $\mathfrak m_n$ \emph{weakly converges to $\mathfrak m$ in $\Omega$}.

Equivalently, the weak convergence can be defined using the uniform convergence of functions.
We say that  a sequence $f_n\in C_c(A_n)$
\emph{uniformly converges} to $f\in C_c(A)$
if their supports are uniformly bounded and
\[\sup_{x\in A_n}\{\,|f_n(x)-f\circ a_n(x)|\,\}\to 0.\]
Then  $\mathfrak m_n\rightharpoonup \mathfrak m$
if for any sequence $f_n\in C_c(A_n)$
with uniformly bounded supports and
uniformly converging to $f\in C_c(A)$
we have $\langle \mathfrak m_n|f_n\rangle \to \langle \mathfrak m|f\rangle $.

%We will denote by $\langle h|f\rangle $ the value of $h\in C^*(A)$ on $f\in C(A)$.







\subsection{ $C^1$ structure and convergence}\label{sec:concept}. 

$\smoothto$

%Here we will describe a more natural definition of convergence that is equivalent to our definition. It is given for purely aesthetic purposes --- we did not use it and do not see its application. By that reason, we give only definitions and do not prove the equivalence.

\parbf{Convergence of vectors.}
Let $A$ be an Alexandrov space, we denote by $\T A$ the set of all tangent vectors at all points.
So far $\T A$ is a disjoint union of all tangent cones;
let us define convergence on it.

Recall that gradient exponent $\gexp\: \T A\to A$ is defined in \cite{AKP}.
Given a vector $V\in \T A$, it defines its gradient curve $\gamma_{V}\:t\mapsto \gexp (t\cdot V)$.
We say that a sequence of vectors $V_n\in \T A$ converges to $V\in \T A$ if $\gamma_{V_n}$ converges to $\gamma_V$ pointwise.
Since the gradient curve $\gamma_V$ is $|V|$-Lipschitz, we get that for any bounded sequence of vectors with base points in a bounded set have a converging subsequence of $\gamma_{V_n}$.
Further, the pointwise limit of such curves is a gradient curve as well.
Therefore, any bounded sequence of tangent vectors with base points in a bounded set has a converging sequence.

A direct analog holds for sequences of Alexandrov spaces $A_n$ that converge to $A$.
That is, if $V_n\in \T A_n$ is a bounded sequence of tangent vectors with a bounded set of base points, then it has a subsequence that converges to some vector $V\in \T A$.

Note that 
\[|V|\le \lim |V_n|\]
and the inequality might be strict.




\parbf{$\bm{C^1}$-delta smooth functions and their convergence.}
Given a function $f\:A\to \RR$ and a vector $V\in \T A$, set
\[Vf=(f\circ\gamma_V(t))'|_{t=0}.\]
Note that $Vf$ is defined for all DC functions.

Two vectors $V,W\in \T_pA$ will be called $\delta$-opposite if
$1-\delta< |V|\le 1$,
$1-\delta< |W|\le 1$,
and $|\langle X,V\rangle +\langle X,W\rangle|<\delta$ for any unit vector $X\in T_p A$.
We say that $V,W\in \T_pA$ are opposite if they are $\delta$-opposite for any $\delta>0$;
in this case, they are both unit vectors and make angle $\pi$ to each other.  

A Lipschitz function $f\:A\to \RR$ will be called \emph{$C^1$-delta smooth} if $Vf$ is defined for any unit vector $V$ and it satisfies the following continuity property:
for any $\eps>0$ there is $\delta>0$ such that if unit vectors $V_n$ converge to $V$ and $V$ has a $\delta$-opposite vector, then 
\[|Vf-\lim V_nf|<\eps.\eqlbl{eq:Vf}\]

Similarly, we can define $C^1$-delta converging sequence of functions.
Namely, suppose $A_n\GHto A$, then a sequence of $C^1$-delta smooth functions $f_n\:A_n\to\RR$ \emph{$C^1$-delta converges} to $f\:A\to \RR$ if for any $\eps>0$ there is $\delta>0$ such that any sequence of unit vectors $V_n\in \T A_n$ that converges to a vector $V\in \T A$ that has a $\delta$-opposite vector \ref{eq:Vf} holds. 

\parbf{Weak convergence of tensor fields.}
{\color{blue} DELETE THIS Paragraf? It is easy to see that test-convergence implies $C^1$-delta convergence.
The latter notion seems to be more natural.
If we replace test convergence
by $C^1$-delta convergence
in the definition of the weak convergence of tensor fields and
formulate the main theorem using this definition, we get a formally stronger
statement; plus our proof still works.

For the part with DC-calculus and 
Key lemma we still use test sequences
(note that $C^1$-delta smooth functions are not DC in general).
}
Then we need the following property:
\begin{thm}{Proposition}
\begin{itemize}
 \item \textit{For any  $C^1$-delta converging sequences $f_n, g_n$ the function
$\langle \nabla f_n , \nabla g_n
\rangle$ delta-converges.}
\end{itemize}
\end{thm}
Its proof reminds the proof of differentiability of a function that has continuous partial derivatives. 

 
\subsection{Tensors}

In this subsection, we define measure-valued tensors on Alexandrov spaces.
Basically, we reuse the \emph{derivation} approach to vector fields in classical differential geometry. 

Let $A\in \Al^m$.
Recall that $\mathfrak M(A)$
denotes the set of signed Radon measures on $A$.
A \emph{measure-valued vector field} $\mathfrak{v}$  on $A$
is a  continuous (with respect to 
$C^1$-delta 
convergence) linear map
$\mathfrak{v}$ that takes a $C^1$-delta smooth function,
spits a measure in $\mathfrak M(A)$,
and satisfies the \emph{chain rule};
that is, for any collection of $C^1$-delta smooth functions  $f_1,\dots,f_k$
and a $C^1$-smooth function $\phi\:\RR^k\to\RR$ we have
$$\mathfrak{v}(\phi(f_1,\dots,f_n))
=
\sum_{i=1}^n (\partial_i\phi)(f_1,\dots,f_n)\cdot\mathfrak{v}(f_i)$$
where $\partial_i$ denotes partial derivatives in $\RR^n$.

In the same way, we define \emph{(contravariant)} measure-valued tensor fields.
Namely, a \emph{measure-valued tensor field} $\mathfrak{t}$ of valence $k$ on $A$ is a multilinear map that takes $k$ $C^1$-delta smooth  functions, spits a measure in $\mathfrak M(A)$, and satisfies the chain rule in each of its arguments.

Suppose that $x_1,\dots,x_m$ are local coordinates in an $m$-dimensional Riemannian manifold $M$.
Then a measure-valued vector field $\mathfrak{v}$ on $M$ can be described by $m$ components, these are measures $(\vv(x_1),\dots,\vv(x_m))$;
these components transform by contravariant rule under change of coordinates.

By the definition of measure-valued vector field, we get
\[\mathfrak{v}(f)=\sum_{i}\partial_i f\cdot \vv(x_i).\]
Similarly, for arbitrary $k$, a measure-valued tensor field of valence $k$ is defined by $m^k$ components 
$\mathfrak{t}(x_{i_1},\dots,x_{i_k})$; namely,
\[\mathfrak{t}(f_1,\dots,f_k)
=
\sum_{i_1,\dots,i_k}
\partial_{i_1} f_1 
\cdots 
\partial_{i_k} f_k
\cdot \mathfrak{t}(x_{i_1},\dots,x_{i_k}).\]

Note that if $T$ is a smooth contravariant tensor field then $\mathfrak{t}=T\cdot \vol$ is measure-valued tensor field.
In other words, usual tensor fields naturally form a subspace of measure-valued tensor fields.

\begin{rdef} {Definition}
Let $M_n\smooths{} A$ be a smoothing.
Assume that $\mathfrak{t}_n$ is a sequence of %uniformly bounded
 measure-valued tensor fields on $M_n$  and $\mathfrak{t}$ is a
measure-valued tensor field on $A$,
all of the same valence $k$.
We say that $\mathfrak{t}_n$ \emph{weakly converges} to  $\mathfrak{t}$
(briefly $\mathfrak{t}_n\rightharpoonup\mathfrak{t}$) if for arbitrary $k$
sequences of $C^1$-delta smooth functions 
$f_{i,n}\smoothto f_i$, the measures $\mathfrak{t}_n(f_{1,n},\dots,f_{k,n})$ weakly converge to $\mathfrak{t}(f_{1},\dots,f_{k})$.
\end{rdef}

\subsection{Dual curvature tensor}

The curvature of Riemannian manifold $M$ is usually described by a tensor of valence 4 that will be denoted by $\Rm$.
We will use a \emph{dual curvature tensor} --- 
a curvature tensor written in a dual form that is denoted by $\Qm$;
it is a tensor field of valence $2\cdot(m-2)$ defined the following way:
\[
\Qm(X_1,\dots,X_{m-2},Y_1,\dots,Y_{m-2})
= 
\Rm(*(X_1\wedge\dots\wedge X_{m-2}),*(Y_1\wedge\dots\wedge Y_{m-2})),
\]
where $X_i,Y_i$ are vector fields on $M$ and  ${*}\:(\bigwedge^{m-2}\T)M\to(\bigwedge^2\T)M$ is the  Hodge star operator.
{\color{blue}
This definition will be used further mostly for gradient vector fields of ??? test functions.}

In addition, we will need a measure-valued version of $\Qm$ denoted by $\qm$;
it will be called \emph{dual measure-valued curvature tensor}.
Namely, we define 
\[\qm(f_1,\zz\dots,f_{m-2},g_1,\zz\dots,g_{m-2})\]
as the measure with density
\[\Qm(\nabla f_1,\dots,\nabla f_{m-2},\nabla g_1,\dots,\nabla g_{m-2}): M\to\R.\]

\parbf{Remarks.}
Note that 
$$\Qm(X_1,\dots,X_{m-2},X_1,\dots,X_{m-2})
=
|X_1\wedge\dots \wedge X_{m-2}|^2\cdot K_\sigma, $$
where $K_\sigma$ is the sectional curvature of $M$ 
on a plane $\sigma$ orthogonal to $(m-2)$-vector
$X_1\wedge\zz\dots \wedge X_{m-2}$.
Hence, the sectional curvatures of $M$ and therefore its curvature tensor $\Rm$ can  be computed from
$\qm$.
By the symmetry
$$\qm(f_1,\dots,f_{m-2},g_1,\dots,g_{m-2})=
\qm(g_1,\dots,g_{m-2},f_1,\dots,f_{m-2}),$$
the density of $\qm$ is defined by the sectional curvature.
Therefore measure tensor $\qm$
gives an equivalent description of curvature of Riemannian manifold.

As you will see further, the described dual form of curvature tensor behaves better in the limit;
in particular, it makes it possible to formulate \ref{prop:3parts:codim2+}.

In the 2-dimensional case, the valence of $\qm$ is $0$;
in this case, $\qm$ coincides with the curvature measure --- the standard way to describe the curvature of surfaces, 
\cite{Resh,AZ}.
For a smooth surface, the density of this curvature measure with respect to the area
is its Gauss curvature.
In this case, it is known that \emph{curvature measures are stable under smoothing} \cite[VII \S13]{AZ};
in other words, our main theorem is known in the two-dimensional case.
