\section{Localization}\label{sec:local}

In this subsection we formulate a local version of the main theorem.
This version is more general, but its proof requires just a slight change of language.
A couple of times we had to use this local version in the proof.
In a perfect world, we had to rewire the whole paper using this language.
However, this is not a principle moment,
so we decided to keep the paper more readable at the cost of being not fully rigorous.
A more systematic discussion of this topic is given in \cite{LNep}.

First, we need to define Alexandrov region;
its main example is an open set in Alexandrov space.

\begin{thm}{Definition}
Let $A$ be a locally compact metric space. 
We say that a point  $p\in A$
is  \emph{ $\eps$-inner point} if
the closed ball $\bar B(x,2\cdot\eps)$ is compact.
\end{thm}

\begin{thm}{Definition}
We say that a locally compact inner metric space $A$ of finite Hausdorff dimension is an \emph{Alexandrov region} if any point has a neighborhood where the Alexandrov comparison for curvature $\ge -1$ holds.

The \emph{comparison radius} $r_c(p)$ for $p\in A$ is defined as the maximal number $r$ such that $p$ is $r$-inner point and Alexandrov comparison for curvature $\ge -1$ holds in $B(x,r)$.
\end{thm}

Any point $p$ in an Alexandrov region admits a convex neighborhood.
Moreover, its size can be controlled in terms of dimension, $r_c(p)$, and a lower bound on the volume of ball $B(p,r_c)$.
The construction is the same as for Alexandrov space \cite[4.3]{perelman-petrunin}.

By the globalization theorem (see, for example, \cite{AKP}), a compact convex subset in an Alexandrov region is an Alexandrov space.
So the statement above makes it possible to apply most of the arguments and constructions for Alexandrov spaces to Alexandrov regions. 
Moreover, in the case when an Alexandrov region is a Riemannian manifold (possibly noncomplete) it is possible to take the doubling of a convex neighborhood from the proposition and smooth it with almost the same lower curvature bound.
This allows us to apply the main result from \cite{petrunin-SC}, where the complete manifold can be replaced by a convex domain in a possibly open manifold. 

Further, let us define a local version of smoothing.
Let us denote by
$\M_{\ge -1}^m$ a class of $m$-dimensional Riemannian 
manifolds without boundary, but possibly non-complete, with sectional curvature bounded
from below by $-1$.

\begin{thm}{Definition}
Let 
$M_n\in\M_{\ge -1}^m$ (with corresponding intrinsic metric)
converge in Gromov--Hausdorff sense to some metric space $A$ via
approximation.
Suppose that $M_n\ni x_n\to x\in A$
and $r_c(x_n)\zz\ge c_0>0$. Let $c_{conv}(m)c_0 > R>0$ and set
$U_n=B(x_n,R)_{M_n}$.
Then we say that $U_n$ is a local smoothing of $U=B(x,R)_A$ (briefly, $U_n\smooths{} U$).
\end{thm}

It is straightforward to redefine test functions and weak convergence for local smoothings.
Using this language we can make a local version for each statement in this paper, the proofs go without changes.
As a result, we get the following local version of the main theorem \ref{main}.
 
\begin{thm}{Local version of the main theorem}\label{mainloc}
Let   
$M_n\in\M_{\ge -1}^m$,
$M_n\GHto A$, $U_n\subset M_m$,
  $U\subset A$, and $U_n\smooths{} U$ be a local smoothing.
  
Denote by $\qm_n$ the dual measure-valued curvature tensor on $U_n$.
Then there is a measure-valued tensor $\qm$ on $U$ such that $\qm_n\rightharpoonup \qm$.
\end{thm} 
