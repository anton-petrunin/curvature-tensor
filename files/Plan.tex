\section{Regular points}\label{sec:ref}

\subsection{Common chart and delta-convergence}

Choose a smoothing $M_n\smooths{} A$ of an $m$-dimensional Alexandrov space $A$.
Let $p\in A$ be a point of rank $m$; that is, there are $m+1$ points $a_0,\dots, a_m\in A$ such that 
$\angk p{a_i}{a_j}>\tfrac\pi2$ for all $i\ne j$.

Recall \cite[Sec. 7]{petrunin-conc} that we can choose a small $r>0$,
finite set of points $\bm{a}_i$ near $a_i$,
and a smooth concave increasing real-to-real function $\phi$ defined on an open interval such that 
\[f_i=\sum_{x\in \bm{a}_i}\phi\circ\widetilde{\dist}_{x,r}\]
is strongly concave function that is defined in a neighborhood $U\ni p$.

Since $r$ is small and $\bm{a}_i$ is near $a_i$ we get that the functions $f_0, \dots, f_m$ are tight in $U$; see the definition in \cite{petrunin-conc}.
In particular, the map $U\to\RR^m$ defined by $x\mapsto (f_1(x),\dots,f_m(x))$ is a coordinate system in $U$.

The presented construction can be lifted to $M_n$.
As a result we obtain a chart of an open set $U_n\subset M_n$.
Passing to smaller sets we may assume that $U$ and each $U_n$ is mapped to a fixed open set $\Omega\subset\RR^m$ for all large $n$.
Further, we assume that it holds for all $n$; it could be achieved by cutting off the beginning of the sequence $M_n$.

The obtained collection of charts $\bm{x}_n\:U_n\to \Omega$ and $\bm{x}\:U\to \Omega$ will be called \emph{common chart} at $p$.
It will be used to identify points of $\Omega$, $M_n$ and $A$; 
in addition we will use it to identify the tangent spaces $\T M_n$ and $\T A$ with $\RR^m$.
For example, we will use the same notation for function $M_n\to \RR$ and its composition $\Omega\to \RR$ with the inverse of the chart $U_n\to \Omega$.
We will use index $n$ or skip it to indicate that the calculations are performed in $M_n$ or $A$ respectively.
For example, given a function $f\:\Omega\to \RR$, we denote by $\nabla_nf$ and $\nabla f$ the gradients of $f\circ \bm{x}_n$ in $M_n$ and $f\circ \bm{x}$ in $A$ respectively.

Recall that $A^\delta$ denotes the set of $\delta$-strained points in $A$.
For a fixed common chart $\bm{x}$ we will use notation $A^\delta_\Omega$ for the image $\bm{x}(A^\delta)\subset \Omega$.

The part \ref{prop:3parts}(\ref{prop:3parts:reg}) will be reduced to certain estimates in one common chart.

\begin{thm}{Definitions}\label{def:delta-converge}
Let $M_n\smooths{} A$, $\dim A=m$;
choose a common chart with range $\Omega\subset \RR^m$.

A sequence of measures $\mathfrak n_n$ defined on $\Omega$ is called \emph{weakly delta-converging}
if the following conditions hold:
\begin{itemize}
 \item Every subsequence of $\mathfrak n_n$ has a weak partial limit.
 \item For any $\eps>0$ there is $\delta>0$ such that for any two weak partial limits $\mathfrak m_1$ and $\mathfrak m_2$ of $(\mathfrak n_n)$ we have  
\[|(\mathfrak m_1-\mathfrak m_2)(S)|<\eps\]
for any Borel set $S\subset A^\delta_\Omega$.
\end{itemize}

A sequence of functions $f_n$ defined on $\Omega$ is called \emph{uniformly delta-converging}
if the following conditions hold:
\begin{itemize}
 \item For any $\eps>0$ there is $\delta>0$ such that such that 
\[\limsup_{n\to\infty} \{f_n(x)\}-\liminf_{n\to\infty}\{f_n(x)\}<  \eps\]
for any $x\in A^\delta_\Omega$.
\end{itemize}

\end{thm}

\begin{thm}{Observation}\label{obs:delta-weak-uniform}
If $f_n$ is uniformly delta-converging and $\mathfrak n_n$ is weakly delta-converging,
then $f_n\cdot \mathfrak n_n$ is weakly delta-converging.
\end{thm}





%The following proposition lists the properties of the common charts.

%\begin{thm}{ Proposition}\label{Prop:chart}
%Let $M_n\smooths{} A$, $\dim A=m$, and
%$\bm{x}_n\: U_n\zz\to \Omega$ and $\bm{x}\: U\zz\to \Omega$ is a common chart.
%
%\begin{enumerate}[(a)]
%
%\item\label{obtuse}
%The coordinates functions $x_i$ are tight concave function in $M_n$ and $A$;
%in addition, $x_i$ smooth in $M_n$.
%
%\item The charts are bi-Lipschitz  with uniform  bi-Lipschitz 
%constants.
%
%\item\label{metric} 
%Denote by $S_\Omega$ the image of singular locus of $A$ in $\Omega$.
%There exists a continuous Riemannian metric
%$g$ on $\Omega\setminus S_\Omega$ which locally 
%realizes distance on $A$.
%Moreover, $g$ has bounded variation
%on $\Omega$. 
%
%\item\label{metricseq}
%$(g_{ij,n}, g_{ij})\in BV_0^{seq}(\Omega, S_\Omega)$, where $g_n$ is the metric tensor of $M_n$ on $\Omega$.
%Moreover, $\det(g_n)$ is bounded and bounded away from 0.
%
%
%\item\label{funktioninchart}
%For any
%sequence
%of concave smooth functions
%$f_n:B(p_n,r_n)\to\R$ 
%which roughly $C^1$-converges to  $f:B(p,r)\to\R$ we have for its coordinate
%expression
%$(f_n\circ\bm{x}^{-1}, f\circ\bm{x}^{-1})\in DC_0^{seq}(\Omega, S_\Omega)$
%
%\end{enumerate}
%\end{thm}

\subsection{Test convergence}

The following lemma relies on DC-calculus that is discussed Section~\ref{sec:DC}.

\begin{thm}{Lemma}\label{lem:test-delta}
Let $M_n\smooths{} A$, $\dim A=m$;
choose a common chart with range $\Omega\subset \RR^m$.
Let $f_n\: M_n\to\RR$ be a sequence of test function such that $f_n\zz\testto f\:A\to \RR$.
Let us denote by $\partial_1,\dots,\partial_m$ the partial derivatives on $\Omega\subset \RR^m$.
Denote by $g_{ij,n}$ and $g^{ij}_n$ the components of the metric tensors on $M_n$.
Then 
\begin{enumerate}[(i)]
\item\label{lem:test-delta-f} $f_n$ uniformly converges to $f$ on $\Omega$;
\item\label{lem:test-delta-partial} $\partial_if_n$ are uniformly delta-converging;
\item\label{lem:test-delta-g}  $g_{ij,n}$ and $g^{ij}_n$ are uniformly delta-converging for all $i,j$;
moreover, $\det g_{ij,n}$ is bounded away from zero;
\item\label{lem:test-delta|nabla|} $|\nabla_n f_n|$ uniformly delta-converges on $\Omega$; %???do we need it???
\item\label{lem:test-delta-partial-g} the partial derivatives $\partial_kg_{ij,n}$, $\partial_k g^{ij}_n$,  $\partial_j\partial_if_n$, as well as their products to uniformly delta-converging functions,  are weakly converging.
\end{enumerate}

\end{thm}

\parit{Proof.} Part (\ref{lem:test-delta-f}) is trivial.

\parit{(\ref{lem:test-delta-partial}).}
Note that for any point $p\in A$ and almost all points $y,z\in A$ the geodesics $[py]$ and $[pz]$ are uniquely defined.
In this case for any sequence of points $p_n,y_n,z_n\in M_n$ such that $p_n$, $y_n$, and $z_n$ converge to $p$, $y$, and $z$ respectively we have
\[\lim_{n\to\infty}\measuredangle \hinge{p_n}{y_n}{z_n}\ge \measuredangle \hinge{p}{y}{z}.\]

If $\T_p$ is Euclidean, then we can choose a point $x\in A$ so that $[px]$ is unique, and the sum 
\[\measuredangle \hinge{p}{x}{y}+\measuredangle \hinge{p}{y}{z}+\measuredangle \hinge{p}{z}{x}\]
is arbitrarily close to $2\cdot\pi$.
Applying the comparison, we get that $\lim\measuredangle \hinge{p_n}{y_n}{z_n}$ is arbitrary close to $\measuredangle \hinge{p}{y}{z}$; therefore
\[\measuredangle \hinge{p_n}{y_n}{z_n}\to\measuredangle \hinge{p}{y}{z}\]
as $n\to\infty$.
The latter implies that
\[\langle d_p\widetilde\dist_{y_n,r},d_p\widetilde\dist_{z_n,r}\rangle\to \langle d_p\widetilde\dist_{y,r},d_p\widetilde\dist_{z,r}\rangle\]
for \emph{any} choice of sequences $p_n$, $y_n$, and $z_n$ that converge to $p$, $y$, and $z$ respectively.
Hence,
\[\langle d_pf_n,d_ph_n\rangle\to \langle d_pf,d_ph\rangle\]
if $f_n,h_n\:M_n\to\RR$ are sequences of test functions such that $f_n\testto f$ and $h_n\testto h$.

Note that the partial derivatives $\partial_if_n$ at a regular point $p$ can be expressed in terms of $\langle d_pf_n,d_px_j\rangle_n$ and $\langle d_px_j,d_px_k\rangle_n$, where $x_1,\dots,x_m$ are the coordinate functions of the chart.
Therefore, we get that $\partial_if_n$ converge at any regular point.

Finally, observe that if $p$ is a $\delta$-strained point for sufficiently small $\delta>0$,
then the calculations above go thru with a small error.
Whence the statement follows.

\parit{(\ref{lem:test-delta-g}).} This part follows from the proof of (\ref{lem:test-delta-partial}) since $g^{ij}_n=\langle d_px_j,d_px_k\rangle_n$ and $g_{ij,n}$ can be expressed thru~$g^{ij}_n$.

\parit{(\ref{lem:test-delta|nabla|}).}
Note that $|\nabla_n f_n|$ can be expressed from $g^{ij}_n$ and $\partial_if_n$.
Since these quantities delta-converging, so is $|\nabla_n f_n|$.

\parit{(\ref{lem:test-delta-partial-g}).} The weak convergence of $\partial_kg_{ij,n}$, $\partial_k g^{ij}_n$, and $\partial_j\partial_if_n$
follows from \ref{metricBV}.
By Observation~\ref{obs:delta-weak-uniform}, the products of these partial derivatives to uniforly delta-converging sequences of functions is weakly delta-converging.

By \ref{metricBV}, the limit measures of $\partial_kg_{ij,n}$, $\partial_k g^{ij}_n$, and $\partial_j\partial_if_n$ have vanishing $(m-1)$-density.
Further, a limit of delta-converging sequence is might be discontinues only at the singular set which has vanishing $(m-1)$-Hausdorff measure.
It follows that the measure $\mathfrak{m}_1-\mathfrak{m}_2$ in the definition of weak delta-convergence is vanishing. 
Whence the weak convergence follows.
\qeds

\subsection{Proof modulo key lemma}

\begin{thm}{Key lemma}\label{A^0}
Choose a common chart with range $\Omega\subset \RR^m$ for a smoothing $M_n\smooths{} A$.
Choose a component $\Rm_{ijsr,n}$ of the curvature tensor of $M_n$ in $\Omega$.
Then $\Rm_{ijsr,n}\cdot \vol^m_n$ is a weakly delta-converging sequence of measures.
\end{thm}

The proof of the key lemma will take the remaining part of this section;
in the current subsection we show that it implies \ref{prop:3parts}(\ref{prop:3parts:reg}).

\parit{Proof of \ref{prop:3parts}(\ref{prop:3parts:reg}) modulo \ref{lem:test-delta} and \ref{A^0}.}
Recall that components of $\qm_n$ can be expressed from the components of $\Rm_n$.
Therefore, the key lemma implies delta-convergence of components of $\qm_n$.

Choose sequences of test functions $f_{1,n},\dots,f_{m-2,n},h_{1,n},\zz\dots,h_{m-2,n}$ on $M_n$ that test-converge to $f_{1},\dots,f_{m-2},h_{1},\dots,h_{m-2}\:A\to \RR$.
By \ref{lem:test-delta}, we have delta-convergence of the partial derivatives $\partial_if_{j,n}$ and 
$\partial_ih_{j,n}$ to $\partial_if_{j}$ and 
$\partial_ih_{j}$ respectively.
The measures $\qm_n(f_{1,n},\dots,f_{m-2,n},h_{1,n},\zz\dots,h_{m-2,n})$ 
can be expressed as a linear combination of the components of $\qm_n$ with coefficients that expressed in terms of $\partial_if_{j,n}$.
By \ref{obs:delta-weak-uniform}, it follows that the sequence of measures 
\[\mathfrak m_n=\qm_n(f_{1,n},\dots,f_{m-2,n},h_{1,n},\zz\dots,h_{m-2,n})\]
is delta-converging.

Finally, recall that 
\[A^\circ=\bigcap_{\delta>0}A^\delta.\]
Therefore delta-convergence of $\qm_n(f_{1,n},\dots,f_{m-2,n},h_{1,n},\zz\dots,h_{m-2,n})$ implies \ref{prop:3parts}(\ref{prop:3parts:reg}).
\qeds
