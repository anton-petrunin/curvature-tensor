\section{Regular points}\label{sec:ref}

\subsection{Common chart and delta-convergence}

Choose a smoothing $M_n\smooths{} A$ of an $m$-dimensional Alexandrov space $A$.
Let $p\in A$ be a point of rank $m$; that is, there are $m+1$ points $a_0,\dots, a_m\in A$ such that 
$\angk p{a_i}{a_j}>\tfrac\pi2$ for all $i\ne j$.

Recall that we can choose a small $r>0$ and finite set of points $A_i$ near $a_i$ and a smooth concave increasing real-to-real function $\phi$ defined on an open interval such that the function
\[f_i=\sum_{x\in A_i}\phi\circ\widetilde{\dist}_{a_i,r}\]
is defined in a neighborhood $U\ni p$ and strongly concave.

Since $r$ is small and $A_i$ is near $a_i$ we get that the functions $f_0, \dots, f_m$ are tight in $U$.
In particular, the map $U\to\RR^m$ defined by $x\mapsto (f_1(x),\dots,f_m(x))$ is a coordinate system in $U$.

The presented construction can be naturally lifted to $M_n$.
As a result we obtain a chart of an open set $U_n\subset M_n$.
Passing to smaller sets we may assume that $U$ and each $U_n$ is mapped to a fixed open set $\Omega\subset\RR^m$ for all large $n$.
Further we assume that it holds for all $n$; it could be achieved by cutting off the beginning of the sequence $M_n$.

The obtained collection charts $\bm{x}\: U_n\to \Omega$ and $\bm{x}\: U\to \Omega$ will be called \emph{common chart} at $p$.

The part \ref{prop:3parts}(\ref{prop:3parts:reg}) will be reduced to certain estimates in one common chart that will be used to identify points of $\Omega$, $M_n$ and $A$.
We will use index $n$ or skip it to indicate that the calculations are performed in $M_n$ or $A$ respectively.
For example, given a function $f\:\Omega\to \RR$, we denote by $\nabla_nf$ the gradient of the function $f\circ \bm{x}$ in $M_n$.

Recall that $A^\delta$ denotes the set of $\delta$-strained points in $A$.
For a fixed common chart $\bm{x}$ we will use notation $A^\delta$ also for the image $\bm{x}(A^\delta)$.

\begin{thm}{Definitions}\label{def:delta-converge}
Choose a common chart with range $\Omega\subset \RR^m$ for a smoothing $M_n\smooths{} A$.

A sequence of measures $\mathfrak n_n$ defined on $\Omega$ is called \emph{weakly delta-converging}
if the following conditions hold:
\begin{enumerate}
 \item Every subsequence of $\mathfrak n_n$ has a weak partial limit.
 \item For any $\eps>0$ there is $\delta>0$ such that for any two weak partial limits $\mathfrak m_1$ and $\mathfrak m_2$ of $\mathfrak n_n$ we have  
\[|(\mathfrak m_1-\mathfrak m_2)(S)|<\eps\]
for any Borel set $S\subset \Omega\cap A^\delta$.
\end{enumerate}

A sequence of functions $f_n$ defined on $\Omega$ is called \emph{uniformly delta-converging}
if the following conditions hold:
\begin{enumerate}
 \item Every subsequence of $f_n$ has a partial uniform limit.
 \item For any $\eps>0$ there is $\delta>0$ such that for any two uniform partial limits $h_1$ and $h_2$ of $f_n$  such that 
\[|h_1(x)-h_2(x)|<  \eps\]
for any $x\in \Omega\cap A^\delta$.
\end{enumerate}

\end{thm}

Observe that if $f_n$ is uniformly delta-converging and $\mathfrak n_n$ is weakly delta-converging,
then $f_n\cdot \mathfrak n_n$ is weakly delta-converging.

The proof of \ref{prop:3parts}(\ref{prop:3parts:reg}) relies on the following lemma.

\begin{thm}{Lemma}\label{A^0}
Choose a common chart of a smoothing $M_n\smooths{} A$ and a component $\Rm_{ijsr,n}$ the curvature tensor of $M_n$.
Then $\Rm_{ijsr,n}\cdot \vol$ is a weakly  delta-converging sequence of measures.
\end{thm}

The following proposition lists the properties of the common charts.

\begin{thm}{ Proposition}\label{Prop:chart}
Suppose $\bm{x}_n\: U_n\to \Omega$ and $\bm{x}\: U\to \Omega$ is a common chart of a smoothing $M_n\smooths{} A$.

\begin{enumerate}[(a)]

%\addtocounter{enumi}{2}
\item\label{obtuse}
The coordinates $x_i$ are tight concave function in $M_n$ and $A$;
in addition, $x_i$ smooth in $M_n$.

\item The charts are bi-Lipschitz  with uniform  bi-Lipschitz 
constants.

\item\label{metric} 
Denote by $S_\Omega$ the image of singular locus of $A$ in $\Omega$.
There exists a continuous Riemannian metric
$g$ on $\Omega\setminus S_\Omega$ which locally 
realizes distance on $U$, where
$S_\Omega=\mathfrak X(U\cap A\setminus A^\circ)$.
This metric tensor 
(defined almost everywhere on $\Omega$)
is of bounded variation
on $\Omega$. 

\item\label{metricseq}
Let $g_{ij,n}$ be coordinates of metric tensor of $M_n$ 
in the chart $\X_n$. 
Then
$(g_{ij,n}, g_{ij})\in BV_0^{seq}(\Omega, S_\Omega)$.
Moreover, $\det(g_{ij,n})$ is bounded and bounded away from 0.


\item\label{funktioninchart}
For any
sequence
of concave smooth functions
$f_n:B(p_n,r_n)\to\R$ 
which roughly $C^1$-converges to  $f:B(p,r)\to\R$ we have for its coordinate
expression
$(f_n\circ\X_n^{-1}, f\circ\X^{-1})\in DC_0^{seq}(\Omega, S_\Omega)$

\end{enumerate}
\end{thm}
