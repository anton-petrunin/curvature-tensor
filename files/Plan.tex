\section{Regular points}\label{sec:ref}

\begin{thm}{Definitions}\label{def:delta-converge}
Consider a smoothing $M_n\smooths{} A$.

A sequence of measures $\nu_n$ defined on $M_n$ is called \emph{weakly $\delta$-converging}
if the following conditions hold:
\begin{enumerate}
 \item Every subsequence of $\nu_n$ has a weak partial limit.
 \item For any two weak partial limits $\mu_1$ and $\mu_2$ of $\nu_n$ and any bounded set $\Omega\subset A$ there is a constant $\const$ such that 
\[|(\mu_1-\mu_2)(S)|\le \const\cdot \delta\]
for any Borel set $S\subset \Omega\cap A^\delta$.
\end{enumerate}

A sequence of functions $f_n$ defined on $M_n$ is called \emph{uniformly $\delta$-converging}
if the following conditions hold:
\begin{enumerate}
 \item Every subsequence of $f_n$ has a partial uniform limit.
 \item For any two uniform partial limits $h_1$ and $h_2$ of $f_n$ and any bounded set $\Omega\subset A$ there is a constant $\const$ such that 
\[|h_1(x)-h_2(x)|\le\const\cdot \delta\]
for any $x\in \Omega\cap A^\delta$.
\end{enumerate}

\end{thm}

Observe that if $f_n$ is uniformly $\delta$-converging and $\nu_n$ is weakly $\delta$-converging,
then $f_n\cdot \nu_n$ is weakly $\delta$-converging.

Since 
\[A^\circ=\bigcap_{\delta>0}A^\delta,\]
part \ref{prop:3parts:reg} of \ref{prop:3parts} follows from the next lemma.

\begin{thm}{Lemma}\label{A^0}
Let $M_n\smooths{} A$ and $\lambda,L\in\RR$.
Suppose $h_n\:M_n\to\RR$ be a sequence of $\lambda$-concave $L$-Lipscitz smooth functions that converges to a function $h\:A\to R$.
Then the sequence of measures $q_{M_n}(\nabla h_n,\nabla h_n)$ is \emph{$\delta$-converging}.
\end{thm}

The 3-dimensional case is the main part of its the proof.
First, we introduce nice chart --- a common parametrisation of the limit space and approximating sequence.
Using nice charts, we prove $\delta$-convergence of the so called \emph{strange curvature tenor}.
Here we essentially use a Bochner-type formula that gives an expression for strange curvature \ref{thm:bochner-formula}.
Further we show that $\delta$-convergence of strange curvature implies $\delta$-convergence of dual curvature tensor.

The proof in higher dimensions is done by induction on dimension.
It relies on a local version of the main theorem described in Section~\ref{sec:local}.
This version of theorem deals with a minor technical issue that the formulation of main theorem is global --- it use test functions and their convergence that changes if we pass to a smaller domain. 
