\section{Regular points}

\subsection{Partition into three subsets}
 Let   
$M_n\in\M_{\ge -1}^m$,
$M_n\GHto A$ 
and  $U\subset A$ is a good domain for this convergence.
Because of the symmetry 
 of curvature tensor it is sufficient
to  prove  the weak convergence 
of measures 
$\mathfrak{q}_{M_n}(f_{1,n},\dots,f_{m-2,n},f_{1,n},\dots,f_{m-2,n})$
with densities
$$\omega_n
=
{q}_{M_n}(f_{1,n},\dots,f_{m-2,n},f_{1,n},\dots,f_{m-2,n})
\:M_n\to\RR$$
for any collection of  sequences
$f_{i,n}\in C^1(M_n)$,  $i\in\{1,2,\dots,m-2\}$,
such that
$f_{i,n}\ccto f_i  \in rC^1(A) $. Let us fix such a collection
and keep the notation $\omega_n$ for densities above.
Recall that
$${q}_{M_n}(f_{1,n},\dots,f_{m-2,n},f_{1,n},\dots,f_{m-2,n})(x)=
|\nabla f_{1,n}(x)\wedge\dots \wedge\nabla f_{m-2,n}(x)|^2\cdot K_\sigma, $$
where $K_\sigma $ is the sectional curvature at $x\in M_n$
on a plane orthogonal to 
$\nabla f_{1,n}(x)\wedge\dots \wedge\nabla f_{m-2,n}(x)$.
Since gradients are bounded the
Corollary~\ref{Kbound} 
implies that
 for any point $p\in M_n$ and $r<1$,
$$\int\limits_{B_r(p)}|\omega_n|d\vol_n\le \const\cdot r^{m-2},\eqlbl{sc-pet}$$
i.e.  locally measures  $\omega_nd\vol_n$ are uniformly bounded.
This boundness of measures
is the crucial point for our proof.
Firstly this makes possible
 passing to a subsequence of $M_n$ such that
measures  $\omega_nd\vol_n$  weakly converge.
Now
let $\mathfrak{r}_1, \mathfrak{r}_2\in\mathfrak M(A)$ be two weak partial limits, we want to show that $\mathfrak{r}_1=\mathfrak{r}_2$.
To do this
we partition $A$ into three subsets $A^\circ$, $A'$ and $A''$ and separately prove that $\mathfrak{r}_1$ and $\mathfrak{r}_2$ coincide on each of these subsets (claims~\ref{A''}, \ref{A'} and \ref{A^0}).
The partition is constructed as follows
\begin{enumerate}
\item $A^\circ=\set{x\in A}{\T_x\ \text{is isometric to}\  \RR^m}$ --- the set of regular points.
\item $A'=\set{x\in A\backslash A^{\circ}}{\T_x\ \text{is isometric to}\  \RR^{m-2}\times \Cone^2}$ the set of singular points of codimension two.
\item $A''$ --- the rest, i.e. $A''=A\backslash (A^{\circ}\cup A')$ the set of singular points of codimension three and larger.
\end{enumerate}


 Proofs for $A'$ and $A''$ (for exact formulation see \ref{A''}, \ref{A'})
 are self-contained and don't differ  for different dimensions.
  For the set  $A^\circ$  we use a sort of induction (for plan see \ref{sec:A0} ).
  
  Passing to partial limits
 $\mathfrak{r}_1$, $\mathfrak{r}_2$
is convenient for formulations and on the last step of the proof. In the proof we don't pass to
convergent subsequences, but verify weak convergence of
measures directly.
While there is still
the point (in the proof of Lemma~\ref{A^0}) where we strongly rely on uniform boundness 
of measures: we prove convergence of integrals 
not for all continuous functions as a test functions, but
only for
smooth functions, i.e. we prove convergence in the sense 
of distributions. Then we  obtain weak
convergence  provided by the uniform boundedness of measures.   




\subsection{The set $A''$ of  singularities of higher codimension }
According to \cite[10.6]{BGP}, the $(m-2)$-hausdorff measure of $A''$ vanishes.
Thus, from \ref{sc-pet}, we get

\begin{thm}{Claim}\label{A''}
$\mathfrak{r}_1|_{A''}\equiv\mathfrak{r}_2|_{A''}\equiv0$.
\end{thm}

\subsection{The set $A'$ of codimension 2 singularities  }
Consider function $\omega\:A\to\RR$ defined as
$$\omega(p)=2\cdot\pi\cdot\l(1-\frac{\vol\Sigma_p}{\vol\SS^{m-1}}\r).$$
%Clearly $\omega|_{A^\circ}\equiv0$, further 
According to \cite[7.14]{BGP}, the
function $\omega\:A\to\RR$ is lower-semicontinuous.
If $p\in A'$, i.e. $\T_p=\RR^{m-2}\times \Cone^2_p$, then the total angle of $\Cone^2_x$ is equal to $2\pi-\omega(x)$.

Remind that we have a  collection of 
sequences $f_1^n,f_2^n,\dots,f_{m-2}^n$,  $C^1$-converging in Alexandrov sense.
For $x\in A''$ the tangent cone splits as
$\T_xA=\RR^{m-2}\times \Cone$.  
$C^1$-convergence implies in particular 
that 
in some sense there exists  a "limit projection" of $\nabla f_i^n$ onto $\RR^{m-2}$-factor for all $i\in \{1, 2, \dots, m-2\}$
(for exact statement see ??). We denote these projections by
$(\nabla f_i)^\bot(x)\in\RR^{m-2}\times \{O\}\subset T_xA$
and define $\theta_x(f_1,f_2,\dots,f_{m-2})=
|(\nabla f_1)^\bot \wedge (\nabla f_2)^\bot\wedge\dots
  \wedge (\nabla f_{m-2})^\bot|^2$. 

We denote by $h_\alpha$ the $\alpha$-dimensional Hausdorff measure. Measures
$\mathfrak{r}_1$ and
$\mathfrak{r}_2$ are
 absolutely continuous with respect to
 $h_{m-2}$ (this follows from \ref{sc-pet}).
Moreover,
densities of
 $\mathfrak{r}_1$ and
$\mathfrak{r}_2$
coincide on $A'$ and equal 
$ \omega(x)\cdot\theta_x(f_1,f_2,\dots,f_{m-2})$:
\begin{thm}{Claim}\label{A'}
 
For any subset $X\subset A'$
 
 $$\int_Xd\mathfrak{r}_1=\int_Xd\mathfrak{r}_2=\int_X
\omega(x)\cdot\theta_x(f_1,f_2,\dots,f_{m-2}) d h_{m-2} $$
 
  
 %$\mathfrak{r}_1|_{A'}=\mathfrak{r}_2|_{A'}=\omega(x)\cdot\theta_x(f_1,f_2,\dots,f_{m-2})\cdot h_{m-2}|_{A'}$.
\end{thm}

Firstly we prove this claim in the particular case, when
the limit Alexandrov space is a cone of the form $A=\Cone\times\RR^{m-2}$. Then for general $A$ we blow up the 
neighborhood of  $x\in A'$ and apply result for a cone.

\subsection{The set $A^\circ$  of regular points }\label{sec:A0}
This part of the proof is the most technical.
We recall that the set of regular points can be presented as
$$A^{\circ}=\bigcap_{\delta>0} A^\delta,$$
where $A^\delta$ denotes the set of $\delta$-strained points of $A$.
Then to show that
$\mathfrak{r}_1|_{A^o}\equiv\mathfrak{r}_2|_{A^o}$
it is enough to show the following

\begin{thm}{Lemma}\label{A^0}
Let $\nu=|\mathfrak{r}_1-\mathfrak{r}_2|$.
Then for any $x\in A^\circ$, there is a neighbourhood $U_x\ni x$ such that
$$\nu({A^\delta\cap U_x})\le \const\cdot\delta$$
for some fixed $\const\in\RR$, independent of $\delta$.
\end{thm}


We prove firstly Lemma~\ref{A^0} for dimension 3. In the proof
we use the main theorem (local) for dimension 2. 
 Then we go through the proof and obtain the main theorem (local) for dimension 3.
 Then we reduce higher dimensional case of Lemma~\ref{A^0} to 
   the main theorem (local) for dimension 3. 
    
 
%Let note that the lemma asserts 
%weak $ c\delta$-convergence 
%of measures with densities $r_n$ on $A^\delta\cap U_x$ in the sense
%of definition~\ref{deltaconv2},  but  in the proof we don't pass to patial limits and verify convergence for smooth function as a test function (Lemma~\ref{smoothmeasure}).
 
 

  Now we sketch the proof of this lemma in dimension 3. 
 For the proof
 we introduce 
 a new tensor 
 for Riemannian manifold $M$ and call it
 Strange curvature:  $Str(w,w)=\Sc\cdot |w|^2-Ric(w,w)$, $w\in TM$.
 In 3-dimensional case
 Strange curvature
  completely defines curvature tensor
  and we reduce the proof of convergence of curvature tensor
  on $A^\delta$ to the convergence of Strange curvature.
 For the proof of this convergence we use 
 expression for Strange curvature for Riemannian 
 manifold,
namely
 a Bochner-type formula \ref{Bochner}.

To prove convergence of
integrals in this
 formula  we use
special charts with common domain $\Omega$ for all the sequence and the limit space
 around a  point $x\in A^0$.
These charts are special
diffeomorphisms
$\mathfrak X_n:U_n\to\Omega\subset\R^3$, $U_n\subset M_n$ and
homeomorphism
$\mathfrak X:U_x\to\Omega\subset\R^3$, $U_x\subset A$.
Here we  generalize methods introduced in
\cite{PerDC}.  In  this paper G. Perelman 
constructed $DC$-atlas ($DC$=difference concave)
for Alexandrov space and developed calculus for
$DC$-functions. We adjust these for a sequence
of manifolds converging to Alexandrov space.



 
The proof of Lemma~\ref{A^0} for higher dimensions
relies on the main theorem (local version) for dimension 3.



\subsection{Locally defined Alexadrov space structure }

\begin{rdef}{Definition}
Let $A$ be a metric space. 
We say that an open  $U\subset A$
is a \emph{strongly inner domain} if
for some 
$R>0$ we have
$U\subset B_R(x)$
and $\overline{B_{10R}(x_n)}$ is compact.

\end{rdef}

\begin{rdef}{Definition}
We say that
 a locally compact inner metric space $A$
is an \emph{ Alexandrov region}
any point has a neighborhood where Alexandrov
comparison for curvature $\ge -1$ holds.
We say that $U\subset A$
is an \emph{ Alexandrov domain}
 if $U$ is 
 a strongly inner domain.

\end{rdef}

It is possible to show  that 
Toponogov comparison holds for sets with local structure of Alexandrov space and most of arguments and constructions for Alexandorov spaces could be applied to Alexandrov domain.
In particular we need main result from
 \cite{petrunin-SC}, where complete manifold can be replaced by 
strongly inner domain in a possibly open manifold. 
 
 
 
 Next we describe smoothing for Alexandrov domains.
 We denote by
$\M_{\ge -1}^m$ a class of $m$-dimensional Riemannian 
manifolds without boundary, but possibly non-complete, with sectional curvature bounded
from below by $-1$.

\begin{rdef}{Definition}
Let sequence
$M_n\in\M_{\ge -1}^m$ (with corresponding intrinsic metric)
converges in Gromov--Hausdorff sense to some metric space $A$.
We say that an open $U\subset A$ is \emph{good} with respect to this convergence
 if
$\dim U=m$ and for some 
$R>0$, $M_n\ni x_n\to x\in A$ we have
$U\subset B_R(x)$
and ${B_{R}(x_n)}$ is a strongly inner domain in
$ M_n$
. 
\end{rdef}
 Let note that in this case $B_{10R}(x)$ is Alexandrov region
 and $U$ is an Alexandrov domain.
 
 We will say that open $U_n\subset M_n$
approximates good subset $U$ and write $U_n\to U$
if $U_n\GHto U$ and for any $p\in U$ there is $M_n\ni p_n\to p$
and $r>0$ such that $B_r(p_n)\subset U_n$.
 
 \subsection{Rough $C^1$ structure }\label{sec:rC}
 
 
Let $f$ be a continuous semiconcave function defined on an open subset of Alexandrov space $A$.
Given $p\in \Dom f$, set 
\begin{align*}
\mathfrak{a}_f(p)&=\limsup_{x\to p}\frac{f(x)-f(p)}{|x-p|},
&
\mathfrak{b}_f(p)&=\limsup_{x\to p}\frac{f(p)-f(x)}{|p-x|}.
\end{align*}
Note that $\mathfrak{a}_f(p)=|\nabla_pf|$.



Recall that a function $f$ defined on Alexandrov space is called differentable at $p\in \Dom f$ if its differential at $p$ is a linear function;
that is, $d_pf\:\T_p\to\RR$ is concave and convex at the same time.

\begin{thm}{Claim} Let $f$ be a continuous semiconcave function defined on an open subset of Alexandrov space $A$.
Then  $f$ is differentiable at $p\in\Dom f$ if and only if $\mathfrak{a}_f(p)=\mathfrak{b}_f(p)$.
\end{thm}





\begin{thm}{Definition}

We say that a continuous semiconcave function $f$  defined
on an open subset $U$ of an Alexandrov space $A$
is \emph { roughly smooth}
and write $f\in rC^1(U)$
if 
for any compact $K\subset A$ there
is a constant $c(K)>0$, such that
$|\mathfrak{a}_f(p)-\mathfrak{b}_f(p)|< c(K)   $
for $p\in A^\delta\cap K$.
\end{thm}

\begin{thm}{Proposition}\label{prop:ab}
Let $A_n\to A$ be a convergence of Alexandrov spaces with uniform lower curvature bound
and $f_n$ be a sequence of continuous $\lambda$-concave functions defined on open domains in  $A_n$ that converges to a function $f$ defined on open domains in  $A$.

Then for any sequence of points $A_n\ni p_n\to p\in A$ we have
that 
\[\liminf_{n\to\infty}|\nabla_{p_n}f_n|\ge \mathfrak{a}_{f}(p)\]
and 
\[\limsup_{n\to\infty}|\nabla_{p_n}f_n|\le \mathfrak{b}_{f}(p)\]

In particular, if $f$ is differentiable at $p$, then 
\[\lim_{n\to\infty}|\nabla_{p_n}f_n|= |\nabla_{p}f|\]


and for   roughly smooth function $f$
and $p\in K$ we have
\[
\limsup_{n\to\infty}|\nabla_{p_n}f_n|-
\liminf_{n\to\infty}|\nabla_{p_n}f_n|
\le c(K)\delta. \]
 

\end{thm}




%Let    $M_n\in\M_{\ge -1}^m$,
% $M_n\GHto A$  and  $U\subset A$ be a good domain for this convergence.


 
 
 
 \subsection{Local version of the main Theorem }\label{sec:loc}
 
 
\begin{thm}{Main theorem (local)}\label{mainloc}
Let   
$M_n\in\M_{\ge -1}^m$,
$M_n\GHto A$, 
  $U\subset A$ be a good domain for this convergence
  and $U_n\to U$.
  Let sequences $f_{1,n},\dots,f_{m-2,n},\\ g_{1,n},\dots,g_{m-2,n}\in C^1(U_n)$ 
 converge to
   $f_1,\dots, f_{m-2}, g_1,\dots, g_{m-2}\in rC^1(U)$ correspondingly.
Then measures 
$${q}_M(\nabla f_{1,n},\nabla f_{2,n},\dots,\nabla f_{m-2,n},
\nabla g_{1,n},\nabla g_{2,n},\dots,\nabla g_{m-2,n})d\vol_m$$ weakly converges to some
measure on $U$.
\end{thm}

Theorem~\ref{main} could be reduced to the local version
in view of Lemma~\ref{testrC}.


\subsection{Measures are uniformly bounded}
Our proof strongly relies on the following result from
\cite{petrunin-SC}.
\begin{thm}{Theorem}\label{scPet}
Let $M$ be a complete Riemannian m-manifold with sectional
curvature $ \ge -1$. Then for any $p \in M$, $r<1$
$$\int_{B_r(p)} \Sc\le \const(m)\cdot r^{m-2},$$
where $B_r(p)$ denotes the ball of radius $r$ centered at $p\in M$ and $\Sc$ is a scalar curvature
of M.
\end{thm}

This result can be proved in the same way for
strongly inner subsets in $M\in\M_{\ge -1}^m$.
We denote by $K_{max}(x)=\sup_{\sigma\subset T_xM} |K_\sigma|$
for any point $x\in M$.
The straightforward consequence that we need is the following:

\begin{thm}{Corollary}\label{Kbound}
Let $M\in\M_{\ge -1}^m$ ,
$U\subset M$ is a strongly inner domain and
 $r<\min\{\diam U, 1\}$.
 Then for any $p\in U$,

$$\int_{B_r^M(p)} K_{max}\le \const(m)\cdot r^{m-2}.$$


\end{thm}






