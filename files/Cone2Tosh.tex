\section{Singularities of codimension 2}

Let $M$ be a 3-dimensional Riemannian manifold.
Given a smooth function $f$ denote, denote by $L_{(f=c)}$ the level set of $f$;
that is,
\[L_{(f=c)}=\set{x\in M}{f(x)=c}.\]
If the level set $L_{(f=c)}$ is a smooth surface in a neighborhood of $x\in L_{(f=c)}$,
then denote by $k_1(x)\le k_2(x)$ the principle curvatures of $L_{(f=c)}$ at $x$
and set 
\begin{align*}
G_f(x)&=k_1(x)\cdot k_2(x),
\\
H_f(x)&=k_1(x)+ k_2(x);
\end{align*}
that is, $G_f(x)$ and $H_f(x)$ are Gauss and mean curvature of $L_{(f=c)}$ at $x$.

\begin{thm}{Theorem}\label{thm:HG-converge}
Let $M_n$ be a sequence of $3$-dimensional Riemannian manifolds
and $f_n\: M_n\to \RR$ be a sequence of strongly concave functions.
Suppose that $\sec M_n\ge -\tfrac1n$ for each $n$, $M_n\zz\to \RR\times A$ without collapsing and $f_n$ converges to the $\RR$-coordinate in $\RR\times A$ as $n\to \infty$.
Then $G_{f_n}$ and $H_{f_n}$ weakly converge to zero.
\end{thm}

\parit{Proof.}
Choose $p\in \RR\times A$; set $a=f(p)$.

By theorem of Artem Nepechiy \cite{Nepechiy},
there is a $(-2)$-concave function $\rho$ defined in an $r$-neighborhood of $p$ such that $\rho(x)=-|p-x|^2+o(|p-x|^2)$.
Moreover, the function $\rho$ is \emph{liftable};
that is, there is a sequence of $(-2)$-concave $\rho_n\:M_n\to\RR$ that converges to $\rho$.

Consider a point $q\in \RR\times A$ above $p$; that is, its $\RR$-coordinate of is larger and its $A$-coordinate is the same.
If $\RR$-coordinate of $q$ is large then $\dist_q+f$ is $\lambda$-concave for small $\lambda>0$ and it has a nonstrict minimum at $p$.
Therefore, given $\lambda>0$, we can find $q$ so that the sum $s=f+\dist_q+\lambda\cdot \rho$ is $(-\lambda)$-concave and has strict maximum at $p$.
Moreover
\[-\lambda\cdot|p-x|_A^2\ge s(x)-s(p)\ge -\tfrac12\cdot\lambda\cdot|p-x|_A^2\]
and therefore
\[|\nabla_xs|\le 10\cdot\lambda\cdot|p-x|_A
\eqlbl{O(p-x)}\]
if $|p-x|_A$ is sufficiently small; say if $|p-x|_A\le \tfrac r{10}$.

Choose a sequence of points $q_n\in M_n$ that converges to $q$ and set $h_n\zz=\dist_{q_n}+\lambda\cdot \rho_n$;
this is a sequence of liftings of $h=\dist_q+\lambda\cdot \rho$. 
Observe that \ref{O(p-x)} implies that the first condition in \ref{thm:extimage-of-G-and-H} is met for all large $n$ in an $\tfrac r2$-neighborhood of $p_n$ with $\eps=10\cdot\lambda\cdot r$.
Moreover one can choose $b$ so that the second condition is satisfied and $B_n=B(p_n,r/10)\cap L_{(f=a)}\subset W_{a,b}$.
Applying \ref{thm:extimage-of-G-and-H}, we get that for any $\delta>0$, we have 
\[
\int_{B_n}G_{f_n}<\delta,
\quad\text{and}\quad
\int_{B_n}H_{f_n}<\delta.
\]
for all large $n$.
It remains to integrate the obtained inequalities by $a$ and pass to a limit as $n\to\infty$.
\qeds

\begin{thm}{Corollary}\label{cor:Ricci}
Suppose that $M_n\to  \RR\times A$ and $f_n\: M_n\to \RR$ be as in \ref{thm:HG-converge}.
Set $u_n=\tfrac{\nabla f_n}{|\nabla f_n|}$.
Then 

\begin{enumerate}
\item\label{cor:Ricci:Ricci} $\langle \Ric u_n,u_n\rangle$ weakly converges to zero.
\item\label{cor:Ricci:vw}
Let $v_n$ and $w_n$ are sequences of uniformly bounded, continuous vector fields on $M_n$.
Suppose that $\langle v_n,u_n\rangle$ and $\langle w_n,u_n\rangle$ converge uniformly as $n\to \infty$ to some constants $a$ and $b$ respectively.
Then 
\[q(v_n,w_n)\rightharpoonup a\cdot b\cdot \mathcal{H}_1\oplus \kappa,\]
where $\kappa$ is the curvature measure on $A$.

\end{enumerate}

\end{thm}

\parit{Proof; (\ref{cor:Ricci:Ricci}).}
Passing to a subsequence if nesessary, we can assume weak convergence $\langle \Ric u_n,u_n\rangle\rightharpoonup\mu$ to a measure $\mu$ on $A$.
Since $\sec M_n\ge -\tfrac1n$, we have that $\mu\ge 0$.
Therefore it is sufficient to show that $\mu\le 0$.

By \ref{thm:bochner-formula} we have that the following equality
\[\int\limits_\Omega \phi_n\cdot \langle \Ric u_n,u_n\rangle =
\int\limits_\Omega [\phi_n \cdot G_{f_n}+H_{f_n}\cdot\<u_n,\nabla\phi_n\>-\<\nabla\phi_n,\nabla_{u_n} {u_n}\>]
\]
holds for any function $\phi_n$ with compact support on $M_n$,
assuming that all expressions in the formula have sense.

It is sufficient to find a sequence of nonnegative function $\phi_n\:M_n\to \RR$ 
with compact support 
that converge to a $\phi\:A\to \RR$ such that (1) $\phi$ is unit in a neighborhood of a given point $p\in A$ and (2) we have a control the three terms on the right hand side of the formula; that is, we have the following weak convergences:
\[
\begin{aligned}
\phi_n \cdot G_{f_n}&\rightharpoonup 0,
&
H_{f_n}\cdot\<u_n,\nabla\phi_n\>&\rightharpoonup 0,
&
\<\nabla\phi_n,\nabla_{u_n} {u_n}\>&\rightharpoonup 0.
\end{aligned}
\eqlbl{eq:3->0}
\]

For the first convergence, it is sufficient to choose the sequence $\phi_n$ so that in addition it is universally bounded.
Indeed, according to \ref{lem:smooth-covergence}, $|\nabla f_n|\to 1$.
Therefore \ref{thm:HG-converge} and the coarea formula implies the first convergence in \ref{eq:3->0}.

Similarly, to prove the second convergence in \ref{eq:3->0}, it is sufficient to assume in addition that $|\nabla\phi_n|$ is universally bounded and apply \ref{thm:HG-converge} together with the coarea formula.

To prove the last convergence in \ref{eq:3->0}, note that \ref{lem:smooth-covergence} implies weak convergence
$|\nabla u|\rightharpoonup0$ away from the singular locus.
This observation is used to control the term $\<\nabla\phi_n,\nabla_{u_n} {u_n}\>$ at the points far from the singular locus of $A$ --- we only need to assume that $|\nabla\phi_n|$ is bounded.

Further, since $|u_n|=1$, we have $\nabla_{u_n} u_n\perp \nabla f_n$.
Therefore if $\nabla \phi_n$ is proportional to $\nabla f_n$ at some point, then at this point we have $\<\nabla\phi_n,\nabla_{u_n} {u_n}\>=0$.
This observation makes it possible to choose $\phi_n$ so that the term $\<\nabla\phi_n,\nabla_{u_n} {u_n}\>$ vanish around the singular locus of $A$.
Namely in addition to the above conditions on $\phi_n$ we have to assume that the identity $\phi_n=\psi\circ f_n$ holds at the points of $M_n$ that are sufficiently close to the singular locus of $A$.

Finally observe that the needed sequence exists ---
one can take 
\[\phi_n=(\sigma\circ \dist_q)\cdot \psi\cdot f_n\]
for appropriately chosen fixed mollifiers $\sigma,\psi\:\RR\to \RR$.

\parit{(\ref{cor:Ricci:vw}).}
Since $G_{f_n}\rightharpoonup 0$, we get that curvature measure of level sets of $f_n$ weakly converges to the curvaure of $A$.
It follows that 
\[q(u_n,u_n)\rightharpoonup \mathcal{H}_1\oplus \kappa.\]

Suppose $v'_n\perp u_n$ for all $n$.
Part (\ref{cor:Ricci:Ricci}) implies that  $q(v'_n,v'_n)\rightharpoonup 0$.

Fix $t\in \RR$.
Since the lower bound on sectional curvature of $M_n$ converges to $0$, any partial weak limit of $q(v'_n+t\cdot w_n,v'_n+t\cdot w_n)$ is nonnegative.
It follows that 
\[q(v'_n, w_n)\rightharpoonup 0\] for any sequence of fields $v'_n,w_n$ such that $v'_n\perp u_n$.

Consider the vector fields $v_n',w_n'$ such that 
\begin{align*}
v'_n&\perp u_n,
&
v_n&=a_n\cdot u_n+v_n',
\\
w'_n&\perp u_n,
&
w_n&=b_n\cdot u_n+w_n'.
\end{align*}
Since $q$ is bilinear, we get that
\[q(v_n,w_n)=q(v_n',w_n) + a_n\cdot [q(u_n,w_n')+ b_n\cdot q(u_n,u_n)].\]
By assumption, that $a_n=\langle u_n,v_n\rangle $ and $b_n=\langle u_n,w_n\rangle$ uniformly converge to $a$ and $b$ respectively.
Whence the statement follows.
 

\qeds


